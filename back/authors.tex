\addcontentsline{toc}{part}{Contributors}
\pagestyle{authors}
\chapter*{Contributors}
\protect\thispagestyle{chaptertitlepage}

% Short Bios of authors start here
% Each author gets 1 \paragraph
% Each \paragraph starts with author name
% This author name is formatted as the \paragraph title to make it pop out more (bold and some space)

\paragraph{Christian Baier} is an Associate Professor in the Department of German
Language and Literature at Seoul National University. He received his
PhD from the University of Bamberg, Germany, in 2011. In his
dissertation he analysed the concepts of genius in three major novels by
Thomas Mann, after which he published several articles on both the works
of Mann and the aesthetics of genius. Other research interests are the
theories of fictionality and narration, German Romanticism and the works
of Franz Kafka and Günter Grass. His current research focuses on the
concept of narrative, especially in its non-textual, discursive
variations.

\paragraph{Anne Baillot} is a full professor in German Studies at Le Mans Université (France). She teaches Literary Studies, History of Ideas and Digital Humanities. Her research focuses on Enlightenment, German Romanticism, Cultural Heritage Studies and Open Access. She is the main editor of the digital edition \emph{Letters and texts: Intellectual Berlin around 1800}.

\paragraph{Lamyk Bekius} is a PhD candidate in the project ‘Track Changes: Textual Scholarship and the Challenge of Digital Literary Writing’, which is a collaboration between the University of Antwerp and the Huygens ING, a research institute of The Royal Netherlands Academy of Arts and Sciences (KNAW) in Amsterdam. Her research focuses on how genetic criticism can be applied to born-digital material, and specifically to keystroke logging data. Since June 2021 she is also the coordinator of the University of Antwerp’s division of the CLARIAH-VL consortium, as well as that of the platform{DH}.

\paragraph{Manuela Bertone} is Professor of Italian Studies at the
Université Côte d'Azur, Nice. She is the author of \emph{Il romanzo come
sistema: Molteplicità e differenza in C.E. Gadda} (1993). With Robert S.
Dombroski she edited the collection of essays \emph{Carlo Emilio Gadda:
Contemporary Perspectives} (1997). She also published Gadda's \emph{I
Littoriali del lavoro e altri scritti giornalistici 1932-1941} (2005).
She has written extensively on Gadda for scholarly journals. She is a
member of the board of editors of \emph{Cahiers de Narratologie},
\emph{Cahiers d'études italiennes,} and \emph{The Edinburgh Journal of
Gadda Studies}.

\paragraph{Elli Bleeker} is a researcher at the Huygens Institute for the History of the Netherlands. She specializes in digital scholarly editing and computational philology, with a focus on modern manuscripts and genetic criticism. Elli completed her PhD in 2017 at the University of Antwerp on the role of the scholarly editor in the digital environment. During her Early Stage Research Fellowship in the Marie Skłodowska-Curie funded Digital Scholarly Editions Initial Training Network (DiXiT ITN, 2014–2017), she received advanced training in manuscript studies, text modeling, and XML technologies. She has participated in the organization and teaching of workshops on scholarly editing with a focus on knowledge transfer and the application of computational methods in a humanities environment. She is also the Associate Editor of Variants from this issue onwards.

\paragraph{Mark Bland} is an independent bibliographical scholar specialising on the London book-trade 1560-1640 and the circulation of manuscript poetry in that period. He is the author of \textit{A Guide to Early Printed Books and Manuscripts} and numerous articles on paper, the book-trade, and manuscript studies as well as Ben Jonson. He is the Oxford editor of a new edition of \textit{The Poems of Ben Jonson} and is presently finishing a monograph on \textit{The World of Simon Waterson, Stationer}.

\paragraph{Ronald Broude} is principal of the music publisher Broude Brothers Limited and Founding Trustee of The Broude Trust for the Publication of Musicological Editions, a non-profit publisher of scholarly editions of music. His research has appeared in musicological journals such as \textit{Early Music, Notes}, and \textit{The Musical Times}, and text-critical periodicals such as \textit{Textual Cultures} and \textit{Scholarly Editing}. He is a member of the Executive Committee of the Society for Textual Scholarship, and he served as the Executive Director of that organization from 2004 to 2005. In 2009, his paper on the sources of the Savoy Operas was awarded the Association for Documentary Editing’s Boydston Prize.

\paragraph{Anna Busch} is a postdoctoral researcher at the Theodor-Fontane-Archive at the University of Potsdam. She manages the digital projects of the archive and is responsible for the digital presentation and analysis of the collections. She works on digital editions, collation tools, visualizations of text geneses and cultural heritage data. She is an editor for the digital edition \emph{Letters and texts: Intellectual Berlin around 1800}.

\paragraph{Barbara Cooke} is Co-Executive Editor of the Oxford University Press's
\emph{Complete Works of Evelyn Waugh}. She has co-edited Waugh's
autobiography \emph{A Little Learning} (1964) for the edition, and is at
work on the semi-autobiographical novel \emph{The Ordeal of Gilbert
Pinfold} (1957). Her chapter on ``Organising a Large Edition'' is
published in \emph{A Handbook of Editing Early Modern Texts} (2017). She
is a lecturer in English at Loughborough University, where she leads on
research strategy for Textual Editing and Interpretation.

\paragraph{Wout Dillen} is a Senior Lecturer in Library and Information Science at the University of Borås, Sweden. In 2015, he defended his PhD in Literature on ``Digital Scholarly Editing for the Genetic Orientation'' at the the University of Antwerp. As part of his PhD project, Wout also developed the Lexicon for Scholarly Editing (\url{https://lexiconse.uantwerpen.be}
). From 2016 to 2017, Wout held a Marie Skłodowska-Curie Experienced Research Fellowship in the Digital Scholarly Editions Initial Training Network (DiXiT ITN) at the University of Borås. From 2016 to 2021, he was the Antwerp Coordinator of CLARIAH-VL, a Flemish digital research infrastructure project that contributes to DARIAH and CLARIN. Wout currently serves as the Secretary of the European Society of Textual Scholarship (ESTS), and as a member of the Executive Committee of the DH Benelux. Wout was an Associate Editor of \emph{Variants} 14, and is the General Editor from the current issue onwards. Besides these, he is also on the editorial boards of the Review Journal for Digital Editions and Resources (RIDE) and the Journal of the DH Benelux.

\paragraph{Michelle Doran} is a postdoctoral Research Fellow at Trinity Long Room Hub and Project Officer for the Trinity Centre of Digital Humanities. She holds a PhD in Medieval Irish Studies, and her principal research interests lie in the field of (digital) humanities research and the underlying epistemological and ideological premises. She is module coordinator of the Digital Scholarship and Skills workshop series hosted by the Trinity Long Room Hub and facilitates a number of workshops on the subjects of Digital Humanities, Data Management Planning and Digital Scholarly Editing.

\paragraph{Hans Walter Gabler} is Professor of English Literature (retired) at the
University of Munich, Germany, Senior Research Fellow of the Institute
of English Studies, School of Advanced Study, University of London, and
Doctor of Literature, \emph{honoris causa}, from the National University
of Ireland, Maynooth. From 1996 to 2002 in Munich, he directed an
interdisciplinary graduate programme on ``Textual Criticism as
Foundation and Method of the Historical Disciplines''. He is
editor-in-chief of the critical editions of James Joyce's \emph{Ulysses}
(1984/86), \emph{A Portrait of the Artist as a Young Man}, and
\emph{Dubliners} (both 1993). His present research continues to be
directed towards the writing processes in draft manuscripts and their
representation in the digital medium. Editorial theory, digital editing
and genetic criticism have become the main focus of his professional
writing.

\paragraph{Laura Esteban-Segura} is a Senior Lecturer at the Department of English, French and German Philology of the University of Málaga (Spain). She holds a Master of Letters in English Language and English Linguistics awarded by the University of Glasgow and a PhD in English Philology by the University of Málaga. From 2008 to 2015 she was based at the University of Murcia (Spain). She has been a Visiting Researcher at the Department of English Language of the University of Glasgow on several occasions, at the Research Unit for Variation, Contacts and Change in English (University of Helsinki), and at the Department of Cultural Studies and Languages of the University of Stavanger. She has been a member of the board of SELIM (the Spanish Society for Mediaeval English Language and Literature) (2012-2016) and Managing Editor of \emph{Atlantis. Journal of the Spanish Association of Anglo-American Studies} (2015-2018). Her research interests are English Historical Linguistics, Manuscript Studies and Textual Editing. She is a member of several research projects devoted to the electronic editing of Late Middle, Early Modern and Late Modern English \emph{Fachprosa}.


\paragraph{Anthony John Lappin} was born 1968 in Liverpool, gained his doctorate at the University of Oxford, and has taught in universities in the UK, Ireland and latterly Sweden, where he currently lives. He has also published a long article on the transmission of \emph{Donne’s Valediction: forbidding mourning in Studia Neophilologica (2020)}.

\paragraph{Hugo Maat} is a guest researcher in the Research and Development Team at the Humanities Cluster, part of the Royal Netherlands Academy of Arts and Sciences. He specializes in Early Modern Cultural History and completed his MA in 2020.

\paragraph{Dariusz Pachocki} is a professor at the John Paul II Catholic University of Lublin, and the author of academic works and articles in text criticism, literary criticism, and literary studies. He edited the works by Edward Stachura, Józef Czechowicz, Bolesław Leśmian, Stanisław Czycz, Władysław Broniewski, and Leopold Tyrmand. He received the Feniks award from the Association of Catholic Publishers twice for his editorial work, and a medal of President of the City of Lublin. He received scholarships from: Foundation for Polish Science, the Minister of Culture and National Heritage, the Kosciuszko Foundation, Harry Ransom Center (Austin, USA), Hoover Institution (Stanford, USA).

\paragraph{Jonas Rosenbrück} is currently a Postdoctoral Fellow in Comparative
Literary Studies at Northwestern University (Evanston, USA). His work on
German, French, and Anglophone literatures, aesthetics, and poetics has
appeared or is forthcoming in \emph{Comparative Literature}, \emph{CR:
The New Centennial Review}, and \emph{The Germanic Review: Literature,
Culture, Theory.} His current book project is titled \emph{A Revolution
of the Senses: Odor and Modern Poetry.} Until recently, Rosenbrück has
served as Director of Volunteer Development for the Northwestern Prison
Education Program.

\paragraph{Stefano Rosignoli} received an MA in Modern Literature (2006) and an
MPhil in Publishing Studies (2008) from the University of Bologna. From
2008 to 2015 he focused on trade publishing in Italy and the UK while
taking the first steps towards his PhD in English, which he is
completing at Trinity College Dublin. Stefano's academic education is
grounded in textual studies at large, from philology to genetic
criticism, balanced by formalism, structuralism and the semiotics of
texts, and his research examines the philosophical exogenesis of Irish
literature in English. He has recent or forthcoming publications on
Samuel Beckett, T. S. Eliot and James Joyce; he teaches modern
literature and theory at Trinity College and University College, in
Dublin; and serves as Review Editor for \emph{Variants: The Journal of
the European Society for Textual Scholarship}. In 2018, he has been a
James Joyce Visiting Fellow and J--1 Short-Term Scholar at the
Humanities Institute, State University of New York at Buffalo, and a
visiting research scholar at Rare and Manuscript Collections, Cornell
University Library, Cornell University.

\paragraph{Paulius V. Subačius} is a professor at Vilnius University, born in Lithuania, in 1968. He studied Lithuanian language and literature and defended a doctoral thesis in 1996. As a participant in the European Society for Textual Scholarship (form 2003 onwards), he has presented papers at thirteen of its last conferences. He is the author or editor of sixteen books (incl. \emph{Textual Criticism: Guidelines of Theory and Practice}, 2001; \emph{Antanas Baranauskas: The Text of Life and the Lives of Texts}, 2010; \emph{Twenty-five Years of Religious Freedom, 2016}; and several critical editions of Lithuanian authors, all in Lithuanian). He has also published multiple articles in English in \textit{Variants}, \textit{Editio}, \textit{Textual Cultures}, \textit{Filologia} XXI.

\paragraph{Dirk Van Hulle} is Professor of Bibliography and Modern Book History at the University of Oxford and director of the Centre for Manuscript Genetics at the University of Antwerp. With Mark Nixon, he is co-director of the Beckett Digital Manuscript Project (\url{www.beckettarchive.org}), series editor of the Cambridge UP series ‘Elements in Beckett Studies’ and editor of the \textit{Journal of Beckett Studies}. His publications include \textit{Textual Awareness} (2004), \textit{Modern Manuscripts} (2014), \textit{Samuel Beckett’s Library} (2013, with Mark Nixon), \textit{The New Cambridge Companion to Samuel Beckett} (2015), \textit{James Joyce’s Work in Progress} (2016), the \textit{Beckett Digital Library} and a number of volumes in the ‘Making of’ series (Bloomsbury) and genetic editions in the \textit{Beckett Digital Manuscript Project}, which won the 2019 Prize for a Bibliography, Archive or Digital Project of the Modern Language Association (MLA).