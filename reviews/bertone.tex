%%%%%%%%%%%%%%
%% METADATA %%
%%%%%%%%%%%%%%

\contributor{Manuela Bertone}

\contribution{Carlo Emilio Gadda, \emph{Quer pasticciaccio brutto de via Merulana}}

\begin{review}
\renewcommand*{\pagemark}{}

%%%%%%%%%%%%%%%%%%%%%%%%%%%%%%%%%%%%%%
%% DESCRIPTION OF THE REVIEWED BOOK %%
%%%%%%%%%%%%%%%%%%%%%%%%%%%%%%%%%%%%%%

\begin{reviewed}
Review of \thecontribution. Ed. Giorgio Pinotti. Milano: Adelphi, 2018. 370 pp. ISBN:
978--8--84--593306--6.
\end{reviewed}

%%%%%%%%%%%%%%%%%%%%%%%%%%%%%
%% YOUR REVIEW STARTS HERE %%
%%%%%%%%%%%%%%%%%%%%%%%%%%%%%

% remove asterisk (*) if you want to number your sections
% add a title for your section in between the {curly brackets} if you need one
\section*{} 
Giorgio Pinotti, a refined connoisseur of Carlo Emilio Gadda, offers again
in the new edition of the works of the Milanese writer, which Adelphi has been publishing for a few years, a text of \emph{Quer pasticciaccio brutto de
via Merulana} {[}That Awful Mess on Via Merulana{]} which is
identical to the text which we are already familiar with, being
based on the second reprint, revised and amended by the author, issued
as a volume-publication in September 1957, soon after the first one,
available from the end of July of that same year.

In the absence of textual variants suited to entice Pinotti's
philological talent, he still gifts us a useful
``Nota al testo'' {[}Note on the Text{]} of approximately fifty
pages, where he revisits the troubled composition of the novel,
beginning with the 1945 Roman crime news that inspired Gadda to pen
the novel, up until the present edition.

As Pinotti unravels the textual history for us (a sort of novel
within the novel, one might say) in an impeccable chronological order,
abounding in dramatic twists, he shares several novelties
consisting in a few flashes from documents rich in sketches and working
notes, recently unearthed among the Gadda papers preserved at the
Archivio Liberati in Villafranca di Verona. Among the most significant,
we recommend four photographs in the Ager Romanus by Gadda
himself (1953), while he was exploring the Latian countryside in order
to achieve a more mimetic representation of the territory. A
curiously suggestive ``finale `imperfetto'\,'' {[}``imperfect'' ending{]} \citep[322; quoting Gadda]{pinotti_nota_2018}, handwritten on a squared
sheet, is another find. It was most likely intended to seal a hypothetical second volume of the
novel which, as we know, Gadda never authored, hence disappointing the
expectations of his insistent and impatient publisher Livio Garzanti. A
``splendido brano'' {[}magnificent passage{]} of
high lyrical tone, a single printed page long, whose reading we strongly
recommend \citep[323]{pinotti_nota_2018}.

From the papers now resurfaced, attentively quoted but not exhibited by
Pinotti, emerges the confirmation that Gadda strived to
overcome structural concerns and imagine solutions regarding the ending
not just of the never released second volume but of even that one, the
only one, we are familiar with. However, Gadda's adumbrated hypotheses
and his cherished solutions (honestly, not many) were, as a matter of
fact, set aside and the \emph{Pasticciaccio}, among short schemas
and early drafts, pursuits of connections, completions and mends, ended
up being left untouched as it is, with a conclusion that might seem
provisional, but which Gadda decided, with unambiguous authority, to
make definitive.

To tell the truth, upon examining the documentary fragments produced by
Pinotti, we are rather confirmed in our belief that Gadda had no
intention to send a sequel to press. Ready to make any pledge to
his publishers as long as he could break free from their pressing
demands and injunctions, the writer did not feel, like Garzanti, the
necessity to repeat the commercial success of the novel. Pinotti, one of
the few interpreters of the \emph{Pasticciaccio} who, once again,
believes that Gadda intended to end his crime fiction novel by revealing
the culprit, as already argued in his annotated edition of the
correspondence between Gadda and Pietro Citati \citep{gadda_gomitolo_2013}, insists
here that the auto-exegetic pronouncements which Gadda offered through
the years, aiming to defend his choice not to close the book, are
pretextuous. Pinotti, therefore, confirms that in his opinion the
author's explanations should be construed as misleading statements, as
attempts to ascribe, from the sixties onwards, a narrative
\emph{défaillance} ``a un meditato disegno, trasformando così la resa in
vittoria'' {[}to a pondered design, in order to turn
surrender into victory{]} \citep[347]{pinotti_nota_2018}. However, Pinotti later admits, with delightful
acrobatics, that the authorial statements are justified by the fearless
solution which Gadda devised for the ending, and recovers, surprisingly,
a famous pronouncement by Giancarlo Roscioni --- who, for that matter,
was hardly \emph{flatteur} towards a narrator of so travelled craftiness
such as Gadda --- according to whom the unfinished nature of the work
should be ascribed to the fatigue of the chess player who, ``dopo aver
cercato di prevedere il maggior numero possibile di mosse {[}\dots{}{]},
sposti la prima pedina che lo liberi dal compito di ulteriormente
riflettere e decidere''
{[}having attempted to envisage the highest number of moves {[}\dots{}{]},
opts for the first piece which can free him from the task of thinking
and deciding any further{]} (\citealt[91--92]{roscioni_disarmonia_1975}; see \citealt[347]{pinotti_nota_2018}). But is it really likely that Gadda put
so much effort retrospectively in fabricating and then disseminating
nothing more than a captivating meta-textual tale in order to cover up a
creative defeat and avoid the admission of guilt? As a matter of fact,
the conveyor of the most plausible answer to this question is still
Gianfranco Contini, who suggested taking authorial intention duly into
account, observing the following:

\begin{quote}

Tutti sanno che il \emph{Pasticciaccio} è, anche nell'ultima edizione,
un libro incompiuto; ma, precisazione ben più importante, un libro, se
non proprio così impostato intenzionalmente, accettato deliberatamente
come incompiuto. Se la fine sia stata soltanto abbozzata o non abbia
proceduto oltre la concezione mentale, è cosa del tutto secondaria di
fronte a quest'accettazione, che Gadda difese (per la verità senza
troppa fatica) contro le insistenze editoriali.

{[}Everybody knows that the \emph{Pasticciaccio} is, even in its last
edition, an unfinished book; but, as it is far more important to
clarify, a book which was, if not quite intentionally so configured,
still deliberately accepted as unfinished. Whether the ending was only
outlined or it never progressed beyond the mental conception, is
definitely of secondary importance \emph{vis-à-vis} such acceptance,
which Gadda defended (as a matter of fact without too much effort),
against the insistence of his publisher.{]}

\begin{flushright}
\citep[45--46]{contini_quarantanni_1989}
\end{flushright}

\end{quote} 

\noindent Ultimately, it is not surprising that the great scholar's assessment radically differs from that of great editors such as
Citati (in the past) and Pinotti (nowadays): exactly like the great
publishing houses which employ them, they ardently desire to
unearth abundant unpublished materials to be sent to press, to the
extent of always hoping that a manuscript perhaps vaguely promised
one day materializes as a substantial rediscovered autograph. Contini
instead assessed Gadda's discourse around the novel for what it is:
perhaps just an act, but an authorial act, more assertive than evasive.
And indeed, if one considers Gadda's statements about his
novel, it definitely does not look like his thoughts on the conclusion
of the \emph{Pasticciaccio} tended towards the digression or the
misdirection, nor even towards the \emph{pis-aller} of the drained chess
player. On the contrary, one might argue that Gadda attempted to
rigorously support the reasons for an originality not suffered and far
from casual. And in any case, Pinotti himself, for the sake of
transparency, reminds us that since as early as 1958 Gadda wrote to his
cousin Piero Gadda Conti that he did not wish to hear ever again of the
\emph{Pasticciaccio}, also acknowledging: ``resterà dunque una
chimera la continuazione-fine auspicata dall'editore'' {[}the continuation-ending yearned for
by the publisher will therefore remain a chimera{]} (\citealt[161]{gadda_conti_confessioni_1974}; see \citealt[346]{pinotti_nota_2018}).

It is not necessary to eulogize Gadda's Roman
novel yet again, since it has already been widely reviewed through the decades and has been the object of numerous
critical studies produced by specialists and mostly meant for them. To
those who intend to approach this indispensable
classic of European literature for the first time --- a work which has come back to the fore
precisely due to Adelphi's well-timed new edition, but which
remains arduously decipherable for the reader not accustomed to Gadda's
extremely difficult prose --- it may be suggested to draw on tools which,
together with the already quoted ``Nota al testo'', can further the understanding of the novel. In the first instance the sturdy and valuable
\emph{Commento} {[}Commentary{]} of the \emph{Pasticciaccio}, an
impressive work (2 vols., 1184 pp.) coordinated by Maria Antonietta
Terzoli (\citeyear{terzoli_commento_2015}), which eviscerates the text line by line, with glosses of
various length and intensity: a useful encyclopaedia that may be
consulted to escape doubts and uncertainties, but also, if one
wishes to go further in-depth, in order to discover extraordinary
layerings of meaning not immediately perceptible. Whoever wishes to
tackle the novel using more manageable aids will instead have the chance
to draw on two additional volumes, prepared again by Terzoli together
with the pupils attending her Romanisches Seminar at the University of
Basel: to the first one, \emph{Un meraviglioso ordegno. Paradigmi e
modelli nel} Pasticciaccio \emph{di Gadda} {[}A Marvelous Device:
Paradigms and Models in Gadda's \emph{Pasticciaccio}{]} (\citeyear{terzoli_meraviglioso_2013}), is entrusted a
broad and new critical analysis, involving multiple voices, which
preludes to the publication of the \emph{Commento}; the second one,
\emph{Gadda: guida al} Pasticciaccio {[}Gadda: A Guide to the
\emph{Pasticciaccio}{]} (\citeyear{terzoli_gadda_2016}), suggests a first approach to the novel,
offering a chapter-by-chapter reading which allows even the most
inexperienced to find their way through the maze-like plot and the equally
maze-like nooks and crannies of the Latian countryside and of the
tenement in Via Merulana where the narrative yarn tangles into an awful mess.

\begin{flushleft}
\bibliography{references/bertone}  
\end{flushleft}

\end{review}