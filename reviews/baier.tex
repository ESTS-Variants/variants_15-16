\contributor{Christian Baier}

\contribution{Thomas Mann, \emph{Joseph und seine Brüder. Text und Kommentar}}

\begin{review}
\renewcommand*{\pagemark}{}

%%%%%%%%%%%%%%%%%%%%%%%%%%%%%%%%%%%%%%
%% DESCRIPTION OF THE REVIEWED BOOK %%
%%%%%%%%%%%%%%%%%%%%%%%%%%%%%%%%%%%%%%

\begin{reviewed}
Review of \thecontribution. 2 vols. in 2 pts. Eds. Jan Assmann, et al. (Vols. 7(1--2) and 8(1--2) of \emph{Große kommentierte Frankfurter Ausgabe. Werke --- Briefe ---
Tagebücher}. Eds. Heinrich Detering, et al.) Frankfurt am Main: S.
Fischer Verlag, 2018. 1923 pp. (vols. 7(1)--8(1), continuously paginated) and 2091 pp. (vols. 7(2)--8(2), continuously paginated). ISBN 978--3--10--048330--0 (vol. 7) and 978--3--10--048333--1 (vol. 8).
\end{reviewed}


%%%%%%%%%%%%%%%%%%%%%%%%%%%%%
%% YOUR REVIEW STARTS HERE %%
%%%%%%%%%%%%%%%%%%%%%%%%%%%%%

% remove asterisk (*) if you want to number your sections
% add a title for your section in between the {curly brackets} if you need one
\section*{} 
The \emph{Große kommentierte Frankfurter Ausgabe} {[}Large
Annotated Frankfurt Edition{]} of Thomas Mann's collected writings is a
monumental editorial project. Having started in 2001 with the
publication of \emph{Buddenbrooks}, the complete edition is projected to
encompass thirty-eight volumes of critical texts including literary
works, essays, letters, and diaries. Each of these volumes also contains
an extensive commentary, annotations, and selected materials. In the
case of \emph{Joseph und seine Brüder} {[}Joseph and His
Brothers{]}, these materials include excerpts from Thomas Mann's work
notes, the Joseph story (\emph{Gen.} 27--50) from Mann's family bible
complete with marginalia, and a selection of letters from experts
advising the author on relevant topics. The volumes are curated and
edited by renowned Thomas Mann expert Dieter Borchmeyer, Egyptologist
and religious study scholar Jan Assmann, and Stephan Stachorski, who
previously co-edited with Hermann Kurzke the six-volume edition of
Mann's \emph{Essays} (\citeyear{mann_essays_1993}).

Thomas Mann's \emph{opus maximum}, the biblical tetralogy \emph{Joseph
und seine Brüder}, is the most recent addition to the Frankfurt
annotated editorial project. The editors sensibly decided to divide the
two-thousand-page novel into two volumes, with \emph{Die Geschichten
Jaakobs} {[}The Stories of Jacob{]} (1933) and \emph{Der junge
Joseph} {[}Young Joseph{]} (1934) comprising volume 7(1), while
volume 8(1) includes \emph{Joseph in Ägypten} {[}Joseph in
Egypt{]} (1936) and \emph{Joseph der Ernährer} {[}Joseph the
Provider{]} (1943). This seemingly obvious distribution already
constitutes a significant improvement over the \emph{Gesammelte Werke in
dreizehn Bänden} {[}Collected Works in Thirteen Volumes{]}
(\citeyear{mann_gesammelte_1974}): in this earlier authoritative edition, the biblical novel is
similarly split between volumes IV and V, but for some incomprehensible
reason volume IV randomly ends with Part Four of \emph{Joseph in
Ägypten}.

In the Frankfurt edition of \emph{Joseph und seine Brüder}, the critical
text is based on the first printed edition of the novel rather than on
the manuscripts. The textual emendations to previous editions, while
numerous, are mostly inconspicuous, if frequently amusing: in \emph{Der
junge Joseph}, for example, the editors correct the blunder of one of
Thomas Mann's more infamous secretaries, who had mistakenly transcribed
the phrase ``Verkehrstrubel der Reisestraße'' {[}bustling traffic on
Egypt's highway \citep[1055]{mann_joseph_2005}{]} as
``Reisetrubel der Verkehrsstraße'' {[}bustling crowd on Egypt's traffic
way{]} (Mann 2018, 7(2): 213). There are, however, two
notable exceptions: for the first time since the very first edition, the
\emph{Vorspiel: Höllenfahrt} {[}Prelude: Descent into Hell{]},
preceding \emph{Die Geschichten Jaakobs}, is separately paginated in
Roman numerals. Secondly, a single comma was deliberately deleted in the
title of the fourth and final novel, changing it from \emph{Joseph, der
Ernährer} to \emph{Joseph der Ernährer} --- a significant difference
according to the editors, since it turns a nominal attribute into a part
of a proper name, following the style of rulers such as William the
Conqueror (see 7(2): 211).

The overall sparseness of significant textual corrections is all the
more notable if one takes into account the work's rather turbulent
genesis in the years between 1926 and 1943. On a personal level, this
period saw Mann's decision to not return to Germany in 1933, the loss of
his German citizenship in 1936, and his life in exile, first in
Switzerland and then in the U.S. Considering these circumstances, it is
surprising that the first two volumes of the novel could still be
published in Berlin in 1933 and 1934, while the third and fourth volume
had to be published in Vienna in 1936 and Stockholm in 1943 respectively
after the forced relocation of Thomas Mann's Jewish publisher Gottfried
Bermann Fischer.

More complex even than the publication process was the tetralogy's
reception. As the Frankfurt Edition documents it extensively, it
organizes the material geographically while also considering the
reviewers' socio-political or ideological backgrounds: domestic
criticism is kept separate from the reactions of the German exile press,
which in turn is distinguished from the reception in German-speaking
countries as well as in the rest of Europe and in the U.S.
Simultaneously, the editors quote bourgeois-conservative reviewers next
to right-wing polemicists and Catholic critics, contrasting their
opinions with the overwhelmingly positive reactions from Jewish
intellectuals to \emph{Die Geschichten Jaakobs} --- reactions that,
surprisingly, could still be voiced in Germany in 1933.

This edition's most important contribution to Thomas Mann scholarship,
however, lies in the extensive annotations that constitute the core of
the commentary in each volume: a total of almost one thousand pages of
detailed explanations of words, concepts and circumstances related to
(among countless other topics) the biblical source material and its
variations in the Islamic tradition; Thomas Mann's archaic use of
language; his eclectic research on Ancient Egypt, the religions of the
Middle East, and the gnosis; as well as the influence of Richard Wagner,
Johann Wolfgang von Goethe, and Franklin D. Roosevelt on Mann's epic
re-imagining of the Joseph story. The depth and variety of these
elucidations are especially apparent when juxtaposed with the previously
authoritative reference work, Bernd-Jürgen Fischer's \emph{Handbuch zu
Thomas Manns ``Josephsromanen''} {[}Handbook of Thomas Mann's
``Joseph Novels''{]} (\citeyear{fischer_handbuch_2002}). This comparison is not, in any way,
intended as a criticism of Fischer's work: his compilation is a valuable
resource, at the same time clearly illustrating that no single scholar
is able to do justice to all the topics, motives and interwoven
references constituting the fictional world of \emph{Joseph und seine
Brüder} --- a task which Assmann, Borchmeyer and Stachorski accomplish
with erudition and diligence.

The new edition of Thomas Mann's biblical tetralogy upholds in every way
the high editorial standards that have distinguished the \emph{Große
kommentierte Frankfurter Ausgabe} since its inception in 2002. With its
well-structured abundance of factual knowledge, interpretative insights
and selected materials, it is a treasure trove for any Thomas Mann
enthusiast, and an invaluable resource for all future scholarship on
\emph{Joseph und seine Brüder}.

\begin{flushleft}
\bibliography{references/baier}  
\end{flushleft}

\end{review}