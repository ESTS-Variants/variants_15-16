%%%%%%%%%%%%%%
%% METADATA %%
%%%%%%%%%%%%%%

\contributor{Stefano Rosignoli}
\contribution{Tracing ``Auto(bio)graphy'' in ``Three Novels'' by Samuel
Beckett: A Review Essay}

\shortcontributor{Stefano Rosignoli}
\shortcontribution{Tracing ``Auto(bio)graphy'' in ``Three Novels''}

\begin{paper}
\renewcommand*{\pagemark}{}

%%%%%%%%%%%%%%%%%%%%%%%%%%%%%%%%%%%%%%
%% DESCRIPTION OF THE REVIEWED BOOK %%
%%%%%%%%%%%%%%%%%%%%%%%%%%%%%%%%%%%%%%

\begin{reviewed}

Review essay of Samuel Beckett, \emph{The Beckett Digital Manuscript Project}. Instalments 2, 4, and 5 (digital modules and accompanying printed volumes). Directed by Dirk Van Hulle and Mark Nixon. Technical Realisation by Vincent Neyt. Brussels: University Press Antwerp (ASP/UPA); London: Bloomsbury, 2013--17. 

\paragraph{\footnotesize Instalment 2: \emph{L'Innommable / The Unnamable}}
    \begin{itemize}
        \item Samuel Beckett, \emph{\emph{L'Innommable / The Unnamable}: A Digital Genetic
Edition}. Eds. Dirk Van Hulle, Shane Weller and Vincent Neyt. Brussels:
University Press Antwerp (ASP/UPA), 2013\newline
\textless{}\url{http://www.beckettarchive.org}\textgreater.
        \item Dirk Van Hulle and Shane Weller, \emph{The Making of Samuel Beckett's}
L'Innommable \emph{/} The Unnamable. Brussels: University Press Antwerp;
London: Bloomsbury, 2014. 272 pp. ISBNs: 978--90--5718--181--8 (ASP/UPA);
978--1--4725--2951--0 (Bloomsbury).
    \end{itemize}
\paragraph{\footnotesize Instalment 4: \emph{Molloy}}
    \begin{itemize}
        \item Samuel Beckett, \emph{\emph{Molloy}: A Digital Genetic Edition}. Eds. Édouard Magessa
O'Reilly, Dirk Van Hulle, Pim Verhulst and Vincent Neyt. Brussels:
University Press Antwerp (ASP/UPA), 2016\newline
\textless{}\url{http://www.beckettarchive.org}\textgreater.
        \item Édouard Magessa O'Reilly, Dirk Van Hulle and Pim Verhulst, \emph{The
Making of Samuel Beckett's} Molloy. Brussels: University Press Antwerp;
London: Bloomsbury, 2017. 416 pp. ISBNs: 978--90--5718--536--6 (ASP/UPA);
978--1--4725--3256--5 (Bloomsbury).
    \end{itemize}
\paragraph{\footnotesize Instalment 5: \emph{Malone meurt / Malone Dies}}
    \begin{itemize}
        \item Samuel Beckett, \emph{\emph{Malone meurt / Malone Dies}: A Digital Genetic
Edition}. Eds. Dirk Van Hulle, Pim Verhulst and Vincent Neyt. Brussels:
University Press Antwerp (ASP/UPA), 2017\newline
\textless{}\url{http://www.beckettarchive.org}\textgreater.
        \item Dirk Van Hulle and Pim Verhulst, \emph{The Making of Samuel Beckett's}
Malone meurt \emph{/} Malone Dies. Brussels: University Press Antwerp;
London: Bloomsbury, 2017. 336 pp. ISBNs: 978--90--5718--537--3 (ASP/UPA);
978--1--4725--2344--0 (Bloomsbury).
    \end{itemize}

\end{reviewed}

%%%%%%%%%%%%%%%%%%%%%%%%%%%%%
%% YOUR REVIEW STARTS HERE %%
%%%%%%%%%%%%%%%%%%%%%%%%%%%%%

% remove asterisk (*) if you want to number your sections
% add a title for your section in between the {curly brackets} if you need one
\section*{\centerheading{I}}
\vskip 1em 

Between 2014 and 2017, University Press Antwerp and Bloomsbury have
gifted the community of Beckett studies with three essential research
tools which\emph{,} as part of the \emph{BDMP --- Beckett Digital
Manuscript Project,} examine \emph{Molloy}, \emph{Malone meurt} /
\emph{Malone Dies} and \emph{L'Innommable} / \emph{The Unnamable} from
the perspective of their textual development. Those readers who are
not yet familiar with the \emph{BDMP} might rely on two reviews already
published by \emph{Variants} \citep{bailey_review_2013,mcmullan_review_2019}, but
most importantly on the ``Series Preface'' authored by the co-directors
of the project. The \emph{BDMP} is presented there as a collaborative
research endeavour in twenty-six instalments which involves the universities of
Antwerp, Reading and Texas at Austin, grows out of two genetic or
variorum editorial initiatives, and is dedicated to the study of
Beckett's manuscripts in the light of genetic criticism and digital
scholarship  \citep[7]{van_hulle_series_2011}. Since the publication of its
first research output ten years ago \citep{beckett_stirrings_2011,van_hulle_making_2011},
this extensive effort of description and interpretation of archival
sources has become the most prominent attempt to trace and address the
genesis of Beckett's original works, with a focus on the
\emph{avant-texte} in the digital modules of the project.

While the \emph{BDMP}, as a whole, is the result of two turnings in
Beckett scholarship --- the archival and the digital --- the instalments
of the \emph{BDMP} dedicated to his ``three novels'', specifically,
revolve around the two turnings in Beckett's literary path which marked
his ``frenzy of writing'' or ``siege in the room'' in 1946\emph{--}50 \citep[qtd. in][309]{oreilly_making_2017}:

\begin{enumerate}
\def\labelenumi{\arabic{enumi}.}
\item
  \emph{The turn from omnipotence to impotence, in the contents
  expressed:} reacting against James Joyce's attempt to grasp all
  existence by endlessly enriching the text, Beckett aimed to express an
  existential impasse by impoverishing his work in terms of characters,
  plots and motives (\citealt[351--53]{knowlson_damned_1996}; see also \citealt[25]{oreilly_making_2017}).
\item
  \emph{The turn from English to French language, in the form of
  expression:} Beckett begun to write in French not merely due to his
  decision to live in France permanently, but also in an attempt to
  pursue, at stylistic level, the same impoverishment which he strived for
  in his narratives (\citealt[356--58]{knowlson_damned_1996}; see also \citealt[25]{oreilly_making_2017}).
\end{enumerate}

\noindent An outcome of both turnings, \emph{Molloy}, \emph{Malone meurt} and
\emph{L'Innommable} were composed in 1947\emph{--}50 and first published
in the original French in 1951\emph{--}53, establishing a long-lasting
commercial bond with Jérôme Lindon at Les Éditions de Minuit, followed
by a set of international publishers which issued Beckett's work for the
remainder of his literary career. Beckett translated \emph{Molloy} in
English with Patrick Bowles, but he worked alone on \emph{Malone meurt},
initially deeming the task ``child's play after the Bowles revision'' in
a letter to Barney Rosset of 18 October 1954 \citep[2: 507]{beckett_letters_2009},
and on \emph{L'Innommable}, calling the demanding endeavour an
``impossible job'' in a letter to Pamela Mitchell of 12 March 1956 \citep[2: 606]{beckett_letters_2009}, and he had the three English texts first
published in book format in 1955--58. The English translations
were collected by Grove Press, Olympia Press and John Calder
(Publishers) in 1959\emph{--}60: a decision which pleased Beckett, who,
writing to Judith Schmidt on 5 November 1959, claimed to have ``always
wanted to see the three together'' \citep[qtd. in][118]{oreilly_making_2017},
although he could also not ``bear the thought of word trilogy appearing
anywhere'', as reinstated in a letter of 5 May 1959 to Barney Rosset \citep[3: 230]{beckett_letters_2009}. Beckett generally referred to the collected
edition as ``the 3 in 1'', as in the aforementioned letter to Barney
Rosset, and at least once as ``pseudo-trilogy'', in a letter to Con
Leventhal of 26 May 1959 \citep[qtd. in][185]{cohn_beckett_2001}, but the word
``trilogy'' eventually slipped into the cover of the Olympia Press
edition and the blurb of the Grove and Calder editions (\citealt[81--82]{van_hulle_making_2014}; see also \citealt[107]{van_hulle_making_2017}; \citealt[100 and 121]{oreilly_making_2017}). The ``three novels'', as the present
review essay will call them, borrowing a rather neutral definition
occurring as a subtitle and later as a title in Grove's collected
editions (1959 and 1965), also mark Beckett's turn from coherent
storylines to meta-fictional narratives, which has been dated to the
watershed between \emph{Malone meurt} and \emph{L'Innommable} or, more
precisely, to the watershed between \emph{L'Innommable} and \emph{Textes
pour rien} \citep[23]{van_hulle_making_2017}.

The three instalments of the \emph{BDMP} dedicated to Beckett's ``three
novels'' offer both a description of the genetic dossiers, partially
displayed and transcribed in the digital modules but fully examined in
the documentary section of the printed volumes, as well as an
interpretation of the genetic dossiers, solely present in the critical section of the
printed volumes. ``Description'' and ``interpretation'' derive from
Hans Zeller's terms \emph{Befund} {[}record{]} and \emph{Deutung}
{[}interpretation{]}, the second of which the \emph{BDMP} borrows with
the purpose to shed light on Beckett's novels as both a product and a
process \citep[25]{oreilly_making_2017}, which suggests that the \emph{BDMP}
also applies editorial theory to a critico-genetic purpose. In Zeller's
essay ``Record and Interpretation'', the terms are defined within the
boundaries of a methodological enquiry on the reliability of editorial
practice in the philology of modern texts, addressing the subjective
element which underlies any given edition and exercises a crucial role
during the reception of an edited text \citep[18--20]{zeller_record_1995}. The essay
does not recommend to clear textual editing of its subjective element,
which means of the individuality of the editor --- an unachievable but
also unadvisable goal, due to the hermeneutic nature of textual editing
itself --- but rather to lay that subjectivity out on the page in the
wake of Aristarchus' classical edition of the \emph{Iliad}, structured
around a clear division between mechanical \emph{recensio} (objective)
and conjectural \emph{emendatio} (subjective) \citep[20--22]{zeller_record_1995}. While
discussing the rationale of editions, the essay endorses a pursuit of
the will of the author only as evidenced in the witnesses, prescribing
to examine the constitution of texts drawing on authorized versions
rather than authorial intention, which requires the collation of
printings approved by the author and of ``all manuscripts of a work in
whose production the author was involved or that were produced under his
instructions'' \citep[25--26 and 53n22]{zeller_record_1995}. If, strictly speaking,
the \emph{Befund} {[}record{]} is the manuscript itself, then the only
aspect of an edition which preserves most of its objectivity is the
documentation of the record by utilizing photomechanical reproduction or at
least verbal description, whereas all the editorial work aiming to
produce a text should be deemed \emph{Deutung} {[}interpretation{]} of
the manuscript, which preludes to the interpretation of the text offered
by literary criticism \citep[42--45]{zeller_record_1995}. In the \emph{BDMP}, the
digital modules and the documentary section of the printed volumes
provide photomechanical reproductions of the \emph{avant-textes} and
verbal descriptions of the complete genetic dossiers available,
respectively, but the same documentary section, which is enriched by a
detailed bio-bibliographical element already, is also followed by an
interpretive section which conflates editorial and literary
interpretation in line with the tradition of genetic criticism. Therefore, in terms
of structure the \emph{BDMP} stems from a binary distinction
rooted in German \emph{Editionswissenschaft} {[}editorial studies{]},
but in terms of methodology and purpose it is clearly indebted to French
\emph{critique génétique} {[}genetic criticism{]}. Consequently, this
review essay will address the interaction of the two traditions in the
description and interpretation of the genetic dossiers of Beckett's
``three novels'', offered both in the printed and in the digital sides
of the \emph{BDMP}.

\section*{\centerheading{II}}
\vskip 1em

The description of the genetic dossiers of Beckett's ``three novels''
does not separate French and English geneses, since the digital modules
display the \emph{avant-textes} chronologically as facsimiles with
transcriptions and the documentary section of the printed volumes offers
verbal descriptions of the entire genetic dossiers organized by type of
document examined. The descriptions begin with the French manuscripts,
handwritten respectively on four notebooks in Foxrock, Paris and Menton
between 2 May and 1 November 1947 \citep[33, 37--38, 47 and 51]{oreilly_making_2017}; on two notebooks in Paris between 27 November 1947 and 30 May
1948 \citep[47--48]{van_hulle_making_2017}; and on two notebooks mostly
in Paris and Ussy between 29 March 1949 and an unspecified date in
January 1950 \citep[32--33]{van_hulle_making_2014}. The descriptions then
progress towards either the full manuscript or the surviving fragments
of the English translations; they gather the extant French and English
typescripts, together with the galleys and proofs if available; along with dealing
with pre-book publications, French, UK and US editions, and broadcasting
scripts almost exclusively in the printed volumes. Speaking of the BBC
broadcasts with extracts from the ``three novels'', I could not readily
find a mention of the broadcast from \emph{The Unnamable}, first aired on
19 January 1959, in the corresponding volume of the \emph{BDMP}, whereas
the volume dedicated to \emph{Malone Dies} maintains that the recording
of the broadcast from the novel, first aired on 18 June 1958, has been
lost, and hence no comparison with the surviving scripts would be possible \citep[108 and 116]{van_hulle_making_2017}. Although this was certainly
true until recent years, I was lucky enough to find the recording in
February 2015, thanks to the invaluable assistance of Steven Dryden
(Broadcast Recordings Curator, The British Library). Subsequently,
the recording has also been digitized by the Sound Conservation Team and
made available in the library reading rooms. The documentary section of
the printed volumes is consistently marked by two useful features: the
quantitative surveys of textual endogenesis, which let us follow the
chronology and pace of Beckett's writing when the autograph is
thoroughly dated (as in the case of the French manuscript of
\emph{Molloy}), and the comparative surveys of textual epigenesis, which
offer either a selection or a complete list of variants between
different editions of Beckett's ``three novels''.

The outlined arrangement generates homogeneity throughout the
description of the genetic dossiers of Beckett's ``three novels'',
which, however, also details the many specificities of each dossier,
which can be mentioned here only in passing. \emph{The Making of Samuel
Beckett's} Molloy highlights that the first page of the French
manuscript, containing the beginning of the novel, was written on the
last day of composition; it suggests that the expunction of the
notorious passage on the economy of Ballyba in the French typescript
might have been Beckett's, or a result of his recommendation; it expands
on Beckett's enduring issues with censorship, in the run up to a
pre-book publication in the New World Writing series of the New American
Library; and it wonders if Beckett had the chance to give his
\emph{imprimatur} to any of the first collected editions of his ``three
novels'' in English at all \citep[33, 65, 84--86 and 103]{oreilly_making_2017}. \emph{The Making of Samuel Beckett's} Malone meurt \emph{/} Malone
Dies publishes a detailed collation report of the first notebook of the
French manuscript, in an attempt to find the reason of a textual
lacuna; it offers a full break-down of the notebook containing the
surviving manuscript fragment of the English translation, together with
draft letters and prose fragments also pertinent to the genesis of
\emph{Foirades} / \emph{Fizzles}; and it speculates on the lost French
typescripts, while providing a summary of the discovery of Beckett's
work by Minuit \citep[40--43, 50--61 and 62--64]{van_hulle_making_2017}.
\emph{The Making of Samuel Beckett's} L'Innommable \emph{/} The
Unnamable shows that the text of the French manuscript closes at the end
of the physical manuscript itself, ``as if Beckett set himself the task,
not so much to write a novel as to fill two notebooks''; it points at
Beckett's unusual decision to write the body of the text on the verso
rather than on the recto of the pages in the same manuscript, using the
recto rather than the verso for facing-leaf additions, paralipomena or
sparse doodles; and it provides a detailed account of one additional
case of censorship suffered by Beckett when Jean Paulhan published a
bowdlerized extract from the novel in the second issue of his \emph{NNRF
--- Nouvelle Nouvelle Revue Française} (February 1953), under the title
``Mahood'' \citep[31, 32--33 and 59--67]{van_hulle_making_2014}.

The interpretation of the genetic dossiers of Beckett's ``three novels''
in French is strongly interdisciplinary, being rooted in genetic
criticism and theory of literature (or of the visual arts), and is
articulated in a paragraph-by-paragraph (or section-by-section) textual
analysis, followed by a study of Beckett's authorial translations (or
co-translation) in English.

The interpretive section dedicated to \emph{Molloy} in French draws on
the definitions of \emph{autograph} (or \emph{holograph}, ``écrit de la
main de l'auteur'' {[}handwritten by the author{]}; \citealt[241]{gresillon_elements_1994},
and \citealt[271]{kline_guide_1998}), of \emph{autography} (``literally
self-life-writing''; \citealt[x]{abbott_beckett_1996}) and of \emph{autographic} (said of
an artwork, and by extension of its artform, if ``the distinction
between original and forgery of it is significant''; \citealt[113]{goodman_languages_1968}) \citep[qtd. in][25--26]{oreilly_making_2017}. In brief, the analysis of the
autographs of \emph{Molloy} in French has the purpose of reconsidering the
autographic dimension of literature, in general, and to read Beckett's
novel, in particular, as autography: the study of the genesis of
\emph{Molloy}, even more simply, sheds light on the uniqueness of
literature and leads to consider \emph{Molloy} as self-writing. And
indeed, the autographs of the novel carry meta-fictional passages,
omitted or blurred in the published text, which can be read as
``auto(bio)graphical'' traces, such as scatological imagery symbolizing
handwriting, identifications between author and narrator or between
narrator and characters, and narrative turns matching the pagination of
the physical manuscript or its internal division into paragraphs \citep[149--51, 167--68, 185 and 236--37]{oreilly_making_2017}. The textual
analysis of the genesis of \emph{Molloy} in French is followed by a
thematic analysis of its creative co-translation in English: a
full-fledged ``\emph{re-writing}'', conceived as a ``writing
\emph{again}'' and a ``writing \emph{anew}'' \citep[338]{oreilly_making_2017}, which grew out of a collaboration which became quite tense, for a
while, after the completion of the first half of the translation.

The interpretive section dedicated to \emph{Malone meurt} dwells again
on the notions of \emph{autograph}, \emph{autography} and
\emph{autographic}, but focuses on the shift from \emph{story} to
\emph{discourse} in the ``three novels'', read as a transition through a
``Coda'' in the manuscript rather than as a sudden break along Beckett's
literary path \citep[23--24]{van_hulle_making_2017}. The analysis
characterizes the published \emph{Malone meurt} as ``a `surface' text
whose particulars are in the manuscript'', going beyond a conception of
the novel as a parody of Honoré de Balzac's realism and tracing in the
autographs the decline of the autographical tradition of Jules Renard's
\emph{journal intime} \citep[27--29 and 158]{van_hulle_making_2017}. The
meta-fictional passages, largely introduced in the autographs by way of
revision as elsewhere in the ``three novels'', are frequently
identifications between the author and the narrator, even more often than in
\emph{Molloy}, but the peculiarity of \emph{Malone meurt} are the
narrator's comments on his own storytelling, and particularly the
expressions of boredom which replace entire sections of the
Saposcat/Macmann tale \citep[123 and \emph{passim}]{van_hulle_making_2017}. The thematic analysis of \emph{Malone Dies}, the first
of the ``three novels'' which Beckett translated alone and which turned
out to be a demanding task because of his increasingly hectic schedule,
finds an extension of Beckett's autography again in the continuation of
the genesis of the novel as authorial self-translation \citep[275]{van_hulle_making_2017}.

The interpretive section dedicated to \emph{L'Innommable} centres on the
manifestations of the negative in the novel, addressed as traces of a
development towards the ``Literatur des Unworts'' {[}literature of the
unword/non-word{]} which Beckett set forth in his ``German letter'' to
Axel Kaun of 9 July 1937 (\citealt[4 and 173]{beckett_disjecta_1983}; see also \citealt[1: 515 and 520]{beckett_letters_2009}) and which he pursued under the influence of
language scepticism \citep[19 and 22]{van_hulle_making_2014}. Since this
instalment of the \emph{BDMP} was published a few years prior to the other
two examined by the present review essay, the analysis here prefigures
the aforementioned comparison between \emph{autograph},
\emph{autography} and \emph{autographic.} The meta-fictional passages
are deemed to prove ``the essential metaphoricity of thought'' not only
in the contents expressed but also in the style expressing them,
especially by way of epanorthosis, or rhetorical self-correction, used
elsewhere in the ``three novels'' but not as extensively as in
\emph{L'Innommable} \citep[26 and 103]{van_hulle_making_2014}. The
thematic analysis of \emph{The Unnamable} reads Beckett's wearying
self-translation as a continuation of his ``unwording'' project: a
self-decomposition of language in the aporetic pursuit to express the
inexpressible (or ``unnamable'') in the morphology, syntax and lexicon
of the novel, or by way of rhetorical devices \citep[194--95]{van_hulle_making_2014}.

\section*{\centerheading{III}}
\vskip 1em

The value of the instalments of the \emph{BDMP} dedicated to Beckett's
``three novels'' is chiefly a result of their focus on textual evidence,
since they aim in the first instance to collate, arrange, display and
detail three genetic dossiers. This descriptive effort --- summed up in
each essential ``Genetic Map'', in the printed volumes, or ``Manuscript
Chronology'', in the digital modules, as well as in the useful pie
charts which can be found in the ``Statistics'' section under ``Free
Features'', in the digital modules --- leads to circulate documents
scattered around the globe and at times largely unknown, among which can
be found textual fragments expunged along the genesis of the ``three
novels''. The passage on the economy of Ballyba, in the French
manuscript and typescript of \emph{Molloy} (FN3 {[}HRC, MS SB/4/7{]},
65r--78r, and FT {[}HRC, MS SB/17/6{]}, 214r--24r; see also \citealt[49--50, 62--67, 262--76 and 380--87]{oreilly_making_2017}); the excized segments of
the Saposcat/Macmann tale, in the French manuscript of \emph{Malone
meurt} (FN1 {[}HRC, MS SB/7/2{]}, 68r--71r, 72r, 76r--82r and
\emph{passim}; see also \citealt[150--56, 157--58, 164--78 and \emph{passim}]{van_hulle_making_2017}); and the ``Coda'' which might well be a
prelude to \emph{L'Innommable}, again in the French manuscript of
\emph{Malone meurt} (FN2 {[}HRC, MS SB/7/4{]}, 110v--12r; see also \citealt[41, 253--56 and 304--07]{van_hulle_making_2017}), are all notable
examples. The description of the genetic dossiers also summarizes the
circumstances of Beckett's composition and examines Beckett's practice
of composition itself, reinstating, for instance, his habit to leave
blank versos (or rectos) for facing-leaf additions, to jot down
paralipomena later developed in the body of the text, and to produce
typescript carbon copies. The instalments of the \emph{BDMP} under
examination here, however, also acquire value due to their interpretive
nature, since they aim to locate the textual scars produced by Beckett's
``vaguening'' of his work in order to recover the \emph{mémoire du
context} {[}contextual memory{]} of the work itself (\citealt[121]{ferrer_logiques_2011};
see also \citealt[289--90]{oreilly_making_2017}). According to the line of
interpretation presented in the \emph{BDMP}, it is logical to argue that
only the autographs can unveil that memory, which reveals the
autographic dimension of literature and initiates a textual analysis
which ultimately leads to read as autography Beckett's non-programmatic
\emph{écriture a processus} in his ``three novels'' \citep{hay_troisieme_1986}. The
two-fold nature of the \emph{BDMP}, which this review essay has
attempted to examine with specific reference to three of its
instalments, makes it essential for consultation and research, leaving
space for further enquiries focusing on the exogenesis and epigenesis of
Beckett's works, in an attempt, for instance, to fill the lacunae which
currently exist in the \emph{avant-textes} of the English \emph{Molloy}
and \emph{Malone Dies} (\citealt[55-61]{oreilly_making_2017}; \citealt[56--58]{van_hulle_making_2017}).

\begin{flushleft}
\bibliography{references/rosignoli}  
\end{flushleft}

\end{paper}