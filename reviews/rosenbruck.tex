\contributor{Jonas Rosenbrück}

\contribution{Walter Benjamin, \emph{Berliner Chronik / Berliner Kindheit um neunzehnhundert}}

\begin{review}
\renewcommand*{\pagemark}{}

\begin{reviewed}
Review of \thecontribution. 2 vols. Eds. Burkhardt Lindner and Nadine
Werner. (Vol. 11(1--2) of \emph{Werke und Nachlaß. Kritische
Gesamtausgabe}. Eds. Christoph Gödde and Henri Lonitz.) Berlin: Suhrkamp Verlag, 2019. 652 and 466 pp. ISBN: 978--3--518--58728--7.
\end{reviewed}

\section*{} 
Presented as a ``complete critical edition'', Walter Benjamin's
\emph{Werke und Nachlaß} {[}Writings and Literary Estate{]}, published
by Suhrkamp since 2008, has opened a new phase in the convoluted
reception history of this seminal thinker's work, suggesting that a
definitive edition might have finally arrived after decades of belated
discoveries of unpublished texts. On behalf of the Hamburger Stiftung
zur Förderung von Wissenschaft und Kultur and under the general
editorship of Christoph Gödde and Henri Lonitz, as well as in
cooperation with the Walter Benjamin Archiv, ten volumes have been
published thus far, with eleven still to come. The volume under review
here, \emph{Berliner Chronik / Berliner Kindheit um neunzehnhundert}
{[}A Berlin Chronicle / Berlin Childhood around 1900{]}, published in
2019, gathers an impressive and remarkably useful range of contents: not
only all the extant versions of Benjamin's collection of
autobiographical vignettes --- manuscripts, typescripts, and published
versions --- but also a lengthy and thorough account of its genesis and
publication history --- more than a hundred pages of relevant letters,
charts establishing meticulous comparisons between texts, and copious
annotations. This volume constitutes a crucial test case for the
editorial principles of the complete critical edition published by
Suhrkamp, for two main reasons: on the one hand, \emph{Berlin Childhood}
occupies a privileged place in the history of German editions of
Benjamin's works. It was the first book published in post-war Germany
(1950, ed. Theodor W. Adorno) and thus marked the beginning of
Benjamin's return to scholarly and public discourse. On the other hand,
and more importantly, \emph{Berlin Childhood} is concerned with two
central matters, both of which directly relate to editorial principles
and textual scholarship at large: a theory of language (in particular of
\emph{Schrift} {[}writing{]}) and a theory of memory and tradition, as
it comes to bear on the transmission of texts.

The following passage crystallizes Benjamin's understanding of memory
and simultaneously sheds light on the value of the critical edition of
\emph{Berlin Childhood}:

\begin{quote}
Wer sich der eignen verschütteten Vergangenheit zu nähern trachtet, muß
sich verhalten wie ein Mann, der gräbt. Vor allem darf er sich nicht
scheuen, immer wieder auf einen und denselben Sachverhalt zurückzukommen
--- ihn auszustreuen wie man Erde ausstreut, ihn umzuwühlen, wie man
Erdreich umwühlt.

{[}Whoever desires to approach his own buried past must behave like a
man who digs. Above all he must not shy away from returning again and
again to the same state of affairs --- disseminating it like one
disseminates earth, turning it over like one turns over soil.{]}

\begin{flushright}
(1: 367)
\end{flushright}
\end{quote}

\noindent Benjamin conceived the work of memory in archaeological terms: a
repeated digging up, rummaging through layered memories, and a
subsequent dispersal of what has been dug up. The critical edition of
\emph{Berlin Childhood} enables the reader to perform a similar
archaeological search, as far as the textual development of Benjamin's
collection is concerned: it displays the various layers of the
collection's ``soil'', allowing the reader to rummage through them. The
layering of textual variants and their reoccurrence demystify the idea
of a stable transmission of an unaltered text and instead presents the
texts as loose amalgamations --- still rigorously structured and clearly
formed, at times --- inviting the reader to delve into them. In short,
the editorial principles of the critical edition line up closely and
productively with the theory of memory in the collection, specifically
with the definition of the relationship which we entertain with our past
and of the strategies to recover that past in our present.

More can also be said about the consistency between Benjamin's theory of
writing and the editorial principles of the critical edition of
\emph{Berlin Childhood}. Instead of relegating to an apparatus the
textual variants which Benjamin produced during almost an entire decade,
the critical edition displays in the main text all its emendations and
insertions, using symbols, changing fonts, and adding marginalia in
order to replicate the visual appearance of the original document.
Through these visual disruptions of the reader's gaze, attention is
drawn to the materiality of language as an obstacle, suggesting that
reading is not a straightforward process of semantic extraction of
meaning from a text. This is precisely what Benjamin presents as the
heart of his theory of language, in \emph{Berlin Childhood}. For a
child, it is above all the graphic and phonetic appearance of words that
matters, not necessarily to the exclusion of meaning, but as a
diverging, alternative access to language. In one episode recounted by
the narrator, for instance, the child encounters potboilers in his
school's library and loses himself in the ``Gestöber der Lettern''
{[}flurry of letters{]} (1: 514). This flurry of letters offers the
child --- and by extension the reader --- the opportunity to envelop
themselves in words: ``Beizeiten lernte ich es, in die Worte, die
eigentlich Wolken waren, mich zu mummen'' {[}In good time, I learned how
to wrap myself into the words that were actually clouds{]} (1: 538).
Words become nebulous, enveloping clouds: it is this unsettling of the
text, brought out by the editorial principles of the critical edition,
that forecloses the text from being read as a charming, nostalgic
reminiscence which merely entertains or delights the reader with its
style. Instead, the reader is forced to engage with Benjamin's theory of
language through the very encounter with Benjamin's own language.

The importance attributed to the materiality of Benjamin's writing
reaches a highpoint in one of the most impressive aspects of the
editorial project published by Suhrkamp: the accompanying online digital
database \emph{Walter Benjamin Digital} (\citeyear{hamburger_stiftung_zur_forderung_von_wissenschaft_und_kultur_walter_2016}). This resource is set to pair
the publication of volumes 17, 18, and 20 as well --- that is, the
volumes gathering the unfinished late works --- and the advantages which
it offers partially counterweight the almost prohibitive retail price of
the printed volumes. The range of features of the online digital
database combines practicality with the possibility to wrap oneself into
the text as if it were a nebulous cloud: the manuscripts are displayed
as digital facsimiles overlaid with diplomatic transcriptions which can
be smoothly crossfaded. A search tool allows us to trace individual
words or sentences and to arrange the texts according to their edition,
their repository, or the project to which they belong. The search tool
is supplemented by bookmark and download tools as well as by a
trilingual (German, English, French) description. While there certainly
exists the danger of a fetishization of the ``original'' text in the
form of the facsimile, this online resource is not only extremely useful
to the reader, but also helps to counteract the occasional resurfacing
of hagiographic reverence in Benjamin studies by presenting texts as
mobile creations which exist to be rummaged through, dug up, and
disseminated.



\begin{flushleft}
\bibliography{references/rosenbruck}
\end{flushleft}

\end{review}