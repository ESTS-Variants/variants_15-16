%%%%%%%%%%%%%%
%% METADATA %%
%%%%%%%%%%%%%%

\contributor{Hans Walter Gabler}

\contribution{Paul Eggert, \emph{The Work and the Reader in Literary Studies: Scholarly Editing and Book History}}

\begin{review}
\renewcommand*{\pagemark}{}

%%%%%%%%%%%%%%%%%%%%%%%%%%%%%%%%%%%%%%
%% DESCRIPTION OF THE REVIEWED BOOK %%
%%%%%%%%%%%%%%%%%%%%%%%%%%%%%%%%%%%%%%

\begin{reviewed}
Review of \thecontribution. Cambridge: Cambridge University Press, 2019. 242 pp. ISBN: 978--1--10--864101--2.
\end{reviewed}

%%%%%%%%%%%%%%%%%%%%%%%%%%%%%
%% YOUR REVIEW STARTS HERE %%
%%%%%%%%%%%%%%%%%%%%%%%%%%%%%

% remove asterisk (*) if you want to number your sections
% add a title for your section in between the {curly brackets} if you need one
\section*{} 
Which academy, one wonders, would venture a summer school or
sabbatical-term training camp for experienced textual scholars together
with early-career hands-on (analogue) and ``keys-on'' (digital) editors
for the purpose of exchanging practical experience from their variegated
individual tasks and projects in the scholarly editing and digital
exploration of texts and works of literature? The circle of participants
should be international, representing distinctly different schools of
expertise. It should also, for an essential backgrounding of theoretical
conceptions and methodological reflections, comprise the literary and
literary-genetic critic, as well as the interdisciplinary digital
humanist. An assembly on such scale would convene the competence for
searching dynamic investigations of the literary foundations of our
cultural heritage and their future, as well as past, medial unlocking.
It would be well served with Paul Eggert's monograph of 2019 from
Cambridge University Press, \emph{The Work and the Reader in Literary
Studies} to kick it off, and from there to develop exchanges to
exemplify and counter-exemplify, substantiate, modify, or critique the
tenets, positions, and systematizations the book offers.

The Introduction towards laying out in manifold detail that ``{[}t{]}he
present book {[}\dots{}{]} centres itself around the much overlooked
concept of the work'' (9) is first a warmingly personal, then a
stimulatingly sweeping appetizer to the book's perspectives and
argument. Before anatomising the centre, the second chapter, in a
rhetorical sweep typical of Paul Eggert, ``revives the work concept''
triadically via music, literature, and historic buildings. The work
concept is so at once holistically modelled on the art form of music in
a ``continuity from composer to score to performers to audience'' (22).
The third chapter draws us fully into the book's core area of expertise,
that of textual criticism and scholarly editing. As guided, though, by a
seasoned practitioner and theorist of Anglo-American textual criticism,
we are taken on a route of proper schooling in the template of that
discipline, Shakespeare editing in printed scholarly editions. Once
more, the rhetorical strategy is thus delightfully oblique (``By
indirections find directions out'', as Polonius in \emph{Hamlet},
II.i.65, phrased it). Eggert's opening gambit for this chapter is to
imagine a digital native's first encounter with that dinosaur from
analogue times, the editing of Shakespeare, touchstone for principles
and practice of editing in the Anglo-American environment. In a
crash-course manner, the chapter climbs rapidly to the book's full
professional heights. It culminates in analytics suited to enlighten,
even redirect Shakespeare editing itself --- and so unashamedly leaves
the digital neophyte well behind.

In its two subsequent chapters, under the headings ``The Reader-Oriented
Scholarly Edition'' (4) and ``Digital Editions: The Archival Impulse and
the Editorial Impulse'' (5), the book reaches its core systematics.
Significantly, the ``reader-orientation'', essential as it is for
Eggert's overall triadic conception encapsuled in the book's title, is
established and argued first in Chapter 4. This sets out from the
definition that editions are arguments, and that they are transactional
(64). While Eggert concedes that ``editions may have succeeded
brilliantly in their presentational role'', he also emphasises that
``they have failed, or at least partially failed, in their transactional
one'' (67). The vision therefore ``is a more reader-oriented scholarly
edition than has been feasible in print'' (76). The sights are thus
directed on support from the digital medium. Chapter 5, ``Digital
Editions'', builds up to devise design and format for digital editing
and digital editions of the present and for the future. To order
theoretically and pragmatically what in documents and transmissions
editors encounter and what, in combination, their preservational,
presentational and transactional responsibilities should comprise,
Eggert models the editorial and interpretatively critical, and thus
always readerly, trajectory of engagement with text, work, and reading
onto a sliding scale between an archival and an editorial impulse. The
slider is, at one level, theory visualised. It brings into an, as it
were, living image a coherent conception of the centrality that textual
criticism and scholarly editing occupy, or should occupy, for literary
studies. At another level, essential for the book's developing argument,
it positively models our present and future commitment to scholarly
editing into the digital medium. Eggert recognises ``an expanded remit
for scholarly editors in the digital medium, one that might begin to
resolve, through practice, the old stand-off between literary
scholarship and literary criticism'' (80). The model and device of the
slider becomes the template for situating the book's work concept in
literary studies, pivoted on the editor:

\begin{quote}
If the edition is to be seen as an argument then it is necessarily one
that is addressed \emph{to} an audience \emph{in respect of} the
documents gathered and analysed for the editorial project, usually
documents deemed to witness the textual transmission. If the edition is
an argument addressed to a readership then it must anticipate the needs
of that readership. This transactional view of things places the editor
in a medial or Janus-faced position, looking in one direction towards
the relevant documents and looking in the other towards the audience.
\begin{flushright}
(82)
\end{flushright}
\end{quote}

The four chapters to follow are pragmatic in their main orientation.
They live from Paul Eggert's stunningly rich and variegated practical
experience of editing and from his capacity to differentiate, generalise
and theorise from it. From his intense work of conceptual and editorial
cooperation on the Australian \emph{Charles Harpur Critical Archive} (\citeyear{harpur_charles_2019}), he
refines, in Chapter 6, the book's ``work concept'' through introducing
its sub-granularity ``version''. The version is fore-grounded as a
significant element in an edition's work-directed argument. Chapters 7
and 8 address ``Book History and Literary Study'' in an increasingly
urgent plea for the transactional dimension of the ``work edition'' of
the future. At the present moment in the field of literary studies, as
well as in its foundational sub-disciplines of textual criticism and
editorial scholarship, it is indeed urgent to establish, or
re-establish, book history as a main tributary to knowledge and
historical insight. Book history today is no longer the positivistic
discipline that it once was. Drawing it in as a main factor in his
pragmatic concerns, Eggert decisively strengthens its relevance to
literary studies, and consequently to scholarly editing, in
comprehensive commitment to the reader. Chapter 7 on such terms works
out the commercial and ultimately cultural relevance both ``at home in
the colony'' and ``far away in England'' of the Australian writer Rolf
Boldrewood. Chapter 8 addresses the literary creativity of Joseph Conrad
and D.H. Lawrence, once more in terms of book history. We partake again
of Paul Eggert's first-hand editorial experience from co-operating in
these editions, conceptually refined now in terms of his monograph's
lode-stars of a version-refined work concept and the transactional
commitment postulated for the editorial enterprise. To cap this pursuit,
Chapter 9 engages with adaptations. The description of the manifold
variations of the ``Ned Kelly Story'' under a Robin Hood template easily
persuades that this chapter belongs and is indeed a logical follow-up of
the book's over-all argument. Adaptation, accordingly, is a mode of
reception engendering new production, ``but the new production will be
subject to the different cultural valencies of its time and place, when
and where it must establish its own readership or audience or
participatory user base'' (157). The literary historian will easily
appreciate that this is a constant in cultural history and applies for
instance to medieval epics, Shakespearean plays, or Dickensian novels
alike.

The Conclusion stringently draws together the perspectives, lines of
argument and postulates the book has developed. What I was yet unaware
of when I volunteered to review it for \emph{Variants} was that a series
of articles I had written since 2006 would serve Paul Eggert's monograph
as springboard for giving the book's tenets their final edge and
profile. I greatly appreciate that I should have in effect moulded an
echo chamber for them. The overtones of resonance between us in that
chamber will and should be heard by the community of fellow researchers
at large --- and they should, moreover, of necessity come from the
entire ranges of different schools of criticism, genetic criticism,
textual criticism, scholarly editing, genetic editing, and digital
editing, both scholarly-at-large and genetic. Preliminarily to
establishing a common platform of understanding, some basic terms would
need to be reflected and their definitional differences in the several
disciplines be laid open. Paul Eggert, for example, while aware of such
differences (sometimes more, sometimes less) still operates with terms
such as ``document'', ``witness'', ``version'', ``intention'', much on
an Anglo-American system of coordinates. This serves his case, the set
of his cases, well, and is not detrimental. Yet some conclusions drawn
in the course of the book would be re-accentuated were other basic
definitions and co-ordinations within the frameworks of the disciplines
concerned admitted or considered. To admit to the full the notion that
``text'' is logically separable from ``document'', for example, would
re-calibrate the systemic relationship between document and text; it
might moreover, render ``witness'' obsolete. For the concept of the
``version'', so rightly essential to Eggert's case, it would be
important to pursue the implications of the circumstance that Eggert
from the Anglo-American angle can perceive it as largely intrinsic to a
work's progression, while its definitional frame in German editorial
scholarship is essentially extrinsic: what is a ``version'' is there due
less to the work's progression than to the editor's decision to single
out a given so-declared version on grounds of external circumstances of
transmission and reception. Above all: the reviewer of \emph{The Work
and the Reader in Literary Studies} is --- that is, I am, and remain to
the end, deeply puzzled that Paul Eggert throughout staunchly sticks to
the notion and over-all concept of ``intention'' very much still in its
Anglo-American application. His deeply perceptive discussion of Joseph
Conrad's mind- and health-racking struggle to achieve \emph{Under
Western Eyes} (1911; see also \citealt{conrad_under_2013}) should have rendered inescapable the conclusion
that the sense and significance of ``intention'' needs a thorough
overhaul. None of Eggert's examples more than that of Conrad's
\emph{Under Western Eyes} cries out for a genetic approach. This, Eggert
acknowledges. But to adhere, at the same time, to the notion of
``intention'' locks the genetic progression within a teleological
framework that the progressive writing and especially the over-and-over
re-reading, re-thinking and revising of that writing thoroughly
countermands. Eggert establishes, in his monograph, that the future of
scholarly editing, transactionally aimed comprehensively at literary
studies, lies in the digital medium and through it in the digital
edition. His discussion of Joseph Conrad's \emph{Under Western Eyes}
constitutes his best case that the digital edition requires the genetic
dimension. The book does not resolve how it is to be attained. The
genetic digital edition in full potential yet stands to be modelled. It
is where Paul Eggert's notion of an ``editing on the level of the work''
should become realisable. This is a task yet ahead of us all.

To conclude, then: \emph{The Work and the Reader in Literary Studies} is
the most substantial book I am aware of today to lay out the land of
literary study on foundations of documented transmission of works of
literature: works and the texts that adumbrate them, written and
re-written, read and re-read, and ever safeguarded by the manifold
agencies of authors, scribes, typists and type-setters, digital
key-strokers, publisher's editors, book historians, commercial or
scholarly editors, and ever and ever again readers. In scope, the book
unfolds on its choice of soundly reflected theoretical foundations an
amazingly encompassing pragmatics of literary study in its specifics of
textual criticism, editorial scholarship and richly variegated
experience in mind-on and hands-on editing. It is a book indeed over
which literary critics, textual critics and editors with long experience
of their own, as well as early researchers, should ideally get together
to pick up on its multitude of definitions, reflections and lines of
argument. The historical moment is opportune: we are on the threshold or
indeed in the very throes of the medial shift from the materially
analogue to the binary digital curation both of our literary heritage
and our culture's literary future. The disciplines we term today
literary study, textual criticism, scholarly editing, digital humanities
should feel called upon to de-compartmentalise themselves --- also
de-nationalise themselves --- and unite in sustained debates to shoulder
their united responsibility for reassessment and renewal. \emph{The Work
and the Reader in Literary Studies} forms an important point of entry to
re-conceptualisings of literary study.

\begin{flushleft}
\bibliography{references/gabler}  
\end{flushleft}

\end{review}