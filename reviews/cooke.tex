%%%%%%%%%%%%%%
%% METADATA %%
%%%%%%%%%%%%%%

\contributor{Barbara Cooke}

\contribution{% complete information on the reviewed work
Peter Shillingsburg, \emph{Textuality and Knowledge: Essays}}

\begin{review}
\renewcommand*{\pagemark}{}

%%%%%%%%%%%%%%%%%%%%%%%%%%%%%%%%%%%%%%
%% DESCRIPTION OF THE REVIEWED BOOK %%
%%%%%%%%%%%%%%%%%%%%%%%%%%%%%%%%%%%%%%

\begin{reviewed-wider}
Review of \thecontribution. University Park (PA): Pennsylvania State University Press, 2017. 240 pp. ISBN: 978--0--271--07850--2.
\end{reviewed-wider}

%%%%%%%%%%%%%%%%%%%%%%%%%%%%%
%% YOUR REVIEW STARTS HERE %%
%%%%%%%%%%%%%%%%%%%%%%%%%%%%%
\section*{} 

In ``Some Functions of Textual Criticism'', Peter Shillingsburg comments
the homogenising effect of the poetry anthology. By effacing the social
and material contexts in which its poems were first produced, and by
reproducing them all ``in the same type font with approximately the same
margins and spaces between the lines'' (60) they belie the individuality
of each individual poem. The same observation can be made of an essay
collection. The thirteen essays contained in \emph{Textuality and
Knowledge} started life in a number of formats for different audiences,
including journal articles (``Textual Criticism, the Humanities, and
J.M. Coetzee'', 13--27; ``Convenient Scholarly Editions'', 134--44),
conference papers (``Responsibility for Textual Changes in Long-Distance
Revisions'', 64--82; ``Text as Communication'', 83--93), and keynote
speeches (``The Semiotics of Bibliography'', 28--47; ``Some Functions of
Textual Criticism'', 48--63). According to the book jacket, as collated
in a single volume these essays distil ``decades of {[}\dots{}{]} thought
on literary history and criticism'' from one of the most experienced and
well-travelled textual editors in the business.

Some of the texts have been revised, and some melded from more than one
earlier iteration for this collection. Were Shillingsburg not a textual
editor, it might not occur to a reader to ask exactly where such
revisions have been made. However, remarks such as ``We have had enough
of electronic editions and archives created as look-don't-touch products
{[}\dots{}{]} ultimately abandoned'' (192) read differently from the rival
viewpoints of 2007, 2010 and 2017. Shillingsburg making this observation
in 2007 is a Cassandra; in 2017, he is underlining the overdue need for
a sea-change in the ways in which electronic editions are planned,
funded and executed. Similarly, the hope that ``If we cannot police
ourselves, perhaps the public will'' as voiced in a pre-Brexit,
pre-Trump era now carries a weight of dramatic irony that does not apply
if it is, in fact, a contemporary revision. The sociology of critical
texts matters as much as that of the literature they critique.

It would be facile to weigh the rival ``importance'' of such diverse
outputs. However, the manner of their grouping can make for uneven
reading. Shillingsburg's inaugural professorial lecture given at De
Montfort University, for example, is directed at a lay audience and
argues the case for textual criticism with passion and concision. This
eminently accessible piece succeeds ``The Semiotics of Bibliography'',
an involved and detailed argument for the re-evaluation of authorial intention
after its almost wholesale disavowal by D.F. McKenzie, reviewing the
20\textsuperscript{th} century history of textual criticism
controversies as it goes along. Shillingsburg's view of authorial intent
here is closely allied to his Platonic concept of the ``work'', which
runs throughout the collection but is delineated in most detail in ``How
Literary Works Exist'' (115--33).

These theoretical enquiries are applied to case studies in ``Some
Functions of Textual Criticism'' and ``Responsibility for Textual
Changes in Long-Distance Revisions''. In particular, Shillingsburg's
discussion (originally presented to an audience at Loyola University
Chicago) of a ``lost'' two-and-a-half pages of \emph{To the Lighthouse}
(1927) cut immediately prior to publication raises multiple questions
related to authorial intent. These two-and-a-half pages were,
Shillingsburg tells us, cut by Woolf herself. They were not, however,
omitted primarily for aesthetic concerns but because the last signature
of printed pages was simply too long for the format in which the book
was to be bound. That in itself makes Shillingsburg ask whether the
longer or the cut version of the text is closer to that Woolf thought
best (71). Furthermore, if the cut is accepted, then subsequent
editorial difficulties arise from the fact that there are more than one
set of ``cut'' proofs in existence. These are not identical, and carry a
number of parallel, smaller emendations. Shillingsburg's enquiry as read
here is mostly pragmatic: if one is engaged in producing an eclectic
text of \emph{To the Lighthouse} then which if any of these emendations,
large and small, should be adopted --- and from which text(s)?

The practical conundrums of this scenario notwithstanding, for literary
biographers it is perhaps more revealing as a study in the
\emph{process} of authorial decision-making. Authors revise and emend
their texts for a whole range of reasons, which might span from ``purely
aesthetic'' to ``purely practical'' on one axis and ``wholly conscious''
to ``wholly unconscious'' on another. Frequently, authors are less
precious about the contours of their ``work'' than their readers and
critics: witness Woolf's apparent willingness to cut those
two-and-a-half pages for convenience of printing. That is a conscious,
pragmatic decision. Less consciously, but equally pragmatically, authors
might cut a description short or revise less than usual because they are
getting a headache, running low on paper or need to put the bins out.
Such decisions leave no trace. Divining ``intent'' in this network of
influences foxes behavioural psychologists, let alone literary critics.
Shillingsburg, particularly perhaps the younger Shillingsburg, might
tell us it's cowardice not to try. But to those editors inclined to what
Shillingsburg characterizes as ``critical archiving'', that is
presenting historically intact versions of text with the minimum of
editorial intervention, organising editorial practice around perceived
authorial motivation remains a precarious and tangled exercise.

However, there is more at stake in this collection than the (admittedly
fraught) questions of whether to restore this or that reading, correct
that spelling or insert that comma in a critical edition. Throughout the
essays, Shillingsburg asks two key questions about the acquisition of
knowledge: ``where does that come from?'' and ``how do you know?''.
These two questions override any in-fighting between specific groups of
textual editors, and Shillingsburg deploys them time and again to
demonstrate how textual criticism must be the bedrock of the humanities.
In ``Publishers' Records and the History of Book Production'' (178--92),
Shillingsburg shows that, as early as 1805, American presidents were
denigrating the scourge of ``false facts'' (191--92). The term finds its
corollary in our contemporary battle with ``fake news''. Shillingsburg's
insistence that \emph{we} insist on the importance of provenance in our
classrooms and editions is timely, urgent and --- as we would expect ---
supported by the soundest available textual evidence.
\end{review}