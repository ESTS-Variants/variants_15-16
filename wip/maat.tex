\contributor{Hugo Maat}
\contribution{Restricted Translation of Historical Dutch
Text}
\shortcontributor{Hugo Maat}
\shortcontribution{Restricted Translation of Historical Dutch
Text}

\defcitealias{arab_audientie_nodate}{ARAB: Audiëntie 554} 
\defcitealias{hua_staten_nodate}{HUA: Staten van Utrecht Landsheerlijke Tijd}
\defcitealias{kha_e30_nodate}{KHA: A 11/XIV E/30}
\defcitealias{kha_i12_nodate}{KHA: A 11/XIV I/12}
\defcitealias{rad_stadsarchieven_nodate}{RAD: 3 Stadsarchieven}


\begin{paper}

\begin{abstract}
This article proposes an experimental approach to the diachronic translation of document sources written in historical variants of the Dutch language. It argues that, in order to make textual sources more accessible to read for modern audiences, it is possible to alter grammatical aspects of the text while preserving the lexical aspects of the historical language variant. Because of this restricted translation, this approach is less time consuming compared to a total translation. The article also describes a method for producing such texts, illustrated with examples taken from the sixteenth-century correspondence of William of Orange.\end{abstract}

\section{Statement of Purpose} 
This article presents and explains a novel approach to increase the
accessibility of textual sources written in historical variants of the
Dutch language. This approach was developed as part of a source edition
project at the Huygens Institute for Dutch History (Huygens ING) in
Amsterdam, in December 2019.\footnote{The author worked as a research
  assistant for E. Dijkhof at the Huygens Institute in Amsterdam from
  October to December 2019. The project in question, to produce a new
  digital edition for the correspondence of William of Orange, was in
  early stages of development. The article was written for a research project for
  the department of Digital Infrastructure for the Royal Netherlands
  Academy of Arts and Sciences (KNAW).} The initial purpose was to
develop a translation method for lay-oriented versions of historical
source texts, specifically concerning the Dutch-language correspondence
of William of Orange (1533--1584). Because of the large size of the text
corpus, the project looked for an efficient method that would not
require too much labour. A well-crafted translation in modern, legible
Dutch, of a letter from the corpus in question, could take a translator
as much as a full day to produce. The method discussed here, in first
estimates, could process a thousand words in forty minutes, by instead
rendering a ``restricted translation'', only targeting specific aspects of
the historical language for the translation. The goal became to produce
a text with increased legibility rather than a ``total translation'' (a
translation whereby every aspect of the language in the source text is
changed in the translated text), and thereby increase the accessibility
of historical text. It requires manual translation, because automated
translation would first require training data.

The following example is a transcript from a letter written in 1581 to
the Prince of Orange and his council by the \emph{Landraad Beoosten
Maas}, one of the wartime provisional governing bodies in the
Netherlands. The example includes the opening formula to illustrate that
the letter contains only commas by way of punctuation. This remains true
for the text from beginning to end, and is standard fare for this corpus
of documentary sources.

\begin{quote}
Doerluchtige hoochgeboren furst genadige heer, ende Edele, Erentfeste,
hoochgeleerde, wijse seer voorsienige heerene

Wij houden Uwe Furstliche Genade ende Edelen genaede Indachtich vandie
deliberatien hier beuorens bij den Staten generall dick ende menichmall
gehouden, op het toelaten offte geheelick verbieden vanden trafficque,
commercie offte toeganck opdie Prouincien ende Steden, die bij onse
vianden sijn geoccupeert, welcke deliberatien genoechsam Indien state
gebleuen sijn dat geraden waere intselue geheelick mede op grote poenen
te verbieden, doch Indien verstande, dat sulx van wegen dien van
Franckrijck oick aenpaarlick soude hebben te geschieden Nu Ist sulx dat
vermits eenige occurrentien die ons dienthaluen sijn voer gecomen ten
aensiene vande Stede van Groeningenn shertogenbosch ende Breda, wesende
binnen landtsche Steden ende denwelcken alle toeganck lichtelicker dan
van andere souden mogen beleth worden, {[}\ldots{}{]}
\begin{flushright}
\citepalias[f. 179 r-180 v.]{arab_audientie_nodate}
\end{flushright}

\end{quote}

\noindent The translation method alters the sentence structure and syntax while
leaving most idiomatic elements of the text intact, rendering a version
of the text that is not recognizable as present-day Dutch but
considerably easier to interpret, in the sense that it requires less
effort on the part of the reader.

\begin{quote}
{[}\ldots{}{]}

Wij houden Uwe Furstliche Genade ende Edelen genaede Indachtich vandie
deliberatien, hier beuorens bij den Staten generall dick ende menichmall
gehouden, op het toelaten offte geheelick verbieden vanden trafficque,
commercie offte toeganck opdie Prouincien ende Steden, die bij onse
vianden geoccupeert sijn.

Welcke deliberatien sijn genoechsam gebleuen Indien state, dat geraden
waere intselue geheelick mede op grote poenen te verbieden, doch Indien
verstande, dat sulx van wegen dien van Franckrijck oick aenpaarlick te
geschieden soude hebben.

Nu Ist sulx: Eenige occurrentien sijn ons dienthaluen voer gecomen, ten
aensiene vande Stede van Groeningenn shertogenbosch ende Breda, wesende
binnen landtsche Steden, ende denwelcken alle toeganck lichtelicker dan
van andere souden mogen worden beleth.
\end{quote}

\noindent The initial project intended to produce a translation of the historical
textual sources that was accessible to an audience of lay readers, for
instance for educational purposes. The proposed method, which was
developed as a spin-off, is primarily directed towards making historical
texts more legible for scholars and researchers. This restricted
translation specifically targets textual structure and grammar in
historical language text. These are considered the main challenges
because they prevent readers from employing reading strategies based on
text structure, and because reference material is more suitable for
resolving lexical elements of the text. A modern reader can relatively
easily find the meaning of individual words through online resources
such as the \emph{Historische Woordenboeken} (Historical Dictionaries) on the website of the
Institute for the Dutch Language (INT). The \emph{Historische Woordenboeken} is an online dictionary that incorporates several historical Dutch dictionaries, accounts for variant spelling, and groups search results by time period.\footnote{See \url{https://ivdnt.org/onderzoek-a-onderwijs/lexicologie-a-lexicografie/historische-woordenboeken}, last accessed at August 7, 2020.}

The key idea behind this approach to translating historical text is that
there is an imaginary ``tipping point'' for the degree to which a
translator alters a text in the process of translation. In a total
translation, the text is altered on every level. With the translation of
historical text to a rendition in modern language, that covers spelling,
vocabulary, grammar, structure and punctuation, and idiomatic elements.
The tipping point as part of that design denoted the point at which
the translation method would cover so many aspects that there would be
no significant difference with a total translation in regard to effort
required. A not-insignificant motivation for the development of and
research into such a translation method is an interest in cost-effective
solutions. Translations, after all, require human labour, and a
translation of historical text requires specific additional skills. The
question addressed in this article is whether it is possible to improve
accessibility of a historical text through restricted translation, and
what such a translation method would entail.

This article prefaces the explanation of the translation method with a
discussion of intralingual translation, both manual and automatic, and
the linguistic concepts used in the design. Intralingual translation is
a relatively overlooked topic within translation studies, and textual
scholarship tends not to incorporate these disciplines. Editions of
historical texts are published in modern language translations, but I
have not been able to find theoretical discussions or linguistic
analyses of these practices. There are discussions about whether or not
such translations are possible or allowed, and discussions about dealing
with stylistic textual characteristics, but there is a lack of
comprehensive theory for \emph{how} a diachronic translation is produced
within textual scholarship. The theoretical basis is therefore the
research into intralingual translation practice by Aage Hill-Madsen and
automated machine translations, most of which were developed for modern
languages. The proposed method in this article is divided into three
parts, which are: \emph{parsing} (dividing the original text into
component clauses), \emph{clausal shift} (changing clause ranks to
create shorter sentences), and \emph{modal shift} (resolving certain
forms of grammar specific to historical Dutch). The steps in the method
are illustrated with examples from historical Dutch source texts from
the William of Orange project.

\section{State of the art}
\subsection{The discussion of intralingual translation in Dutch
literary studies}

The idea, though not the exact term, of intralingual translation has
been a topic for debate amongst literary scholars in the Netherlands.
The discussion pertains specifically to modernizing editions and
translations of works considered to be part of Dutch literary heritage.
From time to time, modernizing translations of Dutch literary works have
drawn the ire of literary scholars.\footnote{See \citealt{van_oostendorp_handen_2013,jansen_hom_2007,kuipers_pinkeltje_1996}. A non-Dutch discussion on a similar topic can be found in \citealt{taylor_translation_1996}.} More often than not, the attitude of literary
scholars is cautious or negative towards this practice. An interlingual
translation or \emph{hertaling} would after all alter the rhythm and
tempo of a text, and because words are embedded in semantic fields it is
rarely possible to find perfect equivalents for expressions or lexical
words that have gone ``out of date''. A historical text remains a
historical text in translation because of the customs and mentalities
present in the text, which are themselves historical. At times, modern
language conflicts with this. One such Dutch critic of translation
practices, Marita Mathijsen, advocated that a \emph{hertaling} should
provide footnotes for words that lack true equivalents in modern
language rather than translating them, and that the norms and structures
of the original textual style should be maintained. She described this
approach as ``bending the knee to a lazy reading public'' but also the
only way to save historical literature in the Netherlands \citep[129]{mathijsen_een_2003}. This attitude was met with criticism from other scholars who
felt that this did not describe a \emph{hertaling}, but a critical
edition of a historical text \citep[247--57]{van_strien_enig_2004}.

The approach to translation as presented in this method is not very
suitable for literary texts. A literary translation has to account for
stylistic elements; it treats both the source text and target text as
works of art. By contrast, the method proposed here is concerned with
the accessibility of the contents of the text, treating certain
stylistic elements as obstructions. Take, for example, the rhythm of the
text. A common element in many historical texts, when compared to a
modern writing style, is the length of the sentences. In some cases, a
16\textsuperscript{th} century sentence can be the length of a paragraph, or even fill a
page. When creating a literary translation, the translator should take
into account whether this serves a stylistic function and might even
consider authorial intent. This is not meant to say that a literary
translator could never edit the structure of the text, because they
might and this might be justified. The point here is not to define and
prescribe a translation method for literature, but to note the
complexity or even impossibility to create hard and fast rules for
editing such texts. The proposed method, however, \emph{is} meant as a
limited set of hard and fast rules for textual editing. The source text
will not be respected as an aesthetic object, and the method should be
applied accordingly.

\subsection{Estimating readability}

There are no objective parameters to determine the accessibility or
readability of a particular text. Research into textual comprehension in
combination with language acquisition or education suggests there are
several relevant factors at work, such as complexity and length of
individual words, the use of anaphores (a word referring back to a word
used earlier in a text to avoid repetition), and the length of sentences \citep[64]{benjamin_reconstructing_2012}. Several of these factors are common-sense
observations. There are other, more complex notions, such as
intertextual cohesion, whereby the readability of any particular text is
connected to the reader's access or familiarity with texts that are
semantically similar. Studies suggest that the subjective, individual
variation between readers and their assessment of the readability of a
particular text. It is also important to incorporate the target audience
as a factor into the assessment of readability. The efficacy of formulas
used to estimate the readability for any particular text is contested,
though quantitative analysis methods are promising (Benjamin 2012, 78).
The application of such methods is however difficult for the subject at
hand, as they can only measure relative accessibility of a text. ``Will a
proposed method produce a text that is more readable than the source
text?'' is not the relevant question; ``Is the translated text
(sufficiently) readable?'' is.

\subsection{Automated Text Simplification and Statistical Machine
Translation}

In recent years there have been promising research projects on
diachronic translation through quantitative methods. Several of these
specifically used the Dutch language and its historical variants for
developing Statistical Machine Translation (SMT) tools for ``modernizing''
translations (\citealt[55]{tjong_kim_sang_clin27_2017}; \citealt[299]{domingo_historical_2017}). The reason
for this is that there exists a significant corpus of the same texts in
different historical variants of the Dutch language, made available
through the CLIN27 project for translating historical text. The texts
used for these studies were Bible translations from different time
periods and Dutch literary classics from the 17\textsuperscript{th} century, which had
counterparts in modern Dutch translations. The existence of such a
corpus allowed for automated methods to be trained and refined, by
having human and machine translators work together.

The projects that develop tools for automated diachronic translation
produce clear advantages for translators. The results suggest that this
method is quite successful at resolving the spelling variants of
historical language variants for individual words and recognizing the
words which were no longer part of the modern lexicon \citep[303]{domingo_historical_2017}. The shared tests of the CLIN27 project have however not succeeded
in approaching the quality of human translation or of gold standard
translation methods in other fields, going by the error margins.
Semantic shift over time is also still a challenge quantitative methods
cannot yet sufficiently solve \citep[60]{tjong_kim_sang_clin27_2017}. Additionally,
it should be noted that these studies were performed with favourable
conditions, because of the corpus of historical documents and modern
translations thereof. This is usually not possible with historical
documents. Finally it is important to recognize the difference between
correspondence, for which the proposed method was developed, and texts
such as the Bible or literary works, most notably the difference in
textual structure. It is unclear whether the techniques developed in the
mentioned studies would be applicable, though the research into
automated diachronic translation is ongoing.

There has been research into Automated Text Simplification (ATS) on a
syntactic level for modern languages, which rely on shortening sentences
and resolving complex grammatical structures. The syntactic approach to
ATS (lexical and hybrid approaches for text simplification exist as
well) has much in common with the proposed translation method for
historical text. However, these ATS applications were developed for
second language readers or readers with aphasia, and each application is
language specific. (Modern) Dutch was in fact one of the early languages
for which ATS applications were developed \citep[63]{shardlow_survey_2014}. As with
the above examples of SMT, this approach would require annotated corpora
of text to train a system, in order to automatically generate rules. In
order to take advantage of the progress in automated translation, any
diachronic translation will have to invest manual labour.


\subsection{Lay-oriented INTRA}

Despite being explicitly oriented towards the translation of historical
texts, the proposed method does not use a specifically diachronic type
of intralingual translation as a foundation. The diastratic or
lay-oriented approach is used as the base concept instead. There is
definitely a diachronic aspect to the translation method, but the
intention and purpose of the target text are those of a diastratic
translation. The primary goal of a diachronic translation is to create a
text that is contemporary with its audience in the sense of language.
The method proposed in this article is not meant to make texts read as if they were modern
texts, but to make the source texts more accessible for readers.

A notable advantage of intralingual translation in specific and the
linguistic approach to translation in general is the availability of
terminology. In order to discuss and develop ways to translate texts it
is vital to have a language system that describes this activity. The
concepts and terminology used to create this method is that of Aage
Hill-Madsen's 2014 dissertation \emph{Derivation and Transformation}, which
offers a descriptive linguistic study of translation practice in the
language used in the field of commercial pharmaceutics. Hill-Madsen's
study produced a typology of ``shifts'', the various microstrategies that
translators can employ (\citealt[134]{hill-madsen_derivation_2014}; see also \citealt[269]{baker_routledge_2009}).

To Hill-Madsen's knowledge (and mine) no similar research exists for
diachronic intralingual translations. He attributes this to the lack of
academic interest in the field of intralingual translation. An
additional problem is the lack of data to use for research. It was
possible for his study to build a corpus for the study of lay-oriented
translation, because it was being done on a large scale with specific
parameters for each individual text. In order to research translation
strategies for diachronic translation one would need a corpus of source
texts and target texts. Danish translation scholar Karen Zethsen, who
incidentally was the supervisor of Hill-Madsen, made a cursory attempt
in 2009 on a small scale using several generations of new Bible
translations, similar to the CLIN27 project \citep[797]{zethsen_intralingual_2009}. This
was a qualitative study, by far not as broad or thorough as
Hill-Madsen's dissertation, meant to provide an example of the kind of
research that could be done on intralingual translation.

\section{Definitions and concepts}

\subsection{Types of translation}

Within the field of translation studies, the type of translation that
this method is designed to create is considered a \emph{diachronic
intralingual translation}. This means that it is a translation from a
source text (ST) written in one variant of a language to a target text
(TT) written in another variant of the same language, and the source
language is a historical variant. The distinction between interlingual
and intralingual translations is in certain cases open for debate, as
the difference between two historical variants of the same language can
sometimes be as great as the difference between two different
contemporary languages.

In addition it is necessary to make a distinction between what is known
as a \emph{total translation} and a \emph{restricted
translation}.\footnote{The distinction between ``total translation'' and
  ``restricted translation'' is not to be confused with that between a
  ``partial translation'' and a ``full translation''. A partial
  translation denotes a translation of a fragment of the source text,
  while a full translation denotes the translation of the source text
  in its entirety.} In the case of a total translation, the target text
is rendered completely in the target language, affecting orthography,
which words are used, and the grammar. A targeted or restricted
translation only affects a specific level (or multiple specific levels)
of language in the source text \citep[22]{catford_linguistic_1965}. The
translation may focus on {[}X{]}, but leave {[}Y{]} intact. For the
purpose of this paper, the proposed method for improving the
accessibility of the language of historical text will be treated as a
restricted intralingual translation.

The concept of intralingual translation was described by Roman Jakobson
in his 1959 essay ``On Linguistic Aspects of Translation''. In his essay,
Jakobson distinguished three types of translation: he defined
intralingual translation as translation to another variant of the same
language; interlingual translation as translation to another distinct
language; and intersemiotic translation as translation to a different
medium \citep[233--34]{jakobson_linguistic_1959}. The focus here lies on the first
category. Since Jakobson there have been a number of subcategories added
to intralingual translation. The term dialectical translation is used to
denote the process of rendering a text in a different dialect, for
instance translating from British to American English. Diachronic
translation is used for translation between historical variants, though
this virtually always means changing the language of the text to the
modern variation of that language, while the inverse is generally
nonexistent. Interlingual translations of historical texts are usually
diachronic. This has led to peculiar situations wherein a historical
text has become more accessible in translation, even to native speakers
of the language of the original text \citep[71]{hill-madsen_derivation_2014}. Diastratic
translation (also sometimes called ``paraphrase'') describes the process
by which a text is changed from expert-oriented to lay-oriented or vice
versa.

Compared to interlingual translation, intralingual translation has been
less examined by academia and held in less regard. This began already
with Jakobson, who also described interlingual translation as
``translation proper'' in his seminal essay. Criticism has been leveled
at the distinction of these two kinds of translation. For one, the
matter of whether two languages are truly distinct or that one is a
dialect of the other is arbitrary and can be a contentious subject. When
it comes to diachronic translation, a case can be made that some
historical variants of a language can be less accessible to a modern
user of that language than contemporary versions of other languages. The
difference in accessibility between \emph{Althochdeutsch} and
contemporary German is at least as great as the differences between
modern Danish, Swedish and Norwegian \citep[25--27]{schreiber_ubersetzung_1993}.
Furthermore, the distinction between intralingual and interlingual is
far less relevant to automated machine translations, which treat a
language variant no different than the languages that are considered to
be languages in their own right, simply as sets of data \citep[64]{shardlow_survey_2014}.

\subsection{Intralingual translation outside of
TS}

The process of rewriting a text to another variant of the same language,
specifically for the accessibility of older texts for a contemporary
audience, is known by several names depending on field of expertise or
language. Within Dutch literary studies it is referred to as
``\emph{hertaling}'', a portmanteau of ``\emph{vertaling}''
(translation) and the prefix ``\emph{her-}'' (comparable to the English
``re-'' in the sense of ``anew''). This term does not translate easily.
The same phenomenon has several terms in neighbouring language areas. In
French, such as in the digital archives of the Sorbonne, the term
\emph{actualiser} is used, which has quite a broad meaning.
\emph{Traduit en français moderne} is also common. German terms include
\emph{Aktualisierung} and \emph{Umformulierung}, of which the former is
also used to denote a digital ``update'' and the latter means as much as
``rewording''. In the English language, this type of textual alteration is
sometimes called rewording or ``modernizing''. In translation science
textbooks some of these terms are used to describe intralingual
translation or to provide alternative terms for it. There is a great
deal of ambiguity, as each of these words have other meanings, which
furthermore tend to be the meanings far more commonly used. When the
concept ``intralingual translation'' is used, or its counterparts in the
other languages mentioned, which are easily recognizable
(\emph{intralinguale Übersetzung}, \emph{intralinguale vertaling}, or
\emph{traduction intralinguale}) it is virtually always within the field
of translation science. Exceptions exist, as with the French medievalist
and literary historian Michel Zink, who pioneered research into
translations of historical variants of French language texts as well as
into the medieval practice of similar types of translation \citep[16]{galderisi_lancien_2015}.

Another reason to employ terminology derived from translation studies
and linguistics is that it enables precise and detailed description of
what actually happens in the course of the translation act. In other
words, it provided the tools to develop and describe a method. It is
important to note that scholars in the fields of translation science and
intralingual translation generally work descriptively, rather than
prescriptively. The studies used as reference material in the
development of the proposed translation method were not written as
instruction manuals for translators. Indeed, the reason why the proposed method was developed in the first place is because there is no instruction manual available for the type of textual alteration intended for the Correspondence of William of Orange-project. In some fields, the work
of the translator is almost treated as if it were a type of black box,
due to its inherently hermeneutic and subjective character. This does
not mean that these types of translations are not made, because they
are, in different language areas. There is as of yet, however, very
little academic attention for this phenomenon. Furthermore, translations
of older literary texts are at times carried out by commercial
publishers who may not share considerations or practices with academic
or scholarly translators. This practice varies in different parts of the world \citep[577]{berk_albachten_intralingual_571}.

\subsection{Structural, grammatical and
lexical}

A restricted translation as proposed here presupposes that we can
distinguish grammatical aspects from lexical ones in the ST. This
distinction is made for pragmatic reasons and is arbitrary to a degree
when it comes to translating a text. The lexical function or meaning of
a word or morpheme can vary depending on the grammatical order of the
sentence. These aspects of the text are intertwined in natural language.
In the strictest sense, the process of translation cannot affect only
the one element and not the other. As such, this method is designed to
operate on the level of the grammar of the ST while leaving the lexical
elements unchanged, while not assuming that such a distinction can be
fully maintained in the translation act. When the adjustments on the
grammatical level cause changes on the level of individual words present
in the ST, this is acceptable provided that it does not reduce internal
consistency of the resulting translation.

Grammar consists of syntax and morphology. The syntax determines which
combinations of words form a sentence, while morphology determines how
words are formed, and how they function. Within the translation method,
the \emph{clausal shift} is a translation strategy concerning syntax,
while the \emph{modal shift} concerns morphology. In addition, the
\emph{clausal shift} affects punctuation, which is not a grammatical
aspect of the text but rather a structural one, operating on a level
above the sentence. This is a necessary part of this method of
translation because the historical language variant follows a different
logic for the formation and visual representation of sentence structure.

\subsection{Shifts}

A ``shift'' in Translation Studies is a change that occurs in a text
because of translation. The distinction of shift types in \emph{Derivation
and Transformation} serves to create the most granular descriptions of
individual types of changes translators make. Hill-Madsen advises as
such that the shifts themselves or that level of analysis should not be
used for didactic or prescriptive purposes, only for analytical
purposes. Hill-Madsen groups the numerous shifts distinguished in the
course of his research by two different systems: ``species'' by frequency
and ``genera'' by function. The genera are then ordered in a cline of
``lay-orientedness''. These are \emph{structural reorganization},
\emph{ideational variation}, \emph{clarification},
\emph{concretization}, \emph{de-compacting}, \emph{neutralization} and
\emph{personalization} \citep[262--69]{hill-madsen_derivation_2014}.

Conceived as such, these \emph{shifts} each represent a specific
microstrategy for translators, used to describe the various
small-grained actions as part of a translation act. In this context, the
two shifts of this method are used in a prescriptive rather than a
descriptive sense, and cover several different textual adjustments. In a
descriptive analysis of translation strategies, a shift indicates the
smallest and most granular component of a translator's actions \citep[3--12]{hill-madsen_derivation_2014}. This applies to the prescriptive approach to
translation as well, i.e. this method and instructional documents, but
the two shifts of this method each have a potential ripple effect.
Because of the interconnected nature of aspects of language, altering
one aspect for a translation may necessitate further changes to the
text. With the \emph{clausal shift}, for instance, changing the class of
a clause within a sentence structure may be linked to a change in word
order or the inference of a subject. These changes can be described as
shifts in their own right, but since they are the result of the initial
\emph{clausal} or \emph{modal} shift they are not distinguished as parts
of the method in their own right.

\section{The translation method}

The proposed translation method consists of three steps. The following
section provides a summary of the method. The first step is
\emph{parsing}, in which the text is broken up into clauses, treating
the verb as a point of orientation. The second step is \emph{clausal
shift}, in which new (shorter) sentences are formed by changing some of
the subordinate clauses into main clauses, adding or altering
punctuation to create a visible new structure. The third step is
\emph{modal shift}, in which specific conjugations, indicative of
historical variants of Dutch, are normalized to their equivalents in
modern common use.

The examples used in the following section are drawn from the
correspondence of William of Orange. The corpus contains incoming and
outgoing letters, and the Dutch language documents within it mostly
concern matters of diplomacy, law, and warfare \citep[117]{hoekstra_correspondentie_2007}. The
texts have been converted from handwritten sources by myself, through
the transcription software of Transkribus.\footnote{See \url{https://transkribus.eu} (accessed on August 7, 2020).} Because of this, the source
text has not been normalized in spelling, though the abbreviations in
the original text have been resolved. While it is not required for the
transcription to be normalized, it is helpful if abbreviations have been
resolved as part of the transcription process, as the translation method
does not incorporate this element.\footnote{The William of Orange
  project at the Huygens Institute uses the transcription software
  Transkribus, which offers transcribers options to annotate and resolve
  abbreviations.}

\subsection{First step: Parsing}

The proposed method restructures the ST on the level of clauses, which
requires the translator to first identify and distinguish the clauses in
the text. In the context of this translation method, this step is called
``parsing''. Both historical texts and source language variants contain
inconsistent use of punctuation, and certain verb modes that differ from
modern Dutch. Because of this, the parsing has to be done manually, at
least until sufficient training data is available to develop automated
solutions. The translator uses the verb or verb groups to determine the
boundaries of each clause. No distinction is to be made between main
clauses and subordinate clauses as part of this operation.

\subsection{Parsing example}

Returning to the excerpt from the introduction, this example shows the
transition from the ST to a set of clauses. The source document (a
manuscript letter) is only structured in lines of text, with no
recognizable start or end of sentences or clauses.

\begin{quote}
Doerluchtige hoochgeboren furst genadige heer, ende Edele, Erentfeste,
hoochgeleerde, wijse seer voorsienige heerene\\
Wij houden Uwe Furstliche Genade ende Edelen genaede Indachtich vandie
deliberatien hier beuorens bij den Staten generall dick ende menichmall
gehouden, op het toelaten offte geheelick verbieden vanden trafficque,
commercie offte toeganck opdie Prouincien ende Steden, die bij onse
vianden sijn geoccupeert, welcke deliberatien genoechsam Indien state
gebleuen sijn dat geraden waere intselue geheelick mede op grote poenen
te verbieden, doch Indien verstande, dat sulx van wegen dien van
Franckrijck oick aenpaarlick soude hebben te geschieden Nu Ist sulx dat
vermits eenige occurrentien die ons dienthaluen sijn voer gecomen ten
aensiene vande Stede van Groeningenn shertogenbosch ende Breda, wesende
binnen landtsche Steden ende denwelcken alle toeganck lichtelicker dan
van andere souden mogen beleth worden, {[}\ldots{}{]}
\begin{flushright}
\citepalias[f. 179 r-180 v.]{arab_audientie_nodate}
\end{flushright}
\end{quote}

\noindent Note that the capitalization of words appears to be inconsistent. This
is partially because of the transcription process, as there are multiple
variations for the same grapheme in a manuscript, which cannot be
consistently translated into the binary of upper and lower case. In the
parsed version of this excerpt below, the verb groups have been
underlined. Note that the line breaks made here do not always coincide
with the ST commas.

\begin{quote}
Doerluchtige hoochgeboren furst genadige heer, ende Edele, Erentfeste,
hoochgeleerde, wijse seer voorsienige heerene\\
/Wij \underline{houden} Uwe Furstliche Genade ende Edelen genaede Indachtich
vandie deliberatien\\
/hier beuorens bij den Staten generall dick ende menichmall \underline{gehouden},
op het \emph{toelaten} offte geheelick \emph{verbieden} vanden
trafficque, commercie offte toeganck opdie Prouincien ende Steden,\\
/die bij onse vianden \underline{sijn geoccupeert},\\
/welcke deliberatien genoechsam Indien state \underline{gebleuen sijn}\\
/dat \underline{geraden waere}\\
/intselue geheelick mede op grote poenen te \underline{verbinden},\\
/doch Indien verstande, dat sulx van wegen dien van Franckrijck oick
aenpaarlick \underline{soude hebben te geschieden}\\
/Nu {Ist} sulx\\
/dat vermits eenige occurrentien die ons dienthaluen {sijn voer gecomen}
ten aensiene vande Stede van Groeningenn shertogenbosch ende Breda,\\
/\underline{wesende} binnen landtsche Steden\\
/ende denwelcken alle toeganck lichtelicker dan van andere \underline{souden mogen
beleth worden}, {[}\ldots{}{]}
\end{quote}

\noindent There are only two main clauses in this ST excerpt. The first starts
right after the address, with the SVO structure of ``Wij houden Uwe
{[}etc.{]}'', the second being ``Nu Ist sulx'', which would mean as much
as ``The situation now is:'', which is followed by subclauses. The parsing
follows the verbs in the text, as mentioned. There are additional verbs
within the third line, set in cursive. These are gerunds or infinitives
functioning as nouns. They are not part of the verb group, and therefore
not taken into account for the parsing process. On the whole, this step
requires only little editing judgement. There is a word group without a
verb, ``doch Indien verstande'', which could stand on its own (assuming
an unwritten but implied verb) or be included in the preceding or
following part. Because many clauses are reconnected later on in the
translation process, there is no need to be overly cautious with the few
uncertain cases.

\subsection{Second step: Clausal
shift}

This part of the translation is based on the shift in clause ranking.
The objective of the first two steps in the translation method is for
the TT to consist of shorter sentences than the ST. In order to achieve
that, the ratio of subordinate clauses to main clauses needs to be
changed in favour of main clauses. This means that a number of existing
subordinate clauses become main clauses, and that these are combined
with remaining subordinate clauses to form sentences. The sequence of
clauses relative to one another in the text is to remain unchanged.
``Clausal (ranking) shift'' is the descriptor for this change, but in
practice such a change requires a number of different actions from the
translator caused by the shift in clause rank. Generally it will require
a shift in the position of the verb within the clause, towards the
subject and the start of the clause. Present-day Dutch favours the
subject-verb-object sentence (SVO) structure, which the translator
should strive to produce in the TT. The component clauses may rely on
anaphora or lack a proper subject entirely, which can be resolved by
giving the new sentences grammatical subjects in the form of proper
nouns or names. This can be accomplished by repeating the appropriate
subject from a previous sentence. Finally, the translator utilizes
punctuation to indicate the new sentence structure.

This element of the method is the most susceptible to interpretation and
requires a translator to make choices. In these choices, the translator
follows the \emph{skopos}-rule, to translate in a way that enables the
text to function in the situation or for the people using it \citep[55]{hill-madsen_derivation_2014}. This means that the translator strives to rank
the clauses in such a way that the information in the text becomes more
accessible. The shortening of sentences affords the reader breaks in the
reading process and compartmentalizes information \citep[62]{shardlow_survey_2014}.\footnote{It is beyond the scope of this research to investigate the
  reason why the historical Dutch writing style employs much longer
  sentences in comparison to modern use.}

\subsection{Clausal shift examples}

The example from the previous step is used again here to illustrate the
restructuring of clausal and sentence structure.

\begin{quote}
Wij houden Uwe Furstliche Genade ende Edelen genaede Indachtich vandie
deliberatien, hier beuorens bij den Staten generall dick ende menichmall
gehouden, op het toelaten offte geheelick verbieden vanden trafficque,
commercie offte toeganck opdie Prouincien ende Steden, die bij onse
vianden geoccupeert {sijn}.\\
Welcke deliberatien {sijn} genoechsam {gebleuen} Indien state, dat
geraden waere intselue geheelick mede op grote poenen te verbieden, doch
Indien verstande, dat sulx van wegen dien van Franckrijck oick
aenpaarlick {te geschieden} soude hebben.\\
Nu Ist sulx {[}dat{]}: {[}vermits{]} Eenige occurrentien {sijn} ons
dienthaluen voer gecomen, ten aensiene vande Stede van Groeningenn
shertogenbosch ende Breda, wesende binnen landtsche Steden, ende
denwelcken alle toeganck lichtelicker dan van andere souden mogen
{worden} beleth.
\end{quote}

\noindent The (new) clausal rank in this excerpt is indicated by moving the
relevant verb or verb groups in either of two directions. For the main
clauses, to follow standard practice in modern Dutch, the finite verb
should be close to the subject. In subclauses, on the other hand, the
finite verb is placed at the end of the clause, including in combined
verb phrases. The historical language variant of these documents
resembles the German language more than modern Dutch in this regard.
Note also that there are transition words that are rendered obsolete by
the new structure, in which case they can be removed by the translator.
In some cases this may apply to entire short phrases (like ``Soo ist
dat\ldots{}'', which translates to ``It is thus:'') , which exist to mark
transitions in the original text where in modern convention one might
use paragraphs or line breaks instead.

The following excerpt, taken from a letter from the Prince's court from
1573, illustrates how the structural reorganization of the parsed text
can require the inference of a new subject in the creation of new main
clauses. The inferred subjects can be pronouns or repetitions of
appropriate noun groups from earlier in the text. The goal for the
translator is to reduce the amount of anaphora, similar to the syntactic
approaches within ATS. The following excerpt has already been parsed
following the first step of the translation method. It is worth noting
that this specific ST does not contain punctuation of any kind, which is
an infrequent but existing phenomenon in the corpus of the William of
Orange correspondence. Punctuation has been added in the TT to improve
the readability of the text.

\begin{quote}
{[}\ldots{}{]}\\
/Ghevende hem Remonstrant macht avthoriteyt ende speciael bevel by desen
de selven by alle middelen ende wegen volgens de voorschreven
versoucke\\
/by hem remonstrant aen ons gedaen\\
/daer toe te constringeren ende bedwingen\\
/Ordonneren ende bevelen eerhalve onse admiralen allen onsen Oversten
Capiteynen hoops ende bevelsluyden\\
/Versoucken oock allen govverneurs Magistraten ende anderen dient
aengaen mach /ende voor wye hy Remonstrant sich addresseren ende dese
jegenwoirdighe thoonen sal {[}\ldots{}{]}
\begin{flushright}
\citepalias[f. 261 r-v]{kha_i12_nodate}
\end{flushright}
\end{quote}

\begin{quote}
{Wij} ghevende hem Remonstrant macht avthoriteyt ende speciael bevel by
desen de selven, by alle middelen ende wegen volgens de voorschreven
versoucke by hem remonstrant aen ons gedaen, daer toe te constringeren
ende bedwingen.\\
{Wij} Ordonneren ende bevelen eerhalve onse admiralen, allen onsen
Oversten, Capiteynen, hoops, ende bevelsluyden. {Wij} versoucken oock
allen govverneurs Magistraten ende anderen dient aengaen mach, ende voor
wye hy Remonstrant sich {sal} addresseren ende dese jegenwoirdighe
thoonen sal.
\end{quote}

\noindent The clausal shift is the most invasive and complex procedure of the
entire translation method, as it involves nearly every sentence of the
source text. It relies on an effective verb-based parsing in the
preceding step. The translator needs to employ judgement in these
alterations as syntactic structures often have multiple possible
translations. Some examples:

\begin{quote}
/Voorts dient dese omme uwer F.G. te aduerteren,\\
/dat aen ons clachtich geweest zijn eenige Landtluden vanden dorpen In
schielandt onder uwer Excellentie protectie ende gouuernemen
gheseten,
\begin{flushright}
\citepalias{kha_e30_nodate}
\end{flushright}
\end{quote}

\noindent Might be translated as:

\begin{enumerate}\footnotesize
\def\labelenumi{\arabic{enumi}.}
\item {[}\ldots{}{]} dat eenige Landtluden vanden dorpen In schielandt, gheseten  onder uwer Excellentie protectie ende gouuernemen, aen ons clachtich geweest zijn.
\item {[}\ldots{}{]} dat eenige Landtluden aen ons clachtich zijn geweest, vanden dorpen In schielandt gheseten onder uwer Excellentie protectie ende gouuernemen.
\item Dese dient omme uwer F.G. te aduerteren. Eenige Landtluden vanden dorpen In schielandt, onder uwer Excellentie protectie ende gouuernemen gheseten, zijn aen ons clachtich geweest.
\end{enumerate}

\noindent While the following: 

\begin{quote}
{[}\ldots{}{]} /hier toe accerdeerde de toecoempste van deesen Jegenwoirdigen
oirloege\\
/ter oirsaecke vanden welcken die Rivieren vander Maese Rhyn waele Lecke
hebben haer corresponden opte stadt van dordrecht\\
/ende (Importeren aldaer het eeniche welvaert ende neeringe)\\
/geslooten zyn geweest\\
/ende die negotiatie vandien gehelycken gecesseert heeft Inder
vorigen\\
\begin{flushright}
\citepalias[621]{rad_stadsarchieven_nodate}
\end{flushright}
\end{quote}

\noindent Might be translated as: 

\begin{enumerate}\footnotesize
\def\labelenumi{\arabic{enumi}.}
\item De toecoempste van deesen Jegenwoirdigen oirloege {accerdeerde hier toe}. Ter oirsaecke vanden welcken {zyn} die Rivieren vander Maese Rhyn waele Lecke, hebben haer corresponden opte stadt van dordrecht, ({zij} Importeren aldaer het eeniche welvaert ende neeringe) geslooten geweest. Die negotiatie vandien {heeft} gehelycken gecesseert Inder vorigen.
\item Hier toe accerdeerde de toecoempste van deesen Jegenwoirdigen
  oirloege, ter oirsaecke vanden welcken die Rivieren vander Maese Rhyn waele Lecke {geslooten zyn geweest}. {Zij} hebben haer corresponden opte stadt van dordrecht ({ende} Importeren aldaer het eeniche welvaert ende neeringe), ende die negotiatie vandien {heeft Inder vorigen} gehelycken gecesseert.
\end{enumerate}

\subsection{Third step: Modal shift}

Whereas the previous step was mostly concerned with syntax, the third
step is concerned with the morphology of the verb. There are certain
verb modes that have either gone out of use in current-day Dutch or are
used in a different way. These specific verb modes are easily
recognizable and can be changed with only a relatively small impact on
the contents of the text. In regards to translating 16\textsuperscript{th} century Dutch,
this method is concerned with two forms of verb conjugation: the
subjunctive or conjunctive mood (\emph{aanvoegende wijs}) and the
present participle (\emph{onvoltooid deelwoord}) \citep[308 and 321]{van_den_toorn_geschiedenis_1997}.

The subjunctive mode has gone out of use in modern Dutch, beyond a
specific set of expressions, which themselves can evoke an antiquated
feel. It is used, for instance, in liturgical language (\emph{Uw naam
{worde} geheiligd} --- Hallowed be thy name, from the Lord's Prayer) or
traditional phrases (\emph{{Leve} de koning} --- Long live the king). Most
examples of subjunctive mood in the source texts can be replaced with
indicative or imperative forms of the same verb without severely
impacting the meaning of the text. No instance of subjunctive mood
should be retained in the TT because this grammatical aspect is no
longer used productively by Dutch readers.

The present participle is more complicated to deal with in a
translation. Contrary to the subjunctive mood, it is a verb type that is
still used productively in present-day Dutch. What has changed is that
certain applications of the present participle have gone out of use and
are difficult to understand for a modern reader \citep[97--138]{duinhoven_deelwoorden_nodate}. Inversely, this means that some instances of the present
participle being used in the ST can be left unchanged in the
translation. The distinction is generally left to the judgement of the
translator, but some general guidelines can be formulated. First, all
instances of present participles based on the verbs \emph{zijn} (to be)
and \emph{hebben} (to have), i.e. \emph{zijnde} and \emph{hebbende} are
to be considered characteristics of a historical language variant and
should be changed. Second, modern Dutch uses present participles as
adjectives or adverbs, not in absolute form. An absolute present
participle often occurs in subclauses, and can be changed to another
verb tense based on the clause rank.

\subsection{Modal shift examples}
\vskip 1em

\begin{quote}
ST: ``{[}Wij{]} \emph{wesende} nochtans oick well vanden aduise'' \citepalias[554]{arab_audientie_nodate}

TT: Wij zijn nochtans oick well vanden aduise

\vspace{1em}

ST: ``\emph{wesende} binnen landtsche Steden,'' \emph{ibid.}

TT: Wat binnen landtsche Steden zijn

\vspace{1em}

ST: ``{[}zij{]} \emph{verzoekende} daeromme aen ons'' \citepalias{kha_e30_nodate}

TT: zij verzoeken daeromme aen ons

\vspace{1em}

ST: ``Wy ende onse lieve Neve de Grave vanden Marck hadden hem te dien
eynde diversche opene missyve verleent ende gegeven,
\emph{addresserende} aende voorschreven capiteynen,'' \citepalias{kha_e30_nodate}

TT: {[}\ldots{}{]} verleent ende gegeven, geadresseerd aende voorschreven capiteynen,

\vspace{1em}

ST: ``sich te transporteren aende bovengemeld capitain Jan claess spiegel ende anderen {[}die{]} de voorschreven Schepe ende goede genomen
\emph{hebbende},'' (\emph{ibid.})

\vspace{1em}

TT: {[}\ldots{}{]} ende anderen die de voorschreven Schepe ende goede genomen hebben,
\end{quote}

\noindent The translation of verbs from the conjunctive to the indicative mood
tends to be rather straightforward. In modern Dutch, an irrealis mood
like the conjunctive is usually expressed periphrastically, through the
inclusion of some form of the verb \emph{zullen}/\emph{zouden} \citep[304]{van_den_toorn_geschiedenis_1997}. Interestingly, this particular verb was already
in use as an alternative to indicate irrealis (subjunctive or optative,
there is no morphological distinction in Dutch grammar) without using
the conjunctive mood in the 16\textsuperscript{th} century, as evident in the first
example below:

\begin{quote}
ST: ``Zijluden zouden (zoe verre het \emph{gebeurde} dat den viandt
hemluden tot eenighen tijden \emph{ouerliepe} ende \emph{quame} ijemandt
te gecrijghen) hem te beter moghen excuseren,'' \citepalias{kha_e30_nodate}

TT: Zijluden zouden (zoe verre het zou gebeuren dat den viandt
hemluden tot eenighen tijden zou ouerlopen ende ijemandt zou komen
te gecrijghen) hem te beter moghen excuseren,
\vspace{1em}
ST: ``dat de vyant overmits zyne macht ende volck, twelck hy te velde
\emph{hadde},'' \citepalias[38]{hua_staten_nodate}

TT: {[}\ldots{}{]}, twelck hy te velde zou hebben
\vspace{1em}
ST: ``ten zy hemluiiden tot conservatie der selver stadt voersien
\emph{werde} van behoerlycke gratie'' \citepalias[621]{rad_stadsarchieven_nodate}

TT: ten zy hemluiiden tot conservatie der selver stadt voersien zouden
worden van behoerlycke gratie
\vspace{1em}
ST: ``datmen die sonder vertreck \emph{stelle} tot volcoemen
delivrantie'' (\emph{ibid}.)

TT: datmen die sonder vertreck tot volcoemen delivrantie zullen
stellen
\end{quote}

\section{Discussion}

A deliberately restrictive and prescriptive approach to translation may
seem contrary to the nature of the translator's craft. The proposed
method however has a different goal when compared to conventional
translation. A successful translation in the conventional sense consumes
the source text and destroys the source language. The restricted
translation described here resembles the work of an editor or a
publisher proof-reading a manuscript almost as much as the work of an
interpreter. The target text and target language remain historical to an
extent and do not obscure the origins or most of the semantics of the
source, while becoming significantly more accessible. At the same time,
one of the advantages of the digital age and the ease at which textual
data can be stored is that the unmodified transcription can be retained
for an audience. Both the method itself and its products may, because of
this, provide translation with transparency.

The proposed method has its virtues for use in textual scholarship,
though the extent of its applicability will vary depending on the nature
of the textual project. The restricted translation requires a limited
amount of labour from a translator and requires less expertise in its
translator than a total translation would. After all, the method puts
the greater focus on morphology over lexical aspects, which are
unchanged from the ST. This makes the proposed method advantageous for
use in textual scholarship projects with large corpora of historical
textual sources. In the case of a historical ST a reader is required to
parse and interpret the historical syntax in addition to engaging with
other aspects of the historical text. This is the part that can be
resolved through the described alterations.

The restricted translation is not without its downsides or challenges.
First, there is no escape from the subjectivity of the translator. The
proposed method prescribes several operations, but the translator
retains a significant amount of responsibility when it comes to
interpreting the text and choosing how to implement the prescribed
changes. To be fair, this is true of all forms of translation, including
machine translations, which reflect the subjectivity of the translators
that developed it. At best, the proposed method can offer up an
explanation for the nature of the changes and the reasoning behind the
choices made in the reader's stead. The primary choice in this regard is
the decision to mostly disregard lexical elements and spelling in favour
of grammar in the course of translating. Such a question must be
resolved through experimentation as restricted translation is at this
time a new approach.

\section{Closing remarks}

In summation, the proposed method prescribes how to turn a text written
in a historical variant of Dutch into a restricted translation that
follows modern Dutch grammatical and syntactical structure, including
such elements as punctuation, while not altering lexical aspects that
belong to the historical language variant. This is accomplished by
identifying the grammatical clauses in the text, which are not otherwise
indicated by punctuation or capitalization (\emph{parsing}),
reorganizing them into a sentence structure that reflects modern textual
conventions (\emph{clausal shift}) and resolving certain grammatical
elements that have fallen into disuse in modern Dutch (\emph{modal
shift}). The purpose of this proposed method is to improve the
readability of historical source texts while not expending the time and
effort required for a total translation. While a total translation would
be more readable than a restricted translation, the option to produce
total translations may not always be feasible or cost-effective.

This approach to textual editing is connected to the notion that the
divide between historical texts and present-day readers goes deeper than
historical variants in spelling, lexicon and grammar. Through the
application of this method, the texts are given a visual structure that
they did not originally have. Reading the original sources is much like
listening to a spoken rendition, and to receive the information, one has
to consume the text from beginning to end. Modern readers are accustomed
to a visual style of text composition that uses elements like
punctuation, sentences, paragraphs, line breaks, headers. All these
textual elements benefit an audience that is visually oriented. This
enables reading practices such as skimming, which a modern reader cannot
apply to the historical text.

There are several opportunities remaining for future research. The first
objective would be to explore the effectiveness of the research method
and the extent to which the method can be carried out by other
translators using the instructions. This will be accomplished by having
volunteers make a translation of Dutch historical texts (from the
aforementioned corpus) using an instruction document detailing the
translation method. These participants are not instructed beforehand
about the contents of the method. Multiple rounds of tests have been
planned, and the feedback from participants on the clarity of the
instructions and the effectiveness of the method are used to improve
these tools in between tests. This way, the method and the accompanying
instruction document undergo a series of iterative test
phases.\footnote{Due to the COVID-19 outbreak and security measures, it
  was not possible to perform sufficient tests with volunteers in a
  reliable fashion and controlled environment in time for the submission
  of this paper.}

Secondly, future research would serve to widen the scope of the
translation method. This method was geared towards a historical variant
of the Dutch language connected to a specific time period. Further
research would endeavour to apply the same principles of diachronic
translation to other time periods, or to historical variants of other
languages. Efforts in this vein would require collaboration with
researchers from applicable fields of expertise, especially concerning
other modern languages. Finally, once a certain amount of text has been
translated in this way, digital tools like NLP can be developed based on
the training data to automate the translation process to some degree.
This restricted translation is less hermeneutic than a total
translation, and should be expected to be more receptive to
machine-learning solutions.

\begin{flushleft}
\bibliography{references/maat}  
\end{flushleft}

\end{paper}