\contributor{Mark Bland}
\contribution{Some Answer Poems and their Manuscript History: Jonson, Herrick, and the Circulation of Verse}
\shortcontributor{Mark Bland}
\shortcontribution{Some Answer Poems and their Manuscript History}

\begin{paper}

\begin{abstract}
This article uses manuscript sources and stemmatics to illuminate the history of the three answer poems that are discussed. The first example from Ben Jonson shows how he repeatedly tinkered with the text of a poem he wrote in answer to Sir William Burlase. The second example demonstrates that ``Of Inconstancy'' is an exchange between Jonson and Sir Edward Herbert, thereby adding a new poem to Jonson’s canon. The third example illustrates how Robert Herrick took a poem by another and then altered it to write a polemic against it. In both the last two cases, the article resolves issues of uncertain authorship.
\end{abstract}

\section*{} 
\textsc{The circulation of manuscript poetry} in the early-mid
seventeenth century was closely linked to notions of conviviality and
civility; even if, sometimes, a poem might invite a trenchant answer. Hence, the texts in any miscellany reflect both personal taste and an
interest in the broader social conversation to which those texts were
connected. That sense of engagement with the literary utterance extended
to the composition of poems in response to others that were in
circulation. Sometimes answers were direct and personal, sometimes they
were exercises in wit, and sometimes they were deliberately public
responses to a text that was either in circulation or to a text to which
the person concerned had had access. All these different kinds of answer
poetry are to be found in the examples that are to be discussed
presently.

In an important article on answer poetry, Scott Nixon made the point that
certain poems invited a direct engagement with their concerns either
through the style of their argument, or through the way in which they allowed
for debate and even parody \citep{nixon_aske_1999}. Such poems are quite different in tone and intention from those
written in more than one voice, such as Jonson's ``Epode'' (\emph{Forrest}
\textsc{xi}) or ``A Musicall Strife'' (\emph{Underwood} 3), where the
debate is the poem.\footnote{The numbering of the \textit{Epigrammes} and \textit{The Forrest} was established by Jonson, and has been followed by all modern editors; that for \textit{The Underwood} is the accepted convention followed by both Simpson (Oxford: 1925-52) and the Cambridge editors (Bevington et 2013), as well as for the forthcoming Oxford edition.} The obvious corollary to this is that authors
usually did not answer their own poems. Nevertheless, there are examples
where both the original and the answer poem have been attributed to the
same author; and others where a single ``text'' has merged both the
proposition and the answer. In both cases, there are clearly attribution
and identity issues to resolve, and it is here that stemmatic analysis
may help by providing evidence of the relationship between transmission,
revision, and response --- and thus issues of origin and authorship.

The point about stemmatic analysis is that maps of
transmission do not only establish how specific copies of a text are related and through which networks they passed: they also cumulatively map how
miscellanies were pieced together. Thus, it is particularly useful if
multiple texts by different authors are collated and analysed as this
adds precision and complexity to the study of the social networks within
which the texts are located. It is for that reason that scholars have
begun to realize that the scribal networks of manuscript circulation in
the early modern period are more informative and significant than had
been assumed \citep{bland_stemmatics_2013}. Since most of the work that has been done
involving manuscript transmission has been concerned with the poetry of
Donne, the mapping of textual transmission for other authors such as
Jonson, Roe, Beaumont, Herrick, Strode and Randolph could offer further
insight into the social networks that a single author study
obscures. This is especially so when answer poems are concerned, because they are
by their nature multiple author texts. Hence, what was once dismissed
as the corrupt residue of literary fashion is now understood as an
important by-product of, and witness to, the histories of revision,
transmission, and reception. This, in turn, has led to a renewed
interest in the miscellanies and their relationship with one another.

The following discussion will turn first to two examples from Jonson,
including a previously unattributed poem, and then to one that as been
attributed to Herrick but is not by him. Jonson and Herrick are useful
counterpoints to Donne as their initial entry into circulation occurs at
different stages in the transmission process to each other and, thus, as
witnesses to the history of the underlying documents they do not
duplicate the patterns of circulation. That broader matter, however, of
how multiple authors may be used to interpret a manuscript's history is
beyond the scope of the present article. For now, it will suffice to
resolve the more specific issues of authorship and the circulation of
these poems and thus lay the foundation for future work.

\section*{\centerheading{I}}

Jonson's exchange with Sir William Burlase (\emph{Underwood} 52; see Figure \ref{fig:bland:poemetext}) is both
one of his most appealing poems, and a classic example of the ``answer
poem'' genre. For all its brevity and simplicity, it is beautifully
pitched between self-effacing irony and genial praise. Both Burlase and
Jonson suggest that their own command of their mistresses, Art and Poetry,
are inadequate to do their praise for the other justice. Burlase was a friend of
Dudley Carleton and John Chamberlain (a member of parliament), and he
painted for pleasure. He modestly suggests that because he had less than
``half'' of Jonson's art, he will never be able to truly capture Jonson's ``worth'' to
his friends and their admiration of him. Jonson answers by drawing on
the Senecan notion that true benefits are reciprocal. He mocks his own
obesity by suggesting that a ``blot'' would be quite sufficient to portray
him, and laments that as a poet he only has the black and white of the page, and not
the full colour palette of the painter. This allows Jonson to open up
the distinction between his claim to honesty and the ``flatt'ring Colours and
false light'' of the painter. He ends by returning the ample praise of
his friend: conviviality, modesty, and the reciprocation of admiration
are the true benefits that are shared in their exchange, and hence the
poems epitomize the friendship between the two men.

\begin{figure}
    \centering
    \includegraphics[width=.70\textwidth]{media/Bland1.jpg} \caption{Ben Jonson, ``A Poeme sent mee by Sir William Burlase'', \emph{Workes} (1640), 2G3\textsuperscript{v}-4\textsuperscript{r}.}
    \label{fig:bland:poemetext}
\end{figure}

Jonson clearly identifies Burlase as the author of the poem that he
answers, and hence it has been accepted into editions of Jonson's poetry
as the contextual material against which the answer ought to be
understood. There are eleven manuscript copies of the poem to Burlase,
as well as the text in the 1640 \emph{Workes}, the two Benson piracies
(the \emph{Execration upon Vulcan} and \emph{Horace his Art of Poetry},
both published in 1640), and another copy in \emph{Parnassus Biceps}
(Wing W3686 1656, C7\textsuperscript{r-v}). At no time does Burlase's initial poem circulate separately from Jonson's answer. Two witnesses,
Bodleian Library Rawlinson Poetry MS 142 (ll.1-15) and \emph{Parnassus
Biceps}, have Jonson's answer only. These two ``answer only'' versions
descend from different lines within the first state of the poem. Among the
other witnesses, Edinburgh University Library MS Dc.7.94 is a manuscript
copy of the printed Benson duo-decimo piracy, and Bodleian Library
English Poetry MS c.50 is very corrupt and omits the first six lines of
Jonson's poem. As elsewhere, the Benson piracies conflate stolen sheets
from the 1640 \emph{Workes} with an earlier manuscript version. The
sigla used in the stemmatic diagram (see Figure \ref{fig:bland:poemestemma}) combines the STC location
symbol for the library (i.e. \textsc{l} = London), with the first
letter(s) of the manuscript series (e.g\textsc{. a} = Additional), and
the full manuscript number.

The recent Cambridge Jonson edition \citep{bevington_cambridge_2013} did not
attempt textual reconstructions of this kind. The stemma demonstrates
that Jonson tinkered with the poem and revised it four times. One of the
revisions was to line 4 of the poem by Burlase, where ``the trier'' was
altered to ``will tire'': this alteration must have been made by Jonson
around the time of the death of Burlase, as it occurs only in the
Newcastle manuscript (\textsc{lh4955}), and the three editions printed
in 1640. The same line of recension also alters line 18 of Jonson's
text, substituting ``form'd'' for ``drawne''. The other principal revisions
affect lines 1, 12, 15, 21 and 24 of Jonson's answer, and it is the
gradual evolution of these that separates the earlier stages of
revision.

The original tradition, as it was circulated, survives through seven
witnesses, and has twenty-three small-scale variants that document the
history of its reception: of these, one group of three manuscripts reads
``What'' as the first word of the poem while the rest, through all lines
of descent, read ``Why''. The ``What'' group includes the fifteen-line
Jonson-only fragment, Rawlinson Poetry MS 142. The other two manuscripts
(Yale Osborn MS b200, and British Library Harley MS 6931) are united by the
variants of ``be'' for ``are'' in line 3, ``had'' for ``haue'' in line 15 and
``I'de'' for ``I would'' in line 24. The Rawlinson manuscript has the
correct reading ``haue'' in line 15 and is further distinguished by
reading ``w\textsuperscript{th} w\textsuperscript{ch} I am'' for
``wherewith I may'' in line 3. On the other side of this line of descent,
Folger V.a.97 and \emph{Parnassus Biceps} are united by ``of'' for ``at'' in
line 6 and the ``but'' of ``describ'd but by'' in line 11, whilst British
Library Harley MS 6917 reads ``be [...] by which I may'' in line 3 and
``while'' for ``whilst'' in line 11. The final manuscript, Bodleian MS Eng.
Poet c.50 has multiple corrupt readings as well as lacking the first six
lines of Jonson's answer, but it does read both ``whilst'' and ``describ'd
by'' thereby indicating a further line of transmission within this group.

\begin{sidewaysfigure}
    \centering
    \includegraphics[width=\textwidth]{media/Bland2.jpg}
    \caption{`A Poeme sent me by Sir William Burlase'.}
    \label{fig:bland:poemestemma}
\end{sidewaysfigure}


Jonson's first revision was to lines 15 and 18, replacing ``Y'haue made''
with ``You made'' and the passive ``draw, behold, and take delight'' with
the more active ``draw, and take hold, and delight''. This tradition has
two extant descendants (British Library Sloane MS 1792 and Additional MS
30892), a sibling of thee latter of which was used to ``correct'' the
stolen sheets of the 1640 \emph{Workes} that were used as copy for the
Benson piracies, changing ``Yet'' to ``But'' in line 22 and reverting to ``I
would write'' for ``I will'' in line 24. Both these witnesses have the
variant ``drawe'' for ``drawne'' in line 12, with the Sloane manuscript
reading ``bee'' for ``are'' in line 3 and ``Yet'' in line 22.

Jonson then made three further revisions by the mid-1620s. The first was
to alter the fourth word of the first line from ``Why? Though I be of a
prodigious wast'' to ``Why? though I seeme {[}\ldots{}{]}''; In line 12 ``With one
great blot yo'haue drawne me as I am'' was altered to ``y'had drawne''; and
in line 24, ``I would write \emph{Burlase}'' was rendered absolute with ``I
will write {[}\ldots{}{]}''. Only Leeds Archives MS MX237 exists in this state.
The Newcastle manuscript then alters line 4 of the Burlase poem to ``will
tire'' and ``Ne'' for ``Nor'' in line 21. The final revision of the poem for
the 1640 \emph{Workes} involved a further tweak to line 12 with ``drawne''
altered to the more physical ``form'd''.

The final issue involves the history of the Benson piracies. Here, as
elsewhere, the piracies are demonstrably a conflation of stolen sheets
of the 1640 \emph{Workes} emended by reference to another manuscript
source, in this case a relative of British Library Additional MS 30982.
Hence, the Benson texts have the final readings ``will tire'', ``form'd'',
``You made'', ``draw, and take hold, and delight'', and ``Ne''. combined with
``But'' in line 22 and ``I would'' in line 24: a fact indicative of their
conflated status. This is recorded by the dotted line that links them
with their source.

What this stemma illustrates is the history of an evolving text that is
circulated early on and was then revised on multiple occasions with
seemingly minor adjustments. Under the circumstances, corruption is only
one part of the issue, for the history of the poem is as much horizontal
across the tradition as it is vertical within each state. To edit the
poem without a sense of this evolution, as the Cambridge edition does,
failing to separate revision from error for those variants listed in the
commentary, is a dereliction of editorial responsibility.

\section*{\centerheading{II}}

If the Burlase exchange is an obvious example of the answer poem,
another, ``Of Inconstancy'', is quite the opposite. It is found in three
manuscripts: the siblings British Library Harley MS 4064 and Bodleian
Library Rawlinson Poetry MS 31, and in University of Nottingham Portland
MS Pw V37. In the Rawlinson and Harley manuscripts the poem is laid out
as a single text; it is only in the Portland manuscript that the stanzas
are divided by the subtitle, ``The Aunswere, in praise of it''. This copy
of the poem was not known when it was last discussed in connection with
the poems of Edward, Lord Herbert of Cherbury in the early twentieth
century \citep[119 and 167--68]{moore_smith_poems_1923}.

Most of the poems of Edward Herbert were printed for the first time
some seventeen years after his death, in 1665 (Wing H1508). There
exists, as well, an important manuscript of the poems with corrections
in Herbert's hand (British Library Additional MS 37157), and there are a
number of other poems that survive only in manuscript copies \citep[\textsc{ii}: 167--68]{beal_index_1980}. The most recent edition was prepared shortly
before the First World War, though it was published in 1923. The editor,
Moore-Smith, based it on the 1665 text because, he argued, ``the printed
book is based on a manuscript which represented Herbert's second
thoughts'' \citep[xxvii]{moore_smith_poems_1923}. Moore Smith's edition is generally
a good one, and he certainly knew of Herbert's corrected copy. He also
collated some of the more obvious coterie manuscripts, though he did not
do so thoroughly. Instead of using British Library Harley MS 4064, he
printed several poems that had not been included in \emph{1665} from
Bodleian Rawlinson Poetry MS 31, and he treated one of them, ``Of
Inconstancy'' as dubia.

In Rawlinson Poetry MS 31, the text is copied on f.36\textsuperscript{r}
with the title, and the attribution, ``Sir Ed\={w}: Harbert''. It reads:

\begin{figure}[H]
\begin{quote}
Inconstancy's, the greatest of synns \\
It neyther endes well, nor beginns \\
All other ffaultes, wee simplye doe \\
This 'tis the same ffaulte, and next to: \\

Inconstancye, noe synn will proue \\
yf wee consider that wee Love \\
But the same beautye in another fface \\
lyke the same Bodye, in another place:\textsubscript{/}
\end{quote}
\end{figure}

\noindent Moore-Smith did not record the Harley manuscript, or its variants, and
must have overlooked it. Instead, he reproduced the above with some
minor adjustments, and noted in his commentary that ``Perhaps they {[}the
lines{]} are by two authors --- the second stanza being an answer to the
former. In this case it is the answer one would assign to Lord Herbert'' \citep[167]{moore_smith_poems_1923}. His guess, of course, was to be proved by the
Portland manuscript.

The copy of ``Of Inconstancy'' in Harley 4064 f.259\textsuperscript{v} is
in the hand of the second scribe, who is not particularly
careful --- although he is less idiosyncratic in his spelling than the
Feathery scribe, who prepared Rawlinson Poetry 31 \citep{beal_praise_1998}. On the
whole, the second Harley scribe punctuates less heavily, though he does
record a comma after ``This'' in line four, implying that there should be
a colon after ``doe'', a period after ``beginns'', and a semi-colon after
``synns''. Other minor adjustments include the reading ``too'' at the end of
line four, the words ``loue'' (l.6) and ``body'' (l.8) with an initial lower
case, and the absence of the initial double ``ff'' that is characteristic
of Feathery. The most substantial difference is in the first line, which
the second Harley scribe reduces to a regular tetrameter at the expense
of the meaning of the poem. As well as removing ``of'', he introduces a
hyphen into ``Inconstancy'':

\begin{quote}
In-constancy the greatest sins \\
It neither ends well, nor beginns \\ 
All other faults wee simply doe \\
This, tis the sam fault and next too \\ \\
Inconstancy no sinn will proue \\
If wee Consider that we loue \\
But the same bewty in another face \\
like the same body in another place.  
\end{quote}

\noindent The final manuscript unknown to Moore-Smith came from Welbeck Abbey. On
the whole, the text of Nottingham MS Pw V37 p.204 is better than that
found in the Rawlinson-Harley pair, though too many words have an
initial capital. This suggests that there may be a lost intermediary
between the answer as it first circulated and the copy that is preserved
in this source. In line three, the Portland manuscript has the obviously
correct reading ``singly'' instead of ``simply'', as it is found in the
other pair; and in line four, it reads ``first'', where the other
tradition has the weaker ``same''; elsewhere, the reading ``Inconstancy
is'', in the first line, renders the verse as a regular pentameter. Most
importantly, the manuscript demonstrates that Moore-Smith's hunch was
correct, and that the two stanzas are separate poems:

\begin{quote}
\-\hspace{1cm}Of Inconstancy \\ \\
Inconstancy is y\textsuperscript{e} greatest of Sinnes; \\
Itt neither Ends well, nor beginnes: \\
\-\hspace{.5cm}All other Faults Wee singly doe, \\
\-\hspace{.5cm}This is y\textsuperscript{e} first Fault, and Next too. \\ \\
\-\hspace{1cm}The Aunswere, in praise of it \\ \\
Inconstancy, noe sinne will proue \\
If wee consider y\textsuperscript{t} wee loue \\
\-\hspace{.5cm}But y\textsuperscript{e} same Beauty in another face \\
\-\hspace{.5cm}Like y\textsuperscript{e} same Body, in another place.
\end{quote}

This version makes it possible to resolve the problem of attribution,
for there is nothing unusual about a poem being circulated with an
answer to it in verse miscellanies of the period \citep{hart_answer-poem_1956,nixon_aske_1999}. With the support, therefore, of Rawlinson Poetry 31, and the
judgment of Moore Smith, the ``Aunswere'' may safely be attributed to
Herbert. This leaves the question as to who wrote the other four lines
of the text.

On external grounds alone, Jonson would have to be considered a likely
candidate as author of the epigram. At the end of the first decade of
the seventeenth century, he was obviously closely involved with Herbert:
his epigram to Herbert, Herbert's dedication of the satire ``Of
Travellers'' to Jonson, the gift by Herbert to Jonson of a copy of
Tertullian, and the manuscript of \emph{Biathanatos}, are all indicative
of Herbert having become an intimate part of Jonson's literary
circle.\footnote{The copy of Tertullian's \emph{Opera} (Franeker, 1598)
  is at Charlecote House, Warwickshire, shelfmark L6-22; the manuscript
  of \emph{Biathanatos} is Bodleian Library, Oxford, MS e Musaeo 131
 \citep{bland_jonson_1998}.} This is re-inforced by the arrangement of Harley 4064
and Rawlinson Poetry 31. In Harley 4064, the poem is part of a sequence
that includes Jonson's verse letter to Sir Robert Wroth in praise of a
country life, followed by Donne's ``Twickenham Garden'', followed by the
epigram and its answer, followed by Donne, Jonson, and then Herbert on
the death of Cecilia Bulstrode. With Rawlinson Poetry 31, the two Donne
poems have been removed, leaving the Jonson and Herbert poems as a
sequence together.

When ``Of Inconstancy'' is read against Jonson's other epigrams, style,
sentiment, subject matter, and metre, all cohere to identify his hand.
The idea of constancy and fortitude as manly virtues that require a Stoic
centredness, are to be found throughout the poems. Sir John Roe, for
instance, is one who has endured ``His often change of clime (though not
of mind)'' (\emph{Epig}. 32); William Roe is reminded that ``Delay is bad,
doubt worse, depending worst'', and Jonson goes on to add ``Each best day
of our life escapes vs, first'' (\emph{Epig}. 70); Sir Henry Goodyere is
complimented for his ``wel-made choise of friends, and bookes''
(\emph{Epig}. 86); Sir Thomas Roe is told ``He that is round within
himselfe, and streight, \textbar{} Need seeke no other strength, no
other height'' (\emph{Epig}. 98); and Herbert is praised for ``Thy
standing vpright to thy selfe'' (\emph{Epig}. 106). If the theme is
insistent, there are other echoes as well that are more subtle, but
which show the same turn of mind, whether it be ``Th'expence in odours is
a most vaine sinne'' (\emph{Epig}. 20), or the structure and rhythm of
the final couplet to Sir Henry Savile (\emph{Epig}. 95):

\begin{quote}
Although to write be lesser then to doo, \\
It is the next deed, and a great one too.      
\end{quote}

\noindent Even if, on internal evidence alone, one might hesitate (it is, after
all, a short poem), what makes the attribution convincing is that the
literary judgments about style and substance combine with the evidence
from the Portland manuscript that the two stanzas are two poems, and the
fact that the Rawlinson-Harley pair of manuscripts are so clearly
identifiable with the verse of Donne, Jonson, and Herbert. The poem is
not by a later imitator, nor is it characteristic of Donne or Sir John
Roe, but rather it has much in common with a group of early fragments by
Jonson, on ``Murder'', ``Peace'', and ``Riches'', that are to be found in
\emph{Englands Parnassus} (STC 378-80 1600, P2\textsuperscript{r},
Q2\textsuperscript{v}-3\textsuperscript{r}, S1\textsuperscript{v}). It
ought not be surprising that from time to time Jonson might sketch out
an idea, or that friends might engage in a little literary bandinage.

Hence there are, perhaps, three reasons why Jonson may have chosen not
to preserve this poem: first, if Jonson had started early on with the
idea of writing epigrams about abstract moral ideas, it is a stage that
he developed through, and it is possible he cannibalized much of this
material for his \emph{Epigrammes}. Second, the poem and its answer do
not particularly fit either the \emph{Epigrammes} or the \emph{Forrest},
and it is evident elsewhere that Jonson did not include a number of his
early poems for this reason. Third, Jonson regularly treated of the
theme elsewhere, and he may have felt that these four lines were rather
a sketch that might be regarded as unfinished. If Jonson had intended a
longer meditation on inconstancy, for instance, then the next logical
step would have been to turn the poem towards the praise of patience,
fortitude, and the golden mean with a specific person in mind (e.g. ``Yet
you, H\textsc{erbert}, who understand such things, \textbar{} And need
not my advice when each day brings \textbar{} New trials to test one's
patience and the will {[}\ldots{}{]}'', and so on --- these lines, I should add,
are mine, not those of Jonson). It may simply be that Jonson never found
the right moment for an extensive meditation on Lipsian notions of
constancy and neo-Stoicism.

For present purposes, the broader significance of ``Of Inconstancy'' and
its answer, beyond their inclusion in the forthcoming Oxford edition of
Jonson's \emph{Poems}, is that it is another example of a widespread
literary practice. Neither Jonson nor Herbert are any less identifiably
themselves for all that they have exchanged poems. On the other hand,
neither exists in absolute isolation from the other: they are part of a
community of friends, and texts. What stands out about this coterie of
the first decade of the seventeenth century is the way in which the
people involved record their sociability through the exchange of verse
letters, satires, epigrams, and answer poems: Jonson was not always
seeking patronage, sometimes his poems are genuine expressions of
\emph{amicitias}. This is quite different from the social circulation of
verse in Oxford in the 1620s and 1630s which, in many ways, is less
intimate.

\section*{\centerheading{III}}

When Herrick was edited in the mid-twentieth century, the first edition
of \emph{Hesperides} was taken as the copy-text where possible (Martin
1956), whilst the full extent of Herrick's circulation in manuscript was
not established by the editor. Rather, decisions about copy-text for
known manuscript poems were based on the convenience of the Bodleian and
British Libraries, rather than a strict analysis of textual traditions.
Herrick was not alone in being edited in this way: Simpson edited
Jonson's \emph{The Underwood} from the 1640 \emph{Workes} \citep{herford_ben_1925} --- a decision that was repeated in the recent Cambridge
edition; Ayton was edited from a couple of important manuscript
collections, with miscellany copies ignored \citep{gullans_english_1963}; even Henry
King (edited by the manuscript scholar Margaret Crum) got the light
touch and was primarily edited from the first printed edition \citep{crum_poems_1966}. Similarly, Helen Gardner, who was a formidable manuscript
scholar, edited Donne from the 1633 \emph{Poems} and a small group of
related manuscripts \citep{gardner_elegies_1965,gardner_divine_1978}. Hence the stemmatic analysis
of manuscript traditions has been neglected, especially following some
poor work by \citet{leishman_you_1945} and \citet{wolf_ii_if_1948}.

Something approaching the full extent of the surviving manuscript
evidence for Herrick was established by Peter Beal more than thirty
years ago \citep[\textsc{i}: 527--66]{beal_index_1987}. What Beal revealed was a
Herrick who was deeply involved in the manuscript culture of the 1620s
and 1630s, and whose manuscript texts were variant from the printed
texts in the \emph{Hesperides} in significant ways. In particular, it
became evident that Herrick did not necessarily revise in a straight
progression through various stages, but that he might also revise by
returning to his original papers and re-editing a poem in a completely
different way, altering different lines, and hence creating two or more
independent lines of transmission \citep{cain_lords_2011,connolly_editing_2012}. The most recent Oxford edition of \citet{cain_complete_2013} has
addressed this by mapping the evolution of the poems in manuscript in a
second volume, and providing some with stemmatic diagrams (though not
all).

What follows is a discussion of the transmission history of a group of
manuscripts for a specific problem text called ``To his False Mistress''
that has been attributed to Herrick, and for which Herrick wrote an
answer ``Goe perjured man''. It is a poem for which the recent editors
provided a stemmatic diagram and copy-text, both of which are at issue
here, and in which they acknowledge that ``there is no real evidence to
suggest that Herrick wrote this poem'' although they do observe that
``Herrick might well have been involved with tweaking the attributed
version'' \citep[\textsc{ii}: 59]{cain_complete_2013}. Nevertheless, their
analysis fails to identify that point of transition from an original to
an altered version and what those changes involved. As will be shown,
although Herrick did not write the poem, he did tweak a total of four
words. The rest of the manuscript history is one of scribal transmission
where the variants were neither instigated by the original author nor by
Herrick. Nevertheless, owing both to their reconstruction of the
evidence and their adoption of a specific manuscript as their witness,
the copy-text selected by the editors is variant by a further twelve
words to the changes that were introduced by Herrick.

As will become evident, the reconstruction of scribal manuscript
transmission without the survival of authorial papers poses a particular
challenge to editorial practice because error is both endemic to the
evidence and particular to the witness. Manuscript is different from
print. When an editor selects a printed witness, removing the variants
that were identified during stop-press correction, that choice reflects
what several hundred or thousand people first read as the text. When an
editor selects a scribal manuscript because there is no other witness,
without removing the variants generated via scribal transmission, that
text reflects what one person read and, in this instance, what
twenty-one did not. The issue is both of philosophical and practical
import.

The answer part of the exchange, sometimes known as ``The Curse'', is well
attested as by Herrick in manuscript and was printed in the 1648
\emph{Hesperides}: it survives in at least 59 manuscript copies (Beal
1987, HeR 49-107). ``To his False Mistress'', on the other hand, is not in
\emph{Hesperides} and survives in 22 known copies \citep[HeR 379--400]{beal_index_1987}: only eleven of which are copied with ``The Curse'' directly
following. A further five witnesses have both ``To his False Mistress''
and ``The Curse'' present in the same miscellany, but at some distance
from one another, and hence there stemmatic relationships belong to
different traditions. One of the first issues that arises, therefore, is
why ``To his False Mistress'' was excluded from \emph{Hesperides} if it
was written by Herrick.

The obvious and logical answer to this question is that Herrick's poem, ``The Curse'', is
an answer poem, and that such poems were not usually written by the same
person as the one that gave rise to the response. ``The Curse'' is also
self-contained, and it does not require knowledge of the other poem to
appreciate its moral stricture. In other words, we are dealing with an
example of social verse, similar to Jonson's answer poems, although it
is not as convivial a response as the others. By removing the source
poem, Herrick turns the direct trenchancy of the answer into an abstract
and distanced comment: the ``perjur'd man'' could be any man and not
simply the complainant of ``To his False Mistress'': by removing the cause
of is animus, Herrick allowed conviviality to be restored.

\begin{sidewaysfigure}
    \centering
    \includegraphics[width=\textwidth]{media/Bland3.jpg}
    \caption{Anon, ``To his False Mistress''.}
    \label{fig:bland:mistress}
\end{sidewaysfigure}


The stemmatic diagram for ``To his False Mistress'' (see Figure \ref{fig:bland:mistress}) maps the
transmission of the poem: it describes an early version of the text, and
then two main lines of descent, and establishes that the attribution to
Herrick occurs at the bottom of one line of descent of the revised
version, and that both these copies descend from the same source (MS 9).
For a valid attribution, an early state of the poem ought to have a
clear association, and the further away that attribution is from the
original copy, the less confidence we can have in the association (Bland
2013). As can be seen the initial state of the text (MS 1) has no
attribution and, while ``The Curse'' is present in these miscellanies, it
is does not follow on from ``To his False Mistress'' --- in fact, in both
cases, it occurs earlier in the collection as a whole. Those manuscripts
that do have ``The Curse'' immediately following are marked with an
asterisk.

Perhaps as useful from a textual point of view is the fact that the only
variants between British Library Sloane MS 1446 and West Yorkshire
Archives (Leeds) MS 156/237 are differences involving spelling and
punctuation, but no substantive disagreements. Further, both copies
agree that the poem was first written as three quatrains. As a
copy-text, therefore, it represents the poem as Herrick received it (and
thus as close as we can establish to the original author's intention),
before Herrick made changes to suit his own purposes. As recorded in the
Leeds manuscript, the text of the poem originally read:

\begin{quote}
Whether are all her false oathes blowne? \\
Or in what Region doe they liue? \\
I'me sure no place, where fayth is knowne \\
Dare any harbor to them giue: \\

My wither'd hart which loue doth burne \\
Shall venter one sigh with the winde, \\
And neuer back againe returne \\
Till one of her lost vowes it finde. \\

There may they wrastle in the Skye, \\
Till they doe both one lightning proue; \\
Then falling lett it blast her eye, \\
That was so periur'd in her loue.
\end{quote}

\noindent When he got hold of the poem, Herrick altered the archaic ``doth'' in line
five to ``did'', changing the present into the past; the ``lost vowes'' in
line eight once again became the ``false oathes'' of line one; and in the
final line ``so'' was altered to ``thus'', changing the egregious emphasis
with the rhetoric of an argument. Overall, the changes shift the
sentiment from one of present suffering to past hurt, more clearly
enabling Herrick to level the accusation of insincerity. To argue, as
the Oxford editors have implicitly done, that Herrick then made a
further twelve changes to such a short poem, none of which are
indicative of a more refined intelligence revising it, is difficult to
credit as plausible.

The critical reading to determine is ``venter'' (to sell) in line 6 and
its variant ``venture'' (to invest, put forth): this occurs on both sides
of the MS 2 line of descent, at stages independent of one
another, and simply reflects the scribal habit of substituting the more
familiar word. This is not problematic because it is the logical
variant: what is important about ``venture'' is that once it has occurred
there is no way back to ``venter''. It would, in other words be much
harder to argue that three separate scribes changed ``venture' to ``venter''
than the other way round, which confirms the primacy of ``venter'' as the
original reading. One other manuscript reads ``tender'', a variant that
would not be repeated independently because it involves a more
significant change. There are a couple of other independent agreements
that are straightforward in their cause, but for present purposes what
matters for reconstructing the stemma is that ``venter/venture'' is
removed from the primary analysis beyond placing ``venter'' at the top of
the tree. When this is done, the line of recension can be established
fairly easily.

All stemma involve an element of critical taste in their estimation of
the facts: this is particularly true when determining the source and
direction of the primary readings before scribal changes took place \citep{love_ranking_1984}: for instance, did line 3 first read ``I'm sure'' or ``I know'':
the later is both more emphatic and prosaic; to be ``sure'' means that the
possibility of doubt has been entertained and examined before a secure
knowledge is asserted; the claim to ``know'' requires no such critical
assessment of the facts. For that reason, ``I'm sure'' was preferred as
the primary reading as was ``which'' over ``that'' in line 5: these
decisions subsequently meant that the link between what became the MS 1
group and the source did not depend on any other line of recension, and
hence opened up the possibility of that group being the original
version. In stemmatic analysis, simplicity is the greatest virtue.

If we begin with the larger group of manuscripts descending from
MS 2, the reading ``I'me sure'' occurs in all the manuscripts on
the left side, as well as in Rosenbach MS 239/23 p.156, and all but one
of those on the left side read ``Dare'' in line 4, while the other
manuscript reads ``Will''. George Morley's text in Westminster Abbey MS 41
is variant in both cases, reading ``Others'' and ``Does'': the latter an
understandable palaeographical misreading of ``Dare''. What marks off the
``Will'' manuscript, Rosenbach MS 239/23 p.156, as belonging on the
right-hand side is that it shares the readings ``that'', ``skies'' and
``eyes'' with the ``I know'' manuscripts of MS 4. It is in other respects
highly variant and we can only assume that we have lost part of the
chain of evidence that generated such variants as ``grapple'' for
``wrestle'' in line 9. This was the copy-text chosen by the Oxford editors
despite what was so clearly an erroneous reading.

On the left-hand side, descending from MS 3, the equally variant
Morley-Westminster manuscript is linked by the shared reading ``And'' for
``Then'' in line 11 and to Huntington MS 198 part 1 by the reading ``Where''
for ``Whither'' in line 1. Once again the extent of the Morley-Westminster
variants suggests that we are missing some intermediary witnesses that
might account for how that text became as corrupt as it did: in this
particular instance the textual history of ``The Curse'' might provide
some of the missing clues.

Once the two oddities are accounted for, the remaining manuscripts
present few difficulties. On the right hand side, descending from
MS 4 through MS 7, all the remaining eleven manuscripts are
linked by reading ``I know'' for ``I'm sure'' in line 3, while expanding
``till'' to ``Vntill'' and omitting ``false'' from ``false Oathes'' in line 8.
Three of these manuscripts, via MS 11, substitute ``Loue'' for ``faith'' in
line 2: all read ``was'' in line 12, with one manuscript preserving ``was
thus'', and two reading ``was so'' via MS 14, two of these witnesses are
also in quatrains; five of the other manuscripts are grouped by ``is
thus'' in the final line via MS 12, while the other three retain ``was
thus'' but substitute ``Whither'' for ``Whether'' in line 1 via MS 13. Within
the ``is thus'' tradition, the remaining witnesses independently
substitute of ``Whither'' for ``Whether'' and ``venture'' for ``venter''.

The final six manuscripts on the left-hand side, descending from
MS 5, also form a coherent group. They are all linked by
reading ``pierce'' for ``blast'' and ``let them'' for ``may they''; two of the
manuscripts read ``venture'' for ``venter'' (this is why this reading needed
to be excluded from the initial analysis) and two read ``with'' for
``which'' in line 5. Of the remaining two manuscripts Yale University
Osborn MS b205 has two variants that suggest that again we may be
missing an intermediary, whilst Folger MS V.a.96 p.72 turns out to be
the best text of this tradition, and it includes capitalisation and
italics.

The most significant other line of descent is MS 9 that has the only two
manuscripts that attribute ``To his false Mistress'' to Herrick: these are
linked by the reading ``venture'': what is important about this group is
that the attribution to Herrick is demonstrably late; it descends from a
common source, and it only occurs at this point across the entire
stemmatic tradition. Further, there is no other attribution to Herrick
on the right-hand side that would suggest the attribution derives from
the top of the tree. On the other hand, the manuscripts marked with an
asterisk occur on both sides of the stemma, suggesting that ``The Curse''
was circulated with ``To his False Mistress'' from the outset.

When the stemma was first drawn the MS 1 group were placed hanging in
the middle, between the two more extended groups. What was evident,
however, was the unusual nature of two of the variants and by the fact
that neither of the manuscripts circulated directly with ``The Curse''
(although both have it elsewhere in their volumes). This was suggestive.
Elsewhere in this analysis, it has implicitly been suggested that the
old fashioned variant is probably earlier and the more current word
later \citep[177--82]{bland_guide_2010}: hence ``venter'' is earlier than ``venture'' and
``Whether'' (as used) earlier than ``Whither''. By the same logic, ``doth''
ought to be earlier than ``did'' and it is difficult to see why a later
scribe would deliberately introduce an old-fashioned form. Equally,
``lost vowes'' is an unusual substitution for ``false oathes'' as it has
neither a palaeographical cause nor a memorial one, and it cannot be
argued to be the ``more common word''. The most likely explanation for
such a variant is revision.

The issue is therefore one of direction: while Herrick might have
revised ``was thus'' to ``was so'', it is unlikely that he would have
revised ``did'' to ``doth''. Further, given the circulation of the
MS 2 tradition, it seems unusual that the MS 1
tradition did not circulate as widely. If, on the other hand, the
MS 1 readings represent the original version of the poem, then
what happened becomes explicable: Herrick received an early version of
``To his False Mistress'', tweaked the text, and circulated it with his
answer, ``The Curse''. When it came to printing \emph{Hesperides}, he
simply excluded ``To his False Mistress'' because he knew he was not the
author of that poem, and that ``The Curse'' could stand without it.

One final comment: it is possible to reconstruct the stemma in another
way. The issue is that if ``lost vowes'' were a sign of revision, then
``pierce'' for ``blast'' could be as well. This would involve swinging the
left-hand MS 3 side out to the right and up across the top line. To do
so, we would have to argue that the poem was revised not once, but
twice: first changing ``Then'' to ``And'' in line 11, and next changing ``may
they'' to ``let them'' in line 9 and ``blast'' to ``pierce'' in line 11. If
this were the case, then Folger MS V.a.96 p.72 would require no
modification other than a small adjustment to its punctuation as the
final revised copy-text. This is not an option that the Oxford editors
discussed. The hesitation about doing so is that we are clearly missing
something from the manuscript tradition to account for the witnesses we
have. Against this doubt, one has to acknowledge that ``pierce'' could be
a further revision by Herrick.

\section*{\centerheading{IV}}

It was observed at the outset of this article that the conjunction of
the textual histories of different authors in a single miscellany could
be informative about how and when the various texts were circulating. In
the case of MS 156/237 from the West Yorkshire Archive in Leeds, we have
a late version of Jonson's ``A Poem sent me by Sir William Burlase'' and
the earliest version of ``To his False Mistress''. The sequence in which
they are found has a great many poems by William Strode and Thomas
Carew, a couple by Aurelian Townshend, one by John Grange, a few poems
by Donne and a scattering of poems by Jonson written at different stages
of his career. The immediate sequence that follows after a copy of
Jonson's ``Eupheme'', written in 1633, involves three poems by Carew, one
by Townshend, three more by Carew, then ``To his False Mistress'' followed
by lines from Jonson's ``Satyricall Shrub'' and the song from Epicoene,
and then again Carew. This clearly points to someone in the Jonson-Carew
nexus as being responsible for the poem. In British Library Sloane MS
1446, ``To His False Mistress'' also follows on from three poems by Carew,
but of these it is only the last, ``To T. H. a Lady resembling my
Mistress'', that is the same. In other words, the source of the poem is
someone associated with Carew.

The Burlase poems do not occur in the Sloane manuscript, and they appear
some twelve pages on from ``To his False Mistress'' in the Leeds material.
In between are a medley of poems by Strode, William Herbert,
Townshend and Donne. That section clearly involves a different, looser
set of underlying papers than those in which ``To his False Mistress'' is
found. For the record, ``The Curse'' occurs much earlier, at the start of
the volume, among a sequence of poems by Henry King and Carew, mixed in
with a couple of poems by Donne. There is no Jonson material in that
part of the miscellany. The volume was prepared by a cousin of Sir John
Reresby of Thribergh Hall.

Although it is evident that ``To his False Mistress'' is not by Herrick,
that does not mean it ought to be excluded from a scholarly edition of
his verse. ``To his False Mistress'' and ``The Curse'' share a similar, if
more detached, relationship as the Burlase-Jonson exchange; the
difference is the tone of the answer and the fact that the we do not
know the author of the poem that Herrick attacked, although we do know
the circle from which it emerged. In that sense ``To his False Mistress''
joins the massed ranks of anonymous manuscript verse texts of the early
seventeenth century \citep{north_anonymous_2003}.

The story, however, does not quite end with the de-attribution of ``To
his False Mistress'' from the canon of Herrick's verse. Herrick was
directly or indirectly responsible for the existence of twenty of the
twenty-two surviving copies for the poem that only became well known
because he circulated it with his answer. Hence, we need to understand
``The Curse'' from the perspective of both its history as an answer, and
as a poem that circulated separately in its own right. As 48 out of 59
copies of ``The Curse'' exist without direct connection to ``To his False
Mistress'' it is evident that Herrick continued to revise that poem
independently of the poem with which he first circulated it. That is a
larger question, summarized but not fully answered by the Oxford editors
and, again, it is one that raises further and separate questions as well as
editorial issues.

For the present, it is enough to have demonstrated the ways in which
stemmatic analysis can be useful in resolving issues of origin and
authorship. All stemma are maps of social networks \citep{bland_stemmatics_2013}. Many of
the manuscripts discussed have known compilers, and those people were
often members of Oxford and Cambridge colleges, or the Inns of Court,
and had extended familial relationships. It would take too long here and
now to begin to describe and detail all those connections, but what such
an analysis does is invite that kind of historical reconstruction. The
more we build up maps of such networks, the more subtle our
understanding of scribal transmission will become. When we use terms
such as scribal networks, community, and conviviality, we need to
recognize that they are not abstractions and that the history of
specific contexts will leave traces in the textual and archival records
that most interest us. Part of the problem is unlocking the information
in a way that allows us to understand more acutely what the primary
evidence is, and what it represents. Further studies of other texts will
no doubt elucidate much that remains unstudied and unclear, especially
with regard to the anonymous and doubtfully attributed texts of the
period.

\begin{flushleft}
\bibliography{references/bland}  
\end{flushleft}
\end{paper}