\contributor{Dariusz Pachoki}
\contribution{Does the Editor Know Better? The Editorial Vicissitudes of the
20th Century Polish writers}
\shortcontributor{Dariusz Pachoki}
\shortcontribution{Does the Editor Know Better?}

\begin{paper}

\begin{abstract}
The main purpose of this article is to consider the question: Do authors
always decide on the final form of their works? To explore the answer, this essay
describes relations between authors and editors in chief. To this end, the essay provides some examples of the various forms this activity can take, and reflects on what we can learn about a text's creation after the author regards it as finished. The examples will be derived from the manuscripts of the Polish author Leopold Tyrmand. His literary output was studied by many scholars. So far, however, such analyses of his works were mainly of a literary nature --- and only examined rarely from a philological point of view. Nevertheless, it would not be fair to blame the scholars for this lacuna: surely many of them would consider such an inquiry if they did not lack the proper source materials. As a result, in the research of Tyrmand's literary output practically no examination of Tyrmand's creative process, no closer examination of the 
subsequent phases of the formation of text, and no compilation of its various
variants or attribution problems has existed until now.
\end{abstract}

\section*{} 
Any literary message coming from a writer's hand has always been
in need of an intermediary who would aid it in reaching the reader. In this function, editors in chief of magazines and publishers are often tempted to introduce
changes to the pieces they were about to publish. Some considered it
their obvious right --- and one that has been awarded to them by means of contractual agreements. However, there have also been those whose interventions ventured far beyond their
competence. The examples I am about to present in order to illustrate
the discussed issue come from the practice of professionals working at
the two largest publishing houses that gathered primarily Polish \textit{émigré}
writers. Because freedom of speech was non-existent under the communist regime
in Poland after WWII, many Polish writers decided to live
as emigrants. Communist Poland's censorship often exercied the power to
block every form of public art expression, such as books or films. Censorship
in Communist Poland (1945-1989) was performed by the Main Office of
Control of Press, Publications and Shows, an institution created in 1946 by
the pro-Soviet Provisional Government of National Unity with Stalin's
approval and backing. In 1950 a list of prohibited publications and
boycotted writers was created and subsequently modified many times by
the communist authorities. Articles and books with dozens of cuts might have
had a greater impact on the readers' minds. An alternative strategy for writers was to use the Aesopian language which was less understood by censors. The
censorship law was eliminated after the fall of communism in Poland
(1990).

Susan Greenberg \emph{\emph{has noticed}} that: 

\begin{quote}
Editing as a distinct
activity or role tends to be pushed into the margins of our attention.
This is partly because it takes place behind the scenes, more hidden
than the work of the bylined author and branded publisher, and partly
because it is everywhere and, therefore, nowhere. 
\begin{flushright}
\citep[8]{greenberg_when_2010} 
\end{flushright}
\end{quote}

\noindent Indeed, the  the
negotiations between editors and their writers are rarely made
public. Yet they warrant a closer look, since they can shed a new light on the degree
of influence  editors in chief had on their writer's publications. In addition,  the changes they introduced were not
always made with noble motives.

Jerzy Giedroyc,\footnote{Jerzy Władysław Giedroyc (1906--2000) was a Polish
  writer and political activist, and for many years the editor in chief of the
  Paris-based periodical \emph{Kultura} (sometimes referred to as
  ``Paris-based Culture''), which was published from 1947 to 2000 by the Literary
  Institute in Maisons-Laffitte (near Paris). It used to play a
  significant role in Polish literary life. Its authors published 
  polemics and original articles, including pieces by Nobel laureates
  Czesław Miłosz and Wisława Szymborska.} who was in charge of the
\emph{Kultura} (Maisons-Laffite) journal, and Mieczysław
Grydzewski\footnote{Mieczysław Grydzewski (1894---1970) was a Polish historian,
  publisher, literary critic, and the editor in chief of \emph{Wiadomości}
  (referred to as ``London-based The News''). The magazine was as a
  major \textit{émigré} journal from 1946 until 1981.} who was the editor in chief of
the London-based \emph{Wiadomości} both created islands of freedom for Polish
writers. And although these two varied substantially in their assessment
of the events in Poland, they were undoubtedly an important and
respected voice of ``free Poland''.\footnote{Both \emph{Wiadomości }and
  \emph{Kultura }were forbidden in Poland.} As a result, they were
quite successful in gathering interesting and vital writers, who had no other choice for a publication venue, if they wished to publish their pieces in Polish as
emigrants \citep{borejsza_totalitarian_2006}. Their
employers, Giedroyc and Grydzewski, were both competent, expressive and
powerful personalities. Both of them were also men of their own specific
vision and program which they were successfully executing. Editing the
pieces which had been accepted for publication was part of their job.\footnote{Quite
  often they combined various editorial roles and behaved as
  editor in chief, developmental editor, copy editor, proofreader or
  managing editor. For a snapshot of these editorial roles, see \citealt[11]{greenberg_poetics_2018}}
Writers who were collaborating with them quickly became familiar with their methods, when their texts were subjected to various
types of alterations. Frequently, these practices included the implementation of changes that 
often seriously altered the text's artistic side. Moreover, it happened
quite often that changes were introduced without the authors' awareness, or
despite their firm protests.

While it is not my aim to suggest that such interventions were common practice, it is nevertheless important to address them and put them in their temporal context. Before World War II, Jerzy Giedroyc was part of the editorial board of a popular magazine
\emph{Polityka}, together with
Mieczysław Pruszyński (secretary) and Adolf Bocheński (foreign
department). Apart from being in charge of
the magazine, Giedroyc was also responsible for Polish internal
affairs \citep[26]{habielski_dokad_2006}. We do not
have a lot of accurate data on Giedroyc's editorial practices from these
times, but from the sources that did survive it appears that Giedroyc ``held the magazine with an iron hand, that is he
interfered in the content of the articles adjusting them to his own
purposes'' \citep[27]{habielski_dokad_2006}.\footnote{A letter from Adolf Bocheński to Mieczysław Rettinger from
December 26th, 1936. Manuscript at Warsaw University Library, no.
1477.} Adolf Bocheński
complained in 1936 that the editor ``is being a terrible bully''. which
meant that he was mercilessly interfering in his press
reviews: ``as a result, he makes them utterly pointless'' wrote Bocheński.
``[H]e adds some sort of crops of his own mind and then he prints it all.
And some think I'm the one responsible for that. It's awful'' \citep[27]{habielski_dokad_2006}.

After the war, when Giedroyc took charge of \emph{Kultura},
his editorial habits did not change --- one could say that quite the
opposite was true. Instead, he felt a strong urge to modify the articles that were published in \emph{Kultura} (as well as the books that were
published in the \emph{Biblioteka Kultury} literary series) in discussion with their author. In addition, the editor's private beliefs and his
literary taste often gave way to lengthy discussions with his authors on the final form of the works that he would publish.\footnote{Is is easy to categorize this as a form of collaboration, but we have to keep in mind that writers basically had the
  choice between accepting the changes, or not being published anywhere at all.}

One writer who had the opportunity to experience Giedroyc's editorial style was Marek Hłasko. In January 1957, Giedroyc initiated a
meeting between them by inviting Hłasko to collaborate with him on his journal. In his letter to Andrzej Bobkowski Giedroyc wrote that Hłasko's ``legend is completely fake. He is just a boy, hurt and unhappy, who has
decided to rebel. Personally, I find him very nice. {[}\ldots{}{]} I have
every reason to promote him on an international scale and to think we
shall have \emph{poulain} possibly even larger in popularity than Miłosz
himself'' \citep[511--12]{zielinski_j_1996}.

The writer felt honoured and agreed to send Giedroyc his short stories.
The first two of them \emph{Cmentarze [The Graveyard]} and
\emph{Następny do raju [Next Stop, Paradise]} were published
within the \emph{Biblioteka }``\emph{Kultury}'' series in 1958. The
issue loomed large in Hłasko's mind, encouraging him to become bolder in his writing, and giving room to 
certain tensions between the author and his publisher. In a
letter dated April 30\textsuperscript{th} 1962 Giedroyc wrote to Hłasko:

\begin{quote}
 Zygmunt told me about your letter, that the next stories you are
currently preparing are going to be even more drastic and that you are
not going to agree to have anything removed from your pieces. I have a
feeling that there has been some ongoing misunderstanding between us. It
is not my aversion towards drastic words (which, to be true, I detest)
but the rules of censorship. I do not know about the situation in
Germany, but I am obliged to obey the French law. One can print various
things here, but certain language forms have to be kept. And there is
nothing I can do about it, no matter how much I would like to. Please
take this into account and decide for yourself if this shall be enough a
reason not to publish in \emph{Kultura} in the future.\footnote{All quotes from the correspondence between Hłasko and Giedroyc the Archive of Literary
  Institute in Maisons-Laffitte \citep{hlasko_kor_1962}. All  these quotations have been translated by the author, who would like to thank the management of
  the institute for making these resources accessible to him. The topic of the disagreement between Hłasko and Giedroyc was a series of short stories which were to be published in
  \emph{Kultur} and Biblioteka \emph{Kultury}. See, among others: \citealt{hlasko_w_1962,hlasko_drugie_1965,hlasko_wszyscy_1965}.}
  \begin{flushright}
  \citep{hlasko_kor_1962}
  \end{flushright}
\end{quote}

\noindent In a reply dated May 5\textsuperscript{th}
1962, Hłasko tried to find a subtle way to explain his reasoning to Giedroyc. Undoubtedly, Hłasko realized that his collaboration with \emph{Kultura} was
a chance for him to gain a wider readership:

\begin{quote}
You mentioned that you are a not a fan of drastic words. It might sound
strange to you, but I am personally not a fan of them either. But when
writing a short story about castaways bending over backwards to get a
slice of bread, I have the right to assume (not referring to even my own
experience) that such people do use such words, if a literary piece
shall be treated as a synthesis, obviously. I believe this has nothing
to do with pornography, if for the sake of depicting a character, when
trying to create a language which a man uses, I use several drastic
words. Personally, what does infuriate me are the dots {[}\ldots{}{]}.
Why am I allowed to use, in a story, such words as \emph{dick}, but not
\emph{cunt}? This has nothing to do with respect towards women, if we
are allowed to use such words as \emph{bitch},\emph{ son of a
bitch},\emph{ whore},\emph{ prostitute} and so on. There is no
explanation to that and we should not bow our heads to this {[}\ldots{}{]}.
  \begin{flushright}
  \citep{hlasko_kor_1962}
  \end{flushright}
\end{quote}


\noindent Hłasko was also quite surprised that the editor referred to the censorship law in France\footnote{After
  WWII the use of swear words in literature was deemed acceptable in France.} to strengthen
his position,
recalling that it was in France where they ``read and publish \emph{Gargantua
and Pantagruel}''; that it was in France where Faulkner became popular even though his piece
\emph{Light in August} contained words such as ``fuck'' and ``son of a
bitch'' and ``bitch'' in such quantities that --- as Hłasko argued --- ``even I
would be afraid'' (ibid.). Moreover, as he continued, it was in France where
\emph{Ulysses} and Henry Miller's works were first published, and where Marquis de Sade's books were
originally published in English.

In a letter dated on May 26\textsuperscript{th} 1962, Giedroyc replied
that his hands were tied and that, as he had previously stated, he had to
undertake censoring activities because of the existing law in France.
Referring to some particular swear words, he claimed that in the works of Rabelais
or Faulkner, ``they are acceptable, because a word like bitch and
the like have gained citizenship. Others did not'' \citep{hlasko_kor_1962}. As he continued: ``It
might seem funny that you can write: bitch, son of a bitch, ass, but you
cannot write dick or cunt'' (ibid.). The reasons for these he saw in the fact
that French language has centuries-old tradition of erotic terminology
and possesses numerous words and allusive phrases: ``Polish language in
this sphere did not step beyond Kochanowski and Potocki and is awfully
vulgar'' (ibid.). Giedroyc strongly suggested that Hłasko would replace the swear words with
their initial letters, followed by an ellipsis. To this, Hłasko responded that one cannot
replace a vulgar word with a different one, because ``it becomes a
different word then'' (ibid.). It might be crossed out, but in that case ``it changes
the story'' (ibid.). He also pointed out to Giedroyc that the choice of language that is used in a story is motivated by the
convention according to which the text is written; by the person who narrates the story (e.g. ``a scamp, a blighter, a pimp''), and by what the story is
about. In an undated letter (probably from October 1964), the writer mentioned a particular example from the short story
that was soon to be published in \emph{Kultura} --- a character of the
\emph{Drugie zabicie psa [Killing the Second Dog]} piece:

\begin{quote}
I fear that if we change ROBERT when we take his language away --- he
will turn out a different man, not so strong and full-blooded. Of
course, you may cross out the word \emph{cunt}, because it's not pretty
--- but I only do this not to offend our beloved Zosia {[}Hertz{]}; but
if we change Robert's sentence:\\
--- To hell with the son-of-a-bitch\ldots{}{}\\
into:\\
--- To hell with the s\ldots{}\\
Then it's just not the same.
\begin{flushright}
\citep{hlasko_kor_1962}
\end{flushright}
\end{quote}


\noindent It has to be recognized, though, that in this case the editor did accept
Marek Hłasko's points and the original version of this piece was
published. However, in other places the debatable words were replaced
with dots or completely crossed out \citep[81 and 141]{hlasko_drugie_1965}.
In the years that followed new books were published, but the
cooperation did not go smoothly. Certain disagreements over financial
issues were piled on top of the old tensions over language. The relations
between Hłasko and \emph{Kultura} became more loose \citep{tyrchan_marek_2007}, which may be confirmed by the writer's letter (from August
25\textsuperscript{th}, 1967) to the editor of the London-based
\emph{Wiadomości}, Juliusz Sakowski, in which he was complaining about
the arbitrary decisions made by the editor in chief of \emph{Kultura}:

\begin{quote}
Thank you, Sir, for your kind words on my reportages from the USA. I am
sorry to inform you that there will be no more of them. Jerzy Giedroyc
introduces his alterations arbitrarily: not knowing English, he changes
names --- for example he changed the name DOUGLAS (shortened DOUG) to
DOUD which makes no sense whatsoever and puts me in a position of an
idiot who seems not to have even the most rudimentary grasp of the
country in which I'm living and its language {[}\ldots{}{]}. But Giedroyc,
whose editorial gift I respect a lot, also has a gift to turn even the
most faithful people from him.
\begin{flushright}
\citep[114]{kielanowski_kulisy_1967}
\end{flushright}
\end{quote}

\noindent Hłasko was of course not the only person who had to negotiate the final
versions of their works with Giedroyc. Around this time (i.e.: in the mid-1960s) another Polish writer decided to emigrate: 
Leopold Tyrmand.\footnote{Leopold Tyrmand (1920---1985) was a legendary
  figure as a writer, journalist and jazz connoisseur in Poland. After
  WWII he returned to Warsaw, where he started to work as a
  journalist (1946). Tyrmand refused to collaborate with the new regime, and
  suffered from an unofficial ban on his publications as a result. The years from 1957
  until 1965 were a period of a deepening crisis for Tyrmand. His books were banned
  by the censorship, because the communists considered him uncomfortable
  and dangerous. He finally left Poland in 1965, and emigrated
  to the USA in 1966. In English, his works include: \emph{The Man With White Eyes},
  \emph{Notebooks of a Dilettante}, \emph{Seven Long Voyages, Diary
  1954.}} His first stop behind the ``iron curtain'' was the editing
office of \emph{Kultura} in Maissons-Laffitte itself. After having
finished a tour around Europe, the writer decided to move to the USA.
He published a piece in \emph{Kultura} where he burnt all the bridges
that linked him to Poland, decided to become an
English-language author instead. It has to be noted that he became quite a
successful English-language author at that. His columns were pubished in prestigious magazines such as
\emph{The New Yorker}. But the past was not easily forgotten,
and Tyrmand --- as it later turned out --- would reminisce about it
until the end. When he left Poland, he took the novel
\emph{Życie towarzyskie i uczuciowe [A Social and Emotional Life]} with him, the publication of which had been blocked by a censor. He hoped that he would not
encounter any such problems when he tried to publish the piece with Giedroyc. The
latter seemed interested, but he made his conditions. In a letter dated on May
5\textsuperscript{th} 1967, Giedroyc informed Tyrmand that he finds the book
very interesting and brilliant, but also pointed out that it would be
necessary to make some cuts in the original, because it contained too many
longeuers that impeded the reading, and made it too similar in style to his
most popular work, i.e. \emph{Zły [The Man With White Eyes]}, a
crime story. In a letter, Giedroyc proposed the following cuts:

\begin{quote}
p. 232 cut from the beginning of the second paragraph ``finally he met''
to the end of the chapter\\
p. 241 the whole of chapter nine\\
p. 252-256 shorten the story with Giga as much as possible\\
p. 308-325 possibly leave just the thing with Częstochowa. It doesn't
taste that well and it's too similar to \emph{Zły}.\\
380 in the second paragraph from ``Brutus was missing''\ldots{} until ``the
whole life to be erased''\\
{[}\ldots{}{]}\\
695-695 until the end of the chapter\\
717-719 the whole party at Stefania's\\
719-726 cut mercilessly.  
\begin{flushright}
\citep{tyrmand_kor_1966}
\end{flushright}
\end{quote}


\noindent Apart from the suggested cuts, Giedroyc allowed himself to make some
comments of more general nature. He drew the writer's attention to the
fact that in the book ``there was always someone whose back was sticking
with sweat''. He hoped that the author would accept the comments and take
necessary action. It did not take long for Tyrmand to reply, and in a
letter dated 17 May 1967 he emphasized that his
answer came as a result of deep reflection. He assured Giedroyc that the
approach he held towards his own writing was not of a devotional nature,
but he also pointed to the fact that the book had already been edited
four times by him and what he meant by ``edited'' was ``just cuts and
minor changes'' \citep{tyrmand_kor_1966}. After having looked into Giedroyc's suggestions he
decided that the proposed cuts would be too hard for him to accept. After a series of long debates, the book was finally published in 1967. The final version
was around 30\% shorter than the original typescript. Some changes were
never accepted by Tyrmand. However, he did not decide to have them edited once more.

A year later, the writer presented the editor of \emph{Kultura} a piece
which he announced with the following words: 

\begin{quote}
Would you like to have a
look at an article (from me, indeed!) entitled \emph{W szponach
metafizyki [Possessed by Metaphysics]} which is about me, Free
Europe, Zionists, \emph{Kultura}, athletics and of course about me? 
\begin{flushright}
\citep[2 July 1968]{tyrmand_kor_1966}\footnote{All quotes from the correspondence between Tyrmand and Giedroyc the Archive of Literary
  Institute in Maisons-Laffitte \citep{tyrmand_kor_1966}. All  these quotations have been translated by the author, who would like to thank the management of
  the institute for making these resources accessible to him.}
\end{flushright}
\end{quote}

\noindent It was supposed to be a
kind of reconciliation after the strife over getting the book ready for
publishing. Giedroyc reacted positively to the text. A few weeks later, he replied: 

\begin{quote}
I have received also an
article of yours, which I enjoyed and I consider it correct, but which
contains several ``adjectives'' or certain terms which are exaggerated
and which I would suggest to omit. Because surely you do have your own
copy, I will write down the suggested cuts on a separate sheet for you
and I'm looking forward to hearing if you accept them. 
\begin{flushright}
\citep[31 July 1968]{tyrmand_kor_1966}
\end{flushright}

\end{quote}



\noindent This time, Giedroyc's reservations did
not --- like in Hłasko's case --- concern the use of foul language, but rather the words he used to describe Polish communist
politicians. Giedroyc claimed that while he was sympathetic with his writers, he
could not take the risk of calling a politician an ``asslicker'', because the magazine's funds
depended on people whose views leaned towards the political ``Left'' (ibid.). The
magazine was Giedroyc's priority, and he would risk a lot to ensure its
survival. Moreover, as he confessed, his own political views were closer to
the ``Left'' as well. In response, Tyrmand started looking for compromises and agreed
to replace the explicit words with less expressive ones in an attempt to soften his tone:

\begin{quote}
p. 17 I agree to replace \emph{a chicken-brained leader\ldots{}} with the
word \emph{communist}\\
to the long cuts on p. 18 I \emph{cannot}, unfortunately, agree, but,
again we can replace \emph{Polish political leader} with \emph{communist
politician} (omitting \emph{asslicker}\ldots{}).
\begin{flushright}
\citep[8 August 1968]{tyrmand_kor_1966}
\end{flushright}
\end{quote}

\noindent Nevertheless, their reconciliation would not last long. A source of a new conflict was the publication --- in
\emph{The New York Times} --- where  Tyrmand renounced his
Polish citizenship in order to protest the Warsaw Pact's invasion of
Czechoslovakia. The editor in chief of \emph{Kultura} refused to reprint
Tyrmand's declaration which he considered pointless and futile, due to the
writer's residence in America. Further on, he
returned to the controversial article suggesting that Tyrmand's
concessions are insufficient:

\begin{quote}
I also regret that you do not agree to the changes in your article.
This is yet again an exaggerated pushing of a pedal. Where is this
inclination of yours coming from? I am afraid that you are very unaware
of the reality in Poland due to your stay in the USA and you
subconsciously enter the direction dominant in American press. My
tactics need to be based on the process of \emph{normalising} the
relations and that is a very uncompromising fight against the stupidity
of the regime and against the disorders among the society. But articles
such as yours are only an unnecessary act of adding fuel to the fire. We
are not moving out of Poland, our wish is rather to change it or to have
an influence. In this situation I prefer not to publish your article.
\begin{flushright}
\citep[10 September 1968]{tyrmand_kor_1966}

\end{flushright}
\end{quote}

\noindent To Tyrmand, this response indicated that Giedroyc had broken the negotiated agreement, and had changed
rules of the game while it was taking course. In his own response, Tyrmand reminded the editor that he had
been warned about the controversial nature of his article, and that he had alrady agreed to a half of the suggested
changes as a
result of the editorial negotiation process. He decided that Giedroyc's actions were
unfair and aimed against him personally. Soon after receiving Giedroyc's letter, he argued his case:
\vspace{2em}

\begin{quote}
I personally believe that my article is right, justified and useful at
the present moment. If the discrepancy between our assessments of this
fact is so huge, we should surely expect an unfavourable development of
further events {[}\ldots{}{]}. Under these conditions, I consider it a
natural consequence that I will ask you to release me from my duties of
\emph{Kultura}'s correspondent in America and remove my name from the
editorial note {[}\ldots{}{]}. We have simply not managed to \emph{adjust to
each other} as you used to say, we think in a different way about the
same, often the most vital issues, we have diffrent approaches to the
whole system of methods and ways of conduct. {[}\ldots{}{]} I am afraid
that in my case you have failed to make me believe that you really care
about me.  
\begin{flushright}
\citep[13 September 1968]{tyrmand_kor_1966}
\end{flushright}
\end{quote}

\noindent According to Tyrmand, the Polish system was one of Marxist totalitarism, and constituted an audacious evil that continued to spread with impunity throughout much of the world. Overpowering for many, Tyrmand  had no intention to give in to
what he considered to be a state that causes moral inertia. Undoubtedly, Tyrmand believed that
adopting an attitude such as Giedroyc's would only serve to elevate this evil in a way. Surely, this was not what Tyrmand had expected when he decided to flee to the
West. Presumably, he had instead expected to find himself embraced by a brother in arms, fighting for change in the Polish political system --- a change
that would need to be be instigated by condemning the satus quo, and educating the people. For Tyrmand, his
clash with reality must have constituted a great
disillusionment. The lesson he learned from his experience made him decide against  writing
pieces such as those that were dictated by the interests of the USA's New Left. Instead, Tyrmand made a conscious choice to continue expressing his
anti-communist views determinedly and unperturbedly --- which eventually led the
 \emph{The New Yorker} to break off their cooperation with the writer.
Had Tyrmand yielded and softened his viewpoint, this would have contradicted
the values that he held so dear as a citizen, and as an
intellectual. It would have meant denying the cornerstones of his worldview which
informed both his particular moral standpoint and his actions. Far from being opportunistic, these values had made sure that the intellectual side of his works was always greatly influenced by experiences from his past. Undoubtedly, these qualities
amplified his power of expression in his writings, and legitimized his skills as a 
publicist.

What really destroyed Tyrmand's final chances on having a good --- or at least
satisfactory --- relationship with Giedroyc, however, was his involvement in the
editing of \emph{Kultura}'s anthology. Indeed, it was Tyrmand who was supposed to be
responsible for preparing this issue for an audience in the USA. Not only was Giedroyc dissapointed
with the final result, he also believed that Tyrmand had blatantly taken advantage of his position to promote himself. The absolute proof Giedroyc used to support this claim was a huge
photograph of Tyrmand that ended up on the issue's fourth page. These misunderstandings
caused Tyrmand's turn away from \emph{Kultura}, and to focus instead on his relationship with  the editorial board of London-based
\emph{Wiadomości}. This magazine published one of his most
widely-read pieces: \emph{Dziennik 1954 [Diary 1954]}. But here too, 
the story was bound to repeat itself. Stefania Kossowska, who had much influence on
the shape of published pieces, introduced a number of changes (that she found to be \emph{necessary}) to parts of the work. These changes included Tyrmand's
sharply formulated treatment of living people, and excerpts that she
considered erotically too daring for publication. Alongside these changes however, there were other interventions that Tyrmand failed to understand, but accepted nonetheless. In a letter to \emph{Wiadomości}, Tyrmand wrote: 

\begin{quote}
but of course, we will not quarrel over such
details --- \emph{a wife that cheats on him} may be replaced with \emph{a
wife that disagrees with him}, but the whole case smells of grotesque.
\begin{flushright}
\citep[22 February 1975]{tyrmand_kor_1966}
\end{flushright}
\end{quote}

\noindent Reluctantly, Tyrmand complied with his
editor's conditions, but he was not ready to accept just any changes. In Tyrmand's mind, some of the proposed
interventions would distort the whole point of the piece, and destroy the force of
statements he expressed in subsequent parts of the diary. He summed up his
collaboration with \emph{Wiadomości} in one a letter he sent to Juliusz Sakowski --- another editor:

\begin{quote}
I have to be honest and admit to myself that the relationship between
the diary and \emph{Wiadomości} was not a happy one, it did not work
out. Ms. Kossowska, in whose good intentions towards me I do believe and
I have no reasons to detect any personal hostility in her attitude, did
not appreciate the diary somehow {[}\ldots{}{]}. She is not interested in the
personal content of the text, she does not enjoy the social side of it,
and the assessment of the contemporaries --- the most valuable element of
the diaries and the sole privilege of the writer --- meets her stubborn
resistance {[}\ldots{}{]}. As a result, collaboration with \emph{Wiadomości
}may be classified as a fiasco.
\begin{flushright}
(\citealt[4 September 1974]{tyrmand_kor_1966}; \\ see also \citealt[90]{tyrmand_listy_2014})
\end{flushright}
\end{quote}

\noindent Much seems to indicate that Mieczysław Grydzewski's colleagues
were acting in accordance with the style he had once excelled in himself.
Through the eyes of his collaborators, Grydzewski appears as an editor who ``detested longeuers and
mercilessly castrated articles, claiming that all cuts serve them so
well'' \citep[179]{goll_ma_1996}, whereas Stefania Kossowska
noted that 

\begin{quote}
Grydzewski would accept any text on four conditions: they
had to be well-written, ``independence-related'' (there was no space for
compromise here), ``socially-oriented'', although with a good pinch of
liberty, and finally, they could not be anti-religious, because the
editor, despite being indifferent to the subject of religion himself,
had the biggest respect for all confessions {[}\ldots{}{]}. There was
one more rule: Grydzewski would not publish pieces which would be
critical towards the pieces of other authors of the journal or the
authors themselves. 
\begin{flushright}
\citep[121]{tyrmand_listy_2014}
\end{flushright}
\end{quote}

\noindent Theory and practice do not always go hand in hand, however --- in fact
the two frequently contradicted one another. Aleksander Janta for example --- a writer and
an antiquarian --- had no doubts that his collaboration
with Grydzewski was ``the school of writing'' for him, as their discussions
often concerned issues related to language, style, and even layout. At the same time, Janta hinted towards the fact that Grydzewski had quite a strong personality, as was reflected in the alterations he
would make to Janta's text: 

\begin{quote}
you may actually consider those texts sent back to the
authors after having been altered and corrected by Grydz his
compositions, which a literature professor, concerned about accuracy and
quality, partly presents for consideration and partly forces upon his
pupils. The tensions and haggling, here over a word, elsewhere over a
sentence, somewhere over whole paragraphs were making it all more
satisfying, but also it would raise the temperature of a discussion to
the point of boiling sometimes.
\begin{flushright}
\citep[177]{janta_lustra_1982}
\end{flushright}
\end{quote}

\noindent And further on, Janta remarked that:

\begin{quote}
Negotiations and ordeals with the editor of \emph{Wiadomości} were
usually a consequence of his passionate editorial intervention in the
text which had been sent to him. He would manage them in whatever way he
wished to {[}\ldots{}{]}. Sometimes I would not recognize my own writing
after they had been subjected to Grydzewski's mangle and wringer. It was
then that the correspondence-haggling would usually begin over restoring
certain fragments or their original shape {[}\ldots{}{]}.
\begin{flushright}
\citep[199]{janta_lustra_1982}
\end{flushright}
\end{quote}

\noindent Negotiations with Grydzewski were extremely difficult, as he was usually
fully convinced of his own infallibility. The fact that Grydzewski had the
ambition --- not to put too fine a point on it --- to excert his control over all the literary pieces that were
sent to him was widely known and commented on. Reminiscing on his collaboration with
\emph{Wiadomości}, Gustaw
Herling-Grudziński  would point out that:

\begin{quote}
Grydzewski had some
unprecedented ambitions to interfere in every single text, although
usually this would be about minor details. Initially, he also tried to
interfere in my reviews and articles, but eventually I won the battle, I
managed to convince him. 
\begin{flushright}
\citep[164]{madyda_krytyka_1995}.
\end{flushright}
\end{quote}

\noindent Herling--Grudziński
belonged to a small group of writers who worked for \emph{Wiadomości}
but managed to restrain Grydzewski's controling nature. Nevertheles, the
majority of journalists and writers had to contend with the editor's
conviction of his own infallibility, and to try to maintain a decent relationship
with him --- while at the same time arguing for the right to keep their pieces
in their original form. 

One of those writers was Józef Mackiewicz, whose bitterness towards \emph{Wiadomości} (and especially its editor-in-chief) would systematically increase. It started
small, for example by neglecting to mention that Mackiewicz's  book about Katyń, which had been sent to Grydzewski with the hope of having it
mentioned in \emph{Kronika}, had been published in Spanish without his
knowledge. Another example would be more or less serious
arguments about spelling, like when Grydzewski was in favour of capitalising the word ``country'', which
Mickiewicz opposed. In his defense, Mickiewicz argued that:

\begin{quote}

there is
no such rule which orders an author to express their respect towards
things which they do not respect by writing it with a capital letter
{[}\ldots{}{]}. The word ``country'' is written with small letter and
not with a capital one.
\begin{flushright}
\citep[92]{lewandowski_gleboki_1995} 
\end{flushright}
\end{quote}

\noindent Later (18 July 1957), Mackiewicz would attach  a letter to one of his submissions that contained 
and a few sentences that would become the beginning of a more
serious conflict between the two:

\begin{quote}
I am sending you an article which disputes with you {[}\ldots{}{]}. Not
about this article, of course, but in general, I have to unfortunately
declare that I cannot consent any longer to the existing practice of
adding changes and alterations to my articles. It is not a threat I
would address towards you --- rather towards myself {[}\ldots{}{]}. It has
been 35 years now since I started writing; I am an author of a sizeable
pile of books; my hair is going grey. And my articles are being
corrected as if they were some compositions of junior high school pupil.
I am obviously not talking about spelling or some ghastly mistakes.
Everybody makes them. You also make numerous mistakes and it sometimes
gets to the point that I have to correct my own article which you had
written anew. I am not of the opinion that your style is poor. But I
want to write using my own style.
\begin{flushright}
\citep[130--31]{mackiewicz_listy_2010}
\end{flushright}
\end{quote}


\noindent As might be expected, Grydzewski, did not react
calmly to this message, and instead firmly denied that he
corrected the style of anybody's literary pieces \citep[93]{lewandowski_gleboki_1995}. Mackiewicz did not
wish to prolong. In his response, he ended his relationship with \emph{Wiadomości}:

\begin{quote}
Further on you wrote: ``you are completely wrong that I correct
anyone's style''. I am not only not wrong, but I claim that not only you
correct the style, but you tend to distort it completely. And not only
the style, but, by cutting the text in a particular way, you
\emph{often} distort my main thought, accents and the force of
argumentations of my opinions which you do not share. If I write ``the
gentlemen of the Home Army'' and you change it to ``the soldiers of Home
Army'' (as was the case) then this is obviously not about a language
mistake of any sort, but it is an act of reverting the accent and thus
the intention of an author. I could enlist a hundred of such examples,
but I do not have the time {[}\ldots{}{]}.

\newpage

Of course, I am not going to write articles for \emph{Wiadomości} any
longer. I am not going to discuss our qualifications, but I am not of
the opinion that you write better than I do. And having to sit and
correct allegedly my own article written by you {[}\ldots{}{]} I find
equally depressing and ridiculous.
\begin{flushright}
\citep[135--36]{mackiewicz_listy_2010}
\end{flushright}
\end{quote}

\noindent Wacław Lewandowski pointed to the fact that Grydzewski would not treat
declaration such as these, where his authors would threaten to end their collaboration, too seriously. Instead, he regarded the contents of these
letters rather as a list of writers' complaints about petty
details \citep[94--95]{lewandowski_gleboki_1995}. Indeed, Grydzewski was perfectly
aware of the fact that emigrated Polish authors had very limited
possibilities when it came to publishing their works in their native language.

On the other hand, it should be pointed out that some writers allowed the editor a
lot of freedom in the editing of their works. One of the people who noticed this
was Aleksander Madyda, who wrote about Zygmunt Haupt's cooperation with the
London-based \emph{Wiadomości}. From Haupt's letters to Grydzewski, we can deduce that the editor's interventions were primarily meant to reduce
the volume of the texts, and to introduce changes on the lexical level. Not
all of the editor's changes were fully approved of by the writer. Still, in general Haupt valued Grydzewski's contribution to the final shape of the
short stories highly: 

\begin{quote}
Sometimes when a thing I sent you is shortened
too much (and in any case, I have authorized you to do this), I feel
hurt, but there are also cases when (like with \emph{Meerschaum} for
example) you have done me a real favour.
\begin{flushright}
(a letter from December 3\textsuperscript{rd}, 1940; see \citealt[64]{madyda_krytyka_1995})
\end{flushright}
\end{quote}

\noindent In any case, the writer did not have much choice in the matter, since he believed that 
\emph{Wiadomości} was offering him his only chance to publish his works in Polish.

The two main activities which Grydzewski would undertake on the
texts which had been sent to him were a) shortening them, and b) interfering
with the language. Zygmunt Haupt was also confronted with this process. His texts were quite literary slashed into pieces by means of
editorial interventions. The cuts that Haupt's pieces
were subjected to often had destructive effects on their structure and
sometimes even distorted their meaning, or confused particular narrative threads. Nevertheless, as Aleksander Madyda points out:

\begin{quote}
cuts were just one of the categories of innovations introduced by
Grydzewski. Their advantage was that they did not appear in all texts
prepared for publication. However, there wasn't probably a single piece
that would not trigger the will to ``correct'' the language.
\begin{flushright}
\citep[220]{madyda_od_2015}
\end{flushright}
\end{quote}

\noindent Fortunately, typescripts with Grydzewski's handwritten
corrections have survived so that we can look into the nature and extent
of the changes he introduced. To illustrate the issue, I will go on to quote a couple of
instances. Haupt's prose piece entitled \emph{Dzień targowy
[Market day]}, for example, was subjected to an array of various interventions, including over a dozen grammar and stylistic corrections \citep[222]{madyda_od_2015}.

Aleksander Madyda, who meticulously analysed Grydzewski's corrections in
Haupt's typescript noticed: 

\begin{quote}
the examples presented here prove the
existence of a peculiar --- for an editor of a cultural and literary
magazine --- insensitivity to semantic and aesthetic values of a word
used not by himself, which insensitivity could be mischievously
classified as stylistic and linguistic hearing impairment.
\begin{flushright}
\citep[226]{madyda_od_2015}
\end{flushright}
\end{quote}

\noindent After having analysed the manuscripts of Haupt's works that are kept in the Stanford University
Archives and the archives of Polish Literary Institute, Madyda argued
that reading these materials may be shocking for a philologist, since,
as it turned out: 

\begin{quote}
the editor-in-chief of \emph{Wiadomości} treated
Haupt almost as an literary illiterate person, not only not in full
command of his native language, but also unable to construct a narrative
or built a tension within the plot and so on.
\begin{flushright}
\citep[228]{madyda_krytyka_1995}
\end{flushright}
\end{quote}

\noindent In the above, Mayda is specifically refering to the type of changes that were introduced in Haupt's short
story \emph{Powrót [Return]}, such as:

\begin{table}[H]
    \centering\small
    \caption{Examples of Grydzewski Mieczysław's changes to Zygmunt Haupt's \emph{Powrót}.}
    \begin{tabular}{p{4.75cm}|p{4.75cm}}
    \toprule
        która jest {[}\emph{which is}{]} & jaką stanowi {[}\emph{that makes}{]} \\
    \midrule
        w latach osiemdziesiątych, kiedy po światowym kryzysie ekonomicznym {[}\emph{in the `80s, when after the global economic crisis}{]} & pod koniec ub. w., kiedy po światowym kryzysie gospodarczym {[}\emph{at the end of the previous century, when after the global economic crisis}{]} \\
    \midrule
        byliśmy spóźnieni {[}\emph{we were late}{]} & spóźniliśmy się {[}\emph{we got there late}{]} \\
    \midrule
        nonszalancki {[}\emph{nonchalant}{]} & niedbali {[}\emph{neglectful}{]} \\
    \bottomrule
    \end{tabular}
\end{table}

\noindent Looking into the subsequent materials may lead the reader to agree that
writers who were submitting their works to be published
in \emph{Wiadomości} were right to protest against Grydzewski's
interferences, and that their complaints seem very well grounded. The nature
and extent of changes Grydzewski introduced has such a large impact on the writing that --- as Madyda
rightly pointed out --- any research into the style of a particular author's prose as published in
\emph{Wiadomości} will be unreliable, because ``most probably the style
in question is primarily a reflection of linguistic preferences of the
authoritarian publisher'' \citep[231]{madyda_od_2015}.

In addition, Grydzewski's co-workers (such as Michał Chmielowiec, Stefania Kossowska
or Juliusz Sakowski) also had significant influence on the final shape of published texts. For example, it was Sakowski who got involved in a dispute with Marek
Hłasko that almost ended up in court. The source of their disagreement was was --- just
like in Giedroyc's case --- some of the swear words Hłasko wanted to use, and a
single sentence that Sakowski considered a blasphemy. In a letter
dated 15 May 1967, Hłasko defended his case:

\begin{quote}
Indeed, the picture of Mother of God dedicated to Grzegorz is a case of
blasphemy. But the whole book is a search for God in every moment
{[}\ldots{}{]}. Dear Sir, this single blasphemy cannot in any case kill
the spirit of this novel which is a story about love: and every story
about true love is a story about God.
\begin{flushright}
\citep[111]{kielanowski_kulisy_1967}
\end{flushright}
\end{quote}

\noindent The writer also referred to the allegation of swearword abuse: 

\begin{quote}
When it
comes to the word \emph{pierdolić} {[}\emph{fuck}{]}, you
are writing that women do not use such words. Some do, others don't. My
women did use it. This is also what Weronika says when she is drunk. A
word or a sentence taken out of its context means nothing. One needs to
remember the scene and the time and the circumstances in which it is
taking place.
\begin{flushright}
\citep[112]{kielanowski_kulisy_1967}
\end{flushright}
\end{quote}

\noindent In the end, Hłasko categorically refused to change the dedication:

\begin{quote}
You are talking about me being sued by the Catholics? I will tell you
one thing --- praise the Lord, we're going to sell more books. Are
Catholics going to turn from me? Can someone show me a true Polish
Catholic? Do you really believe that Poles are a catholic nation? And
what sort of Catholicism is it? It is Catholicism on the level of a
Sunday school: it is not Catholicism in the sense of a world view.
{[}\ldots{}{]}

Everything that I have written above has nothing to do with the fact
that I am not going to change the dedication and I'd rather not publish
the book.
\begin{flushright}
\citep[113]{kielanowski_kulisy_1967}
\end{flushright}
\end{quote}

\noindent It would appear, however that the publisher did not take the
author's words seriously: instead Sakowski unilaterally removed part of the text during the last proofreading, despite Hłasko's
strong protests. The excerpt that was removed was linked to a scene where
Grzegorz, the protagonist, unpacks a suitcase and takes out a picture of
Our Lady on which Uncle Józef, the central character of the story, had
written: ``To my beloved Grzegorz, in commemoration of the days and
nights we have spent together in the Upper Galilee --- Mother of God''.

Upon seeing the publication, Marek Hłasko protested strongly, and threatened to take the case to court. In addition, he announced that he would publicly
declare that he has nothing to do with the altered text. And he also
demanded to have his name removed from the cover of the book, and that the editors would
change the book's title as it could not be \emph{Sowa, córka piekarza
}{[}\emph{Owl, the Baker's Daughter}{]}\emph{ }nor \emph{Pójdź z nim przez
dolinę }{[}\emph{Go with him through the valley}{]}. Believing that
these actions might not have been enough, Hłasko employed an American
lawyer, who warned Sakowski in the middle of October 1967  about the
consequences of publishing the book in a form which the writer did not
accept. After month-long negotiations they finally came to an agreement, where  the
words ``Mother of God'' were replaced with the word ``Mary''. Obviously, the new
version was more ambiguous, but it was also definitely less expressive.
Satisfied with the change he succeeded in had pushing through, Sakowski failed
to notice another part of the text that could also be considered blasphemous:

\begin{quote}
\ldots{}and then he thought that Uncle Józef told him once about
Christ, who was walking with the cross and did not stop at the stations
of his passion and when someone asked him why he did that, he explained
that he was a hasty Christ; when I thought about it I started laughing
and it sounded like weeping.
\end{quote}

\noindent A year later, Marek Hłasko died. This meant that he never had an
opportunity to publish the original version of the piece without the
changes that had been introduced against his
will. Because of this, all the editions that available on the publishing
market today contain the text in the shape the author opposed.

In contrast, writers such as Haupt, Tyrmand or Mackiewicz, who had
an opportunities to republish their works after their original publication in \emph{Wiadomości},
did so without hesitation. Nevertheless, the editors-in-chief
of the Paris-based \emph{Kultura }and London-based \emph{Wiadomości} had succeeded in gathering some of the most
interesting figures of Polish literature and culture during the years of
their activity. As Giedroyc
argued: an editor is like a film director, and a director rarely makes
a good actor. Maybe this was the reason why the editors could not entirely
understand the intensity with which the writers defended their style,
particular phrases or even single words. Their editorial work quite
often resulted in writers' frustration and resentment and also led to
heated arguments, which brought their cooperation to an end. It is
difficult to make judgments about these decisions, but the scale of the
phenomenon is large enough to provide a subject for analysis --- from those
connected with text criticism, through stylometric studies, to analyzing the socio-political aspects of certain editorial decisions. It could also be worth considering certain editorial work, and an attempt to return some of the works to their original shape --- i.e. the state it was in before all
the editorial interventions writers disagreed with were introduced. But that falls beyond the scope of this essay, and would be the subject of a completely separate enquiry.

\vfill

\begin{flushleft}
\bibliography{references/pachoki}  
\end{flushleft}
\end{paper}