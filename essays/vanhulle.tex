\contributor{Dirk Van Hulle}
\contribution{Creative Concurrence. Gearing Genetic Criticism for the Sociology of Writing}
\shortcontributor{Dirk Van Hulle}
\shortcontribution{Creative Concurrence}

\defcitealias{beckett_bdmp_2011}{BDMP~1} 
\defcitealias{beckett_bdmp_2014}{BDMP~2}
\defcitealias{beckett_bdmp_2015}{BDMP~3}
\defcitealias{beckett_bdmp_2018}{BDMP~7}
\defcitealias{beckett_bdmp_2020}{BDMP~9}


\begin{paper}

\begin{abstract}
This essay is an attempt to come to terms with a
phenomenon that characterizes many authors' oeuvres: the concurrent
composition of several works. My suggestion is to refer to this
phenomenon as \emph{creative concurrence}, drawing on experiences from
two related disciplines, bibliography and the history of the book ---
notably D. F. McKenzie's ``sociology of texts'' and the concept of
``concurrent production'' --- in order to cross-pollinate genetic
criticism.
\end{abstract}

\section*{}

\textsc{Genetic criticism has so far worked with a model} --- most explicitly put
forward in Pierre-Marc de Biasi's functional typology of genetic
documentation \citep{biasi_what_1996} --- in which the so-called \emph{bon à
tirer} or the pass-for-press moment plays a crucial role. This is the
moment the genesis \emph{generally} moves from a private to a public
enterprise.\footnote{The ``Typology of Genetic Documentation'' is
  presented as ``a \emph{general} table of the stages, phases, and
  operational functions that enable the classification of different
  types of manuscripts according to their location and status in the
  process of a work's production'' \citep[32; emphasis added]{biasi_what_1996}.}
Genetic criticism has traditionally focused on the so-called private
component of the writing process, not so much on the public part,
because that public genesis, according to Pierre-Marc de Biasi, follows
a different logic:

\begin{quote}
The mutations of the \emph{avant-texte} took place in a private writing
domain where everything was possible at any time, including the complete
extinction of the writing, even if the work seemed to be moving in the
direction of a publishable text. By contrast, postpublication
modifications are made in a public sphere where the book's reality
cannot be ignored: they successively affect equally definitive textual
versions of the ``same'' work, each of which can claim the status of a
completely separate text each time, without it's {[}sic{]} being in general
possible to recognize the logic of a process comparable to the
pre-textual one between them. 
\begin{flushright}
\citep[40]{biasi_what_1996}
\end{flushright}
\end{quote}

\noindent There is a historical explanation for this focus on the somewhat too
black-and-white dichotomy between the text and the so-called
\emph{avant-texte}, the writing process preceding the \emph{bon à tirer}
moment when the author decides the text is ready to be printed. In the
first few decades of its existence, genetic criticism understandably
emphasized its difference from related fields of study in order to
establish itself as a discipline in its own right. Although this
tendency is understandable, the downside is that there have been
relatively few efforts (from the perspective of genetic criticism) to
see how these related fields of study dealt with modern manuscripts.
Thus, for instance, in \emph{The Study of Modern Manuscripts} (1993),
Donald H. Reiman distinguishes between private, confidential, and public
documents, ``based on the functions their authors intended them to play
in communicating with specific intended readers'' \citep[xi]{reiman_study_1993}.
According to Reiman, ``the primary factor that categorizes a manuscript
as \emph{private, confidential}, or \emph{public}'' is ``the nature and
extent of the writer's intended audience'' \citep[65; emphasis in original]{reiman_study_1993}: ``A
manuscript is \emph{private} if its author intended it to be read only
by one person or a specific small group of people whose identity he knew
in advance; \emph{confidential} if it was intended for a predefined but
larger audience who may --- or may not --- be personally known to or
interested in the author; and \emph{public} only if it was written to be
published or circulated for perusal by a widespread, unspecified
audience'' \citep[65; emphasis in original]{reiman_study_1993}. Possible objections to Reiman's
categorization are its exclusive reliance on authorial intention and the
compartmentalization of a creative space that is fluid and
uncompartmentalized by nature. But at least Reiman's suggestion adds
nuance to a phenomenon that is treated rather implicitly in de Biasi's
typology.

In its generality, de Biasi's more dichotomous understanding of a
movement from private to public in \emph{creative} composition is
similar to an equally general statement in a related field of study,
Bibliography, about another form of composition. In Bibliography,
composition usually refers to the work of the compositor at the
printer's. With reference to this type of composition in the early
modern period, R. B. McKerrow had made the general claim that ``for a
printing house to be carried on economically there must be a definite
correspondence between rate of composition and the output of the machine
room'' (qtd. in \citealt[26]{mckenzie_making_2002}, suggesting that the printing house
functioned as a well-oiled machine, focusing on one book at a time.
Donald F. McKenzie, however, explained how ``McKerrow's valid general
statement is transformed into an invalid particular statement'', by
looking more closely at the day-to-day business of specific printing
houses. For instance, in his bibliographical study of the first sixteen
years of Cambridge University Press, McKenzie concluded that a small
printing house, ``never using more than two presses, {[}\ldots{}{]} habitually printed several books concurrently'' (26). McKenzie was able
to demonstrate that this principle of concurrent production ``applied to
virtually all book manufacture'' \citep[McDonald and Suarez in][14]{mckenzie_making_2002}.

Genetic criticism can benefit from McKenzie's insights in bibliography
and book history. On the one hand, his notion of the sociology of texts
can be an incentive to pay more attention (in genetic criticism) to the
sociology of \emph{writing}. And on the other hand, his notion of
``concurrent production'' can be a source of inspiration to examine a
phenomenon that is understudied in genetic criticism: writers working on
several projects simultaneously, which I propose to call ``creative
concurrence'' and which cannot be studied without its diachronic
counterpart: ``creative recurrence''.

These two principles of the ``sociology of writing'' and ``creative
concurrence and recurrence'' are tied to each other in the sense that
they both tend to challenge traditional forms of compartmentalization:
the ``sociology of writing'' sees writing in terms of a collaborative
ecology and therefore challenges any tendency to compartmentalize
different agents of textual change, such as treating for instance the
author as a self-contained unit, in isolation from editors, translators,
typists, publishers and other agents of textual change; the notion of
``creative concurrence'' sees writing in terms of a textual ecology and
therefore challenges any tendency to study every single work by an
author as a self-contained textual unit, unrelated to their other works
or to the oeuvre as a whole.

\section{The Sociology of Writing}

McKenzie's sociology of texts was mainly conceptualized on the basis of
his study of early modern texts and printing houses, which explains the
relative lack of engagement in genetic criticism with his work. As
discussed above, according to de Biasi's rationale, the ``different
logic'' of the postpublication phase sets it apart from the
\emph{avant-texte} \citep[40--41]{biasi_what_1996}. In the meantime, however, these
postpublication modifications have been recognized as part of a work's
``epigenesis'', the continuation of the genesis after publication \citep{van_hulle_modern_2014}. But there is another aspect in de Biasi's dichotomy that
might need to be nuanced. Not only does the genesis often continue after
publication, the public aspect of texts sometimes also precedes
publication, especially in the case of pre-book publications. As a
consequence, the creative process is often more ``public'' than generally
assumed and therefore calls for some remodelling to accommodate the
sociology of writing.

A good example in twentieth-century literature is the collaboration
between Gordon Lish and Raymond Carver, the author of the short-story
collection \emph{What We Talk About When We Talk About Love?}
Apparently, the minimalistic writing style Carver is famous for is
actually to a large extent the work of his editor, Gordon Lish. As the
fiction editor of the magazine \emph{Esquire} from 1969 to 1977, Lish
edited Carver's story ``Neighbors'' for publication in \emph{Esquire} in
1971. The radical editing is considered to have had such an impact that
it largely created the Carver style. As Tim Groenland notes in The Art of Editing, ``Lish’s mediating and gatekeeping activities'' illustrate how ``editors function as key players in the production of fiction'' \citep[xi]{groeland_art_2019}. Carol Polsgrove describes how
drastic the editorial intervention was: ``On several pages of the
twelve-page manuscript, fewer than half of Carver's words were left
standing. Close to half were cut on several other pages''\citep[241]{polsgrove_it_1995}. Lish's pruning was relentless, as his papers at the Lilly
Library at Indiana University in Bloomington show. In some cases, one
could even regard the result as a rewriting, and it does not come as a
surprise, therefore, that the \citeyear{carver_collected_2009} edition of the \emph{Collected
Stories} chose to publish some of the stories in both
the edited and the unedited versions.

Playwrights often count on directors to help them give shape to the
script, as the case of Alan Bennett and director Nicholas Hytner
illustrates. For instance, in the spring of 1991, Bennett sent the first
draft of his play \emph{The Madness of George III} to Hytner to ask his
feedback, as Hytner recalls in his memoirs: ``It was based on fact. In
1788, George III apparently went mad. {[}\ldots{}{]} But in the spring of
1789, in the nick of time, the king seemed to come to his senses, so the
Regency bill failed'' \citep[117]{hytner_balancing_2018}. Historical facts were
important to Bennett, who had studied History at Oxford. On the other
hand, Hytner's major concern was not historical truth but dramatic
suspense: ``I've never stopped pressing the claims of narrative tension,
urging Alan to channel more of what he wants to say into dramatic
action'' \citep[120]{hytner_balancing_2018}. Although, according to Hytner, Bennett ``cooperated only
up to a point'' \citep[120]{hytner_balancing_2018}, the Bennett papers at the Bodleian
Library in Oxford \citep[588--89]{bennet_play_1989} suggest that he did revise his early
versions quite drastically (especially the ending) due to comments made
by both the director and the lead actors.

As McKenzie noted, ``a book is never simply a remarkable \emph{object}.
Like every other technology it is invariably the product of human agency
in complex and highly volatile contexts which a responsible scholarship
must seek to recover if we are to understand better the creation and
communication of meaning as the defining characteristic of human
societies'' \citep[4]{mckenzie_bibliography_1999}. This insight has had quite an impact on textual
scholarship and book history, and --- with some delay --- also on literary
studies more broadly. For instance, in Joyce studies. Whereas in 1977,
Michael Groden suggested that the writing of \emph{Ulysses} consisted of
three stages (the development of an interior monologue technique; the
abandonment of experimentation with the monologue for ``a series parody
styles''; and the creation of several new styles, followed by the
revision of earlier episodes \citep[4]{groden_ulysses_1977}, a more recent textual
approach replaces this three-phase process by a five-phase scheme,
importantly devoting a separate stage to the serialization: ``(i)
conception, (ii) drafting, (iii) serialization, (iv) continued
post-serial drafting, and (v) the formation of plans for publication in
volume form'' \citep[74]{hutton_serial_2019}. This new attention to serialization as
part of the writing process marks the increased attention to the
sociology of texts, to a large extent thanks to McKenzie's
\emph{Bibliography and the Sociology of Texts} (\citeyear{mckenzie_bibliography_1999}) and Jerome
McGann's work on the ``socialization of the text'' in \emph{The Textual
Condition} (\citeyear{mcgann_textual_1992}).

Joyce's awareness of the ``human agency'' involved in his literary
enterprises seems to be thematized in \emph{Finnegans Wake}, where this
agency is referred to as ``the continually more and less
intermisunderstanding minds of the anticollaborators'' \citep[118.25]{joyce_finnegans_1939}. These textual agents range from typists, secretaries and
amanuenses to proofreaders, patrons, journal editors, printers,
champions, critics and even ``enemies''. Especially this latter,
inimical category seems to have been a type of agency that sparked
Joyce's creativity. In Joyce's case, the self-styled ``enemy'' par
excellence was Wyndham Lewis, who asked Joyce for a contribution to his
new journal \emph{The Enemy}. Joyce obliged and submitted the piece, but
Lewis then decided not to publish it. Moreover, in its stead he
published his own critical ``Analysis of the Mind of James Joyce''.
According to Lewis, Joyce had basically nothing to say; he was merely
interested in ``ways of doing things'' rather than in ``things to be
done''; and he did not have any ``special point of view, or none worth
mentioning'' \citep[88]{lewis_time_1993}. Joyce's point of view in this
matter seems to have been that ``only the second-rate imagine that they
have messages to deliver'' \citep{banville_waiting_1989}. He kept calm and carried on,
cleverly using Lewis's poison to inoculate his own ``work in progress''
with it.

Joyce may be famous for his disrespect of publishers' and printers'
general insistence on not making any changes at proof stage. Indeed, he
often kept adding text wherever he found a white space in the margins.
The ending of \emph{Ulysses}, for instance, kept expanding to such an
extent that it necessitated no less than four sets of placard proofs \citep[179]{sullivan_work_2013}. But on other occasions, he did show great respect
for printers, as the following textual tale illustrates. Partially in
reply to Lewis, he wrote a fable, called ``The Ondt and the
Gracehoper'', loosely based on Aesop's fable of the ant and the
grasshopper, the Lewisian ant who saves up for winter and the
spendthrift grasshopper who is associated with the Penman Joyce. The
fable appeared several times in pre-book publications \citep[143ff.]{van_hulle_james_2016} and expanded with every new version. It ends with a poem, which
was originally fairly short, consisting of only 12 lines. When it was
first published in the magazine \emph{transition} (March 1928), it had
18 lines. The fable subsequently came out as part of \emph{Tales Told of
Shem and Shaun} in a deluxe edition by The Black Sun press (1929), run
by Harry and Caresse Crosby. At proof stage, it had ten extra lines, 28
in total, and it was to expand even further, resulting in 34 lines when
the book appeared. This accretion was partially due to the printer. In
her memoir, Caresse Crosby mentions an ``unexpected incident'' relating
to the master printer Roger Lescaret during the production of
\emph{Tales Told of Shem and Shaun}:

\begin{quote}
The pages were on the press and Lescaret in consternation pedaled over
to the rue de Lilly to show me, to my horror, that on the final ``forme'',
due to a slight error in his calculations, only two lines would fall
\emph{en plaine} {[}sic{]} \emph{page --} this from the typographer's
point of view was a heinous offense to good taste. What could be done at
this late date! Nothing, the other \emph{formes} had all been printed
and the type distributed (we only had enough type for four pages at a
time). Then Lescaret asked me if I wouldn't beg Mr. Joyce to add another
eight lines to help us out. I laughed scornfully at the little man, what
a ludicrous idea, when a great writer has composed each line of his
prose as carefully as a sonnet you don't ask him to inflate a
masterpiece to help out the printer! 
\begin{flushright}
(\citealt[187]{crosby_passionate_1953}; see also \citealt[614--15]{ellmann_james_1983})
\end{flushright}
\end{quote}

\newpage
\noindent But that was exactly what the ``great writer'' did. Behind Crosby's
back, Lescaret secretly went to Joyce, asked him if he could write a few
extra lines, and Joyce indeed produced the requested lines. Evidently,
if \emph{Finnegans Wake} can be considered a masterpiece, that is not
only thanks to ``anticollaborators'' such as Roger Lescaret, but they
did play an active role in the creative development of the work in
progress.

Later in the writing process of \emph{Finnegans Wake}, Joyce even
outsourced his reading. Thus, he asked Samuel Beckett to read a book
that was sent to him by Heinrich Zimmer Jr., called \emph{Maya:}
\emph{Der indische Mythos} (1938). Beckett made notes for Joyce on three
pages, preserved at the University at Buffalo \citep[143ff.]{van_hulle_beckett_1999}.
Joyce also involved his friends in collective notetaking, as Stuart
Gilbert's Paris journal entry on a reading session in preparation of
\emph{Haveth Childers Everywhere} shows: ``Five volumes of the
Encyclopaedia Britannica on his sofa. He has made a list of 30 towns,
New York, Vienna, Budapest, and Mrs. Fleischman has read out the
articles on some of these'' \citep[20--21]{gilbert_reflections_1993}. Gilbert was irked by
the way Joyce, ``curled on his sofa'', kept on ``pondering puns'' while
exploiting his friends, such as Padraic Colum, Helen Fleischman, Paul
Léon, Lucia Joyce, and Gilbert himself.\footnote{``I `finish' Vienna and
  read Christiania and Bucharest. Whenever I come to a name (of a
  street, suburb, park, etc.) I pause. Joyce thinks. If he can Anglicize
  the word, i.e. make a pun on it, Mrs. F. records the name of its
  deformation in the notebook'' (\citealt[20--21]{gilbert_reflections_1993}; qtd. in \citealt[163]{van_hulle_james_2016})} Gilbert gives the example of the word ``Slotspark'',
which became ``Slutsgartern'' in \emph{Finnegans Wake} (Joyce 1939,
532.22--23): ``Thus `Slotspark' (I think) at Christiania becomes Sluts'
park. He collects all queer names in this way and will soon have
notebooks full of them'' \citep[20--21]{gilbert_reflections_1993}. In his journal, Gilbert
ventilated his resentment: ``With foreign words it's too easy. The
provincial Dubliner. Foreign equals funny'' \citep[21]{gilbert_reflections_1993}, thus giving voice --
albeit in the privacy of his personal journal --- to the social tensions
involved in Joyce's literary corporation, in which he assumed the
entrepreneurial role of CEO. This image is at a far remove from the
private enterprise of the writing process as it is presented in de
Biasi's model. Evidently, this model is a conscious generalisation, as
de Biasi readily admits, but now that genetic criticism exists more than
fifty years, the time seems propitious to refine the model and find a
place for the sociology of writing in genetic criticism.

\section{Creative Concurrence}

A second way in which I would like to suggest a remodelling is related
to the phenomena of creative concurrence and recurrence. Traditionally,
genetic criticism tends to approach an author's oeuvre work by work. It
would be useful to also consider a work's genesis within the development
of the author's oeuvre in its entirety, and consider all the geneses of
its components, including vestigial notes, drafts and unpublished works,
constituting the \emph{sous-œuvre} --- not so much in Thomas C.
Connolly's general sense of the marginalized parts of a work that are
traditionally eclipsed \citep{connolly_paul_2018}, but in the sense of the entire
oeuvre's genetic dossier.

The focus on separate works is understandable, given the complexity of
most literary works' geneses. The reality for many writers, however, is
that they divide their time between multiple book projects on a daily
basis. For instance, as Vincent Neyt has shown,\footnote{Vincent Neyt is currently working on a PhD on the genesis of King’s \textit{IT} at the University of Antwerp.} the genesis of Stephen
King's novel \emph{IT} cannot be studied in isolation. In the period of
seven years between 1980, when King started the first draft of
\emph{IT}, and 1986, the year the novel was published, King published no
less than thirteen books, some of them under the pseudonym Richard
Bachman.\footnote{One work of non-fiction, called \emph{Danse Macabre}
  (1981); one collection of four novellas (\emph{Different Seasons}
  (1982)); a collection of short stories (\emph{Skeleton Crew} (1985))
  and ten novels, some of them under a pseudonym: \emph{Firestarter}
  (1980), \emph{Cujo} (1981), \emph{Roadwork} (1981 --- as Richard
  Bachman), \emph{The Gunslinger} (1982), \emph{The Running Man} (1982
  --- as Richard Bachman), \emph{Christine} (1983), \emph{Pet Sematary}
  (1985), \emph{The Talisman} (1984 --- with Peter Straub),
  \emph{Thinner} (1984 --- as Richard Bachman), \emph{The Eyes of the
  Dragon} (1984).} This was possible because King regards writing as a
craft, which requires a certain discipline. Every day, he tries to write
half a dozen of pages, and in the afternoon or evenings, he tends to
revise another work, often to meet a scheduled submission deadline.
These deadlines, imposed by the publisher, are a significant social
element in the writing process, either as an incentive or as a form of
pressure that can stifle creativity. As a consequence, there are
constantly several book projects underway concurrently. Analogous to
McKenzie's principle of concurrent production, I therefore suggest we
introduce a principle of concurrent writing or \emph{creative
concurrence} in genetic criticism, which implies reconstructing the
everyday reality of how a work, in all its draft versions, interacted
with the other works that populated the author's writing desk at any
given moment.

Creative concurrence does not necessarily have a sociological dimension.
A writer can simply be working on several projects at the same time. But
often a social aspect does play a role. To illustrate just how
interwoven the geneses of an author's individual works can be, it is
useful to have a brief look at Samuel Beckett's writing desk in the late
1970s. He was already working on a longer piece of prose, called
\emph{Company}, when he received a commission. In the fall of 1977, the
English actor David Warrilow requested Beckett to write ``a monologue on
death'', to which Beckett replied on 1
October with the line ``My birth was my death'' \citep[471 note 1]{beckett_letters_2016}. The next
day, under the preliminary title ``Gone'', he immediately started
developing the theme: as soon one is born, one starts dying. Thus began
what was to become \emph{A Piece of Monologue}, translated into French
as \emph{Solo} \citep[205]{pilling_samuel_2016}. On a piece of paper, Beckett wrote
some loose jottings on the theme of death, such as ``Not much left to
die'' and ``Dead \& gone / Dying \& going'' \citep[UoR MS 2460, m28, 01v; qtd. in][75]{van_hulle_becketts_2019}. Some of the more concrete
stage settings clearly presage \emph{A Piece of Monologue}, such as this
idea for the ``End: fade out general light. Hold on globe. Fade out
globe.'' And this one for the opening: ``Fade up. 10''. `Birth.' 10''.''
\citep[UoR MS 2460, m28, 01v; qtd. in][75]{van_hulle_becketts_2019} The note ``Birth was his death.
Etc.'' clearly echoes Beckett's letter to Warrilow. The difference is
that the narrative voice has changed from a first-person to a
third-person narrator. This change marks an interesting turn in
Beckett's late work for the theatre, which tends to present itself as
staged narratives, or what Matthijs Engelberts has termed \emph{récit
scénique} \citep[211--12]{engelberts_defis_2001}. In the case of this opening
sentence, the third-person narrative adds depth and ambiguity, because
the sentence ``Birth was his death'' (which was to become ``Birth was
the death of him'' in the published version of \emph{A Piece of
Monologue}) leaves open the possibility that the protagonist's birth
meant the death of someone else. In the case of the author himself, who
was born on Good Friday 1906, that ``someone else'' was Jesus Christ,
whose suffering on the cross therefore always marked Beckett's birthday
-- a concurrence that never stopped leaving an imprint, also on his
works.

The note ``Birth was his death. Etc.'' is not dated, but on the back of
the piece of paper, Beckett drafted a poem (``fleuves et océans'') that
is dated ``Ussy Toussaint 77'', that is, 1 November 1977. The next day,
on 2 November 1977, he told Jocelyn Herbert he was ``taking it very easy
through 2 prose pieces underway --- snail like --- one very limited in
scope and ambition, the other I hope to keep me going (company) for the
duration'' \citep[471]{beckett_letters_2016}. As Beckett's pun indicates,
\emph{Company} was the longer text that was keeping him company, the
shorter one was \emph{A Piece of Monologue}. Beckett does not even
mention the poem, but the manuscripts indicate that, on 1 November 1977,
Beckett was working on at least three literary projects in three
different genres at the same time: poetry, prose and drama. The
commissioned piece of monologue clearly had a thematic impact on the
poem's content, as the opening lines already indicate, playing as they
do with the standard expression ``ils l'ont laissé pour mort'' (they
left him for dead): ``fleuves et océans / l'ont laissé pour vivant''
(rivers and oceans / have left him alive) \citep[217]{beckett_collected_2012}. So life (`vivant') is, again,
presented as a process of dying. This is a clear case of creative
concurrence, materialized on the recto and verso sides of one single
scrap of paper \citep[UoR MS 2460, m18, 01r and 01v; qtd. in][72]{van_hulle_becketts_2019}.

The concurrence between the drafts of \emph{Company} and \emph{A Piece
of Monologue} even became a form of confluence at some point, however
briefly. After having been working on both projects concurrently for
about eight months, Beckett tried on 17 May 1978 to insert this excerpt
from \emph{A Piece of Monologue} into \emph{Company}, as paragraph 54,
written without errors or revisions. Here, the pronoun (`you') is
different yet again, less ambiguous this time:

\vskip 2em

\begin{quote}
Birth was the death of you. At close of day. Sun sunk behind the
larches. Needles turning green. Light dying in the room. Soon none left
to die. No. No such thing as no light. Dies on the dawn \& never dies.
Slowly in the dark a faint hand. It holds high a lighted spill. In light
of spill faintly the hand \& milkwhite globe. Second hand. In light of
spill. It lifts off globe \& disappears. Reappears empty. Lifts off
chimney. Two hands \& chimney in light of spill. Spill to wick. Chimney
back on. 
\begin{flushright}
\citepalias[EM, 25r]{beckett_bdmp_2020}
\end{flushright}
\end{quote}

\noindent To make this confluence possible, however temporarily, Beckett decided
to leave out four sentences from the dramatic fragment, because they all
related to the act of looking \citep[211]{engelberts_defis_2001}. But then again,
Beckett soon decided that this confluence did not work after all, and he
cancelled the paragraph with a large St. Andrew's cross.

In the end, the three concurrent projects were published separately:
\emph{A Piece of Monologue}, \emph{Company} and the poems as part of the
\emph{mirlitonnades}. But the creative concurrence still shows in the
theme of life as a form of dying, ``his'' birth thus being both the death
of the protagonist and that of Christ. The biographical link is even
more explicit in \emph{Company} than in \emph{A Piece of Monologue},
both of which were written while Deirdre Bair was preparing her
biography of Beckett:\footnote{Bair's bibliography was published on 14 September 1978; see \citealt[207]{pilling_samuel_2016}.} ``You first saw the light of day the day Christ died'' \citep[9]{beckett_company_2009}.

Sometimes, the creative concurrence can materialize in a single
document, such as a notebook. Beckett's so-called ``Eté 56'' Notebook is a
good example (``Eté 56'' because that is what Beckett has written on its
cover). The very idea of identifying his notebook with a period (summer
1956) rather than giving it a title (as he did in other cases, such as
the ``Molloy'' Notebooks or the ``Whoroscope'' Notebook), suggests the
remarkable concurrence of several projects at this point in time. To be
correct, the notebook also contains drafts that were written later than
the Summer of 1956, but several drafts do testify to the concurrence of
various literary projects in the same creative space. As a physical
environment, the notebook serves as a creative ecology. It contains
notes and drafts pertaining to several works, as Beckett retrospectively
noted on the front flyleaf: the play \emph{Fin de partie}, the radio
play \emph{All That Fall}, the play \emph{Krapp's Last Tape}, the prose
work \emph{Comment c'est}, the play \emph{Happy Days} \citepalias[`Eté 56' Notebook]{beckett_bdmp_2015}. And he even forgot to mention the radio play
\emph{Words and Music}. The back of the notebook also contains an
attempt to design the typography of a title page for his radio play
\emph{Embers}. As this list already indicates, the creative concurrence
in this notebook involves an interesting generic interaction between
prose writing and dramatic texts, but also between media (as the
notebook also contains notes and texts for the radio medium).

Even if the temporal coincidence is not preserved in one particular
document, it is possible to observe the impact of creative concurrence
and its relevance to genetic criticism. Writers often allude to their
previous works, creating a so-called ``intratextual'' network of
references, a set of ``links established by a reader between at least
two texts written by the same author'' \citep[93]{martel_les_2005}. To see how
concurrent writing on different projects can result in intratextual
interference across versions, the winter of 1957-58 is a particularly
interesting juncture. In the late fall of 1957, Beckett was struggling
with the translation of his own novel \emph{L'Innommable} into English,
and so he interrupted his translation to write a first draft of a radio
play, \emph{Embers}. At that moment (10 December 1957), the BBC
broadcast a fragment from Beckett's novel \emph{Molloy}, read by Patrick
Magee. Beckett was struck by the actor's voice, but the transmission was
not ideal. While Beckett temporarily abandoned his work on the radio
play and continued translating \emph{L'Innommable},\footnote{In his
  manuscript of \emph{The Unnamable}, Beckett marked this moment on page
  23v of the second notebook, referring to \emph{Embers} as ``Henry \&
  Ada'': ``Reprise 21.1.58 après échec de Henry et Ada'' {[}``Taken up
  again 21.1.58 after failure of Henry and Ada''{]} (\citetalias[EN2, 23v]{beckett_bdmp_2014}).} he went to the BBC studios in Paris where they played a
recording of Magee's reading. This was probably the first time Beckett
saw a tape-recorder, which prompted him to start writing the play
\emph{Krapp's Last Tape} (originally called ``Magee Monologue'') even
before the end of his translation of \emph{L'Innommable} --- which he
finished when he was three days into the writing process of
\emph{Krapp's Last Tape} \citep[138--50]{van_hulle_making_2015}.

But that is still just concurrent writing in general, chronological
terms. If we look at the texts, we see that this creative concurrence
has a direct intratextual effect on the content of these works. A
notebook at Harvard University's Houghton Library \citepalias[HU MS THR 70.3]{beckett_bdmp_2018}
contains an early version of the radio play \emph{Embers},\footnote{This
  early English version of the radio play \emph{Embers} starts on folio
  10r in the form of a dialogue between ``He'' and ``She'', which ends
  on page 20r.} which can be read as an inquiry into the workings of the
human mind that was inspired partly by listening to Patrick Magee's
readings of the \emph{Molloy} fragments (10 December 1957)\footnote{On
  11 December 1957 Beckett wrote to Donald McWhinnie at the BBC: ``I was
  in Paris last night and there listened to ``Molloy''. Reception
  execrable, needless to say, but I got enough, knowing the text so
  well, to realize the extraordinary quality of Magee's performance.'' \citep[77]{beckett_letters_2014}} and \emph{From an Abandoned Work} for the
BBC.\footnote{In the same letter of 11 December, Beckett wrote he was
  ``hoping to get it clearer on Friday, and on Saturday FAAW
  {[}\emph{From an Abandoned Work}{]} which I am waiting for with acute
  curiosity'' \citep[77--78]{beckett_letters_2014}.} But \emph{Embers} was also
inspired by the act of translating \emph{L'Innommable}. In the
manuscript of \emph{The Unnamable}, just before ``Basil is becoming
important'' and the narrator decides to ``call him Mahood instead''
\citepalias[EN1, 21r]{beckett_bdmp_2014}, the first-person narrator describes his own
voice, which --- he says --- was ``like the sea'':

\vskip 1em

\begin{quote}
I strained my ear towards what must have been my own voice still, so
weak, so far, that it was \emph{like the sea}, a calm distant sea far
calm sea dying --- no, none of that, no beach, no shore, the sea is
enough, I've had enough of \emph{shingle}, enough of sand, enough of
earth, enough of sea too. 
\begin{flushright}
\citepalias[EN1, 21r; emphasis added]{beckett_bdmp_2014}
\end{flushright}
\end{quote}

\noindent While Beckett was making this translation, he started writing
\emph{Embers}, which opens with the directions: ``\emph{Sea scarcely
audible.} Henry's \emph{boots on shingle}'' \citep[20]{beckett_krapps_1959}. Here,
the creative concurrence obviously had a direct intratextual impact.

But there were also \emph{inter}textual connections that were partly
triggered by concurrence.\footnote{I use the notion of
  ``intertextuality'', as opposed to ``influence'' (which only looks at
  interaction between authors without taking the reader into account),
  in the sense of Michael Riffaterre's definition of intertextuality as
  ``the reader's experience of links'' between different works \citep[4]{riffaterre_trace_1980}, with the explicit proviso that this also includes genetically
  informed readers.} The opening directions of \emph{Embers} with the
sea in the background and the boots on the shingle recall the Joycean
image of Stephen Dedalus walking on Sandymount strand in the ``Proteus''
episode of \emph{Ulysses}. At this point, one might object that this is
reading too much Joyce into the text. But when we look at the chronology
of Beckett's day-to-day business, we see that he had just been reading
Joyce's letters shortly before he started writing
\emph{Embers}\footnote{Faber had sent him ``Joyce's letters'' according
  to a letter to Barbara Bray (18 September 1957) and a month later he
  wrote to Thomas MacGreevy that he had been ``reading his letters''
  \citep[137]{pilling_samuel_2016}.} and that, after interrupting this writing
process and taking \emph{Embers} up again, he read Stanislaus Joyce's
book \emph{My Brother's Keeper}, about which he reported to Con
Leventhal on 29 February 1958 \citep[39]{van_hulle_samuel_2013}. Beckett
thus seems to have been reminded of Joyce's famous experiment with the
interior monologue in \emph{Ulysses}, where Stephen closes his eyes to
concentrate on the ``ineluctable modality of the audible'' (episode 3).
In \emph{Embers}, Beckett chose a similar setting to give a new, equally
experimental shape to the Joycean stream of consciousness, concentrating
on the audible in the ```blind' radio medium'' \citep[251]{beloborodova_broadcasting_2018}.

And there is more going on at the same time. Next to this watery
connection to the sea, another element that connects the different
concurrent creative projects is fire. In \emph{A Portrait of the Artist
as a Young Man} and later in \emph{Ulysses}, Joyce had made his
protagonist refer to Percy Bysshe Shelley's fading coal as a metaphor
for ``the mind in creation'': ``In the intense instant of imagination,
when the mind, Shelley says, is a fading coal'' \citep[9.380]{joyce_ulysses_1986}. The
reference highlights only the first part of Shelley's comparison. In the
second part of the famous ``Defence of Poetry'', Shelley interestingly
suggests a hint of linguistic skepticism, thereby presaging Fritz
Mauthner's \emph{Sprachkritik}, which fascinated both Joyce and Beckett:

\begin{quote}
the mind in creation is as a fading coal, which some invisible
influence, like an inconstant wind, awakens to transitory brightness;
this power arises from within, like the colour of a flower which fades
and changes as it is developed [\ldots{}]. Could this influence be
durable in its original purity and force, it is impossible to predict
the greatness of the results: but when composition begins, inspiration
is already on the decline; and the most glorious poetry that has been
communicated to the world is probably a feeble shadow of the original
conceptions of the poet. 
\begin{flushright}
\citep[505]{shelley_shelleys_1977}
\end{flushright}
\end{quote}

\noindent The moment of inspiration may awaken the coal's brightness, but as soon
as the poet starts putting it into words the coal increasingly
transforms into the state of embers and ashes, which suggests a link
with Beckett's lifelong fascination with the ineluctable modality of the
ineffable. One of the intertextual elements in Beckett's work that
illustrate this fascination is his favourite line from Petrarch, which
he used to quote by heart: ``Chi può dir com' egli arde é in picciol
foco'' \citep[80]{atik_how_2001}. In Beckett's own translation: ``He who knows he
is burning is burning in a small fire''; or in an older translation by
John Nott: ``Faint is the flame that language can express'' \citep[80]{atik_how_2001}. In
other words, if you can still put it into words, it can always get
worse. Beckett used these ``faint fires'' in his translation of
\emph{L'Innommable}: ``Je ne dois mon existence à personne, ces lueurs
ne sont pas de celles qui éclairent ou brûlent'' \citep[13]{beckett_linnommable_1953} / ``I owe
my existence to nobody, these faint fires are not of \sout{the kind}
those that illuminate or burn'' \citepalias[EN, 04r]{beckett_bdmp_2014}. During the
process of translating \emph{L'Innommable}, Beckett worked on the first
manuscript version of \emph{Embers} --- which on its last page tellingly
contains the words ``the fire gone'' \citepalias[HU MS THR 70.3, 20r]{beckett_bdmp_2018} --- and the
first version of \emph{Krapp's Last Tape}, in which the fire
(symbolizing creative power) is a key theme. When Beckett discussed the
play with the actor for whom he had written it (Patrick Magee) he wrote:
``While I think of it a word to be brought out very strong is `burning'
(page 7, line 1), in order that `fire' at the end may carry all its
ambiguity'' \citep[129]{beckett_letters_2014}. Beckett is here referring to a
particular typescript, on which he replaced the word ``panting'' by
``burning'': ``drowned in dreams and \sout{panting}
\textsuperscript{burning} to be gone'' \citepalias[ET5, 07r]{beckett_bdmp_2015}. The
ambiguity Beckett mentions is the tension between the wish to die
(burning to be gone) and the creative fire of the artist Krapp when he
was still younger, full of himself and ``burning'' to write his
\emph{magnum opus}. This cluster of meanings was contained in that one
line from Petrarch, as Beckett explained in a letter to Con Leventhal
shortly after the composition of \emph{Krapp's Last Tape}: after quoting
the line from Petrarch, he writes that he understands ``arde'' not in
the sense of ardent lovers' burning desire but ``more generally, and
less gallantly, than in the Canzoniere'':

\begin{quote}
As thus solicited it can link up with the 3\textsuperscript{rd}
proposition (coup de grâce) of Gorgias in his Nonent:
\begin{enumerate}
\item Nothing is

\item If anything is, it cannot be known.

\item If anything is, and can be known, it cannot be expressed in speech.
\end{enumerate}
\begin{flushright}
\citep[136]{beckett_letters_2014}
\end{flushright}
\end{quote}

\noindent As this letter indicates, Beckett seems to have had a particular
interpretation of Petrarch's line in mind, which focused on linguistic
scepticism. Applied to Krapp, the Petrarchan subtext suggests that he
who can say he is burning (the middle-aged Krapp on the tape, the artist
as a younger man full of himself with ``the fire in {[}him{]} now,''
convinced that he is going to write a magnum opus worth leaving his love
for) is burning in a small fire.

All these faint fires find their place in Beckett's work almost
concurrently in the winter of 1957-58, when Beckett is simultaneously
working on his translation of \emph{L'Innommable}, his first attempt at
writing \emph{Embers}, and \emph{Krapp's Last Tape} (which is then
followed by another bout at writing \emph{Embers}, so that \emph{Krapp's
Last Tape} can be said to have been written as an \emph{entr'acte} in
the genesis of \emph{Embers}). The intratextual connection between the
faint fires in these works is triggered by an intertextual link with
Petrarch. Beckett had been reading Petrarch as a student: his two-volume
edition of \emph{Le Rime del Petrarca} 
contains numerous marginalia and student notes (indicating for instance
the rhyme scheme of some of the sonnets), but remarkably enough, his
favourite line is \emph{not} marked \citep{petrarca_rime_1824}. This raises the question why
Petrarch suddenly enters the scene at this particular moment in 1958.
Most probably, the trigger was actually another author. Beckett's
personal library contains a 1958 edition of Michel de Montaigne's
\emph{Essais}. In his second essay, \emph{De la Tristesse}, Montaigne
quotes precisely this line from Petrarch \citep[32]{montaigne_essais_1958}. In and of
itself, this Pléiade edition's publication in 1958 does not necessarily
corroborate the potential connection with Beckett's output in the same
year. To establish this connection, it may be useful to confront
creative \emph{con}currence with another important force to be reckoned
with in genetic criticism: creative \emph{re}currence.

\section{Creative Recurrence}

To examine the genesis of one particular work, it is often necessary to
zoom out and include the rest of the \emph{œuvre} and the
\emph{sous-œuvre}, both backward and forward in time. In
\emph{retrograde} direction, as indicated above, Beckett's knowledge of
Petrarch dated back to his student days in the 1920s at Trinity College,
Dublin, where he studied French and Italian literature. In the 1930s, he
made copious notes on philosophy, with a special interest in the
Presocratics. Under the heading ``\textsc{{Gorgias of Leontini}} in
Sicily (483-375)'' he noted:

\begin{quote}
Three celebrated propositions --

\begin{enumerate}
\def\labelenumi{\arabic{enumi}.}
\item
  Nothing exists.
\item
  If it did, it could not be known.
\item
  If it could be known, it could not be communicated. (In his {On
  Nature, or, The Non-Existent})
\end{enumerate}

All opinions equally false. 
\begin{flushright}
\citep[TCD MS 10967/48; qtd in][76]{feldman_becketts_2009}
\end{flushright}
\end{quote}

\noindent As the above-mentioned letter to Con Leventhal shows, twenty-five years
after Beckett noted this down among his Philosophy notes, Gorgias's
propositions ``had lost little relevance to Beckett'' \citep[76]{feldman_becketts_2009}.

Looking \emph{forward} in time, Beckett's ``Nonent''-related reading of
his favourite line brings Petrarch's statement closer to another one of
Beckett's favourite lines, this time from Shakespeare's \emph{King
Lear}: ``The worst is not / So long as one can say, This is the worst'' \citep[qtd. in][UoR MS 2910, 14v]{van_hulle_beckett_2010}. Again the underlying idea is similar: as long as you can
still say it, it can always get worse. This line is jotted down in a
much later notebook from the 1970s (the so-called ``Sottisier'' Notebook),
and is the starting point for \emph{Worstward Ho}, a work Beckett wrote
in the 1980s. In another notebook from the 1980s, he jotted down this
reading trace: ``Curae leves loquuntur, ingentes stupent (Seneca)''
(Light sorrows speak, deeper ones are silent.) \citepalias[UoR MS 2934, f. 01r]{beckett_bdmp_2011}. Seneca is not part of Beckett's personal library in Paris.
But there is a noteworthy connection with the same Pléiade edition of
Montaigne's \emph{Essais}. The quote from Seneca appears in the same
second essay, \emph{De la Tristesse}, on page 33, the page facing the
one that contains the quotation from Petrarch. This diachronical axis of
recurrent preoccupations meets the synchronic axis of creative
concurrence in early 1958, giving us a sense of the various literary
projects that were lying on Beckett's desk at the same time, and thus
mutually impacted on each other.

\section{Conclusion}

As mentioned above, now that genetic criticism exists more than fifty
years, the time seems propitious to refine de Biasi's model and find a
place for both the sociology of writing and creative
concurrence/recurrence in genetic criticism. The details of the examples
above should not prevent us from seeing the wood for the trees. They
provide us with the particulars to remodel the complex relationship
between private and public, which has perhaps too easily been
oversimplified as a general dichotomy, and which has in that capacity
played a central role in genetic criticism for several decades. As in
McKenzie's refutation of McKerrow's general statement (quoted above),
the details of everyday life in an author's writing practice can
transform a valid general statement into an invalid particular
statement. As discussed above, in many cases, the text can already
become public before the \emph{bon à tirer} moment, for instance through
pre-book publications. And social events, such as commissions, can lead
to periods of creative concurrence, which is not included in de Biasi's
scheme as its implied focus is on single works as items of study,
relatively isolated from the rest of the oeuvre.\footnote{For instance,
  he defines the collection of genetic documentation {[}\emph{dossier de
  genèse}{]} as ``the whole body of known, classified, and transcribed
  manuscripts and documents connected with \emph{a text} whose form has
  reached, in the opinion of its author, a state of completion or near
  completion'' \citep[31; emphasis added]{biasi_what_1996}. The ``Typology of
  Genetic Documentation'' is presented as ``a general table of the
  stages, phases, and operational functions that enable the
  classification of different types of manuscripts according to their
  location and status in the process of \emph{a work}'s production''
  \citep[32; emphasis added]{biasi_what_1996}.}

To some extent, such a work-by-work approach may have been conditioned
by the print paradigm, which has had consequences for scholarly editing.
Due to the limited space, a printed edition often necessarily presents a
literary oeuvre as a set of works, represented by a critically edited
text and accompanied by annotations and a critical apparatus; and due to
the codex format, the individual works are often contained in separate
volumes. As a result, the writer's complete works appear as a sum of
parts. By means of introductions and annotations, the editor usually has
to remedy this by explaining that the oeuvre also constitutes a
\emph{Gestalt}, a whole that is greater than the sum of its parts. A
\emph{Gestalt} however, is not only greater than, but also
\emph{different} from, the sum of its parts. As Caroll Pratt writes in
the introduction to Wolfgang Köhler's \emph{The Task of Gestalt
Psychology,} it is a common error to leave out the word ``different''
and simply define a \emph{Gestalt} as ``the whole is more than the sum
of the parts'' \citep[9]{pratt_introduction_1969}. This definition mistakenly ignores that a
\emph{relationship between} the parts is itself something that is not
present in the individual parts themselves \citep[10]{pratt_introduction_1969}. If all the parts
of a bike are laid out on the floor of a bike shop, for instance, they
still do not make up the bike. Only when the parts are assembled and
come to take up a specific \emph{relation} to each other, do they become
something different, that is, a bike. If, in the digital age, we treat
the oeuvre as a \emph{Gestalt} to develop a digital complete works
edition that is truly \emph{complete}, we need to think inclusively
about the entirety of genetic dossiers relating to an author's works,
both published and unpublished, providing tools for advanced
chronological searches into both the synchrony and the diachrony of the
\emph{sous-oeuvre}, in which creative concurrence and recurrence play a
significant role.

\begin{flushleft}
\bibliography{references/vanhulle}  
\end{flushleft}

\end{paper}