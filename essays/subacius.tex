\contributor{Paulius V. Subačius}
\contribution{On the Threshold of Editorship. Or From \emph{Collection} to \emph{Oeuvre}}
\shortcontributor{Paulius V. Subačius}
\shortcontribution{On the Threshold of Editorship}

\begin{paper}

\begin{abstract}
The sole collection of verses by Jonas Mačiulis (1862--1932, pen name Maironis), the father of modern
Lithuanian poetry,  went
through five editions in the author's lifetime. The poet continued to
improve the works of his youth until his advanced age, producing
hundreds of textual variants. The first published versions of some
verses, which are ranked as national classics today, cannot be
considered finely crafted works by a long shot. From a retrospective point of view, one is
surprised that such lengthy and tedious corrections could have yielded
such nice final results, while a prospective approach might incite amazement at how
the crude, primary rock of first versions could have concealed the possibility of such poetic gems. Taking a
grandmotherly attitude towards his works, the author did not only polish their
versification, but also applied different strategies throughout his rewriting process to use new editions to
consolidate his own social and cultural position with the changeable standards of
bibliographic codes over time. The problem editors of the Maironis'
posthumous editions face is that the authorial editions of his works
differ in title, textual variants, and arrangement. As this article argues, both
Maironis' high standing in the Lithuanian cultural landscape and the elusive expectation of stability and comprehensiveness
have discouraged editors to plunge into the dynamic nature of Maironis' authorial
manipulations up until today.
\end{abstract}

\section*{} 
\textsc{There are various stories} of how prominent works were written and
published, but most of them understandably contain common traits.\renewcommand*{\thefootnote}{\fnsymbol{footnote}}\footnotemark[1]\footnotetext[1]{This research was funded by a grant (No. S-MOD-17--7) from the Research Council of Lithuania.}\renewcommand*{\thefootnote}{\arabic{footnote}} The first of these would be the
fact that most revisions remain unknown to readers
until their manuscripts are analysed posthumously --- since
authors usually try to publish completed versions. The exception to this rule would be when versions printed in periodicals are followed
by different versions in book editions, or when authors altered their
already successfully circulating works for an edition of their
\emph{oeuvre} or \emph{selected works} to summarize their writing careers. While textual instability is regarded as ``a fact of life'' by contemporary theorists \citep[194]{shillingsburg_textuality_2017}, authors have long preferred to conceal this fact, rather than present it as a goal of their authorship.\footnote{In fact, we know that authors were even required to withdraw their earlier versions upon publishing of a
  new version at least since classical antiquity \citep[see][24]{reynolds_scribes_1991}.}

Another feature of our common assumptions towards the
origins of famous works would be the value we tend to place on them. There is another end to the stick in the teleological approach
to the creative process, as has been discussed by genetic
critics \citep[see][56]{bushell_intention_2005}. If a purposeful relation exists between a work's
earliest drafts and its published text, and if the
writing and revision processes moving in the direction of a final entity, it is only
natural that we also have a look in the reverse direction. In hindsight, a successful
result invites the presumption that --- at least to a
certain degree --- the work's creative potential was already inherent in the earliest variants, which are not only endowed
with historical value in the process, but also with aesthetic value by omission.

However, reality does not always comply with these
presuppositions. How do they reshape our understanding of the way in which literature functions? Do textual
critics or editors have any means, not unlike mineralogists, to perceive and
reveal polishable crystalline structures that are deeply hidden in immature texts?
Or do we perhaps only 
attribute the status of a ``variant'' or ``source'' to a primitive sequence of
words once we know how valuable the final result is? How do we represent the entire genesis of a work if the
author himself changed his attitude towards the work several times along the way, and
applied several different strategies to use it for the consolidation of his
social and cultural position? Some of these questions will remain
rhetorical throughout this essay, intending to inspire further analysis. If I succeed in revealing the
complexity of editorial situations, we may only come some steps closer to devising
simplified answers to these questions. Nevertheless, the potential of digital archives when it comes to  communicating the wholeness of works with complicated geneses to a contemporary audience makes me optimistic for the future.

In this essay, I will discuss these issues by investigating the case of a
Lithuanian writer known under the pseudonym of \emph{Maironis}. Jonas
Mačiulis (1862--1932), a priest, professor of theology and historian,
used this pen name to sign one of the first publications of his poems in
a clandestine periodical \citep{maironis_daina_1891}. He became an icon of the
national revival and was already proclaimed the central figure of modern
Lithuanian literature during his lifetime in the early 20th century \citep[54--55]{seina_maironio_2016}. This official assessment of the poet has not
changed in the last century \citep[see][184--87]{kalnacs_300_2009}. Shortly
after his death, Maironis' house was converted into the national museum
of Lithuanian literature. A 20-litas banknote issued by the Bank of
Lithuania, which was in circulation from 1993 to 2014 (at which point the
national currency was replaced by the Euro), bears his portrait.

Maironis wrote narrative poetry, dramas, and librettos, but it were
his poems that won him the status of a classic. Because the tsarist administration forbade the press to use Latin characters in the 1860s in an attempt to introduce the Cyrillic script, Lithuanian newspapers were published abroad and secretly smuggled into the country until 1904.
Having published eleven poems in these newspapers, Maironis
illegally published a book of his poetry in 1895 \citep{maironis_pavasario_1895}. To this day, this
collection of verses --- the name of which (\emph{Pavasario balsai} \emph{[The Voices of
Spring]}) symbolized the rebirth of the nation's typical nationalism --- remains the most important form for circulating his poetry. At the time
of publication, Maironis already had two published books on his record: one a narrative of the history of Lithuania \citep{maironis_apsakymai_1891}, the other an epic
poem with social overtones \emph{Tarp skausmų į garbę} [\emph{Through
Pain to Glory}] \citep{maironis_terp_1895}. However, it was \emph{The
Voices of Spring} that paved the way to his role as a
national writer. Having realized this, the poet would later cling to the same
perfectly recognizable figurative title, and published three more collections of
his poems appeared as \emph{The Voices of Spring} in his
lifetime \citep{maironis_pavasario_1905, maironis_pavasario_1913, maironis_pavasario_1920}. In addition, a total of 30 posthumous editions of Maironis' poems in Lithuanian were published as
\emph{The Voices of Spring} (the
latest one only yesteryear: see \citealt{maironis_pavasario_2020}).\footnote{The penultimate edition was marked as
  28\textsuperscript{th} \citep{maironis_pavasario_2012}, but for some editions the numbering indicated in
  their subtitle does not coincide with their actual
  numbering in the sequence identified by Maronis' bibliographers. In a discussion
  following my presentation on this collection of poems at the
  26\textsuperscript{th} annual SHARP conference ``From First to Last:
  Texts, Creators, Readers, Agents'' (Western Sidney University, 2018),
  Paul Eggert noted that such a large number of re-editions of a poetry
  book in nine decades is phenomenal, particularly bearing in mind the
  relatively small Lithuanian readership (of \textasciitilde{}3.5 mil
  persons), and the fact that Maironis was hardly known beyond the
  national borders. It should be noted that the poet was indeed
  popular, but that social and political circumstances rather than
  financial interest played an important role in this process of re-editing his works: the community of Lithuanians in exile in the USA, Germany, and Italy (separated from Luthuania in the cold war) were instrumental in publishing a series of local non-commercial
  editions to provide teaching materials for Lithuanian courses and
  Saturday schools. In addition, during Nazi and especially
  Soviet occupations, the government had to replace the editions with more ideologically appropriate
   ones, and some of the poems
  were censored out. That is why, after the change of the regime in
  1990, editors aimed to publish more complete collections of Maironis' poetry.} In this
sense, Maironis is a poet who is identified with a sole collection of poems \citep[62]{daujotyte_intention_1990}. None of the classics of national revival in the
surrounding countries (Janis Rainis in Latvia, Lydia Koidula in Estonia,
Taras Shevchenko in Ukraine) were identified so directly with one
title except for Belarusian short-lived Maksim Bahdanovich (1891--1917),
whose collection of poems \emph{Vianok} [\emph{A Wreath}] (1914) was his
only book of poetry.

It is worth mentioning that when Maironis chose \emph{The Voices of
Spring} as a stable brand name \citep[cf.][251 and 257]{kucinskiene_ankstyvasis_2014},
this decision was exclusively of symbolic rather than financial value. The
publishers' interests and opinions only played a minimal role here, since the publications were funded by the author himself, and by Catholic societies. Due to
the political restrictions of that time, the only thing that the author
may have earned from the first edition of his collection of poems was a
deportation to Siberia. Later, when the ban on the press in Latin
characters was withdrawn, he held well-paid positions of professor of
the Theological Academy and rector of the Seminary; thus remuneration
for his creative work was not and could not have been a source of income
and an underlying motive for this activity.

However, \emph{The Voices of Spring} did not remain the only form in
which Maironis published his collection of poems. Here we can talk about
his secondary initiation: his  initiation to the classics. Six years
before his death, Maironis edited the first volume of his \emph{Oeuvre}
dedicated to verses he chose himself and financed with his own means.\footnote{This
  was not a unique case at that time in Lithuania: by 1927, eight
  living writers had republished or began to republish their works,
  specifically under the title \emph{Oeuvre}. Two of them (Juozas Tumas,
  penname Vaižgantas and Kazimieras Pakalniškis) were friends of
  Maironis.} He did not resume the tradition of \emph{The Voices of
Spring} and titled it \emph{Lyrika} [\emph{Lyrics}] \citep{maironis_maironio_1927}.\footnote{``Sixth edition'' is added under the title, thus
  indicating a direct link to the editions of \emph{The} \emph{Voices of
  Spring}. Bibliographically, Vol. 1 of \emph{Oeuvre} should be the
  fifth edition of Maironis' poetry, but the second edition of
  \emph{The} \emph{Voices of Spring} of 1905 was published with a
  reference that it was the third edition (as the short play \emph{Kame
  išganymas} (\emph{Where is the Salvation}) included in the book after
  the poems was indeed published for the third time), and this is how
  this erroneous sequence became established.} The latter title
expressed the idea that Maironis' works could be regarded as a poetic standard for
Lithuanian literature in general.

A more intense alteration of the poetry collection rises to the surface when
we look at the authorial use of \emph{The Voices of Spring} from the
viewpoint suggested by William Stroebel when he reflected on Constantine
Cavafy's collections: ``not as a one-way communication device
broadcasting the poet's final intentions but as a kind of ongoing, open
workshop, one that continually suspended the finality of its own
production and extended the processes of inscription, assemblage, and
re-assemblage'' \citep[280]{stroebel_assembly_2018}. Extremely important were the
transformation of the arrangement of texts and the revisions in the
poems themselves. In the case under discussion, \emph{a single collection of
poems} does not exactly equal \emph{a collection of the same poems}.
\emph{The Voices of Spring} underwent considerable changes in the poet's
lifetime. Having gone through five editions, it increased three times
from 45 to 131 poems; four poems were once included in the collection,
once removed from it by the author's will.

Far more significant were the revisions in the poems themselves. Even in
those few poems that underwent relatively few alterations by the
author, the spelling was edited and some words or forms were
replaced. The poem ``Nuo Birutės kalno'' [From Birutė Hill],\footnote{Here
  and hereafter I use the traditional titles from the 1927 edition to
  identify the poems since Maironis had revised them many times.}
included in all the editions of the poetry collection, contains the
least amount of revisions. In the author's last version it has 16 lines,
106 words and 535 letters (including the title). During thirty-two years
of revising, in five versions published by Maironis himself, seven
words, eleven morphophonological forms, five punctuation marks, and
twenty-eight cases of spelling were changed.\footnote{From this account
  were excluded variants that originated as the result of three presumed
  typographical errors in the 1895 edition, and three typographical
  errors in the 1927 edition.} While the latter variants are accidentals, the
first ones are undoubtedly substantials.

What then does the versioning of considerably more revised poems look like?
The best poem to illustrate this issue would be ``Lietuva brangi'' [Dear
Lithuania], which has acquired the status of the second, unofficial,
anthem of Lithuania, and is widely learned by heart and sung, which
makes it one of the most popular and famous works by Maironis. First published in a newspaper, the poem consisted of 16 stanzas and 62 lines \citep{maironis_lietuvos_1891}, while the author's last version had 8 stanzas and 32
lines \citep[58--59]{maironis_maironio_1927}. It is only from single words or motifs
that one can approximately guess which stanzas of the early version were
transformed into later variants, as not a single line of the initial
text coincides with the revised one. Not even a hint of eight lines of
the last version appears in the first publication. In all the other
lines both some words and the word order is changed without exception.
If we compare the publications revised by the author from a morphophonological point of view, as few as 40, or one-fourth of 160 words of
the latest text coincide with the first version of the poem. If we add
the differences in punctuation and spelling, we could say that while
revising ``Dear Lithuania'', the poet left a mere 10\% of the text
intact.\footnote{I am using statistics not only because it is quite
  convenient, but also because a much more extensive paper would be needed
  in order to present adequate translations of the variants and to
  compare them. For published samples of Maironis' poetry translations into English, see \citealt{maironis_maironis_1963} and \citealt{maironis_maironis_2002}.}

Many revisions were made to achieve more fluent
versification. Historians of literature unanimously assert that Maironis
was the first Lithuanian poet to use syllabo-tonic versification perfectly, and achieved a high level of poetical precision. In other words, metric schemes are retained in his poems; moreover, they are not in
conflict with the regular accentuation of words and syntactic
combinations, which helps to achieve a harmonious sound. These
statements hold up when analyzing the latest versions revised by the
author --- the first stanza of ``Dear Lithuania'' \citep[58]{maironis_maironio_1927} is
an example of a precise iambic foot with a hypercatalectic caesura \citep[234--5]{girdzijauskas_lietuviu_1966}:

\begin{center}
$\cup - \cup - \cup \| \cup - \cup - \cup$

$\cup - \cup - \cup \| \cup - \cup - \cup$

$\cup - \cup - \cup \| \cup - \cup - \cup$

$\cup - \cup - \cup \| \cup - \cup - \cup$
\end{center}

\noindent In the initial version, by contrast, the first stanza's metric scheme goes as follows --- respecting  the natural accents of the words in standard
Lithuanian \citep{maironis_daina_1891}:

\begin{center}
$- \cup \cup - \cup \| - \cup \cup - \cup$

$\cup - \cup - \cup \| \cup \cup - - \cup$

$\cup - \cup \cup - \| \cup - \cup - \cup$

$\cup - \cup - \cup \| - \cup \cup - \cup$
\end{center}

\noindent The irregularity here is quite obvious, and has been eradicated in subsequent versions through careful polishing.
In fact, many of Maironis' first publications contained versification
errors, which were then successfully avoided in later versions. On the
one hand, there is no doubt that Maironis
could write much more fluent verses than other contemporary poets right from the start. On
the other, a retrospective view of the classical qualities of his poetry
found in textbooks and works on the history of literature has a certain
inconsistency: the final self-revised versions are quoted as proofs of
his poetic elegance, even though the topic of discussion is the young
Maironis and the first edition of \emph{The Voices of Spring} in 1895
\citep[98]{zaborskaite_maironis_1987}, which still contains a large number of sharp edges.

By using a uniform title for the
collections of poetry, Maironis --- perhaps unintentionally --- set up 
the preconditions for his posterior literary critics to paint him as a greater literary pioneer than he
really was. This happened because when critics were analyzing the poems Maironis had written
in the 1910s, which were included in the fourth edition of \emph{The
Voices of Spring}, the newer versions of these poems were mistakenly attributed to the late
nineteenth century --- when their first edition had been published. Due to
very rapid changes that took place in Lithuania, this
quarter-of-a-century anachronism was more than enough to make Maironis
appear as an example to other writers of the first half of the twentieth
century, even though this image was partly based on impressions of poems that were (re)written at
the same time or even later than their works. National ambitions spurred
the declarations that as early as 1895, Lithuanian literature already
had a classical author who wrote poems of refined lyrical form.

When we address another type of
Maironis' self-revisions, an even more distinct shadow of anachronism looms over the critics'
remarks on the poet's modernity. This is because the four decades over which his career spans also coincide
with the period where the most intense formation of the standard Lithuanian
language took place. Already after the first publications of his poems in the
periodical press, publishers generally came to a final agreement
regarding several new letters of the Lithuanian alphabet (sz→š; cz→č; ż,
ź→ž; ē→ė; u{[}u:{]}→ū/ų; \citealt[38--39]{venckiene_bendrine_2006}). In other words, it was
necessary to transliterate the early versions of his
poems in subsequent publications. The first edition of \emph{The Voices
of Spring} was followed by a textbook of Lithuanian grammar \citep{jablonskis_lietuviskos_1901}, which established the principles of selection of prescriptive
paradigms from a variety of dialects and their contemporary orthographic
rules. In other words, part of the morphophonological forms, spelling and punctuation  had to be revised, bit by bit, in each subsequent edition of Maironis' 1895 collection \citep[204--05]{venckiene_jono_2012}. In the last period of Maironis' creative work, after the declaration of independence in 1918, the new state was quick to build a system of general education, and to promote the publishing and other forms of communication in the
official Lithuanian language. Therefore, the removal of
the dialectal vocabulary and morphological inconsistencies, as well as
the establishment of the standards of accentuation and pronunciation of
the language used in a cultured urban milieu, was taking place very
rapidly. Although Maironis relied on himself rather than editors even
for minor corrections, it is obvious that only the author himself could
revise the poems and fully adapt them linguistically to standard modern
usage.

In many cases, the rejection of dialectal forms also meant approaching
regular standard accentuation --- ergo, a more precise metre, and
\emph{vice versa}. However, I dare to assert that doubt in the
motivation for one or another revision does not deny Maironis' basic tendency
to adapt his writings to the changing language standard. These assumptions imply a
methodological remark about a prospective genetic digital edition of
Maironis' poetry. I suggest that in such an edition it would be important
to demonstrate the chronology of the appearance, inclusion, and
disappearance or removal from standard use of the linguistic features
found in the texts along with the change of variants. In addition,
historical sociolinguistic information would be very useful --- explaining which
lexemes and forms became popular or prestigious, when and in which socio-cultural
contexts these changes took place, etc. \citep[cf.][230]{venckiene_maironio_2018}. It does not occur so
frequently that the peak of language standardization would 
coincide with the period of the creative career of a prominent writer,
during which the latter would intensely (and, importantly,
successfully) revise his texts in every decade. The main difficulty is
that such an undertaking requires detailed data of language history.

Let us take a brief detour into a wider socio-cultural context --- the
development of the modern Lithuanian national identity. Here, we can again
state an important chronological coincidence between important events and different stages of
self-revision of Maironis' poetry. The second edition of \emph{The}
\emph{Voices of Spring}, for example, appeared in 1905 --- the year of the Russian revolution, when the social atmosphere became considerably freer. And the fourth
edition was prepared right after the
Republic of Lithuania proclaimed its independence. In both cases, new ideological accents can be indicated in the
poems. Finally, the last authorial edition of the
\emph{Oeuvre} coincided with the abolishing of the state's parliamentary democracy.
In the period of the authoritarian rule of the president and ideologist
of nationalism Antanas Smetona (1874--1944), which lasted from 1926 to
1940, the exemplary collection of national symbols and the historical
narrative promoted in schools acquired its final stable shape.\footnote{Due to
the Soviet occupation, it remained partially conserved up until 1990,
and still has a huge influence on the self-consciousness of the older
and middle generations even in our days.} Reciprocal influence should be
borne in mind --- Maironis textually reacted to the birth of the
national state and, in its turn, his poetry was simultaneously
popularized as texts that unified society.

A short remark on a factor that might seem to counteract the drive for
authorial revisions: several of Maironis first published poems had already been set to music. At first, they were adapted for folk melodies, and
later performed to music created by professionals. Two of the most prominent Lithuanian
composers of that time, Juozas Naujalis (1869--1934) and Česlovas
Sasnauskas (1867--1916), composed songs for mixed choirs to twenty-three
poems from \emph{The Voices of Spring}. After 1905, when the
tsarist administration stopped interfering with public national events,
they began to be performed on a massive scale. Songs that were performed during Lithuanian
gatherings rapidly became extremely popular. The author himself both cooperated
with the composers and took part in these events --- an
amateur musician himself, who held recitals at his home. On the one hand, more
fluent poems that had already been more refined with regard to versification 
in the early stages of their development were set to music. Such texts
were then revised less by the author when he prepared subsequent
editions of the collection. On the other hand, this aspect alone can hardly
explain the fact that it were especially the poems that had been set to music that
Maironis avoided revising. A juxtaposition of the text and its
melodic accents reveals that the latter stabilized the text, therefore
the idea about the conserving role of music should be borne in mind in
this case. The interaction of the melodies and the texts of the early
edition, as well as the inertia of performance when choir singers
committed the text to memory, along with the motility of singing counteracted
the scale of a more radical authorial alteration.

A look at the revisions that Maironis made in order to adapt to the
rapidly changing language and society prompts another observation, namely that any attempt to describe his relation with cultural modernization contains
an internal contradiction. The well-established statement by literary critics
that Maironis brought a modern poetic language into Lithuanian
literature \citep[286]{nyka-niliunas_maironio_1962} is both correct and misleading. The
poet resourcefully used the form of the language that existed at the
time of writing, and contributed to its further development along the way. Maironis was
modern in each stage of writing and revision: the linguistic expression
of his poetry corresponded to the state of the cultural medium for
approximately a decade until a subsequent revised edition of \emph{The
Voices of Spring} came out. However, from our contemporary perspective
the texts of the first publications do not seem to be written in the
modern Lithuanian language. On the other hand, since the state of the
standard Lithuanian language, which was settled in the late 1920s, did
not experience more radical changes in the fields of morphology and
accentuation, Maironis' latest self-revised versions do not seem
\emph{ancient}, and the texts are read as if they ``have lost the vestiges of time''
\citep[37]{zvirgzdas_po_2012}. One or two remaining archaisms, mainly lexical,
add the sheen of nobility that is characteristic of classical works, but do not
impede an easy reading of the works as modern texts. Probably that is
why the topic of the development of Maironis' poetry seems strongly
overlooked even in the Lithuanian context, where the critics' attention
to textual variants and self-revision is extremely scarce in general. They almost
seem like intuitive attempts to prevent a wide audience from discovering a \emph{more archaic} Maironis, so as not to tarnish his
image as a modern poet.

Another aspect of his image --- the halo of solidity, stability, and
wholeness \citep[cf.][15]{slavinskaite_maironis_1987} --- contradicts what we know
about the numerous revisions of his texts. An afflatus would be much
more befitting to a poet of overdue Lithuanian Romanticism than the
meticulous labour of adjusting their metre to perfection. In
Maironis' case a distinct dissonance of reflection was provoked by the
fact that when he published his last self-revised versions in \citeyear{maironis_maironio_1927}, his
poetry had not only been an obligatory read, but had also already occupied one of the top
positions in schools' curricula for a decade. In other words,
a generation of students who knew at least a dozen of Maironis' poems
by heart from their school years, experienced having to
correct their memory and apparently discard the lines which had stuck in
their minds.

In any event, authorized polyvariance has remained a potential aporia of
reception. Naive lovers of Maironis' poetry felt quite embarrassed when the early versions of well-known poems were
reprinted in commemoration of the centennial anniversary of the first edition of
\emph{The Voices of Spring} \citep{maironis_pavasario_1995}. Readers failed to
accept an imperfect, unpolished classic. In the meantime, in order to
trace the genesis of Maironis' most significant stanzas, one would have
to take an even larger step back. It is not difficult to recognize the
rudiments of the already mentioned poem-anthem ``Dear Lithuania'' and
several other poems --- separate lines, phrases, motifs and models of
strophes --- in his earliest work ``Lietuva'' [Lithuania] \citep{maironis_lietuva_1888}. This descriptive long poem of loose structure, which is more
reminiscent of a bundle of verses on the topic of nature and history,
was written by Maironis at the age of twenty-six and never published. It
was not until the end of the twentieth century that it was included as a
supplement to the second volume of \emph{Oeuvre} containing narrative
poems \citep{maironis_rastai_1988}.

Many writers refrain from publishing their first creative attempts, because they consider
them immature. Maironis' ``Lithuania'', however, is not merely an early
attempt at versification: it is an agglomeration of the rudimentary
elements of his entire poetics. Maironis was noted for a kind of
``economy'' of motifs \citep[81]{speicyte_anapus_2012}. For the entirety of his literary career, he made
use of a limited array of images, which appeared already in his first
verses, and with each edition produced an increasingly stronger poetic
effect due to the constant refinement of the texts. A few known rough
copies testify that even in his mature age, Maironis would start a new
work from a very weak version with a chaotic metre and non-matching
rhymes, which would significantly improve after several revisions
\citep{maironis_punes_1925}. A genetic edition could demonstrate the acting of the
agency that transformed a bad poem into several particularly good poetic works. The genetic dossier of Maironis' verses poses a serious
challenge to the premises of \emph{afflatus}, which is particularly
intriguing bearing in mind the fact that the poet was a priest. The
scrutiny of the succession of textual transformations compels us to
think that the poet's entire activity was geared towards perfecting his craftsmanship.
Still, a reader remains perturbed by occasionally glimmering traces of
irreducible creativity.

Eventually, there is a serious dissonance between the two self-editing
tendencies that are typical of Maironis: 1) he was never satisfied with the result,
intensely reworking his poems for each new publication; and 2) in several
cases, he omitted the unsuccessful verses, while never completely discarding even
the weakest of his already published works, but rather including all of them in
the collections and \emph{Oeuvre}. For example, being unable to deal
with the compositional heterogeneity of one of his early poems,
``Lietuvos vargas'' [Misery of Lithuania] \citep{maironis_lietuvos_1885}, he simply
split it up, and included it in \emph{Oeuvre} as two separate works \citep[50--51]{maironis_maironio_1927}. As such, the whole remained very diverse, lacking harmony and
refinement, and even had several inlays of tastelessness \citep[80]{sauka_gedicht_1998}. On the one hand, this almost freed Maironis'  subsequent
editors from the need to collect the
poems that were scattered in periodicals and manuscripts, discarded or forgotten
by the author, which the compilers of the posthumous \emph{opera omnia}
often confront, and find one way or another to arrange them all into a comprehensive whole.\footnote{Only fifteen of the known poetic works did not
  make their way to \emph{Oeuvre} of \citeyear{maironis_maironio_1927}: nine verses written after
  the compilation of \emph{Oeuvre}, three excerpts of the narrative poem
  ``Mūsų vargai'' [Our Hardships] that were included in \emph{The
  Voices of Spring} of \citeyear{maironis_pavasario_1920} as separate pieces, and two poems that
  previously appeared only in the periodicals and one verse published in
  \emph{The Voices of Spring}.} On the other hand, this was offset by serious
complications for his editors, as he left a legacy of two alternative titles of his collected poetry \emph{The Voices of Spring} and \emph{Lyrics}, and kept altering the sequence of their individual poems.

To illustrate this problem, I must describe the process of
composing \emph{The Voices of Spring} at length. Despite the difficult
circumstances of publishing illegally and abroad,
we can already recognize the authorial arrangement of the sequence of poems in the first edition of the collection, which allows us to consider it as a structurally coherent work.
Historians of literature assert that Maironis formed thirteen
quasi-sections according to the thematic or genre affinity (hymns, sonnets,
satires; \citealt{speicyte_maironio_2019}; cf. \citealt{pokorska-iwaniuk_pavasario_2014}). This is merely an
interpretative observation, as the poet did not introduce chapter headings
(with the exception of the \emph{Sonnets} chapter that appeared in the \citeyear{maironis_pavasario_1913} and \citeyear{maironis_pavasario_1920}
editions); nor do the first and subsequent editions contain any
obvious graphic divisions. However, the grouping is evident and is
confirmed by the further development of \emph{The Voices of Spring}. The
fact remains that in three subsequent editions, when he added new poems to the collection,
Maironis inserted them in the existing groups, consistently following
the principle of thematic and genre affinity. With very few exceptions, he did not change the sequence of the already published poems and
quasi-sections, but rather expanded the latter every time. This means that the
principal structure of the collection of poetry  was stable. For a graphical representation, see Figure \ref{fig:subacius:composition}.

\begin{figure}
\centering
\includegraphics[height=.8\textheight]{media/subacius1a.png}
\includegraphics[height=.8\textheight]{media/subacius1b.png}
\includegraphics[height=.8\textheight]{media/subacius1c.png}
\includegraphics[height=.8\textheight]{media/subacius1d.png}
\includegraphics[height=.8\textheight]{media/subacius1e.png}
\caption{Composition of the collection \emph{The Voices of Spring}  \citep[133--41]{speicyte_maironio_2019} and \emph{Oeuvre}. In this representation, each rectangle, except for the white ones, represents a poem; their colours represent the  quasi-sections they belong to; and  black rectangles represent new poems in the \emph{Oeuvre}. From left to right, they represent the editions of \citeyear{maironis_pavasario_1895} (45 poems), \citeyear{maironis_pavasario_1905} (57 poems), \citeyear{maironis_pavasario_1913} (78 poems), \citeyear{maironis_pavasario_1920} (110 poems), and \citeyear{maironis_maironio_1927} (131 poems). This graphic shows how each of the quasi-sections grows over time, to be ultimately reshuffled in the 1927 edition.}
\label{fig:subacius:composition}
\end{figure}

While preparing a new edition for print, the poet would make revisions
to the already published poems in a personal copy of the previous edition. He
would write down additions, or attach them on a separate sheet, and
renumber the poems by hand, indicating their sequence for the
typesetter. There is a surviving publisher's copy of \emph{The Voices of
Spring} from \citeyear{maironis_pavasario_1920}, which the poet himself produced from a copy of the
\citeyear{maironis_pavasario_1913} edition ``with a pair of scissors and glue'' (\citealt{maironis_pavasario_1918}; see
Figure \ref{fig:subacius:maironis1918}), and a copy of \emph{The Voices of Spring} from \citeyear{maironis_pavasario_1920}, from
which Maironis prepared a rough copy for \emph{Oeuvre} \citealt{maironis_pavasario_1926}
in the same way.

\begin{figure}
\includegraphics[width=\textwidth]{media/subacius2.png}
\caption{Authorial additions and revisions in \citealt[pages 10--11]{maironis_pavasario_1918}.}
\label{fig:subacius:maironis1918}
\end{figure}

Maironis' \emph{Oeuvre} was supplemented with nineteen new poems; other works were
finally adapted to standard Lithuanian as already discussed. Moreover,
the author radically changed the former sequence of the poems by
disrupting most quasi-sections, thematic groups and the previous
framing composition of the beginning and end of the collection based on
patriotic accents (see Figure \ref{fig:subacius:composition}). The cyclical nature of \emph{The Voices of
Spring} should be realized not only as the thematic grouping of the
works. The opening poems of each group provide a certain key to the
reading of other works, and the series function as parables. For
example, ``Jo pirmoji meilė'' (``His first love'') that opens the
collections signals ``the love that the lyrical self feels for
Lithuania.`` By doing so, several further nature or love poems already
imply a national-patriotic statement. In \emph{Oeuvre}, the author
disrupted ``the cycle of poems with its allegorical structure and its
decoding technique'' \citep[69]{kessler_maironis_2014}, as presumably the key was no
longer necessary: the codes of reading the classic Maironis were
imposed by the expanding tradition of interpretation, school textbooks,
and broadcast songs with his lyrics. This new structure of
\emph{Oeuvre} was oriented to literary eternity, and its framing
composition consisted of works on universal and existential rather than
patriotic themes. 

As Peter Shillingsburg suggests, Hershel Parker made a
sceptical presumption in his \emph{Flawed Texts and Verbal Icons} (1983) that ``authors lose their authority over a work
after a certain period and that revision often not only violates the
creativity of the original effort but can end in confusion which might
make a text unreadable'' \citep[70]{shillingsburg_text_1991}. In the case of
Maironis, this idea would argue against the latest authorial versions
of the poems (in which the best versification was achieved). From the point of view
of  general composition, however, the collection of \emph{The Voices of
Spring} was more coherent than the volume of \emph{Oeuvre}. In any case,
it should be noted that from an editorial perspective the \emph{collection}
and \emph{Oeuvre} differ both in their use of textual versions, and in the
number and especially the arrangement of poems.

In the years of World War II (and soon
afterwards) the first eight posthumous editions of Maironis' poetry were published in Lithuania, and by war refugees and displaced persons in Germany (Meerbeck and Würzburg) under the title \emph{The Voices of Spring}. Three of these editions did not reach
their readership: two were suspended and destroyed by Soviet censors, and
one perished during the bombing of Weimar. Although each of these eight editions
are slightly different,\footnote{For example, in some of them several poems about the
struggle of Lithuanians with the Teutonic Order were removed in view of
Hitler's censorship.} all of them were compiled according to a model that was
introduced by a single editor: Juozas Ambrazevičius (1903--1974; known as ``Brazaitis'' from 1944 onwards). He took the versions of the poems from \emph{Oeuvre},
divided them into eight thematic or genre chapters, and even gave each chapter a
heading, unlike any collection published in Maironis' lifetime
\citep{maironis_pavasario_1942}. Although two groups of poems (i.e.  the genre groups
\emph{Satires} and \emph{Ballads}) can be detected both in \emph{The
Voices of Spring} from \citeyear{maironis_pavasario_1920} and in \emph{Oeuvre}, in the above-mentioned
editions of the 1940s type, the arrangement of poems within the groups
did not correspond either to the first or the second authentic sequence as
established by Maironis himself.

In the first edition of \emph{The Voices of Spring} that was published in Soviet
Lithuania \citep{maironis_pavasario_1947}, just like in all the other Soviet editions,
35 poems that contained religious motifs were discarded. However, in the
context of this paper, I would like to draw attention to the aspects of
the selection of versions and the structure of the collections rather
than to this blatant act of censorship. In this case, it were the editors who made the structural decision to arrange the poems chronologically, and to divide them into three
parts according to the periods of creative work. This way, they concealed their aim to
disrupt the original division into quasi-sections that enhanced the manifestative effect of
individual texts, and to present Maironis in the shape of historicized
publication of literary heritage. Bearing in mind the general atmosphere
of Stalin's regime and the ideological requirements that were imposed on
all literature, including that of the past, this could be interpreted as
an attempt to push a patriotic poet, albeit watered down, through
Communist censorship. The dates written at each poem insistently
reminded the readers that it was a thing from the bygone
pre-revolutionary era, which should not be regarded as a source of
relevant motifs and ideas. As the poet himself precisely dated only one
of his poems, this expansion of the peritext by dating all the poems made a historicizing (rather than
aestheticizing) impact on the reading practice.\footnote{In the other
  Soviet editions, dates were also added, in some cases in square
  brackets (\citealt{maironis_pavasario_1976}; \citealt{maironis_pavasario_1986}).} Moreover, for the lack of
specialized research and reliable bibliography, the sequence of poems
based on the first publications was imprecise with regard to the
chronology of both writing the poems and their publication.\footnote{For
  example, the poem ``Mergaitė'' [Girl] was erroneously dated to
  1893 \citep[21]{maironis_pavasario_1986}, as the editors \emph{de visu} did not check
  the reference found in bibliographies that it first appeared in  \emph{Lietuviszkas Kalendorius metams 1894 turintiems 365
  dienas}, Vilniuje: Spauda ir iszdas Jůzapo Zavadzkio
  {[}counterfactual, should be: Tilžė: J. Šenkės sp.{]}, 1893, p. 21. In
  fact, the anonymous verse (with  identical title) published there
  should not be related to Maironis, and the above-mentioned poem by the
  latter was first published two years later \citep[34--35]{maironis_pavasario_1895}.}
The versions of the texts were mainly taken from \emph{Oeuvre}, but
certain single variants from the \citeyear{maironis_pavasario_1905} and \citeyear{maironis_pavasario_1920} editions of \emph{The
Voices of Spring} were inserted rather eclectically: ``in some places the primary version of particular
stanzas or particular lines was restored'' \citep[222]{maironis_pavasario_1947}.

A similar eclectic approach was used by the editor of \emph{The Voices
of Spring} of \citeyear{maironis_pavasario_1913}that was published in the emigration milieu in Rome in 1952. Taking
the \mbox{\emph{Oeuvre}} version as the basis, Bernardas Brazdžionis
(1907--2002), a poet himself, inserted the variants of the \citeyear{maironis_pavasario_1920}
edition in some stanzas and explained, ``The restored old words or lines
are very `familiar' to us and have found a place in our hearts'' \citep[279--80]{brazdzionis_redaktoriaus_1952}. The phrase ``in our hearts'' was not just a
poetic expression. When preparing the edition, Brazdžionis reverted to a common practice of philologists living the camps of displaced persons (1945--1950), who endeavoured to restore Lithuanian literary textbooks and anthologies without the use of books, by merely counting on  what they knew by heart \citep[5]{naujokaitis_lietuviu_1948}.\footnote{I
  would like to thank Jurga Dzikaitė, who drew my attention to this
  fact.} Ironically enough, in the editor's commentaries Brazdžionis
denounced the above-mentioned Soviet edition of \citep{maironis_pavasario_1947}, alleging that in
this edition the poems ``were damaged, edited in a rather peculiar
manner {[}\ldots{}{]} of Soviet publications. {[}\ldots{}{]} This kind of
editing changes the shape and form of the collection'' \citep[294]{brazdzionis_redaktoriaus_1952}. In the Roman edition, eight chapters were introduced again,
albeit different ones, that were arranged in another way than those published in the years of World War II. In addition,  the sequence of the poems did not
coincide with the structure of any edition prepared by Maironis himself.

An even stranger composition (already the sixth type of composition) appeared in an
attempt to publish a semi-scholarly edition of Maironis' works, which
included a discussion of the variants \citep{maironis_pavasario_1982}. In this edition, the first 36
poems were arranged in the sequence taken from the \citeyear{maironis_pavasario_1895} edition of
\emph{The Voices of Spring},\footnote{This does not include the satires and texts with
religious motifs, a total of nine poems, which were censored out by the
Soviets.} and then followed the poems that Maironis inserted in the \citeyear{maironis_pavasario_1905}, \citeyear{maironis_pavasario_1913} and \citeyear{maironis_pavasario_1920} editions of \emph{The Voices of
Spring}. The \citeyear{maironis_maironio_1927} versions of \emph{Oeuvre} were given as the basic
text, and the earlier versions were quoted in a fragmentary manner, and discussed
in the commentaries.

Although a close analysis shows that the authorial sequence had its
unique logic, the subsequent history of publishing Maironis' poetry
testifies that the editors created four new structural models of the
collection. Due to the frequent reprints of \emph{The Voices of Spring}
with large print runs, a socio-cultural image of the collection was formed that was quite
different from the authorial image. As such, a contemporary
editor of Maironis faces several problematic alternatives, which can be
summed up in the following pattern:

\begin{enumerate}
\item When publishing the latest authorial version of the poems, they should
be arranged in the sequence of the \citeyear{maironis_maironio_1927} edition of \emph{Oeuvre}, even though this
does not correspond with the dominant, long-standing and meaningful title
\emph{The Voices of Spring}, and the readers of Maironis do not
recognize the genre title \emph{Lyrics}; moreover, the latter
(sub)title correlates with the volume of \emph{Oeuvre} and would be more
suitable for a many-volume edition of the Maironis corpus than for a
separate collection of poems.\footnote{In the \citeyear{maironis_pavasario_1952} edition,
  \emph{Lyrics} was added as a subtitle to \emph{The Voices of Spring}. This created a new combination of a title, subtitle, versions of poems
  and overall composition of the collection. In one of the more recent editions, this combination of a title, subtitle, and
  composition was repeated, but the versions of the poems are presented
  according to the \citeyear{maironis_maironio_1927} edition \citep{maironis_pavasario_2012}. This way, the variety of
  basic components of the textual and bibliographic code is increased
  once again.}

\item When publishing the collection of \emph{The Voices of Spring}, it would
be appropriate to choose the concept of authorial composition that
corresponds to this title (i.e. maintaining the sequence of the \citeyear{maironis_pavasario_1920} edition), but
that would not include nineteen later poems. Furthermore, this sequence has
an inner conflict with the latest authorial versions of the poems.

\item When publishing any earlier versions of the poems as the collection
\emph{The Voices of Spring} rather than versions of the \citeyear{maironis_maironio_1927}
\emph{Oeuvre} edition, the author's will would be ignored, the most
well-polished variants that have ``caught up'' with the language
modernization would have to be discarded, and a conflict with the
long-standing reception of separate poems would arise.
\end{enumerate}

\noindent These alternatives would be partly annihilated in a digital edition,
which would allow the reader to see and compare any authorial sequence
and corresponding versions of the poems, as well as their historical
development. However, the transformation of the collection
\emph{The Voices of Spring} into \emph{Oeuvre} that plays a different
socio-cultural role and has a different bibliographical and linguistic
code, performed by the poet himself, is fundamental. Thus it is an
inconceivable task to find a model of an edition that minimises the
editorial co-authorship that would integrate both the compositional
concept of \emph{The Voices of Spring} and the latest versions of the
texts, i.e. the versions of \emph{Oeuvre}.

Another editorial challenge arises from one more editorial perspective that could consider the
authorial use of his poetry as a means of revenge. In order to pursue
this curious idea, I should first discuss the bibliographical code of
\emph{The Voices of Spring} in greater detail.\footnote{This description is necessary because so far no bibliographic
  works have been published that analyse (or at least described in detail)
  the bibliographic code of Maironis' books. There is only one recent article that
  looks at the design of \emph{Voices of Spring}, but does this for the most part in the
  context of the Lithuanian imagery of spring \citep{jankeviciute_maironis_2019}.} The
first publication of the 1895 collection had a rather modest printing
quality and a small size, which was understandable because of its
clandestine nature \citep{maironis_pavasario_1895}. Nevertheless, the edition contained ornate initial letters and
several vignettes from the printing house's standard kit. The
publication of \citeyear{maironis_pavasario_1905} boasted even more opulent drawn initials and
slightly better paper, but in this case an original visual solution was
not offered either. A coloured carton-paper cover with the author's
portrait (by the famous artist Antanas Žmuidzinavičius, 1876--1966) and
his two other photographs from different periods in inserted plates
allow us to interpret the third edition of \emph{The Voices of Spring}
as giving particular emphasis on Maironis' person. More abundant
vignettes were not created especially for this book, but they are larger
and more complex, and the layout is more spacious. Yet, the first three
editions of the poetry collection are nothing out of the ordinary in the
stylistic and technological context of the design of printed production
of the given period and specific printing houses.

The \citeyear{maironis_pavasario_1920} edition of \emph{The Voices of Spring} has a lot of marked
differences from the earlier editions. The
extraordinary nature of this edition is enhanced by two historical circumstances. Firstly,
unlike in earlier cases, Maironis himself chose and commissioned its
visual materials --- illustrations and photographs --- and took decisions
regarding their arrangement (materials prepared by the author preserved
in the publisher's copy; see \citealt{maironis_pavasario_1918}). Secondly, Maironis'
efforts did not go unnoticed --- the book's design created a stir in the
cultural and religious circles of that time, was criticized in published
reviews, and caused private gossip both for aesthetic and moral
reasons.

What was it that public opinion was so critical about? A \emph{very
colourful} cover blended \emph{art nouveau} elements with
neo-romanticist \emph{country} sugariness (see Figure \ref{fig:subacius:maironis1920cover}). The author of the
cover design and some of the illustrations was an amateur artist Kazys
Šimonis (1887--1978), whose folksy decorations and forthright
symbolism was quite to the taste of the first-generation urban residents
of the young national republic. Added to this, elements of Šimonis's graphic art, details of Raphael's
paintings, primitive national ornaments, \emph{art nouveau}-style
ladies, flowers from greeting cards and photographs, were all mingled
together on the book's pages. Moreover, almost every page had a
different layout (see Figure \ref{fig:subacius:maironis1920spread}), and some copies were printed on better
quality paper with dark green instead of black ink.

\begin{figure}
\centering
\includegraphics[width=.9\textwidth]{media/subacius3.png}
\caption{\citealt{maironis_pavasario_1920}, cover by Kazys Šimonis}
\label{fig:subacius:maironis1920cover}
\end{figure}

\begin{figure}
\centering
\includegraphics[width=.9\textwidth]{media/subacius4.jpg}
\caption{\citealt[18--19]{maironis_pavasario_1920}}
\label{fig:subacius:maironis1920spread}
\end{figure}

``Neither spit nor swallow. There hasn't been a single book in the
\mbox{Lithuanian} language so lavishly illustrated, so pretentious and so
clumsily published'' \citep[120]{sruoga_knygoms_1920}. In his review, the famous
Lithuanian writer and theatre figure Balys Sruoga (1896--1947) resented
the visual cacophony of the edition. Setting aside all the other
oddities of illustrations, I would note that the poem ``Ant Neapolio užtakos''
[In the Bay of Naples], in which Vesuvius is mentioned, was
illustrated with a photograph of a landscape of the Lithuanian plains. 
Why do I assume that by choosing the superfluous, heterogeneous and kitschy
design, the author was taking revenge? And on whom? In 1910, Maironis bought a
late Baroque mansion in the very centre of Kaunas, City Hall Square. At
the poet's request, the interior decoration of the first-floor rooms was
made by an archaeologist, public figure and creator of symbols of the
national state Tadas Daugirdas (1852--1919). On the one hand, the
interiors of Maironis' house, which was much frequented by guests, became a model
of national-style decoration for Lithuanian city dwellers. On the other,
artists who had studied in the West and intellectuals of the younger
generation made ironic remarks about the eclectic \emph{décor} --- a mishmash of attributes of
noble and peasant culture and pompous neo-romanticist paintings.

Around the same time, the poet's works were increasingly more often
termed as old-fashioned. ``Goodnight, Maironis!'' \citep{smulkstys-paparonis_maironio_1920} --- these words summarized a review of an epic work by Maironis
that came out in the same period. Likely, by the bibliographic code of
the 1920 edition of \emph{The Voices of Spring}, the author seemed to
declare that while people might make fun of his work and lifestyle, it was still he who set the trends in Lithuania, and he who would decide what is
beautiful. And that just like his poems, the images he had chosen would pave the way for a canon of national aesthetics. As an indirect confirmation of this
assumption, I would like to refer to a photo of Maironis' house that
was included in the \citeyear{maironis_pavasario_1920} edition of \emph{The Voices of Spring}.

Furthermore, there is an even spicier feature of this book. Maironis dedicated
several of his poems to women with whom he had close contacts in
different periods of his life. In the \citeyear{maironis_pavasario_1920} edition of \emph{The Voices
of Spring}, two more dedications of this kind appeared. Admittedly, all these dedications were marked with initials only, or else the names were arranged in
acrostics. Contrary to this camouflage, however,  this edition contained five women's photographs along with landscape
photographs. Two of
them can be considered mere illustrations of the national costume
accompanying patriotic poems. Yet the other women are easily
recognizable --- it was to them that the poems were dedicated, and these
photographs have survived in Maironis' private archive with ambiguous
inscriptions. Moreover, a portrait of one girl is set in a vignette
representing a lyre, thus very straightforwardly implying that she was
the poet's muse (\citealt[94]{maironis_pavasario_1920}; see Figure \ref{fig:subacius:apolonija}). It is appropriate to
recall that Maironis was a Catholic priest, to whom celibacy was
obligatory. These illustrations provided an even more fertile ground for
the rumours that, out of keeping with his class, Maironis had been having
intimate relations with several women and was secretly taking care of
an illegitimate child.

\begin{figure}
\centering
\includegraphics[height=.33\textheight]{media/subacius5.png}
\includegraphics[height=.33\textheight]{media/subacius6.png}
\caption{Apolonija Petkaitė in \citealt[94]{maironis_pavasario_1920} (left); and in Maironis’s private album, Maironis Lithuanian Literature Museum, No. 2264a (right).}
\label{fig:subacius:apolonija}
\end{figure}

One should bear in mind that although Maironis was 58 at the time, Lithuania
was a very conservative society in the early twentieth century when it comes to social customs, and was rather intolerant of romantic affairs ---
even if they had taken place in the past. So, why did Maironis dare to make
this provocation? It would be respectable to tell a noble literary tale
of the brave poet's rebellion against the hypocrisy of Victorian morals
and a modern individual's declaration of creative and personal freedom.
Unfortunately, the reality was most probably much more banal. I would argue that Maironis decided
to include photos of real women-muses in the edition after his
smouldering hope to be nominated as a bishop had been
shattered.\footnote{Maironis was particularly hurt by the fact that two
  prominent nineteenth-century Lithuanian priest writers (Motiejus
  Valančius and Antanas Baranauskas) had become bishops, just like his
  younger colleagues, professors at the Theological Academy, the
  now-Blessed Jurgis Matulaitis and Juozapas Skvireckas, while his own
  candidacy was rejected several times.} Being a prelate and the most
famous Lithuanian writer alive, he was not afraid of losing his social
status by choosing these illustrations, and could at
least in this way take his revenge on the church dignitaries that had disappointed him.

Still, in \emph{Oeuvre} (the design of which was also ordered by the author)
Maironis removed both the dedications to his muses, their portraits, and
illustrations in general. By the writer's decision, the colourful
dynamics of authorial manipulation of the editions of \emph{The Voices
of Spring} was weighed down by the classical \emph{Oeuvre} of restrained
appearance and temperate composition, which established the image of
Maironis' solidity. Ironically, with this turn of circumstances, it was
not until recently that the genesis of Maironis' verses and the
development of the structure of the collection have been given
consistent research attention.

Keeping three main things in mind, that is: 1) the poetic elaboration of the text; 2) the
continuous restructuring of the composition of his collections of poetry; and 3) the solutions
concerning bibliographical code,  Maironis' practice of editing his own works was radically
situational. Therefore, any traditional edition may, in the best case,
roughly convey only one of those textual constellations. The effort and
ability of a poet to change creatively is as important evidence of his
talent as the final elaborated version of his poems. Thus, a digital
archive that enables a user to encounter every different stage of the
development of \emph{The Voices of Spring} (be that edition, or manuscript) as a
unique entirety, that presents a genetic sequence of variants would no longer merely present Maironis' poetics as fragmented, and deconstructed. Instead, the author would appear as more
appealing to any contemporary reader, and the edition would be representative of
the \emph{peripeteias} of the creation of modern culture. The digital archive and genetic edition of \emph{The
Voices of Spring} is already prepared and will be publicly available
after this article is published.\footnote{To be hosted online at:
  \url{http://www.pb.flf.vu.lt/}.} The case of Maironis, in its turn,
testifies that the initially perceived stability and continuity of literature may also function in another way, that
is, by situational adaptation. In this sense, the sequence of former chameleonic
transformations of \emph{The Voices of Spring} extends when they are transposed
to the digital medium.

\begin{flushleft}
\bibliography{references/subacius}  
\end{flushleft}

\end{paper}