\title{Editors' Preface}
\author{}
\begin{preface}
\maketitle
\section*{}
\label{preface}

\section*{}
\vskip 1em
The editorial team of \emph{Variants: The Journal of the European Society for Textual Scholarship}, is proud to finally present you with its double Issue (\thevolume) titled ``Textual Scholarship in the Twenty-First Century''. 
The title of this Issue was taken from the sixteenth annual ESTS conference of the same name, which  took place in Málaga, Spain on 28--29 November 2019 where it was hosted by the Department of English, French and German Philology at the University of Málaga.

In good ESTS tradition, the Málaga conference brought together specialists in the theory and practice of textual scholarship from a wide range of countries, including Austria, Belgium, Finland, Germany, Hungary, Italy, Lithuania, the Netherlands, Norway, Poland, Portugal, Romania, Russia, Spain, Sweden, the United Kingdom, and Switzerland.
Its 43 papers were organized into 12 sessions, which were arranged according to topics related to the theme of the conference such as: methods and tools in textual scholarship; types of editing; scholarly editions; texts worth editing; digital editing; the editing of historical texts; textual scholarship in the digital humanities; textual scholarship and translation studies; and directions and challenges in editing. 
In addition to the papers presented by members of the ESTS, the conference included two plenary lectures: one by Professor Isabel de la Cruz Cabanillas (University of Alcalá) and another by Professor Jukka Tyrkkö (Linnaeus University and University of Turku). 
While de la Cruz Cabanillas delved into the topic of the editor’s role by focusing on editorial interventions (and their implications) in Middle English texts, Tyrkkö was concerned with annotation practices of visual and paratextual features from both manuscripts and early printed publications, approaching the topic from the perspectives of both textual scholarship and corpus linguistics.

A lot has happened since those stimulating days in the temperate southern Spanish winter. 
Only a few short weeks later, a group of Chinese scientists would identify a new strain of the coronavirus that would rapidly hold the world in its grasp --- and is still wreaking havoc even now, eighteen months later, as nations around the globe are rushing to vaccinate their citizens. 
It has been an especially trying time for all of us, as we become restricted in our movement and personal interactions, as we try to strike a healthy work-life balance, and as we combat illness and hope for the health and safety of those close to us.

In academia, as indeed in all other professional fields, these events have invariably caused a plethora of delays as people struggle to adapt and re-adapt their daily routines to a ``new normal'' that itself continuously keeps evolving until this day. 
Priorities have to change when the infrastructures we have come to rely on suddenly fail us, and we personally have to take care of our elderly, our sickly family members, or our children. 
That is, of course, when we are lucky enough not to get sick ourselves, or have to mourn our loved ones.
At the same time, infrastructures that we require for professional rather than personal reasons have in many cases become periodically inaccessible to us as well.

Indeed, as the world struggles to manage a global pandemic, it has become even more clear just how much our society relies on the affordances of (digital) technologies for the organisation of our personal and professional lives. Now more than ever, access to and proficiency with digital resources determine the extent to which individuals are able (or even allowed) to interact with each other, and to contribute to society. If the COVID-19 pandemic has taught our academic community anything, it is the inescapable fact that the resilience of our research relies upon the development, sustainability, interoperability, and conscientious criticism of our digital tools, technologies, and methodologies. 

These and other exceptional circumstances have also caused the publication of this Issue to be delayed in all facets of our publication pipeline: from submissions to reviewing, from revisions to proofreading, and eventually to publication itself.
As the editorial board of \emph{Variants} \thevolume, we apologize for these delays, which caused us to publish this Issue almost exactly a year later than originally planned. 
We thank our readers for their patience, but also all our contributors and other collaborators for their continued efforts and support in this difficult time.

The result of these efforts is a sizable and strong double issue that we believe will be the start of much further debate, inside and outside the \emph{Variants} publication venue. 
The Issue combines eight full-length essays, three shorter work-in-progress pieces, one review essay, and five traditional literature reviews. 
Besides an obvious joint interest in and relevance to the field of textual scholarship, these contributions share a number of common threads.

The Issue opens with a promising look at what textual scholarship in the twenty-first century may yet hold in store for us, as early career researcher Lamyk Bekius explores the possibilities of critically analysing born digital writing processes in her essay titled ``The Reconstruction of the Author's Movement Through the Text''. 
There, she takes a closer look at the intricately detailed output of a keystroke logger called InputLog that was used by Flemish author Gie Bogaert when he wrote his novel \emph{Roosevelt} (2016). InputLog tracks an author's movements through the text in the smallest level of detail, resulting in information that allows Bekius to map what she calls the ``nanogenesis'' of a literary work. The output of the InputLog software confronts researchers with new challenges: how should we approach the processing and analysis of these logs? To what extent can they be transformed and encoded into TEI-conformant XML? And, on a theoretical level, what constitutes a version? 
As Bekius' remarkable essay elucidates, the difficulty here lies not just in learning to read and understand InputLog's meticulous output, but also for a large part in visualizing the information and making it understandable for others --- a struggle that will ring true to scholarly editors everywhere.

The preliminary implications of these difficulties are investigated further in a Work in Progress contribution by Dirk Van Hulle (one of Bekius' PhD supervisors). Van Hulle suggests to visualize such nanogenetic analyses as ``Dynamic Facsimiles'': filmic renditions of logged keystrokes as animated transcription videos. 
Interested readers will be able to view examples of such dynamic facsimiles in the online version of Van Hulle's Work in Progress contribution on our Open Access journal's website.\footnote{See: \url{https://journals.openedition.org/variants/1450}.}
Van Hulle's own full length essay in this Issue stays in the realm of genetic criticism as he contemplates what happens when a literary author composes several works simultaneously. 
Drawing on similar phenomena that are treated in the disciplines of bibliography and the history of the book, Van Hulle coins the term ``Creative Concurrence'' to denote the cross-pollination that may occur when such writing processes start to inform one another.

From authorial writing processes we then move on to editorial interventions --- for better, or for worse. 
In ``Creators' Intentions and the Realities of Performance'', Ronald Broude provides us with a series of examples of typical problems that are posed by the editing of opera texts by focusing on Gilbert and Sullivan's Savoy Operas. Over time, the opera texts have been altered for various reasons, such as the preference and talents of the different artists, the incorporation of opera traditions, censorship, and updates to outdated material. Mapping the textual development of these operas, writes Broude, offers readers ``an unusual opportunity to study the dynamics of performing works over a substantial period''. 
Paulius V. Subačius goes on to examine the  challenges of editing a collection of poems where the author (in this case: the Lithuanian poet Maironis) keeps rearranging, recomposing, and republishing his canonical collection of poems. Again, these changes were influenced by social and cultural developments. There exist about a hundred different textual variants, from authorial revisions of the metre, language, and length to editorial revisions of structural organisation, visual material, and dedications.
And Dariusz Pachoki follows these inquiries into editorial difficulties by questioning whether the editor always knows best in an exploration of the quarrels between twentieth century Polish \emph{émigré} writers and the editors-in-chief of the only two literary journals that would publish them at the time, \emph{Kultura} and \emph{Wiadomości}. Although the journals, based outside of Poland, did provide islands of intellectual freedom for Polish writers at the time, their editors often made far-reaching changes to the submitted articles. 

The Issue then moves on to two essays where the authors use stemmatological analyses to expose the textual history of the poetic texts they study, propose scholarly edited versions of the poems in question, and consider the new implications their findings may have for our understanding of the texts. 
First, Anthony Lappin offers a thorough treatment of John Donne's poem ``Wilt thou forgive...''. In addition to the transmission history, Lappin's contribution includes a poetic analysis of the text. 
He shows that copyists slowly but surely ``de-Donnified'' it: over time, the poem in which Donne questioned faith was transformed into a religious hymn that fitted better with the author-image of Donne as  a pious sermonizer. 
Secondly, Mark Bland investigates lingering questions of authorship attributions with regard to three seventeenth century answer poems. 
This genre of poems invited engagement or parody through their style and topic. 
Since answer poems often incorporated the original poem or imitated its style, it can be quite challenging to identify their authors. Using stemmatological methods, Bland is able to provide insight into the texts' transmission history and, in doing so, also sheds new light on the networks of manuscript circulation at the time.
And for the final essay in this Issue, we return to the digital world we started from, as Anne Baillot and Anna Busch use their contribution titled ``Editing for Man and Machine'' to explore the affordances of digital scholarly editions. The accessibility of digital scholarly editions is a long-standing topic of discourse but, Baillot and Busch argue, anticipating multiple user scenarios can contribute to an edition's durability. With examples from the edition project ``Briefe und Texte aus dem intellektuellen Berlin 1800–1830'', they illustrate new ways to make the content (textual transcriptions, facsimiles, metadata) of a digital edition accessible for human and algorithmic audiences alike.

After this rich collection of essays, we open the Work in Progress section of the Issue. This section traditionally houses shorter research narratives that are less formal, more practice oriented, and allow researchers to report on more preliminary research findings. 
Besides Van Hulle's aforementioned discussion of digital facsimiles, this Issue's Work in Progress section includes two more pieces. 
In the first one, Hugo Maat proposes an experimental (and arguably less time-consuming) new way of translating historical source materials to improve their readability that he devised in the early stages of the development of a digital edition for the correspondence of William of Orange. 
Maat has created a method to come to a ``restricted translation'': a modern translation of grammatical elements that keeps the lexical elements. 
It raises the question: is it possible (or feasible) to make a source text more accessible without obscuring its historical and semantic character?

In the second essay in this section, Michelle Doran reflects on the influence of Web 2.0 technologies on the field of digital scholarly editing. In particular, she discusses whether Twitter can function as an extra access point into the digital edition of ``The Poems of Blathmac'', an Early Irish poetry text. Could it be a way for editors to interact with a wider audience, to show how the historical text can still be relevant? Her exploration of the current boundaries of digital editions invites a rethinking of what we consider a digital editorial product.

As we move on to the reviewing section of our \emph{Variants} journal, we first encounter a review essay by our review editor Stefano Rosignoli. 
This relatively new section in the journal was first introduced in \emph{Variants} 14, and provides reviewers with more space to offer a closer investigation into what usually comprizes a larger body of work. 
In this case, Rosignoli reviews three connected modules of the \emph{Beckett Digital Manuscript Project} (BDMP) --- a groundbreaking digital genetic edition of Samuel Beckett's works --- as well as their respective monographs in the project's \emph{Making of} series, where the editors offer a first interpretation of each work's writing process. 
Finally, the Issue closes with five traditional literature reviews. 
The first three offer appreciations of scholarly editions: one by Christian Baier on an edition of a work by Thomas Mann, one by Manuela Bertone on an edition of a work by Carlo Emilio Gadda, and one by Jonas Rosenbrück on a new volume in the so-called ``complete critical edition'' of Walter Benjamin's \emph{Werke und Nachlaß}.
The last two reviews tackle new theoretical works that have been published in the field: Barbara Cooke offers a review of a new collection of essays by Peter Shillingsburg, and Hans Walter Gabler comments on Paul Eggert's recent reflection on the intersection of scholarly editing and book history.  

As the Issue's title page reveals, the above collection of academic essays and reviews has been put together and edited by a new editorial board. 
Looking to pass the baton after serving as the journal's General Editor since 2013, \emph{Variants} veteran Wim van Mierlo recruited Elli Bleeker and Wout Dillen to serve as Associate Editors for \emph{Variants} 14 to learn the ropes and prepare to take the lead in publishing the journal's future issues. 
As newly appointed editors, we would both very much like to thank Wim for the many years of hard work that he has put into editing \emph{The Journal of the European Society for Textual Scholarship}, and for all the support he gave us in this transition. 
Not only did he shape six excellent Issues of \emph{Variants}, each meticulously edited and presented, Wim was also instrumental in the journal's move to a new, Open Access publication venue: OpenEdition Journals. 
Since \emph{Variants} 12--13 (2016), all new issues in our journal are now freely available to anyone with an internet connection --- which can only help propel the research in our field forwards.
We believe Wim's legacy speaks for itself, and are well aware that we have big shoes to fill as we follow in his footsteps. 

Thankfully, Wout and Elli had some help taking the first of these steps when they became the new General and Associate Editors of \emph{Variants}, respectively. 
For this Issue, Laura Esteban-Segura --- who organized the 2019 ESTS conference in Málaga --- is acting as a guest editor; and we are also very grateful to Stefano Rosignoli for continuing to take such good care of our journal's review section. 
To make it easier for all editors to collaborate on all aspects of the publication pipeline --- in this Issue as well as future ones --- we devised a new workflow that makes use of more open and collaborative technologies. 
Developing the current Issue provided us with the opportunity to reuse, double check, fine tune, and document our solutions for moving away from the proprietary software package Adobe InDesign\textsuperscript{\textregistered} and towards the open source alternative \LaTeX ~for typesetting the PDF versions of new \emph{Variants} issues. 

To achieve this goal, Wout designed a \LaTeX ~template called varian\TeX ~to resemble the typesetting style of previous issues as closely as possible. 
This template is made available on Overleaf,\footnote{\url{https://www.overleaf.com/latex/templates/variantex/dwjsvmwfcybk}} hosted in a public repository on GitHub,\footnote{\url{https://github.com/WoutDLN/varianTeX}} documented in a GitHub Wiki,\footnote{\url{https://github.com/WoutDLN/varianTeX/wiki}} and archived on Zenodo\footnote{\url{https://doi.org/10.5281/zenodo.3484651}} to ensure the sustainability and reusability of our efforts. 
What is more, from this Issue onwards references to cited works are no longer painstakingly manually formatted, but instead use a newly designed citation style\footnote{\url{https://raw.githubusercontent.com/WoutDLN/varianTeX/main/variantex.bst}} to automatically generate bibliographies from \textsc{Bib}\negthinspace\TeX ~records --- which are now also available in a public Zotero Group Library.\footnote{\url{https://www.zotero.org/groups/2517977/variantsjournal}} 
And finally, Wout has also developed a simplified version of varian\TeX ~for authors (including a similar Overleaf template,\footnote{\url{https://www.overleaf.com/latex/templates/variantex-for-authors/znsqffgrvshv}} GitHub repository,\footnote{\url{https://github.com/WoutDLN/varianTeX-for-authors}} Wiki documentation,\footnote{\url{https://github.com/WoutDLN/varianTeX-for-authors/wiki}} and Zenodo DOI).\footnote{\url{https://doi.org/10.5281/zenodo.4013870}} 

We hope that this will encourage future contributors to \emph{Variants} to submit their articles in \LaTeX, and/or their references in \textsc{Bib}\negthinspace\TeX, because it is through working together and distributing the responsibilities of formatting contributions that we can speed up the journal's turnaround and publish more, qualitative content in issues to come. 
While redesigning this workflow has been a great effort that has brought along its own series of publication delays, we strongly believe that it was a worthwhile endeavour that will save us time in the long run, and that helps us find a more accessible and reusable way of publishing academic research, that may also benefit others. 
Indeed, we are happy to see that the transparency of our approach has already paid off to some extent, as varian\TeX ~is currently being used to carry out the typesetting of another journal in the field: the \emph{DH Benelux Journal}.\footnote{\url{https://journal.dhbenelux.org}}

\label{nword:preface:start}Of course, as the editors of \emph{Variants}, we are not only responsible for the formatting of new issues, but for their contents as well. 
And this also sometimes means that we have to make hard decisions. 
In the present Issue, for example, we needed to decide what to do with racial slurs when they were brought to our attention by one of our reviewers --- for which we are grateful.
In ``Creators' Intentions and the Realities of Performance'', Ronald Broude details the complicated history of the editorial tradition of Gilbert and Sullivan's Savoy Operas.
And part of this history is that at some point in time, changes were made to the original text to avoid the use of racial slurs --- in this case, specifically, the n-word. 

In his essay, Broude compares the old versions with the new ones, to contemplate the context of the changes and the implications they have on the text's transmission. 
Broude's own position in this debate is clear from his essay: that a scholarly editor of a work should not attempt to rewrite history by distorting the original version of the text --- however unpleasant or offensive that version may be, or however far removed from the editor's own social or political views. 
But while this argument can certainly be made for a historical critical edition of a given work, the context of reporting on the research behind the construction of such a text in an academic journal is quite different still, and deserves its own careful consideration.

This is a delicate matter indeed, and as two white people in positions of privilege, who have no first-hand experience of being targeted by such racial slurs, we felt it was important to listen to those who do before making a decision. We sought advice from colleagues who are people of colour, to ensure that we examined the issue from all angles. After listening to their experiences and receiving their advice, we decided to avoid the publication of racial slurs altogether, and acknowledge that there are indeed very few circumstances (if any) where the printing of such words could be considered necessary to convey their meaning. 

Generally speaking, we can treat appearances of racial slurs in citations the same way as we do grammatical inconsistencies or anonymized entities --- i.e. by replacing them with conforming alternatives in [square brackets].
This way, the author can make their argument just as well without printing the original quote in full. 
What is more, because there is a difference between retaining the original phrasing in an official edition of a specific text on the one hand, and retaining it in an academic essay on that edition on the other, we believe that our redaction of Broude's text does not have to contradict his original point. 
As long as the relevant edition is properly referenced so that people may have the opportunity to read the original phrasing if they wish, there is no need to duplicate it in the essay (especially when the original can be derived so easily from context as in this case). 

For these reasons, we decided to use our editorial prerogative to redact the use of the n-word in Broude's essay to conform to our new code of conduct. 
Of course, all of this transpired with the consent of (and, we hasten to add, without any resistance from) the author.
Aware of the fact that this is likely a first in the history of \emph{Variants}, and that some of our readers may find  this practice contradictory to their own sensibilities as scholarly editors, we wanted to not only point out our decision in an editorial footnote attached to the essay in question, but also to explain our reasoning in this Editors' Preface. 
This is not a decision that we have taken lightly, and by being transparent about the process of our decision making we try to become better at dealing with these issues, and to make sure that we are as inclusive and respectful as possible of the needs of all our readers.
\label{nword:preface:stop}

As we close this Editors' Preface, we end with the hope that you will enjoy the contributions in this Issue! 
We have certainly enjoyed putting it together for you, and look forward to receiving more of your research for consideration in future issues. 
We also anticipate the opportnunity to see you all again soon (preferably finally in person again!) possibly at our next annual meeting: ESTS 2022 in Oxford!

\begin{flushright}
\emph{
Wout Dillen, General Editor\\
Elli Bleeker, Associate Editor\\
Laura Esteban-Segura, Guest Editor\\
Stefano Rosignoli, Review Editor
}
\end{flushright}

\end{preface}