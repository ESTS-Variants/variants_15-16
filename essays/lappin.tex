\contributor{Anthony Lappin}
\contribution{From \emph{Christ the Saviour} to \emph{God the Father}: Adjustments to Forgiveness in Donne’s Short Poem, ``Wilt thou forgive\dots{}''}
\shortcontributor{Anthony Lappin}
\shortcontribution{From \emph{Christ the Saviour} to \emph{God the Father}}

\begin{paper}

\defcitealias{celm_cor_540}{\emph{CELM}~CoR~540} 
\defcitealias{celm_cwt_236}{\emph{CELM}~CwT~236} 
\defcitealias{celm_cwt_250}{\emph{CELM}~CwT~250}
\defcitealias{celm_dnj_1584}{\emph{CELM}~DnJ~1584}
\defcitealias{celm_dnj_1578}{\emph{CELM}~DnJ~1578}
\defcitealias{celm_dnj_1577}{\emph{CELM}~DnJ~1577}    

\hyphenation{tende-bant-que}

\begin{abstract}
Donne’s short religious poem, which begins ``Wilt thou forgive\ldots{}'' has undergone numerous changes in its journey through the hands of copyists and into print.  The establishment of a stemma and its analysis shows an early, theologically difficult and questioning poem which was progressively altered by different hands, a process which turned the short verses into a devout affirmation of faith, something quite contrary to Donne’s initial conception of the poem.
\end{abstract}

\section{Introduction} 

\textsc{John Donne (1572--1631) trained as a lawyer,} and made something of a
name for himself as a coterie poet through well-turned erotic and
satyrical verse (Mariotti 1986). He had renounced his familial
allegiance to Catholicism following his recusant brother's death in
prison, and looked set on a diplomatic career, beginning first in the
service of Lord Egerton, keeper of the king's seal. Yet Donne's secret
marriage in 1601 to a teenage heiress, Anne More, destroyed his
prospects, resulted in her disinheritance, and plunged them into
poverty; Anne died during her twelfth lying-in in the summer of
1617. Despite having shown little interest in religious matters other
than as an anti-Catholic polemicist, Donne had been forced into the
clergy two years earlier and had taken up the only means of preferment
then open to him, as one of the royal chaplains, at the insistence of
James VI and I (1566--1625), who, one must assume, thought he had more
than enough lawyers but a dearth of entertaining clergymen.

Donne's early secular poetic expression and his later assumption of a
religious garb --- he was made dean of St Paul's in 1621 and became
something of a superstar preacher --- has caused a certain degree of
difficulty for editors, readers and, as we shall see, copyists. In this
article, then, I shall consider how assumptions regarding the manner in
which religious verse should express itself, and how in particular the
dean of St Paul's should have expressed himself, influenced the
transmission of one particular poem which, quite self-consciously,
escaped from the norms of devotional poetry. Donne preferred most of his
writing --- and particularly his poetry --- to circulate in manuscript.
This, of course, granted a degree of control over whom it might reach,
and in what form it might reach them. Yet it also allowed a certain
\emph{variance} to creep in to the transmission of the poems, through
scribal variants or errors, and scribal rewritings or recomposition.
Such interference on the part of a scribe might remain independent of
other variant versions, or might, in contrast, be designed to resonate
with or correct other incarnations of the poem.

A perfectly licit engagement with textual \emph{variance} is to inquire
after what text (be that hypothetical, arrived at by deducation, or real,
present in a surviving manuscript) might be closest to the original
version --- the one, so to speak, issued by the poet, and therefore the one
most fully expressing their intentions; and, as a corollary of this
activity, to establish whether any variants found in the manuscript
tradition might come from a revision by the author of their original
text. Of course, if this is one's focus, anything extra-authorial, or
even authorial elements that are revised away, should be set aside, as
good for little other than to fill an apparatus whose presence is as
inexplicable as its contents are unread. In this article, however, I
shall use the numerous variants to discuss the evolution of a poem in
copyists' hands, and focus on the variant understandings of a poem ---
variant \emph{instantiations} of a poem --- which were generated in different,
yet still roughly contemporary, contexts. The poem in question has been
usually referred to as ``A Hymne to God the Father'', following the
title provided in the first edition of Donne's poetry, which came to be
inscribed in the Stationers' Register a year and a half after his death
as ``a booke of verses and Poems {[}\ldots{}{]} written by Doctor John
Dunn'' (13 September 1632: \citealt[IV:249]{arber_transcript_1877}). It was published
some six months later; the ``Hymne'' appeared as the final poem of
Donne's \emph{oeuvre} \citep[350]{donne_poems_1633}. The poem maintained this
position, as the final poem of the final section --- that of religious
verse --- in the revised second edition \citep[350]{donne_poems_1635}, and all
subsequent seventeenth-century editions (\citeyear[388]{donne_poems_1639}; \citeyear[368]{donne_poems_1649,donne_poems_1650,donne_poems_1654}). It was this printed text which, with some small alterations,
passed over into Izaac Walton's \emph{Life} of Donne, and so became very
much the textus receptus for those who wished to sample the poetic lyre
of the Anglican divine, as Walton was pleased to present him (see \citealt{martin_izaak_2003}; \citealt{novarr_making_1958}; \citealt[116--17]{smith_john_1983}; further, \citealt{lambert_du_2012}; \citealt{cottegnies_autour_1999}; \citealt[11--13]{haskin_john_2007}).

By the twentieth century, though, some comparison with the variant
readings present in manuscripts had taken place, and the textus receptus
gave way to editorial conflations of disparate readings \citep[see][nn. 6--7, at 26--27]{pebworth_editor_1987}. This eclecticism has been eschewed by the
poem's most recent editor, Robin Robbins (\citeyear[654--56]{robbins_complete_2013}), who adopts
the text as found in Trinity College Dublin, ms. 877, fol. 135r
{[}DT1{]},\footnote{When citing manuscripts, I also provide the
  \emph{sigla} used in the \emph{Donne Variorum} project (1995--),
  provided in their ``Sigla for Textual Sources''.} an election
justified by the observation that the manuscript represents the only
witness ``free of inferior readings''. As an ecdotic definition, it is
rather wanting, and a side effect of the first part of the present
article will be a partial contribution to evaluating this form of
editorial connoisseurship.

The article will develop in the following stages: I shall consider
previous discussion of the poem's textual evolution. I shall then
progress to widen my interrogation of the transmission by using variants
not previously considered, and will use this variation to reconstruct a
hypothetical stemma for the poem. The purpose of this is not solely to
identify those copies as close to the archetype as possible, but also to
gain a comprehensive vision of the overall \emph{variance} of the poem
at the hands of scribes and printers. This overview will then be used to
provide a literary analysis of the poem and its variants. I will attempt
to reconstruct how the poem has been re-interpreted consistently through
its copying history, in relation to various cultural forces.

\section{The First Moments of Evolution: Pebworth 1987}

Over three decades ago, Ted-Larry Pebworth provided a subtle analysis of
the development of the text, tracing its evolution via three significant
stages. ``Text 1'', in his view the earliest, corresponded to the
well-established sub-set of Donne manuscripts, known as Group III;
``Text 2'' to manuscript Group II; and ``Text 3'' to the printed text of
1633. Although I will argue that the manuscript transmission of this
poem is much more complex than Pebworth's analysis allows, and that
variants in other witnesses should be taken much more seriously, it is
important to emphasize the careful work that Pebworth carried out and
upon which depends my own unpicking of an inevitably only
partially-preserved web of transmission.

\subsection{Text 1}

Pebworth's identification of the stages of transmission are
distinguished by, amongst other things, their titles. ``Christo
Saluatori'' {[}To Christ the Saviour{]} is the title born by, in
Pebworth's careful words, ``a fairly accurate verbal record of the
earliest version of the poem'' \citeyearpar[23]{pebworth_editor_1987}, and is represented by four
manuscripts, two dating to the 1620s:


\begin{table}[H]
    \centering\small
    \begin{tabular}{lp{.75\textwidth}}
         B29 & London, British Library, Harleian ms. 3910, fol. 50r \\
         B46 & London, British Library, Stowe ms. 961, fol. 109r
    \end{tabular}
    \caption*{}
\end{table}

\noindent and two to the early 1630s:

\begin{table}[H]
    \centering\small
    \begin{tabular}{lp{.75\textwidth}}
    H6 & Harvard ms. Eng. 966.5, fol. 15v \\
    C10 & Cambridge, University Library, Narcissus Luttrell ms., fol. 93v.
    \end{tabular}
    \caption*{}
\end{table}

\vfill

\subsection{Text 2}
The second, entitled ``To Christ'', is described as a ``grammatical and
aesthetic `tidying up' by either the poet or a copyist'' \citeyearpar[21 and 23]{pebworth_editor_1987}. The attribution is regrettably imprecise, but Pebworth certainly
intended to leave open the possibility that Donne returned to his poem,
and carried out a number of changes to improve it.

The manuscripts concerned are, in the main, from the mid-1620s:

\begin{table}[H]
    \centering\small
    \begin{tabular}{lp{.75\textwidth}}
    B7 & London, British Library, Add. ms. 18647, fol. 91v \\
    CT1 & Cambridge, Trinity College Cambridge, ms R. 3. 12, p. 200 \\
    DT1 & Trinity College Dublin ms. 877, fol. 135r \\
    H4 & Harvard ms. Eng. 966.3, fol. 113r \\
    H5 & Harvard ms. Eng 966.4, fol. 94v
    \end{tabular}
\end{table}

\noindent Nevertheless, even with the evidence that Pebworth marshalls, we must
observe that the process by which ``Text 2'' was reached took place in
two or three moments: a change in title to the vernacular, but
maintenance of the readings of the earlier ``Text 1'' in just one
manuscript, H5, which was then followed by a number of significant
changes in the subsequent manuscripts (H4, DT1; B7, CT1).

We may thus present a representation of something close to the original
text together with variants brought about through the progressive
evolution of the poem: I have adopted a modernized spelling to avoid
insignificant orthographical variation.\footnote{The mis-en-page is
  taken from the majority of witnesses to ``Text 1'': B29, B46, H6. In
  contrast, C10 indents the second, fourth, and fifth lines, but
  maintains the greater indentation of the sixth. In the later
  manuscripts, it is even lines (i.e.: 2, 4, 8, 10, 14, 16), which are
  indented, with a greater indentation for ll. 6, 12, and 18, as they
  end each stanza.} Superscript words indicate the later variants of
the underlined words preceding them.

\vfill
\subsection{A combined edition of the text}

\begin{figure}[H]
    \centering\small
    \begin{tabular}{lc}
    \underline{\textbf{Christo Saluatori}} \textbf{\textsuperscript{To Christ}} \\ \\
    
    Wilt thou forgive that sin where I begun? & \\
    Which is my sin though it \underline{was} \textsuperscript{were} done before, & \\
    Wilt thou forgiue those sins through which I run & \\
    And do them still, though still I do deplore & \\
    When thou hast done, \underline{I have} \textsuperscript{thou} \textsuperscript{hast} not done, & 5 \\
    \-\hspace{2cm}For I have more. & \\ \\
    
    Wilt thou forgiue that sin by which \underline{I} \textsuperscript{I have} won & \\
    Others to sinn and made my sinn their dore & \\
    Wilt thou forgiue that sinn which I did shun & \\
    A year to two, but wallowed in a score &10 \\
    When thou hast \underline{done} \emph{\textsuperscript{done that}}, thou hast not 
    done & \\
    \-\hspace{2cm}for I have more. & \\ \\
    
    I have a sin of feare that when I haue spun & \\
    My last thred I shall perish on the shore & \\
    Sweare by thy selfe that at my death \underline{thy} \textsuperscript{this} sun & 15 \\
    shall shine as it shines now as heretofore & \\
    And having done that, thou hast done & \\ 
    \-\hspace{2cm}I have no more.
    \end{tabular}
    \caption*{}
    \label{fig:lappin:christo}
\end{figure}


The transmission may be presented in tabular form (Table \ref{tab:lappin:1}).

\begin{table}[H]
    \centering\scriptsize\renewcommand{\arraystretch}{1.5}
    \caption{List of variant readings, Texts 1–2}
    \label{tab:lappin:1}
    \begin{tabular}{c|c|c|c|c}
    \toprule
    \emph{line} & \emph{mss.} & \emph{reading} & \emph{variant} &
    \emph{mss.}\\
    \midrule
    \emph{Tit.} & B29 B46 H6 C10 & Christo salvatori & To Christ & H5 H4 DT1
    B7 CT1\tabularnewline
    2 & B29 B46 H6 C10 H5 & was & were & H4 DT1 B7 CT1\tabularnewline
    5 & B29 B46 H6 C10 H5 & I have & thou hast & H4 DT1 B7
    CT1\tabularnewline
    7 & B29 B46 H6 C10 H5 & I & I have & H4 DT1 B7 CT1\tabularnewline
    11 & B29 B46 H6 C10 H5 H4 DT1 & done & done that & B7 CT1\tabularnewline
    15 & B29 B46 H6 C10 H5 & thy & this & H4 DT1 B7 CT1\tabularnewline
    st. II--III & B29 B46 H6 C10 H5 H4 DT1 & II-III & III--II & B7
    CT1\tabularnewline
    subscr. & B29 B46 H4 DT1 & finis & \emph{om}. & H6 C10 H5 B7
    CT1\\
    \bottomrule
    \end{tabular}
\end{table}

\noindent The course of transmission, and the relations of the witnesses, is
relatively evident, with the only uncertainty being generated by the
single-word subscription or colophon, ``finis''. The subscription is
present in both B29 and B46; yet, from the same ``Christo salvatori''
group, it is missing from H6 and C10. Regrettably, it is not possible to
say therefore whether it was contained in the archetype. Any conclusion
would be wholly tentative, since we should allow for a degree of scribal
latitude in choosing or not to omit something which is not part of the
poem proper, but simply a boundary-marker to indicate that the poem had,
in fact, already finished. We have something of a Schrödinger's
archetype, then, which requires some further determination we have not
yet achieved (an autograph, further evidence for Donne's own
subscriptions for poems, for example). \emph{Finis} is dropped by the
first representative of ``To Christ'', H6, but it is preserved by H4 and
DT1. Thus we may say with confidence that the intermediary sub-archetype
which changed the title still kept \emph{finis}, and it was
spontaneously voided by the scribe of H6. This may be transformed into a sequence of emendations:

\begin{table}[H]
    \centering\scriptsize\renewcommand{\arraystretch}{1.5}
    \caption{Sequence of variation, Texts 1--2}
    \label{tab:lappin:2}
    \begin{tabular}{c|p{.2\textwidth}|c|p{.2\textwidth}|p{.2\textwidth}}
    \toprule
    \emph{stage} & \emph{definition} & \emph{witnesses} & \emph{variant} &
    \emph{further development} \\
    \midrule
    χ & text of the archetype & B29 B46 & & \tabularnewline
    \hline 00 & & H6 C10 & \emph{om.} finis / \emph{add.} finis B29 B46
    &\tabularnewline
    \hline 01 & change of title & H5 & To Christ & \emph{om.} finis\tabularnewline
    \hline 02 & variants to ll. 2, 5, 7, 15 & H4 DT1 B7 CT1 & were; \emph{add}.
    hast; \emph{add}. have; this &\tabularnewline
    \hline 03 & l. 11, order of stanzas & B7 CT1 & \emph{add.} that; \emph{transp.}
    st. II--III & \emph{om.} finis\\
    \bottomrule
    \end{tabular}
\end{table}

\section{The Poem in Flux}

Given this result, it would be better to speak provisionally of a
succession of stages in transmission, rather than Pebworth's ``Text 1''
and ``Text 2''. These stages may be further refined by taking in a
larger spread of manuscripts and the first printed version. Some of
these witnesses Pebworth dismissed as being ``memorial constructs'', as
they were present in commonplace books or multiple-author collections
and they escaped from the much tidier pattern he had established. These
witnesses are:

\begin{table}[H]
    \centering\small
    \begin{tabular}{lp{.75\textwidth}}
    B24 & British Library, Egerton ms. 2013 {[}\emph{B24}{]}, fol. 13v, the
    setting of the poem by John Hilton \\
    B47 & London, British Library, Stowe ms. 962, fol. 220r--v, an anthology
    of poetry from various authors \citepalias{celm_dnj_1577}\\
    H9 & Harvard, ms. Eng. 1107, folder 15, a copy by Thomas Gell
    (1595--1667), who was both an MP and member of the Inner Temple, on a
    single folio leaf \citepalias{celm_dnj_1584}\\
    HH6 & San Marino CA, The Huntingdon Library, ms. HM 41456, fol. 183v, an
    anthology volume containing only this work from Donne's oeuvre \\
    O3 & Oxford, Bodleian Library, Ashmole ms. 38, p. 14; like B47, a
    multiple-author florilegium \citepalias{celm_dnj_1578} \\
    1633 & \emph{Poems, by J. D. With Elegies on the Authors Death}. London:
    M{[}iles{]} F{[}lesher{]} for Iohn Marriot, p. 350
    \end{tabular}
    \caption*{}
\end{table}

\noindent A full list of variant readings drawn from all witnesses is given in
Table \ref{tab:lappin:3}; references are to the line (e.g. 14), and then (if necessary)
the subsequent variants within that line (e.g. 14.ii), and finally the
sequence of variants to that point in the text (e.g. 14.ii.b). The
manuscripts are ordered according to their presence in the stemma, which
is established below, in Table \ref{tab:lappin:4}, and shown in Figure \ref{fig:lappin:stemma} Where witnesses
contain a unexpected variant, according to their overall allegiances
within the stemma, this is indicated in italics and commented on
subsequently in discussion of the sub-archetypes.

\begin{center}
\centering\scriptsize\renewcommand{\arraystretch}{2}
\begin{longtable}[]{@{}c|p{.15\textwidth}|p{.22\textwidth}|p{.22\textwidth}|p{.15\textwidth}@{}}
\caption{List of Variant Readings\label{tab:lappin:3}} \\
\toprule
\emph{line} & \emph{mss.} & \emph{reading} & \emph{variant} &
\emph{mss.}\\
\midrule
\endfirsthead

\caption{List of Variant Readings (continued)} \\
\toprule
\emph{line} & \emph{mss.} & \emph{reading} & \emph{variant} &
\emph{mss.}\\
\midrule
\endhead

\bottomrule
\endfoot

Tit\emph{.}a & B29 D46 H6 C10 & Christo salvatori & To Christ & H5 H9 O3
H4 DT1 B7 CT1\tabularnewline
Tit.b & & & \emph{om. } & B47 B24\tabularnewline
Tit.c & & & \textbf{To God Æternall} & HH6\tabularnewline
Tit.d & & & \textbf{A Hymn to God the Father:} & 1633\tabularnewline
1.a & B29 B46 H6 C10 H5 HH6 1633 H4 DT1 B7 CT1 & that sin & \textbf{the}
sin & H9 B47\tabularnewline
1.b & & & \textbf{the sins} & B24\tabularnewline
1.c & & & \textbf{those sins} & O3\tabularnewline
2.i & B29 B46 H6 C10 H5 \emph{B47} B24 HH6 H4 DT1 B7 CT1 & is my sin &
\textbf{was} my sin & H9 O3 \emph{1633}\tabularnewline
2.ii & B29 B46 H6 C10 H5 H9 O3 B47 & was & were & B24 HH6 1633 H4 DT1 B7
CT1\tabularnewline
3.a & B29 B46 H6 C10 H5 H9 O3 B24 HH6 H4 DT1 B7 CT1 & those sins &
\textbf{the} sins & B47\tabularnewline
3.b & & & \textbf{that sin} & 1633\tabularnewline
4.a & B29 B46 H6 C10 H5 H9 O3 B47 B24 HH6 H4 DT1 B7 CT1 & them &
\textbf{run } & 1633\tabularnewline
4.b & B29 B46 H6 C10 H5 H9 O3 B24 HH6 1633 H4 DT1 B7 CT1 & I do & I
t\textbf{hem} & B47\tabularnewline
5.i & B29 B46 H6 C10 H5 H9 B24 HH6 1633 H4 DT1 B7 CT1 & thou hast done &
\textbf{this is} done & O3 B47\tabularnewline
5.ii & B29 B46 H6 C10 H5 & I have & thou hast & H9 O3 B47 B24 HH6 1633
H4 DT1 B7 CT1\tabularnewline
7.i.a & B29 B46 H6 C10 H5 H9 B24 H4 DT1 B7 CT1 & that sin by &
\textbf{the} sin by & B47\tabularnewline
7.i.b & & & \textbf{those sins with} & O3\tabularnewline
7.i.c & & & the sin (\emph{om.} of) & HH6\tabularnewline
7.i.d & & & \textbf{that} sin (\emph{om.} of) & 1633\tabularnewline
7.ii & B29 B46 H6 C10 H5 H9 O3 B47 B24 & I won & I have won & HH6 1633
H4 DT1 B7 CT1\tabularnewline
8.a & B29 B46 H6 C10 H5 \emph{B47} B24 HH6 1633 H4 DT1 B7 CT1 & made my
sin their & made \textbf{their} sin \textbf{my} & O3\tabularnewline
8b & & & \textbf{their} sin \textbf{\textsuperscript{was} my} &
H9\tabularnewline
9a & B29 B46 H6 C10 H5 H9 B24 HH6 1633 H4 DT1 B7 CT1 & that sin &
\textbf{the} sin & B47\tabularnewline
9b & & & \textbf{these sins} & O3\tabularnewline
11.i & B29 B46 H6 C10 H5 H9 B24 HH6 1633 H4 DT1 B7 CT1 & thou hast &
\textbf{this is} & O3 B47\tabularnewline
11.ii & B29 B46 H6 C10 H5 H9 O3 B47 B24 HH6 1633 H4 DT1 & done & done
\textbf{that} & B7 CT1\tabularnewline
12 & B29 D46 H6 C10 H9 B24 HH6 1633 H5 H4 DT1 B7 CT1 & thou hast &
\textbf{I have} & H9 O3 B47\tabularnewline
13.i & B29 D46 H6 C10 H5 B24 HH6 1633 H4 DT1 B7 CT1 & that &
\textbf{least} & H9 O3 B47\tabularnewline
13.ii & B29 D46 H6 C10 H9 B24 HH6 1633 H5 H4 DT1 B7 CT1 & when I have &
\textbf{having} & O3 B47\tabularnewline
14.i & B29 B46 H6 C10 H5 H9 B47 B24 HH6 1633 H4 DT1 B7 CT1 & my last
thread & \emph{om.} last & O3\tabularnewline
14.ii.a & B29 B46 H6 C10 H5 B24 HH6 1633 H4 DT1 B7 CT1 & shall &
\textbf{should} & H9 B47\tabularnewline
14.ii.b & & & \emph{om.} & O3\tabularnewline
14.iii & B29 B46 H6 C10 H5 H9 B47 B24 HH6 1633 H4 DT1 B7 CT1 & the &
\textbf{this} & O3\tabularnewline
15.i & B29 B46 H6 C10 H5 H9 O3 B47 B24 HH6 H4 DT1 B7 CT1 & Swear &
\textbf{But} swear & 1633\tabularnewline
15.ii.a & B29 B46 H6 C10 H5 B47 B24 HH6 1633 & thy sun & \textbf{this}
sun & \emph{O3} H4 DT1 B7 CT1\tabularnewline
15.ii.b & & & \textbf{the} sun & H9\tabularnewline
16.i.a & B29 B46 H6 C10 H5 H9 O3 HH6 H4 DT1 B7 CT1 & as it & as
\textbf{he} & \emph{B24 1633}\tabularnewline
16.i.b & & & \textbf{on me} & B47\tabularnewline
16.ii.a & B29 B46 H6 C10 H5 H4 DT1 B7 CT1 & as heretofore & and
heretofore & H9 O3 \emph{B24 1633}\tabularnewline
16.ii.b & & & or as before & B47\tabularnewline
16.iii & & \emph{-- full line --} & Shall own my soul and cloth it
evermore & HH6\tabularnewline
17.a & B29 B46 H6 C10 H5 1633 H4 DT1 B7 CT1 & And having done that & And
having done \textbf{this then} & H9\tabularnewline
17.b & & & And having \textbf{this done} & HH6\tabularnewline
17.c & & & And having done & B24\tabularnewline
17.d & & & \textbf{When this is} done \textbf{then} & O3
B47\tabularnewline
18.a & B29 B46 H6 C10 H5 H9 H4 DT1 B7 CT1 & I have & I \textbf{ask} &
O3\tabularnewline
18.b & & & I\textbf{'ll ask} & B47\tabularnewline
18.c & & & I \textbf{need} & B24\tabularnewline
18.d & & & I \textbf{fear} & HH6 1633\tabularnewline
st. II--III & B29 B46 H6 C10 H5 H9 O3 B47 B24 HH6 1633 H4 DT1 & II-III &
III--II & B7 CT1\tabularnewline
subscr.a & B29 B46 H4 DT1 & finis & finis D Donn & O3\tabularnewline
subscr.b & & & J:D: & B47\tabularnewline
subscr.c & & & \emph{om}. & H6 C10 H5 H9 HH6 1633 B7 CT1\tabularnewline
subscr.d & & & \emph{John Hilton} & B24\tabularnewline
\end{longtable}
\end{center}

\noindent The analysis of this \emph{variance} is given in Table \ref{tab:lappin:4}, below. In
representing the phenomenon of textual flux in a tabular fashion, I have
maintained (in the far-left column) the numeration of the stages from
Table \ref{tab:lappin:2}; accompanied by the listing of the archetype, here termed χ (as
the first letter of the original title, \emph{Ch}risto), followed by
each sub-archetype (α--λ); in the next column (\emph{justification}) are
the references to those variants which justify that archetype's
existence. The fourth column, \emph{witnesses}, lists those manuscripts
which share the specific variant. These are prefaced by
·\textgreater{}\textsuperscript{\emph{n}}, where \emph{n} indicates the
number of sub-archetypes, or levels of distance, from the archetypal
text. Underneath this listing of manuscripts, aligned to the right, are
any witnesses which depend upon the aforementioned sub-archetype, and in
the final column, \emph{further development}, any other variants which
are solely characteristic of these manuscripts (that is, idiosyncratic
or terminal variants, which do not continue down the stemma by a process
of direct transmission). The symbol `` $\curvearrowbotright$ '' indicates that a variant has
given rise to a further variant reading; `` \textsubscript{?}$\curvearrowbotright$ ''
indicates that such a process is a probable explanation for the presence
of differing variants. An italicized reference to a variant indicates
its non-conformity to the overall picture of transmission. Both
supplementary variation and unexpected variants are discussed in the
clarificatory notes appended to Table \ref{tab:lappin:4}.

\begin{center}
\centering\scriptsize\renewcommand{\arraystretch}{2}
\begin{longtable}[]{@{}p{.025\textwidth}p{.025\textwidth}p{.2\textwidth}p{.2\textwidth}p{.4\textwidth}@{}}
\caption{Stages of Transmission\label{tab:lappin:4}} \\
\toprule
\multicolumn{2}{c}{\emph{stage}} & \emph{justification} & \emph{witness} &
\emph{further development}\\
\midrule
\endfirsthead

\caption{Stages of Transmission (continued)} \\
\toprule
\multicolumn{2}{c}{\emph{stage}} & \emph{justification} & \emph{witness} &
\emph{further development.}\\
\midrule
\endhead

\bottomrule
\endfoot

& χ & archetype & &\tabularnewline
\begin{minipage}[t]{0.17\columnwidth}\raggedright
00\strut
\end{minipage} & \begin{minipage}[t]{0.17\columnwidth}\raggedright
\{α\strut
\end{minipage} & \begin{minipage}[t]{0.17\columnwidth}\raggedright
subscr.a

(subscr.c)\strut
\end{minipage} & \begin{minipage}[t]{0.17\columnwidth}\raggedright
·\textgreater{}\textsuperscript{0} B29 B46

·\textgreater{}\textsuperscript{0} (H6 C10)\strut
\end{minipage} & \begin{minipage}[t]{0.17\columnwidth}\raggedright
\}\strut
\end{minipage}\tabularnewline
01 & β & title.a(b---d) & ·\textgreater{}\textsuperscript{1} H5 H9 O3 H4
DT1 B7 CT1 & (\emph{om.} B24 B47 \emph{var.} HH6, 1633)\tabularnewline
& & & \hfill·\textgreater{} H5 & subscr.c\tabularnewline\hline
& γ & 5.ii & ·\textgreater{}\textsuperscript{2} H9 O3 B47 B24 HH6 1633
H4 DT1 B7 CT1 &\tabularnewline\hline
\begin{minipage}[t]{0.17\columnwidth}\raggedright
\strut
\end{minipage} & \begin{minipage}[t]{0.17\columnwidth}\raggedright
δ\strut
\end{minipage} & \begin{minipage}[t]{0.17\columnwidth}\raggedright
12, 13.i, 17.a/d

(1.a/c, 14.ii.a--b, 16.ii.a--b)\strut
\end{minipage} & \begin{minipage}[t]{.4\columnwidth}\raggedright
·\textgreater{}\textsuperscript{3} H9 O3 B47\strut
\end{minipage} & \begin{minipage}[t]{.4\columnwidth}\raggedright
17.a\textsuperscript{H9} $\curvearrowbotright$ 17.d\textsuperscript{B47--O3} {[}see n. (1.)
and (2e.), below{]};

(1.a\textsuperscript{H9--B47} \textsubscript{?}$\curvearrowbotright$
1.c\textsuperscript{O3}; 14.ii.a\textsuperscript{H9--B47}
\textsubscript{?}$\curvearrowbotright$ 14.ii.b\textsuperscript{O3 (\emph{om.})};
16.ii.a\textsuperscript{H9--O3} \textsubscript{?}$\curvearrowbotright$
16.ii.b\textsuperscript{B47} {[}see nn. 2a--d.{]})\strut
\end{minipage}\tabularnewline
& & & \hfill·\textgreater{} H9 & 8.b, 15.ii.b, subscr.c\tabularnewline
& ε & 5.i, 11.i, 13.ii, 17.d, 18.ab & ·\textgreater{}\textsuperscript{4}
O3 B47 &\tabularnewline
& & & \hfill·\textgreater{} O3 & 1.c, 7.i.b, 8.a, 9.b, 14.i, 14.ii.b, 14.iii,
\emph{15.ii.a} {[}see below, n. (5.){]}, subscr.a\tabularnewline
& & & \hfill·\textgreater{} B47 & title.b, 3.a, 4.b, 7.i.a, 9.a, 16.i.b,
subscr.b\tabularnewline\hline
& ζ & 2.ii & ·\textgreater{}\textsuperscript{3} B24 HH6 1633 H4 DT1 B7
CT1 &\tabularnewline
& & & \hfill·\textgreater{} B24 & 1.b, \emph{16.i.a} {[}see n. (3.) below{]},
\emph{16.ii.a} {[}see n. (2c.){]}, 18.c, subscr.d\tabularnewline
& η & 7.ii & ··\textgreater{}\textsuperscript{4} HH6 1633 H4 DT1 B7 CT1
&\tabularnewline
& θ & 7.i.c--d, 18.d, subscr.c & ·\textgreater{}\textsuperscript{5} HH6
1633 &\tabularnewline
& & & \hfill·\textgreater{} HH6 & Tit.c\tabularnewline
& & & \hfill·\textgreater{} 1633 & Tit.d, \emph{2.i} {[}see below. n. (4.){]},
3.b, 4.a, 15.i, \emph{16.i.a} {[}see n. (3.){]}, \emph{16.ii.a} {[}see
n. (2c.){]}\tabularnewline
02 & κ & 15.ii.a & ·\textgreater{}\textsuperscript{5} H4 DT1 B7 CT1
&\tabularnewline
& & & \hfill·\textgreater{} H4 DT1 & (subscr.a)\tabularnewline
03 & λ & 11.ii, II--III, subscr.c & ·\textgreater{}\textsuperscript{6}
B7 CT1 &\tabularnewline

\end{longtable}
\end{center}

Clarificatory notes.
\begin{center}
\footnotesize\renewcommand{\arraystretch}{2}
\begin{longtable}[]{@{}lp{.75\textwidth}@{}}

    δ -- ε & (1.) 17.a\textsuperscript{H9} ``and having done \textbf{this
    \emph{then}}'' $\curvearrowbotright$ 17.d\textsuperscript{B47--O3} ``\textbf{When this is}
    done \emph{\textbf{then}}''. The initial change in δ but recorded only
    by H9 (\emph{this then} for \emph{that}, perhaps mistaking the latter
    word for abbreviated forms of the two words) created an unmetrical line,
    which was solved in sub-archetype ε by changing the initial words rather
    than suppressing the addition. \\
    
    & (2.) The following three variants do not affect the readings that the
    sub-archetypes would have offered (they do not change), but one
    manuscript, either O3 or B47 produce a variant built upon the previous: \\
    
    & (2a.) 1.a\textsuperscript{H9--B47} ``the sin'' \textsubscript{?}$\curvearrowbotright$
    1.c\textsuperscript{O3} ``those sins'' (archetypal reading: \emph{that
    sin}); the temptation to make the \emph{sin} plural was also felt by B24
    (``the sins''), but there is no real reason to suspect contamination. \\
    
    & (2b.) 14.ii.a\textsuperscript{H9--B47} ``I should perish''
    \textsubscript{?}$\curvearrowbotright$ 14.ii.b\textsuperscript{O3} ``I perish'' (i.e.
    \emph{om}. auxiliary verb; archetypal reading, \emph{I shall perish}).
    The best one can say about the omission is that it does not contradict
    the readings one would expect according to the stemma; and so O3
    certainly omitted \emph{should} rather than \emph{shall}. \\
    
    & (2c.) 16.ii.a\textsuperscript{H9--O3} ``and heretofore''
    \textsubscript{?}$\curvearrowbotright$ 16.ii.b\textsuperscript{B47} ``or as before''; the
    sub-archetypal reading was most probably \emph{and heretofore}, a slight
    misreading of the archetypal \emph{as heretofore}. The variant was
    probably caused by the misreading of \emph{a\longs} as an ampersand
    (\emph{\&}), which is a relatively easy mistake to make owing to the
    ligatures and the slightly bowed shape often given to the long-s (\longs):
    a similar misreading affected both B24 and 1633 at this point. \\
    
    & (2d.) At 8.a, B47 produces the archetypal reading, ``made my sin their
    door'', which contrasts with ``made \emph{their} sin \emph{my} door'' of
    O3. H9 would seem to support O3's version, with the scribe writing
    originally ``their sin my door'' (\emph{om.} made), with a similar
    transposition of possessive pronouns as that found in O3. However, the
    H9 scribe realised that a mistake had been made, and returned to the
    line to offer a superscript correction: ``their sin
    \emph{\textsuperscript{was}} my door''. This is not as easy to
    disentangle as other variants, although the garbled, and then further
    garbled, version of the line in H9 would suggest that sub-archetype δ
    itself bore a confused or confusing reading at this point, which the
    scribe of H9 was attempting to elucidate; this confusion in δ was passed
    on to sub-archetype ε, where again the scribes attempted some remedy:
    B47 guessed correctly; O3 did not. (``(In)correctly'' only if the
    scribe's purpose was to reproduce the previous manuscript tradition; the
    ``incorrect'' reading may have been perfectly purposeful.) \\
    
    & (2e.) H9 is a singleton, but B47 and O3 are both florilegia. Although
    the sub-archetypes --- here, specifically, δ and ε --- are really best
    thought of as hypothetical constructs which act as nodes within the
    stemma, whose existence is merely pending the discovery of further
    witnesses which may cause them to be further multiplied, and so
    redefined, or removed, it is very likely that the transmission of these
    texts passed through other anthologies rather than any collected works
    of the poet himself. Thus we may point to the likelihood of a common
    source for B47 and O3 texts in another, earlier florilegium since both
    exemplars of our poem are found in close vicinity to the Thomas Carew
    poem, \emph{A flye that flew into my Mistris her eye} \citepalias{celm_cwt_236,celm_cwt_250}, although they bear different titles (O3: ``The Amorouse
    Fly'', p. 9; B47: ``An Epitaph vppon a Fly'', fol. 221r--v); that
    titular variation, though, might signify nothing more than that the
    model bore no title for this particular poem. \\
    
    η -- θ & (3.) A particular challenge to the stemmatic dependence of
    witnesses in this section of the apparatus is provided by the variant
    common to 1633 and B24 at 16.i.a, where the sun is referred to as ``he''
    (against the unanimous third-person neuter pronoun throughout the rest
    of the manuscript tradition). There may well be strong theological
    reasons for the change, which were felt by both scribes; alternatively,
    a reminiscence of Hilton's setting (which may have been tolerably
    popular) affected the copyist or printer in setting down the text. \\
    
    δ -- θ & (4.) 2.i: archetypal reading: \emph{is my sin}; variant reading:
    \emph{\textbf{was} my sin}; the variant reading here is shared between
    H9, O3 (but not B47) and 1633. Such variation probably reflects a
    situation in which the sub-archetype originally offered both
    possibilities, one tense of the verb written as a variant of the
    other---such as indeed has been preserved in B46, which adopts the
    reading \emph{was} in the text, yet provides the variant ``is'' in the
    scribe's own hand at the left margin. We may draw a parallel with the
    subscription, ``finis'', which was also maintained by B46 and passed
    down the chains of transmission to H4 and DT1, and, in modified form, to
    O3. Thus we probably have the remnants of scribal annotations preserved
    through sub-archetypes but only occasionally surfacing in the texts that
    are preserved. \\
    
    κ -- O3 & (5.) The common variant between 03 and H4 DT1 B7 CT1 at 15.ii.a,
    \emph{this sun} is surely an example of polygenesis. O3 originally read
    ``thys'', which probably indicates that its model still read the
    archetypal ``thy'', and that the scribe was misled by a ligature or
    flourish to the \textit{--y}. H9, which descends from the sub-archetype above
    O3's own sub-archetype, offers ``the'' (15.ii.b), which suggests that,
    in this side of the transmission, the reading was uncertain or
    indistinct, possibly giving rise to two possibilities in the same text.
    I suggest below that the change may have been attractive as a reaction
    to an allegorical reading of ``sun'' as ``Son'', Second Person of the
    Trinity; both \emph{this} and \emph{the} make the reference resolutely
    this-worldly and so maintain the addressee as Christ.

\end{longtable}
\end{center}

\begin{figure}[H]
\centering
\includegraphics[width=.7\textwidth]{media/lappin1.png}
\caption{A Stemma.}
\label{fig:lappin:stemma}
\end{figure}


\noindent The pattern established, of successive changes at different stages of
copying, evidently argues against any authorial revision producing a
``new version'' of the poem, and argues for a branching out of scribal
innovations and emendation. Significantly differing versions, taken out
of the context of the flow of the manuscript tradition, might well seem
to provide a significant leap which would justify the assumption of a
coherent, single authorial or scribal intervention to create a second
recension, which is essentially what Pebworth produced through his
classifications of Texts 1, 2 and 3, and his corresponding exclusion of
significant variants. Robbins, in his election of DT1, could hardly have
chosen a text further away from the archetype. In contast to the vast
majority of manuscript traditions, where copying only brings about
textual decay, incoherence, loss, deturpation, metrical irregularity
and, eventually, irremedial chaos, the copying of these short poems saw
active, engaged, intrusive copying on the part of the scribes with a
sensitivity to metrical and other forms of seeming errors, and a
consequent preparedness to correct and emend the texts that they were
then themselves partly writing. Robbins' principle, that of looking for
the best text, or the one with fewest obvious errors, was not
necessarily a bad one, particularly if one intends to comment upon all
of Donne's poetry and not just dissect less than twenty lines of it; but
the principle can be very misleading in a textual tradition where
copying is not done by drudges, but by connoisseurs of the poetry
itself. As we can see, the attentive, even playful, copying that
transmitted this poem did not necessarily produce a degraded text ---
far from it; the alteration of the sequence of repetitions found in O3,
for example, produces a different, but no less effective, set of
emphases.

There is now little need to wonder about the earliest state of the text:
four or five witnesses give us a coherent and consistent version; there
are no indications that there is any subsequent interference by Donne in
the manuscript transmission. At most one might argue about the
mis-en-page and punctuation of the archetype; but the sequence of words
is not in doubt. This confirms Pebworth's conclusion regarding the
authority of the \emph{Christo salvatori}-version of the poem, and it is
this --- and only this --- which should be accepted as Donne's own work.
If the purpose of the article had been to identify those texts which
most closely reflected the archetype, and establish Donne's handiwork,
purified from the additions of others, then it might stop here, or
proceed to analyse only the archetypal version, that is the \emph{Christo
salvatori}-version. Yet this was not the purpose; in a manner similar
to my recent analysis of \emph{Valediction: forbidding mourning} \citep{lappin_baroquely_2019}, I propose to study the \emph{variance} of the poem as a function
of scribal presentation and development, in order to provide a global
understanding of the text as it was both written and transmitted; this
article reinforces the conclusions of its twin.

\section{Reading and Writing the Poem}

Robbins, noting the similarity in themes to both sermons and poetry
which Donne preached and wrote around the time, places the composition
of the poem shortly after Anne More's death \citep[654]{robbins_complete_2013}, and
there can be little reasonable disagreement with such an estimation. In
further support of this, we might observe that the connexion with his
wife's demise is particularly strong in a 1620s miscellany volume of
poetical works in both English and Latin, B29 (London, British Library,
Harleian ms. 3910), where the poem (in the \emph{Christo
salvatori}-recension) precedes Donne's own Latin epitaph for Anne More
(foll. 50r, 51r), the two works separated only by Richard Cobbet's
ironic epigram on Lady Arbella Stuart's burial in 1615 (``Vpon the Lady
Arabella'' at fol. 50v; see \citetalias{celm_cor_540}).

\subsection{The title}

The original text of the poem, then, offers us a view of the poet
tormented by grief, and addressing a prayer to Christ, the Saviour,
\emph{Christo Salvatori}. And although ideas of the salvific power of
Christ might be expected to give origin to emotions of tender devotion
and gratitude, invocation of Christ \emph{the Saviour} was done with
at-times theatrical angst, for it was Christ the Saviour who was to
appear also as, at the same moment on the Last Day, Christ the
Judge.\footnote{For example, the angst-ridden 126 lines of the second of
  William Austin's meditations for Good Friday, also entitled ``Christo
  salvatori'' \citep[117--22]{austin_devotionis_1635}; St Augustine counselled fearing
  the Saviour since he will also be the Judge \citep[CCXIII.6]{augustinus_sermones_nodate}.}

Thus the first line, ``Wilt thou forgive {[}\ldots{}{]}'' suggests
immediately the two aspects of Christ, as saviour and judge: saviour
should he offer forgiveness; judge if he withholds it. The vernacular
simplification of the title, then, in removing the concept of the
\emph{Salvator} from the beginning of the poem subtly shifts the
emphasis from a concern with the Last Day to a general prayer to Christ.
The change may simply have been motivated by a shift in audience, away
from the bilinguals of his inner circle for whom Donne generally wrote,
to a more resolutely native, and possibly naïve, readership, for whom it
might have been desirable to remove (together with the Latin) a possibly
unwelcome Catholic insistence upon the importance and unwelcome
uncertainty of that particular \emph{dies illa}.

\subsection{Stanza One}

The poem begins very much at the beginning, however, since the first
request for, or questioning about, forgiveness regards Original Sin, the
sin passed down as damnable inheritance from generation to generation,
from Adam to John: as Psalm 51:5 memorably expressed the fault,
``Behold, I was shapen in iniquity; and in sin did my mother conceive
me''.\footnote{So the King James; thus the Douay-Rheims: ``For behold I
  was conceived in iniquities; and in sins did my mother conceive me''.}
Such an orthodox interpretation is commended in at least one manuscript,
CH1:\footnote{Chester, Chester City Record Office, CR63/2/692/219, fol.
  200r.} a marginal annotation adds ``Originall'' level with these
initial verses.

The most important variant in these lines is undoubtedly the verbal
tense at 2.i: \emph{is} vs. \emph{was}. The difference between these
tenses is not minor. \emph{Was} places the Original Sin as being a sin
from and belonging to the past, and invokes the more Catholic and
Lutheran understanding of the sacrament as ``washing away'' Original
Sin: after baptism (and Donne had, after all, been baptised a Catholic)
that Sin could no longer be his. The present tense, \emph{is}, however,
suggests a much more Calvinist understanding of the role of baptism.
This was not an outward sign of inward grace, but a promise of
forgiveness, to be brought to mind when the faithful yet still
regrettably sinful believers had made themselves anxious over the
possibility of redemption: working oneself up over the certainty of
one's own salvation (or, rather, of its opposite), only to then assuage
the tormented mind by a recollection of the divine promise, is a
characteristic element of Calvinist mental theatre. ``Is'' must be the
primary reading; ``was'' introduced at first as a scribal quibble,
rather than as a simple replacement, a quibble brought about by a
divergent theological sensibility.

Thus, in a contemporary summation of received opinion --- offered by
Samuel Ward, master of Sydney Sussex College, Cambridge, during the
1620s --- baptism was only ``conditional and expectative, of which they
have no benefit till they believe and repent''. Donne's problem, which
he will explore throughout the poem, is what happens if one only sort-of
believes, and only sort-of repents? And the cause for his half-hearted,
or insufficient, repentance may allow us to place Calvinist theology
temporarily in the background, and bring the poet's biography to the
fore: the sin ``where I begun'' recalls the same phrase with which he
ended his \emph{Valediction: forbidding mourning}, which, even in Izaac
Walton's highly sentimentalist reading of the same, was dedicated to Ann
More to commemorate one or other parting \citep[33--34]{walton_lives_1675}:\footnote{Walton entitles the poem \emph{A Valediction, forbidding to
  Mourn}. For discussion of this poem, see, for example, \citealt[118]{smith_john_1983}; \citealt[97--98]{mccolley_poetry_nodate}; \citealt[75--76]{targoff_john_2008}. For the biographical
  inaccuracy of Walton's account, see \citealt[41 note \emph{m}]{walton_lives_1807}.}
here the echo emphasizes that definitive parting of the grave, and the
recollection of ``where I begun'', the triumphant return of the poet to
the beloved in \emph{Valediction}, is applied to \emph{the sin}, his
over-attachment to his wife, perhaps, his inability to abandon his love
and grief for her, his excessive delight in the physicality of their
relationship, or even the guilt he felt for having seduced her in the
first place.\footnote{See also \citealt{lappin_baroquely_2019}, where I argue that rather than
  understanding the poem as an expression of marital attachment (which
  is Walton's presentation of the same), it was written to insist to
  Anne that she should not give any cause for suspicion before their
  secret marriage. ``Where I begun'' is thus a phrase with a particular
  biographical charge, even with a weight of guilt upon it.} She had
died in childbirth, after all, and Donne, despite his many faults, was
not an irresponsible sadist. This emphasis on the human, biographical
experience of the poet --- rather than the theological inheritance from
Adam --- was emphasized through the shift in two unrelated witnesses to
a plurality of sins in the first line (1.b--c: B24, O3), perhaps in part
to avoid the doctrinal question of Calvinist sacramentalism.

The second half of the second line picks up this biographical focus,
``though it was done before'', with the first inkling of the multiple
senses of done / Donne, as contemporary orthography did not consistently
distinguish the words:\footnote{O3, for example, uses \emph{donn} as
  spelling for both \emph{done} and \emph{Donne}; H6 offers \emph{donne}
  throughout, which was the scribe's usual spelling of the past
  participle; for example: \emph{Sappho to Philaenis} (24:
  \emph{Sappho}), l. 52, p. 231; \emph{The Sunne Rising} (36:
  \emph{SunRis}), l. 28, p. 260; \emph{The Triple Fool} (40:
  \emph{Triple}), l. 12, p. 255; \emph{The Extasy} (63: \emph{Ecst}),
  ll. 1, 25, p. 302; \emph{The Blossome} (68: \emph{Blossom}), l. 26, p.
  284; Epistles: \emph{To M\textsuperscript{r} R. W.} (122:
  \emph{RWSlumb}), l. 23, p. 219; \emph{To S\textsuperscript{r} Henry
  Wooton at his going Ambassador to Venice} (129: \emph{HWVenice}), l.
  24, p. 221; \emph{To M\textsuperscript{rs} M. H.} (133:
  \emph{MHPaper}), l. 33, p. 239; \emph{To the Countesse of Bedford}
  (134: \emph{BedfReas}), l. 28, p. 205; idem (136: \emph{BedfHon}), l.
  11, p. 191; {[}\emph{Vpon the death of}{]} \emph{Mrs Boulstred} (151:
  \emph{BoulNar}), l. 16, p. 169; no examples of \emph{done} or
  \emph{donn} were found.} on the one hand, the sin was in the past,
done before the \emph{now} of the poem, committed even before the
existence of the poet. On the other, the sin itself continues to be his,
even though it was much more characteristic of Donne in his youth; and
the nominatively deterministic \emph{done} was of course slang,
referencing the sexual act.\footnote{For example, \emph{Titus
  Andronicus}, IV.ii.75--76 where the accusation, ``Thou hast undone our
  mother!'' is met with the riposte: ``Villain, I have \emph{done} thy
  mother!''} Although he might have put his youthful enjoyments behind
him, they dogged him, even through his name. Robbins adroitly cited an
excerpt of one of Donne's sermons composed four months after his wife's
death: ``We may lose him {[}Christ{]}, by suffering our thoughts to look
back with pleasure upon the sins which we have committed'' \citep[90; citing Sermon 1.245 on Proverbs 8:17]{robbins_complete_2013}. The shift to the
subjunctive mood, ``though it \emph{were} done before'' (2.i: one of the
earliest emendations to the poem in transmission), closes off the
concrete, biographical reference, to leave only hypothetical possibility
or evocation of the distant, pre- and immediately post-lapsarian past.

The parallel senses, of theology and biography, are brought together in
the following lines as the poets' current state is evoked: in the
archetypal text, this is marked by a shift from the previously singular
\emph{sin} to plural \emph{sins}, and their shifting multiplicity
(``through which I run''), serious not simply in number but also in
their commission (``and do them still''). Just as a polyptoton is
performed by \emph{is} reappearing as \emph{was} in ll. 1--2, so
\emph{done} (l. 2) returns as \emph{do} (l. 4): the sins of the past may
no longer characterize him, but that does not mean that he is free of
sins. The alteration of ``do them'' to ``do \emph{run}'' in 1633 (4.a)
allows the pious reader to imagine the late dean of St Paul's not so
much committing sins, but ineffectually fleeing from them, trapped as he
cannot but be by the surrounding world, a world, despite his best
efforts from the pulpit, of sin.

With Original Sin (or Donne's original sins) added to his present sins
--- sins unnamed and elusive of identification --- one passes to the
final verses of the stanza: as the archetypal text has them, ``When thou
hast done, I have not done / For I have more''. Christ is addressed as
``doing'' the forgiveness, but in vain, since Donne himself has more
sins. The evident allusion to Anne More's name has only relatively
lately been accepted by readers of the poem (\citealt[90]{leigh_donnes_1978}; queried by \citealt[291--92]{novarr_amor_1987}; reaffirmed by \citealt[22]{ahl_ars_1988}), no doubt because of
the excessive crudity of the paronomasiae: \emph{have}, like \emph{do},
possesses a raw sexual undertone,\footnote{For example, \citealt[53]{quaife_wanton_1979},
  records the following court deposition: ``coming by a chamber door
  that stood ajar, they thrust him open and stepped into the chamber and
  there they saw Richard Templeton having Agnes Moore against the bed''.
  (Agnes was probably no relation to Anne). Further, \citealt[119]{patridge_shakespeares_1968};
  \emph{OED}, s.v. \emph{have}, §14e.} which --- once accepted --- flows
back into the \emph{thou hast} of the previous line, providing a
shocking juxtaposition to Christ, presenting Christ's forgiveness as a
form of sexual possession. Such a conception, though, of himself
\emph{vis-à-vis} the divinity was in fact a fixed element of Donne's own
spiritual self-presentation \citep{challis_conflicting_2016}. And, although it may present
a transgressive air within the confines of an expected or normative
Protestant religious discourse, it can hardly have been unfamiliar or
outlandish to anyone even slightly acquainted with Spanish religious
poetry, and Donne's acquaintance with the poetry of the Spains was
hardly slight (see \citealt[II.4]{grierson_poems_1912}; \citealt{thompson_mysticism_1921}; \citealt{cora_institution_1996}; for the tradition of meditation: \citealt{roston_donne_2005}).

Nevertheless, the archetypal rendition of the line (``When thou hast
done, \emph{I have} not done'') is not followed by most later witnesses
(5.ii): ``When thou has done, \emph{thou hast} not done''. Since this
variant line simply repeats what is found in the second stanza in the
same place in the archetypal text (l. 11), it is most likely that this
variation was initially the product of eyeskip.\footnote{It is difficult
  to be categorically sure, but the transposition across the first and
  second stanzas of ``I have'' and ``thou hast'' in sub-archetype δ may
  have preceded the adoption of ``thou hast'' for both lines;
  alternatively, ``thou hast'' was the original reading of the copy,
  subsequently corrected by ``I have'' at the wrong position, a
  correction adopted by δ but spurned by ε.} As a principle, mechanical
error is usually a better explanation than purposeful emendation, and so
I will assume that this change is a simple error, unmotivated by any
intention. Nevertheless, the manuscript in which it occurred provided a
bottleneck, and subsequent copies of Donne's poem flowed from that
exemplar.

Yet that is not the only variant at this point. O3 and B47 both provide
a confirmation that the allusion to the poet's own name was an important
element for some readers (5.i): ``\emph{When this is} donn, thou hast
not donn''. \emph{This} does not aid in the extrication of a single
meaning, since it generates a number of referents: the word might invoke
God's forgiveness (ll. 1, 3) which has been granted; or might ---
relying upon the second half of l. 2 (``though \emph{ytt} \emph{was}
\emph{donn} before'') --- point to the completion of his running through
sins, his abandonment of his sinful ways; or, in complete contradiction
of this second meaning, point directly at the poet himself, and so the
phrase would mean ``Whilst Donne behaves in a way characteristic to
himself, that is sinfully''. All these meanings flow seamlessly into the
second half of the line: ``thou hast not done'' (finished forgiving,
gaining Donne to Himself), with the humorous reversion of roles: Christ
is not done, cannot leave off forgiving, precisely because Donne is
Donne.

\subsection{Stanza Two}

The second stanza has suffered least in transmission and offers the
fewest number of variants.

Lines 7--8 look for forgiveness from having brought others to commit
sins; lines 9--10 turn back to (given the allowed time-span of twenty
years), what might be described as the sins of Donne's youth; the
language is suitably condemnatory, with \emph{wallowed} (l. 10), the
go-to word for moralists wanting to describe a sinful, often illicit,
but predominantly sexual, laxity. Thus, in John Taylor's \emph{A
Whore} (ll. 35--36), the eponymous sex-worker is described as ``A
succubus, a damned sinke of sinne, / A mire, where worse than Swine do
wallow in'' \citep[106]{taylor_all_1630}; the enthusiasm for the word relies upon
II Peter 2:22 and Proverbs 26:11, ``A sow that is washed returns to
her wallowing in the mud''.

\emph{Sin}, too, gains a new inflexion here: in the previous stanza it
denoted at first an inherited disposition (ll. 1--2), and its sense was
then broadened to take in individual acts (l. 3), exacerbated by the
consciousness of their sinfulness (l. 4), to subsequently refer to a
characteristic disposition, a type of behaviour (l. 7); and then, in the
present stanza, to epitomize ``sinfulness'', the sinful life, for which
he provided the ``door'' for others to enter into (l. 8): the
orthography of, for example, H6, makes clear that, in the phrase
``won / others to Sinne'', \emph{sin} is understood as a noun rather than
a verb \citep[see][20]{pebworth_editor_1987}, and \emph{door} has perhaps a biblical
sense of an anti-type to the narrow gate which leads to eternal life
(Matthew 7:14). The field of reference is completed by \emph{sin} which
is conceived not as a momentary lapse, but as a repeated surrender to a
specific temptation (l. 10), in full consciousness of its sinfulness
(ll. 9--10, cp. l. 4, ``deplore''). All-in-all, an impressive display of
traductio or antanaclasis using the various meanings of ``sin''.

Through these sets of \emph{foci}, then, we might say that Donne is
fulfilling relatively precisely Calvin's own instruction to the pious
(\emph{Institutes} III.20.9) on how they should conduct their prayers:

\begin{quote}
Finally the beginning and also the preparing of praieng rightly, is
crauing of pardon, with an humble and plaine confession of fault.
{[}\ldots{}{]} For Dauid when he asketh an other thing, saith: Remembre
not they sinnes of my youthe, remember me according to thy mercie for
thy goodnesses sake O lord. {[}\ldots{}{]} Where we also see that it is
not enough, if we euery seuerall day do call our selues accompt for our
new sinnes, if we do not also remembre those sinnes which might seeme to
haue been long agoe forgotten. For, the same Prophet in an other place,
hauing confessed one haynous offense by this occasion returneth euen to
his mothers wombe wherin he had gathered the infection: not to make the
faulte seme lesse by the corruption of nature, but the heaping together
the sinnes of his whole life, how much more rigorous he ys in condemning
himself, so much more easy he maye finde God to entreate.
\begin{flushright}
\citep[207v]{calvin_institution_1561}
\end{flushright}
\end{quote}

\noindent Following Calvin's suggestion, Donne, in order to set himself aright
with God, enumerates the sins of his conception and his youth. The
reason for doing this was to make God more likely to grant his prayers,
which were for forgiveness; however, if forgiveness is to be gained by
individual confession to God, and God is more likely to forgive in
proportion to the number of sins dredged up from the past, then
asserting that one has more sins to confess makes it more likely they
will be forgiven --- \emph{ad infinitum}. This is the theological (and
rhetorical) conundrum which Donne is exploring in the penultimate and
ultimate lines of each stanza. In this second stanza, however, the
\emph{Christo salvatori}-text provides a \emph{variatio} to the
penultimate line: ll. 5--6 offered ``When thou hast done, I have not
done / For I have more''; ll. 10--11 inflects to ``When thou has done,
\emph{thou hast} not done / For I have more''. The expectation would be
that Christ has not finished forgiving: Donne will confess more sins,
which he does. But no longer is it Donne who has not finished sinning
(and so needs to list more sins), rather the sin he will confess is of
another order. And so Christ here will neither finish forgiving (perhaps
He has not even started), nor will He gain Donne to himself, since Donne
has something else up his sleeve.

\subsection{Stanza Three}

What Donne does not do in the first line of this stanza is ask whether
he can be forgiven for the sin he will confess to in the third stanza.
He breaks the expected rhythm he had established with the anaphoric
``Wilt thou forgive {[}\ldots{}{]}'' that begin the first two stanzas. Indeed,
Christ \emph{has not done}, has not finished His activity, since He is
being asked something quite beyond ordinary forgiveness, which in effect
contradicts the possibility of ordinary forgiveness. What is being
requested --- if not demanded --- is a personal promise to the poet that
he will be numbered amongst the elect. For the Calvinist, there would be
a quota even for salvation; those who would be counted amongst that
number were already predestined to eternal bliss; therefore, no amount
of confession of sins could change the brutal fact of divine election
(or divine damnation). As Oliver observed regarding our poem:

\begin{quote}
The Calvinist doctrine of the limited atonement directly challenged the
individual to believe that he or she was among those for whom Christ had
died. Hence the speaker's elaborate and ingenious plan to be sure that
he is one of those benefited by Christ's death. The logical objection to
this is that, if he is among the elect, the merits of Christ's death
will already have been ``applied'' in his case. No amount of wishing for
a personal manifestation of their application will make any difference
to the divine arrangements: Christ cannot be crucified afresh, except in
the imagination. What's more, the elect know that grace is working in
them---that the merits of Christ's death have been applied in their
cases.
\begin{flushright}
\citep[90]{oliver_donnes_1997} 
\end{flushright}
\end{quote}

\noindent Oliver goes on to observe that ``The speaker's prayer thus amounts to an
entirely self-defeating request to be counted among the elect'',
although I feel that Donne's bargaining with the divine may be slightly
more complex, even slightly more cunning. From a theological point of
view, Donne confesses another sin, but one of a different order to the
previous enumeration, a theological sin, as it were, one of fear or
doubt that he will be saved; essentially, a lack of faith; and, within
the Protestant soteriology he had embraced, it was faith alone which
justified; without faith, there could be no salvation; the one Really
Bad Thing a believer could do, that which would utterly derail their
salvation, was doubt that they could be saved. And that is precisely
what Donne insists on confessing. Christ cannot forgive a lack of
belief, since the forgiveness of sins offered by Christ is predicated
upon the individual's believing in that forgiveness (which He gained for
mankind through His single sacrifice upon the Cross). It was pointless
getting angsty over whether you were going to be one of the lucky few
crammed into that ultimate elevator, if you really were unsure you could
actually keep your balance on the narrow \emph{tapis roulant} of faith
which would take you close to those sliding, pearly doors.

This ``sin of fear'' is not unrelated to his previous evocation of
Original Sin; indeed, it was already there \emph{in semine}. As we have
noted, for Calvin, recollection of Original Sin was inseparable from
baptism, and the recollection of baptism re-assured the baptised that
their sins would be forgiven (although they had not been forgiven
through the ritual); forgiveness and salvation were thus postponed to
the moment of death and Judgement. Baptism (for the baptised) was thus
designed to provide a reassurance (in the imagination) of future
salvation. Calvin had opined (\emph{Institutes} VI.15.9):

\begin{quote}
In the Cloude was a signe of cleansyng. For as then the Lorde couered
them with a cloude cast ouer them, and gaue them refreshing colde, least
they should faint and pine away with to cruell burning of the sunne: so
in Baptisme we acknowlege our selues couered and defended with the blood
of Christ, least the seueritie of God, which is in dede an intollerable
flame, shoulde lie vpon vs. 
\begin{flushright}
\citep[102r]{calvin_institution_1561}
\end{flushright}
\end{quote}

\noindent A christening carried out by recusant Catholics may have had rather less
of a consolatory impact upon Donne's imagination than had he been
assured that he had received his baptism surrounded by the righteous
elders of the true church.

Donne's description of his feared death picks up both Calvinist
anxieties: of predestination, and of consequent futility: ``when I have
spun / my last thread, I shall perish on the shore'' (ll. 13--14). The
allusion would seem to be to the silkworm, whose cultivation in London
had been encouraged by James I and VI after 1607, and had become a
long-term success (Peck 2005, 91): after the completion of its cocoon
(its last thread), the caterpillar was killed to gain the silk
(\emph{shore}: river-side sewer, where it would be cast; cp. \citealt[387]{williams_dictionary_1994}, \emph{s.v.} ``common shore''). The grub would die without
fulfilling its (biological) destiny of gaining wings; the poet would die
without transformation to ``become like the angels'', the spiritual
destiny of redeemed humanity. However, whatever the possible
\emph{telos} of each being, neither was predestined for flight: the
cultivated silkworm destined to die as soon as it had begun its process
of transformation; the disbelieving poet, (pre)destined to damnation
even as he died acknowledging his sins (particularly his sinful lack of
belief), his soul perishing on one side without ever reaching the ``ripa
ulterior'', the farthest and last shore.\footnote{Virgil, \emph{Aeneid}
  VI.313--16, ``Stabant orantes primi transmittere cursum / tendebantque
  manus \emph{ripae ulterioris} amore. / Navita sed tristis nunc hos nunc
  accipit illos, / ast alios longe submotos arcet harena'' [They stood,
  begging to be brought on the journey first, holding out their hands
  for love of that \emph{final shore}. The grave boatman yet takes on
  board now these, now those, while he drives off others far from the
  sands].}

The solution to this impasse is to seek a solution from the Old
Testament: a personal covenant with God: Christ is instructed (via a
rather daring imperative) to guarantee, swearing by God, that is, by
Himself (l. 15), to guarantee the poet's salvation. Covenantal theology
(reliance upon God's having bound Himself via an oath or promise) had
become an increasingly popular means of expressing God's relation to the
chosen people (that is, themselves) amongst  proto-Puritans \citep{holifield_covenant_1974,von_rohr_covenant_2002}, and it was probably from this environment that
Donne took his inspiration. Obviously, with Donne, the ``covenant''
could not be just with the people of God to be satisfactory, but must
focus on Dr John Donne --- but, to be fair, it must focus on Dr John
Donne, not because he was especially worthy, but because he was honest
enough to face up to how he himself fell through the cracks of the
Calvinist-Puritan theology of redemption.

Given the later development of the poem into print, where the poem is
addressed to God the Father and the Son that shines is evidently Christ,
it is not surprising that ``thy sunne'' has been elided with
\emph{Christus Oriens}, Christ the Rising Sun, by commentators (\citealt[22--23]{pebworth_editor_1987}; \citealt{robbins_complete_2013}). It was perhaps to prevent such a leap
that, at 15.ii.a, the reading changed to ``this sunne'' at κ, to make
clear that an allegorical reading was \emph{not} being sought. Nor, I
think, was allegory the aim in the archetypal text: ``\emph{thy} sun'';
so not Christ as the sun, but Christ as owner of the sun; the Christ
\emph{through whom} all things were created --- the \emph{Christus
creator} --- together with the Christ \emph{through whom} all things are
saved --- the \emph{Christus salvator} --- (see, \emph{inter alia},
Colossians 1:12--20). \emph{Thy sun}, then,
undoubtedly just originally meant \emph{the} sun (which is the sensible
conclusion of the scribe of H9: 15.ii.b). Donne asks Christ to keep the
sun shining, just as the sun has shone up to the present. The biblical
comparator is, in part, Genesis 22:16 (``And said, By myself have I
sworn, saith the LORD, for because thou hast done this thing, and hast
not withheld thy son, thine only son''),\footnote{The same covenantal
  promise is recalled at Exodus 32:13, ``Remember Abraham, Isaac, and
  Israel, thy servants, to whom thou swarest by thine own self, and
  saidst unto them, I will multiply your seed as the stars of heaven,
  and all this land that I have spoken of will I give unto your seed,
  and they shall inherit~\emph{it}~for ever''. Further Supreme
  Self-swearing is found at Isaiah 45:22, Jeremiah 22:5; and God's
  swearing by His own name, at Jeremiah 44:26.} and, more importantly,
the commentary on the same provided by St Paul at Hebrews 6:13, ``For
when God made promise to Abraham, because he could swear by no greater,
he sware by himself''. Within such a context, the printer's decision to
make the addressee God the Father is particularly understandable.

Yet the promise Donne would see enacted was but a daily phenomenon. How
--- given his ``sin of fear'' --- was he to know that Christ had sworn
anything? Personal assurance was requested: but what certainty could
possibly be received? The final lines, in our archetypal recension, do
nothing to dispel this clouded vision of salvation: ``And having done
that, thou hast done / I have no more''. The traductio of the final lines
to each stanza have been poised between present and future: current
forgiveness of sins (or current potential forgiveness of sins), and
future definitive forgiveness of sins. The play on the poet's name (and
Anne's) remain, but the temporal situation is unclear: does the poet
simply assume that Christ has now sworn to save him, or does he defer
this resolution to the future, a future when Christ will clearly
promise, or a future when Donne will die --- at that point, indeed, Donne
will cease to have his earthly and sinful attachments: \emph{no more}.
The irony --- surely not lost on Donne in evoking himself as a silk-work
--- was that, in contemporary sericulture, the worm was killed whilst
still in the chyrisalis by exposure to the sun (\citealt[28]{bonoeil_his_1622};
further, \citealt[159--61]{staples_clothing_2013}; \citealt{hatch_virginia_1957}).

The variant to the final line provided by the O3 and B47 manuscripts
(18.a--b, ``ask'' for \emph{have}) preserves the bathetic suspension;
the poem ends as a prayer; B47, by using a future tense (\emph{I'll}
ask) also makes clear within its own structuring of the poem that the
resolution is cast towards the poet's death. Hilton's variant,
\emph{need} (18.c), works in a similar fashion.\footnote{Given the
  fixation with death in the final stanza, we may pose the question of
  whether the subscription ``finis'' (subscr.a) is an authorial or
  scribal boundary-marker (as suggested above) or is, in fact, an
  integral part of the poem.}

It is, however, with the intromission of \emph{feare} (18.d) as a
variant to the last line that the ending is significantly changed, and
this alteration cannot be seen outside the shift in the addressee of the
poem: God (the Father) rather than Christ; the latter is now the Son who
shall save him at the last, and this recollection (as it should for a
devout believer) will cast aside the fear that formed the final,
seemingly unreconcilable, sin. This theological emphasis is taken even
further in HH6, where the Son does not simply shine down grace upon the
dying Donne but will ``owne my soule and cloath itt evermore'':
\emph{owne} absorbing and neutralizing the various verbal plays on
\emph{have}; \emph{cloath} draping the heavenly garment of salvation
over the thin and final thread of the poet's existence (for the
``garments of salvation'', see Isaiah 61:10 and Matthew 22:11).

In the hands of these copyists, the poem is thus de-Donnified, turned
much more specifically into a ``hymne'' (as in the title given in the
\emph{princeps}), a devotional song to the first Person of the Trinity.
Its underlying intention is to praise the Most High, with the address to
God cast within a recognizable theological framework. The structure of
the poem thus enacts a crisis, resolved into assurance by a theological
reflexion; it evokes anxiety, only to dispel it, and can take the reader
(or, indeed, singer) through an imaginary process of repentance, doubt
and resolution; the very lack of specificity of the sin or sins
mentioned allows the reader--partaker to apply them to her own
biography, evoking a personal sense of sinfulness which may be assuaged
by a re-affirmed belief in forgiveness. Memory can be briefly
reconfigured within a dominant theological paradigm, and emotions,
similar to those assumed to have led to the composition of the piece,
can be conjured within the reader. Within the hymn-genre, then, the
\emph{I}--function of the text is shared between author and
performer--worshipper, or rather the process of enunciation is taken
over by the devotee, using the author's words to express and suscitate
devotion through a form of ventriloquism. That the transformation of
Donne's own idiosyncratic poem into a genre-following Hymn was not a
sudden rewriting, but an example of consistent normalizing pressure on
the text exerted as scribes and printers altered it in their copying,
shows how important ideological expectations can be in transmission, and
how carefully modern editors must tread in discerning how texts may be
transmitted and altered, even as they are copied into authoritative
manuscript collections.

\section{The author and his image}

The evolution of our poem, too, doubtless reflects the on-going
development of the ``author-function'' (\citealt{wilson_foucault_2004}; for England,
\citealt{armstrong_paratexts_2007}) or ``author-image'' \citep{amossy_double_2009} of Dr Donne, famed
divine and sermonizer; and displays how expectations from his output
also developed in tandem with his increasing absorption into the English
church's hierarchy. It is most certainly one thing to read a racy poem
about love, sex, grief and doubtful forgiveness by a famed roué only
recently of the cloth; quite another --- indeed, somewhat difficult,
perhaps repulsive --- to read a racy poem about love, sex, grief and
doubtful forgiveness by an aged and venerable divine, celebrated for his
learning and his lack of public dalliance.

Yet what is also noticeable, however, is that, despite the canonization
of his works in print, and the presentation of a sanitized version of
``A Hymne {[}\ldots{}{]}'' as the culmination and end of Donne's poetry, some
anthologists a few years later still preferred the rawer, less formally
devout versions that are included in H9, O3 and B47. I would not want to
say that these were acts of resistance to a dominant characterization of
Donne; but they do show an interest in Donne's rather more
characteristic and less normative religiosity, perhaps a taste, even,
for uncertainty, doubt and theological conundra.

Yet even with all the editorial development it underwent, the poem still
had the potential to be somewhat disturbing. The copyist of CT1, whom we
met earlier with his (or her) identification of the sin of the first
lines as being ``originall'', placed two widely-known Latin
\emph{sententiae} warning against hypocrisy and duplicity in the space
left at the bottom of the same page: ``Amicis vitia si feras facis
tuae'' (if you bear your friends' vices, they become your own) and
``Simulata sanctitas, duplex iniquitas'' (holiness feigned, doubled
iniquity). The first originally appeared in a classical assembly of
maxims \citep[4]{duff_minor_1934}; the second was much used, and
often attributed to Jerome, or Gregory, or Augustine. It is difficult
not to suspect that these Latinate admonish­ments were criticisms aimed
directly at that roué turned venerable divine.

Such visible discomfort with Donne's persona may explain the really
quite extraordinary lengths to which Walton went in forging a picture of
a devout, upstanding, protestant individual in his successive revisions
to his \emph{Life} of Donne. Our poem was included as an example of
Donne's devout verse from the very first version of the \emph{Life}
(Walton 1640), and was the only poem that made its way into a popular
abbreviation of Donne's biography \citep[120]{winstanley_lives_1687}. But the poem
was introduced as, explicitly, a hymn written during a bout of serious
illness (\citealt[fol. 84v]{walton_lxxx_1640}; Anne More being long forgotten, and
readers certainly diverted from seeing any puns on her name). Such a
connexion proved attractive to devout readers, and a future archbishop
of Canterbury copied out the poem into his commonplace book, prefacing
the verses with ``Dr Donne in his former sicknesse'' (Oxford, Bodleian
Library, ms. Tanner 466, foll. 4v--5r). Crucially, of course, Donne's
doubting that he could be saved was washed away in the version that had
made itself into print, and which Walton used; and Walton made doubly
sure of this pious inflexion to the poem by adding, after the fallacious
setting on a sick-bed, that Donne ``wrote this heavenly Hymne,
expressing the great joy he then had in the assurance of Gods mercy to
him''. The theological, biographical and spiritual problem, which gives
the poem its force, is simply erased.

This was still not enough, and Walton, in a subsequent return to his
biography, bolstered the account with some reported words of Donne ``to
a friend'' (unnamed, of course) after listening to the setting of the
hymn: ``The words of this Hymne have restored to me the same thoughts of
joy that possest my soule in my sicknesse when I composed it'' (\citealt[77]{walton_life_1658}; \citealt[55]{walton_lives_1670}; Walton was not one for varying an idea once he had
seized upon it). Hilton's setting was thus co-opted to characterize
Donne as having a sincere and heartfelt pious sensitivity. And to
emphasise that the hymn was meant to express and awaken ``joy''.

There are thus wider and longer-lived issues at play in understanding
the variance of a poem, issues that involve the evolution of Donne's
verse over time. Evolution not at the hands of the author, but as it was
wrenched from an individual context of a life as it was being lived, to
be placed in either an implicit or explicit idealized biography, the
\emph{image d'auteur} not only mediating how texts were read, but
directly provoking interference in the transmission of those texts
themselves. Despite the valiant efforts of the \emph{Digital Donne}
project, this process has not been fully rolled back, and much modern
criticism of John Donne has been built upon texts which do not reflect
the real Donne, but rather an amenable construct created, in the main,
post mortem. When it comes to seizing ``Donne'' (\emph{image d'auteur},
\emph{fonction-auteur}, man), we have most definitely not done. Finis.

\begin{flushleft}
\bibliography{references/lappin}  
\end{flushleft}
\end{paper}