\contributor{Anne Baillot and Anna Busch}
\contribution{Editing for Man and Machine. Digital Scholarly Editions and their Users}
\shortcontributor{Baillot and Busch}
\shortcontribution{Editing for Man and Machine}


\begin{paper}
\begin{abstract}
This article explores ways in which digital scholarly editions
can reach new audiences by taking advantage of the
computer-readability of their digital content. Based on the development work on the edition \emph{Briefe und Texte aus dem intellektuellen Berlin 1800--1830}, we present different Open Access-based options that allow for interlinking datasets and facilitate the development of digital editions that go beyond what print editions can achieve on paper.
\end{abstract}

\section*{}
\textsc{Digital scholarly editions} of manuscripts oftentimes present themselves in the form of a scan of the manuscript on one side of the monitor and a transcription on the other side.\footnote{See for example: \citealt{van_gogh_vincent_2009}; \citealt{schlegel_august_2014}; \citealt{ghelardi_burckhardtsource_2019}; or \citealt{jung_digitale_2015}. For a catalog of digital scholarly editions consult \citealt{sahle_catalog_2008}
  or \citealt{franzini_catalogue_2016}.} By presenting this view of the text, the edition caters to
a specific way of reading. But the kind of information it conveys is
--- in spite of its apparent similarity --- not the same as what print
facsimile editions enable. Digital editions are compelled to establish
the chronology of textual genesis (a linearity in the representation of
text production made necessary by the text encoding), while a print
facsimile edition is not necessarily bound to produce this kind of
analysis. But although the encoding process suggests a more linear and
in-depth approach to textual structure on this level, one can observe, on
the other hand, a tendency not to render
all glyphs and signs with the same precision. The reasons behind such editorial decisions are neither solely due to
technical challenges (that would impede the digital reproduction of non-textual elements, for example)
nor to a fundamentally lackadaisical consideration of non-textual
elements. Instead, the editor starts from the assumption
that the reader should be able to make the connection between the
scan and the transcription, and cede to the fact that
one is not identical to the other. In other words, the reader's
understanding of the editorial setting is fundamental to the conception of all editorial
decisions that underlie the edition's development.

The potential affordances of the digital medium affect our conception of
textual structure on different levels. This offers editors of digital scholarly
editions the opportunity to break away from the
page format (see also the first chapter of Grafton \citeyear{grafton_page_2012}). Still,  most editors stick to what they know, be that because of the format of the
scans they acquired for the edition, or due to the lack of a persuasive alternative model that would be as easily
accessible to readers (in terms of readability) as the familiar page
is.

What is more, a digital edition is not just (and perhaps not even primarily)
designed for humans to read in a linear way, but also to be browsed through the
interface, and navigated by means of hyperlinks. This type of interlinking is
another level of reading for which it is difficult to anticipate  the reader's behaviour. On that level too, the editor's quest to direct the reader
along to their own interpretation of the text can be as empowering as it
can be restricting.

Last but not least, digital scholarly editions can also address computers --- or at
least those editions that provide access to their source code and enable
text mining. As such, they facilitate distant reading as
well as close reading.

How can we cater to all these different readers and forms of reading?
That is the question this article tries to answer by building on its authors' experience with
developing a digital edition called \emph{Briefe und Texte aus dem
intellektuellen Berlin 1800--1830} \citep{baillot_briefe_2010}. This edition
was developed in the context of a specific funding program, which meant that the project's funding phase could not exceed five
years' time.\footnote{The project is described on \citealt{baillot_berliner_nodate}; the scholarly contributions
  (publications, presentations) are listed in \citealt{baillot_emmy_nodate}. The project lasted from June 2010 to
  January 2016 and was hosted by the Institute of German Literature of
  the Humboldt-Universität zu Berlin.} This limited time frame and funding
resulted in a pragmatic approach to our editorial tasks, and influenced our workflow
accordingly. Our primary goal was to aggregate resources that
would answer a series of project-specific research questions
\citep{baillot_berliner_2014}. But of course, if answering these questions had been our sole aim,
there would have been no need to develop a user-friendly interface for the data, nor
to make these resources available in the form of an Open Access publication. The edition was hence
conceived of as both a research environment for the research group in which it was situated, and as a
scholarly edition that offers resources to the community at
large. This in turn required to think more about how we would define such a ``community at large'': Who are the readers of a digital scholarly edition,
what are their expectations, and to what extent should we make an effort to meet those expectations?

\section{A short tour through the digital edition \emph{Briefe und
  Texte aus dem intellektuellen Berlin 1800--1830}}

The research group ``Berlin Intellektuelle 1800\emph{\textbf{--}}1830''
at the department of Modern German Literature at the Humboldt-University
of Berlin was dedicated to the analysis of intellectual relationships in
Berlin around 1800. At that time, the Prussian capital was an eminent
place of cultural transfer and knowledge exchange. Form and meaning of
the participation of writers, publishers and scholars in public life
defined the main research axis. To this end, the communication strategies these agents
developed, as well as the close connections they established between circles (such as between
universities, academies, literary clubs and salons) were analysed
on the basis of unpublished or only partially published manuscripts.

While letters constituted the main part of the edition's corpus, 
other sorts of texts such as draft manuscripts of literary texts or
lecture notes are presented as well --- this with the goal of cross-referencing
the treatment of particular topics in different genres. The importance
of letters in general as both a source of information and as a textual format with
an inherent literary nature that adopts complex functions was one of the project's
starting points. That is why the edition was called
\emph{Briefe und Texte aus dem intellektuellen Berlin 1800--1830}.

For the project, the research group designed four key questions to analyse the conditions
and developments that were crucial for intellectual communication in Berlin
between 1800 and 1830 as they were presented in these handwritten documents: 1) To what extent did
the establishment of the Berlin University in 1810 contribute to the
intellectual self-conception of scholars and academics who worked as
lecturers?; 2) To what extent did the presence of the French in Berlin
shape and define the political awareness of intellectuals?; 3) Which
communication strategies did male and female writers use to establish
themselves in the literary milieu?; and 4) How can we extract political statements from a
literary or scholarly corpus?\footnote{The complete list of research
  results (publications, papers, etc.) of the research group ``Berlin
  Intellektuelle 1800--1830'' can be found on \citealt{baillot_emmy_nodate}}

A few months into the project, the research team decided to deviate from its initial plan to publish of a series of smaller print editions, and to 
development an overarching digital scholarly
edition instead. To this end, the team reviewed several digital scholarly
editions of letters that could serve as examples to follow. Great
inspiration was drawn  from ``Vincent van Gogh. The
Letters'' \citep{van_gogh_vincent_2009} and ``Carl Maria von Weber --- Collected Works
(WeGA)'' \citep{von_weber_carl-maria-von-weber-gesamtausgabe_2017}. In many ways, these editions helped the research group develop and implement their editorial practices. In addition to the best practices that were adopted by these digital editions, the editorial team
implemented the standards recommended by the German Research Foundation
(DFG), the project's funding agency for scholarly, web-based text
publications \citep{dfg_informationen_2015}.

Not focusing on a single author, as most editions
(even digital ones) do, the edition provides several possibilities
to approach the manuscripts. This variety of options allows the user to access
the edition in different ways, of which the one via the author-guided approach is just one alongside many. As such, the edited letters and
texts can be accessed via their genre (letter, drama, novella,
reports, lecture notes, etc.) or through a series of topics that are structured according
to the team's four core research questions.\footnote{The
  complete editorial guidelines including all specifications for the
  different types of text genres can be found in German in \citealt{baillot_berliner_2010}.}

The standard view in the editorial interface shows a diplomatic transcription opposite a scan of
the manuscript. These two columns on the screen can be
reduced to a single one in just a click. For each column, it is possible to
visualize the scan, two versions of the document's transcription (either as a diplomatic transcription of the document, or as an edited reading version of the
text), the source code, the metadata, or the identified
entities. It is also possible to generate a PDF of each document.

The diplomatic version provides a transcription of the manuscript's text with as little editorial interference as possible. All
corrections, deletions and additions that are found in the manuscript are
reproduced. Characteristics such as line breaks, the horizontal alignment of
paragraphs, or abbreviations are all retained, and missing parts of the
text are not reconstructed. Therefore, the diplomatic transcription is
suited for textual analysis, interpretation, and investigating the text's genesis.\footnote{Due to the time and funding
  constraints, the TEI encoding that was used in the project does not follow Critical Apparatus model \citep{consortium_critical_2020} --- not even in the case of genetic phenomena. Instead, a project-specific encoding
  was developed that combines a light encoding of genetic phenomena in
  combination with named entities. A project-specific (not completely
  TEI-compatible) \lstinline[language=XML]!<hand>! element  that refers to person
  entities allowed the team to connect both encoding levels (genetic and
  entity-based). The complete project-specific TEI encoding guidelines in English
  can be found in \citealt{baillot_edition-specific_2016}.}

In contrast to the diplomatic transcription, the reading text focuses on readability. While it still provides a reliable textual basis like the diplomatic transcription does, this version aims
to present an easy-to-read view of the transcription that offers a quick
point of reference. This is especially helpful when the document in question contains a lot of
deletions and additions. In this view, the focus lies solely on the ``basic
text'' in the author's hand --- all other hands that might intervene in the
manuscript (such as those of editors, recipients, or archivists) are omitted.
Corrections, deletions and additions are tacitly integrated into the text,
original line breaks are ignored, and abbreviations are written in full.
Parts of the text that are missing but that the team was able to reconstruct with a high
certainty are supplied in brackets. The text is also supplied with annotations,
so as to address different editorial formats: on the one hand, these annotations
reflect on the subtleties of the manuscript and the text; on the other, it offers details concerning the document's wider (and especially
historical) context.

All six of these HTML views (and their downloadable PDF equivalents) are generated from the same TEI-XML file.
Each TEI file contains the transcription of a single letter or
manuscript, its markup and annotations, and its metadata. Additional
information on various entities (persons, groups, places, and publications)
are brought together in one TEI file per entity type, and connected to
the text files via project-specific identification numbers. These TEI
files are the basis for the metadata view, the listing of
entities, and the indexation of entries. Each of these aspects are represented in the encoding
of these elements.

All the different formats the project offers to visualize each text are also available to the reader for download and reuse
under a CC-BY license. In addition, the reader has access to all of the project's
relevant metadata, and to the indexed entries: this includes information on senders and
addressees (in the case of letters); the manuscript's origin, provenance, and current holding
repository; the editors that were involved, etc.

Overall, this edition is not radically different from the major current
digital scholarly editions. What may set it apart, though, is that it is not
focused on a specific author or genre, but rather on a historical context. Its
 qualities lie mostly in the presentation of manuscripts that are chosen because they are
relevant to specific research questions, and in the combination of a
series of features (be they structural, or on the level of the interface)
that address these questions. Developed with four target audiences in mind
(i.e. the research group itself; a broader scholarly audience; the community at
large; and algorithms designed to harvest open data), the edition shows the potential
digital scholarly editions have for opening up their data --- as well as what some of the limitations of such openness can be. Is it even possible to develop
an edition that would be as usable by a knowledgeable scholar as it would be by
a computer?

\section{Editing for man or for machine?}

In general, scholarly (print) editions are developed by scholars, for
scholars. This is especially true in the German context in which the
\emph{Briefe und Texte} edition was developed. Both the format and the price of such editions
make it almost impossible to reach a wider audience. Knowing that one's
edition will be primarily (and most of the time, \emph{exclusively}) used by
scholars who consult them in libraries, editors tend to design their
editorial practices according to their own needs and habits.

Digital scholarly editions, on the other hand, are quite different from most print
editions in terms of their accessibility. While some editions are
password-protected or hidden behind a paywall, many of them are
available in Libre Open Access \citep{suber_open_2012}. This means that virtually everyone is able to
access and use the digital scholarly edition. It does not mean that these editions will automatically have a
significantly wider audience, nor that they were even
developed with such a wider audience in mind. It \emph{does}, however, change the premise of these editions,
in the sense that they may afford editors with the possibility and legitimation to address such a wider audience. But
does the simple fact that the editions are available actually make them
``accessible'' to that wider audience?

It is in fact more difficult to define the expectations of a
non-scholarly audience than those of a scholarly audience. Trying to
reach a wider audience is a business for which scholars are not properly
trained. And what is more: this type of work is often unproductive in
terms of career benefits --- at least in the German academic context. On the
contrary, a higher complexity is usually worth more in terms of academic
reputation and capital than a greater accessibility of the
research results would be. Hence, the aspiration to address a reader other   than
the editor's peers remains mostly unsupported by the academic system in
terms of career evaluation, funding, and training.

Our decision to offer the user a reading version that is not presented as hierarchically
inferior to any of the the other versions represented a major shift for the research team. What especially
mattered during the development of this
digital edition was that we would make all six visualizations
equally easy to reach as the digital image of the manuscript. This included our decision not to transcribe the text at the sign level (e.g. by leaving some
glyphs unrepresented), but instead to give readers the freedom
to establish the connection between the digital image and the
transcription by themselves (as mentioned in the introduction above). The
hermeneutical relevance of this decision is obvious when one realizes
its implications. It means that each reader can bring their own
reading habits to the table. In spite of these varying preconditions, each reader
should still able to make sense of the edition by themselves. This
freedom is based on the assumption that the reader activates their own education and a
critical thinking in the Kantian sense. In addition,  this decision also has more political implications,
since the documents that are displayed in the archives are not
considered any less important or relevant than the transcriptions 
the editors (who are scholars) are presenting to the reader. This act of putting
archival resources and scholarly interpretations on the same level is in
itself already a statement. Finally, these decisions turned out to also have a direct influence on
our choice of corpora, since not all archives would allow us to publish
high quality digital images of their manuscripts in Libre Open Access.

These hermeneutical and, in a wider sense, political choices allowed us to present our edition as a tool that aims to
enable readings of all sorts; to provide a structured space that empowers readers by giving them access to the text, and information to structure themselves. Of
course, the reader is still directed, even in this setup --- indeed, it would be delusional to think that it is possible to set up a ``neutral'' edition that allows for all possible readings. However, \emph{Briefe und Texte} tries to offer its
readers all the critical tools and elements they need to embark on an autonomous reading.
The attempt to offer such an ``enabling'' edition (meaning: an edition that \emph{enables} the reader
to act as the designer of their own readings) is not new, but it does
require us to take into account some specific aspects with regard to digital
scholarly editions.

As editors we anticipate reading scenarios and use them as the basis
for developing the edition's design, drawing mental maps of pathways that can lead the reader to the
text. Here, those might include option such as: searching for a specific author, integrating data for a metadata aggregator,
presenting connections as a network of persons,\footnote{This has become the trademark of
the Schlegel edition in the German-speaking area since \emph{Briefe und
Texte} was developped.} etc. In the case of this edition, however, funding and time constraints
limited our implementation of such reading paths. Taking the conditions in which the edition was
produced into account, the result is overall satisfactory. But the edition is still not fully
satisfactory when it comes to reader-friendliness. Our commitment to accommodate monitors of different sizes and resolutions, 
for instance, required us to fix some
elements (especially frames) in  a way that 
 lacks fluidity. This means that some frame elements appear too dominant on the
vast majority of monitors. This problem could not be overcome in this
funding period. 

The most straight-forward approach that a reader can take to discovering the contents of the
edition, namely by finding the homepage and navigating down the tree structure of data
from there, is a key area where editors can work on building a rapport with their
readers. How do you grab and retain the reader's interest to do so? 
In general, editors are still in dire need of a set of standards that would
give the kind of direction the scholarly book tradition would provide them with.
To start, there are still no established (meaning: widely recognized and actually
used) names for the different types of digital scholarly editions that
exist, and that could contextualize the edition when the reader reaches its
homepage. Most editions usually give themselves a name
from the point of view of their subject, rather than from that of their
method. From the onset, this forms an impediment to establishing a clear reading path for the reader.

But even if editors assumed correctly what the reader's first reaction to the homepage of their edition would be,
it remains impossible for them to anticipate exactly how the reader will move on from there in all possible circumstances. 
It was only after a few years of development and daily
usage that the editorial team of \emph{Briefe und Texte} realized how
unsuitable the edition's design (with its columns and additional
information) is for actually \emph{reading} the text. It is
uncomfortable to read text in HTML, and even more uncomfortable when
additional information pops up around it. The idea to offer a PDF version
alongside the six different HTML displays and the query interface
emerged from the diagnosis that the online edition in itself was not an adequate document for
extensive reading. After making this realization, the team made sure that the reader could download a PDF version for each individual
document, or for a whole corpus. As such, the edition's new PDF generator became a useful way for making the edition more readable.

More could yet be learned about our audience's reading habits
through an analysis of the edition's log-files (as organized
by Anna-Maria Sichani \citeyear{sichani_beyond_2016}). Such an analysis can be undertaken to find
out how people understand and interact with Open Access policies by investigating, for example, how
many users actually use the ``Download XML'' option that \emph{Briefe und
Texte} offers, or how many XML files are downloaded per user
during a certain time frame. This research was inspired by Peter Boot's
seminal study of the log-files of the Van Gogh edition \citep{boot_reading_2011}. His
study mostly confirms well-known reading attitudes: namely that simple queries are used
much more often than advanced queries; that links play a major role in the
way the reader accesses information; etc. It was not easy, however, to extrapolate useful information
from Boot's analysis of the Van Gogh Letters, and translate it into the context of our \emph{Briefe und Texte} edition. 
In his study, Peter Boot demonstrated for example that the first and last letters that are presented on
a web page are those that are consulted the most. Another major anchor point for the users in the case of the Van Gogh edition is to look for
famous paintings. Based on these results, however the question remains how the findings can be used in an edition that does not revolve
around a single author, or a single  corpus --- but that instead decidedly aims to deconstruct
canonical approaches to the history of literature? The results of this
analysis were certainly still useful, if only because of what it
taught us about the possible expectations of the readers in general, and to answer questions such as: To what
extent does the structural design of \emph{Briefe und Texte} differ from that of
other editions? Are there constants in the way readers approach editions, or does their approach
strongly depend on the way the edition is designed, or even on
the edited object? Is it possible to distinguish different reader types or reading
patterns from one another, and if so, what can they tell us about the interest readers are giving
to specific editions, or to editions in general? Although these questions may seem to deal solely with implementation issues, they are in fact representative of
a major question regarding the relationship between the edited text and its
reader, namely: Which role does the editor of a digital scholarly edition play
in guiding the reader's interaction with the text? And how do we balance text with design?

At this point, it is important to reflect on the conceptualization of the
relationship that is to be analysed here. ``Reading'' covers only part
of the way text is accessed by the audience of a digital edition. In
this case, the term ``using'' is in many ways more adequate to the
multiplicity of approaches made possible by the digital media. Firstly, it
can allow us to distinguish between linear reading (reading) and non-linear
reading (navigating, for instance scrolling, as a relevant form of use).
Secondly, it allows us to take design elements into account without necessarily
contrasting them to textual elements. The following section will deploy
a series of additional arguments for this approach to a (positive)
understanding of the ``use'' of a digital scholarly edition.

\section{Accessing the edition from outside the edition}

In a digital scholarly edition such as \emph{Briefe und Texte}, the reader's access to the
text is enabled through different entry points (such as genres, authors, etc.),
but also by the reader's queries (and their corresponding interfaces), and by the hyperlinks it establishes, both within the
edition and outside of it. Each of these ways to access the text
questions the concept of the ``reader''. Specifically in the case of correspondence editions,
dedicated digital tools have also been developed to facilitate these
different types of access to their text, and to offer readers a new level of
user-friendliness.

The accessibility of the edition's data within \emph{Briefe und Texte} ---
and the way it enables its \emph{use} or \emph{re-use} in terms of interoperability --- is ensured by the
implementation of authority files and standards such as: the Integrated Authority
File (GND) for persons \citep{deutsche_national_bibliothek_integrated_2016}; GeoHack for
places \citep{mediawiki_geohack_2020}; XML and TEI for text encoding \citep{consortium_tei_2020}; ISO standards \citep{iso_standards_nodate};
and by making the edition's source code available in Open
Access.\footnote{The edition is licenced under a CC-BY 3.0 license for the editorial work. Each
  of the Cultural Heritage Institution that allowed us to reproduce scans
  of manuscripts specified their re-use conditions, which are mentioned
  below the relevant image.} In addition, it was also crucial to connect the edition with other
repositories and editions, which we achieved in a first stage by
implementing a GND BEACON \citep{wikipedia_wikipediabeacon_2020}. With this simple file format hosted by Wikipedia, it
is possible to link content to one another based on their GND numbers. 
In the case of \emph{Briefe und Texte}, we used this system to connect with historical
agents. When a person appears in \emph{Letters and Texts} (be it in
edited texts, annotation, or metadata) that any other BEACON-using project records as well, 
the system will automatically provide a direct link to the
respective page of this resource. In addition, the other way around,
other resources using this technology will automatically be linked to our \emph{Briefe und
Texte} edition. The German National Library and many regional libraries,
archives, biographical and bibliographical projects, and many others use
the BEACON format. This makes it an easy way to connect the
contents of digital resources with a wide range of scholarly web
services.

Another way to access the information in \emph{Briefe und
Texte}, is at the document level. On this level, all the documents that are edited in \emph{Briefe und
Texte} are also listed in the national German
database and national information system for collections of personal
papers and handwritten manuscripts from German archives and
institutions. This database is named \emph{Kalliope} (\url{http://kalliope-verbund.info/de/index.html}), and points each of these documents
to the corresponding permanent URL in \emph{Briefe und Texte}. This allows a user
to browse through the repository of their favourite small archive
looking for a specific letter, and from there to be guided to its facsimile,
transcription, and annotation in \emph{Letters and Texts} directly from the
aggregated metadata in \emph{Kalliope}. In addition, the
editorial metadata in \emph{Briefe und Texte} also includes links to
\emph{Kalliope} in the other direction, thereby guaranteeing reciprocity in
the way the information is connected. Here, the efforts of cultural heritage institutions 
to catalogue and index data are joined with those of the scholarly editing and research communities, 
in an attempt to venture beyond the archival connection between image and metadata, and instead to include 
also edited transcriptions that are accessible straight from the archival catalogue through hyperlinks.

Finally, more advanced forms of access have been developed specifically for
letters and correspondence since the start of our work on the \emph{Briefe und Texte}
edition. Letters offer particularly interesting opportunities for
interconnecting textual data via the digital medium. One of the developers
of \emph{Briefe und Texte}, Sabine Seifert, is an active member of the
TEI Special Interest Group Correspondence (SIG Correspondence). This
group worked especially on correspondence-specific metadata within the
TEI's \lstinline[language=XML]!<teiHeader>! element working to generate more
interoperable data, and to open TEI-encoded correspondences up to new,
automated uses.\footnote{For the proposal on GitHub, see \citealt{tei_correspondence_sig_correspdesc_2019}; 
in the TEI guidelines, see \citealt{consortium_tei_nodate}. See also Stadler et al. \citeyear{stadler_towards_2016}.}

With regard to encoding correspondence metadata, the SIG Correspondence proposed a
new element called \lstinline[language=XML]!<correspDesc>! (correspondence
description) that contains core correspondence-specific information
that is mentioned on letters or any other piece of correspondence. A more restricted form of the
\lstinline[language=XML]!<correspDesc>! element called the Correspondence Metadata Interchange (CMI) format was then developed to serve 
as the basis for even more standardized metadata exchange. To maximize its interoperability, CMI relies on authority files and standard formats so that it can still meet
with the naturally diverse encoding methods of various letter-based editions and
projects. The web service \emph{CorrespSearch} 
harvests metadata of letters that are 
based on the CMI format and makes the correspondence-specific metadata
of different German-language correspondence editions searchable with a single
query (\citealt{dumont_correspsearch_2020}; see also \citealt{dumont_correspsearch_2018}). For an edition like \emph{Briefe und Texte}, using the CMI framework and
being integrated to the \emph{CorrespSearch} platform provided us with
an additional entry point at the document and author levels for each
letter that is external to the edition itself. This represents one more way to
attend and address different methods the user may employ to access the text and its context (metadata,
annotation). In addition, it also offers the edition a way of making its contents available to
computers, as the edition's standardized metadata become machine-readable. This is especially relevant in the
case where integrating the \emph{GND Beacon}, \emph{Kalliope}, and
\emph{CorrespSearch} serves as a connection between different digital resources. It is
easier to implement such connections on the level of metadata, as it is easier to standardize
this type of data than it is to standardize text annotation. In return, these
metadata then offer the user access to
much deeper information in the form of the fully edited and annotated
digital edition.

To conclude, we found that the best way to address the plurality of users is to break the edition down into different
levels of granularity and units of text conceived as data and
metadata, so as to present the user with a door into the editorial design of each of
these levels that users with different interests are likely to open --- be they archivists, scholars, avid letter readers, or even computers.
The primary architecture designed for the
\emph{Briefe und Texte} edition answers a specific set of research questions. But it also
leaves room --- even taking its limited funding and time into account --- for offering
additional forms of access to users. So while there is one primary way
of reading the text, the edition offers many alternative ways for the user to access it. As such, the  \emph{Briefe und Texte} edition offers
an experimental step in the process of opening up our editions in a way that helps us to
conceive of digital scholarly editions that move beyond the boundaries what print editions have to offer.

Anticipating multiple usability scenarios and
implementing them in a digital edition offers us multiple ways to access the text.
Of course, not even the edition presented here could perfectly anticipate
all user requests, but at least it may provide users with the necessary information to
access the information they need by themselves. As such, the
edition gives users something that a print version cannot, namely
a user-generated and user-specific presentation of its materials. The
resilience of an edition will always depend on how comprehensible its research results are --- mainly in terms of the quality of its scans and the transparency of the editorial decisions that were made, but also by providing detailed links embedding the edition in a relatable scientific context.
In a digital edition, any editorial uncertainties in the wording and syntactic or genetic classification of the
edited material are immediately apparent. While this enables the user to make their own independent
decisions while reading the edition, it also transfers a certain level of philological
responsibility to the user. This democratization of the edition, facilitated by the digital medium, frees the edition from the
ivory tower of philology, where it was allowed to blossom into its most
specialized form for so long. Now, it is up to our users to exploit the new
possibilities these digital scholarly editions have to offer.

\begin{flushleft}
\bibliography{references/baillot}  
\end{flushleft}
%\newpage
\end{paper}