\contributor{Ronald Broude}

\contribution{Creators' Intentions and the Realities of Performance: Some Exemplary Editorial Problems from the Savoy Operas}

\shortcontributor{Ronald Broude}

\shortcontribution{Creators' Intentions and the Realities of Performance}

\begin{paper}

\begin{abstract}
When playwrights, composers and choreographers bring scripts, scores, and choreographies to first rehearsals, these are provisional documents that still need to be tested, transformed into audible sounds and visible movements. Works that are successful enough to be revived go through similar processes of testing and adjustment for each new production. In no genre are such adjustments so many or so far reaching as in opera. The Savoy Operas --- the collaborations of Gilbert and Sullivan --- constitute an excellent repertoire in which to study the dynamics of this process, for their genesis and performing history can be traced in a wealth of documentation. Although performances were controlled for three quarters of a century by the D’Oyly Carte Opera Company, a troupe committed to performing these works exactly as their creators had intended, numerous adjustments were made over the years.  Because few good performers will regard any text as altogether binding, editors should supplement their clear texts with enough information, provided in critical commentary and appendices, to enable performers to exercise intelligently the freedom they will almost certainly claim.
\end{abstract}

\renewcommand*{\thefootnote}{\fnsymbol{footnote}}
\section*{} 
\textsc{The scripts, scores, and choreographies} that playwrights, composers and
choreographers bring with them to first rehearsals are necessarily provisional documents.\footnotemark[1]
\footnotetext[1]{\emph{Editors' Note:} The editors of this issue have exercised their editorial prerogative to redact all racial slurs from this essay. This includes those cases where they were quoted verbatim from the original source text (see page \pageref{nword} below). These changes have been made with the author's consent. For a more detailed justification of how and why these redactions took place, please refer to our editorial \hyperref[preface]{preface} in the current issue (esp. pages \pageref{nword:preface:start}--\pageref{nword:preface:stop}).}
\renewcommand*{\thefootnote}{\arabic{footnote}}Those documents represent the creators' original
intentions, but only when those intentions have been tested in
rehearsal, transformed into audible sounds and visible movements, can
creators know if their conceptions are effective onstage.\footnote{For a
  perceptive discussion of the transformations undergone by a
  playwright's script during rehearsals, see Philip Gaskell's study of
  the textual changes in Tom Stoppard's \emph{Travesties}  \citep[245--62]{gaskell_writer_1978}.} Invariably, adjustments must be made to meet the
realities of performance with specific performers, specific venues and
specific audiences, and the creators' pre-rehearsal intentions are
modified accordingly. The process often extends beyond rehearsals to the
days or weeks following opening night; when all adjustments have been
made, the work is said to have been ``settled'' or ``frozen''. If a work
is successful enough to merit successive productions, the revivals
undergo a similar process of revision. Creators sometimes oversee
revivals of their works, but often revivals take place without them,
either because they are otherwise occupied or because they are dead. In
the absence of a work's creator, new adjustments are made by those
involved in the new production. Sometimes, such adjustments become
firmly fixed elements of performing traditions, so that even
knowledgeable audiences are unable to distinguish them from what was
present in the original production.

These aspects of performing works raise for editors a variety of
problems not usually present with works meant to be read by their
audiences directly off the page, works such as novels, poems and
essays.\footnote{For a discussion of differences between ``page texts''
  (texts meant to be experienced by audiences reading them directly off
  the page) and ``stage texts'' (those of performing works, the texts of
  which audiences need never see), see \citealt[passim, but esp. pages 24--25]{broude_performance_2011}.} The Savoy Operas\footnote{`Savoy Operas' is the collective
  name given to the thirteen collaborations of librettist W. S. Gilbert
  (1836-1911) and composer Arthur Sullivan (1852-1900) for which both
  music and libretto survive. The thirteen are \emph{Trial by Jury}
  (1875), \emph{The Sorcerer} (1877), \emph{H. M. S. Pinafore} (1878),
  \emph{The Pirates of Penzance} (1879), \emph{Patience} (1881),
  \emph{Iolanthe} (1882), \emph{Princess Ida} (\citeyear{gilbert_princess_1884}), \emph{The Mikado}
  (1885), \emph{Ruddigore} (1887), \emph{The Yeomen of the Guard}
  (1888), \emph{The Gondoliers} (1889), \emph{Utopia, Ltd.} (1893), and
  \emph{The Grand Duke} (1896). The operas take their name from the
  Savoy Theatre, expressly built in 1882 by Richard D'Oyly Carte to
  house productions of Gilbert and Sullivan's operas, which by that time
  had become an institution. All the operas that opened before 1882 had
  their premieres at other theatres, although all thirteen were at some
  time performed at the Savoy. A fourteenth work, Gilbert and Sullivan's
  first collaboration, \emph{Thespis} (1871), survives only in part: the
  libretto was printed and so reaches us in its entirety, but only
  snatches of the music have been preserved: ``Climbing over Rocky
  Mountain'' was later recycled in \emph{Pirates} (1:5). Since Sullivan
  was frugal with what he composed, it seems likely that other music
  from \emph{Thespis} may have been recycled. On \emph{Thespis}, see \citealt{rees_thespis_1964}.} --- the collaborations of librettist W. S. Gilbert and
composer Arthur Sullivan --- provide an unusual opportunity to study the
dynamics of performing works over a substantial span of time. They were
performed at first under the direction of the creators themselves and
later, after the creators' deaths, under the control of a single
organization --- the D'Oyly Carte Opera Company --- committed to
performing the works exactly as their creators had intended.\footnote{For
  a history of the D'Oyly Carte Opera Company, see \citealt{joseph_doyly_1994}. The
  casts of 86 years of D'Oyly Carte productions are catalogued in \citealt{rollins_gilbert_1962}. For a recent memoir of the last years of the
  Company, see \citealt{mackie_nothing_2018}.} The D'Oyly Carte Opera Company was formed
by Gilbert, Sullivan and Richard D'Oyly Carte, the impresario who had
brought librettist and composer together and whose diplomatic skills had
kept them so, notwithstanding differences in personality and artistic
aims. The thirteen works for which both music and libretto survive were
composed between 1873 and 1896, and until 1960, when international
copyright in them expired, they were maintained in repertory by the
D'Oyly Carte Opera Company, which tightly controlled both its own
productions and productions by other organizations to which it rented,
under strict licenses, the performing materials needed to stage
full-scale productions. The D'Oyly Carte Opera Company was founded in
1879, when London had two competing productions of \emph{H. M. S.
Pinafore}, Gilbert and Sullivan's fourth collaboration and first real
hit; both then and later (when pirates in America were mounting
unauthorized productions), D'Oyly Carte defined itself as the
organization that performed the operas exactly as their creators had
intended.\footnote{On the origins of the D'Oyly Carte Opera Company and
  its establishment as the organization performing the operas in
  accordance with the creators' wishes, see \citealt[Part B, pages  14--15]{young_introduction_2003}, and \citealt[pages 21--24]{joseph_doyly_1994}.} Gilbert and Sullivan, both known
as sticklers for detail, had shaped the original productions, and the
performing traditions that the D'Oyly Carte Company maintained until its
unhappy last days in the 1980s therefore went back in an unbroken line
to the performances that Gilbert and Sullivan themselves had directed.
Yet notwithstanding the efforts to maintain stability, the operas
changed over the years.\footnote{Opera is a notoriously unstable genre,
  probably because so many interests figure in productions: there are
  composers, conductors, singers, instrumentalists, librettists,
  directors, choreographers, and scene designers --- not to mention the
  producers who look after financial concerns. On the instability of
  operatic works, written by a scholar concerned with both texts and
  performance, see \citealt{gosset_divas_2006}.} The following paragraphs are intended
to look at some of those changes and to raise some questions that those
changes imply.

Until 1958, when Reginal Allen's \emph{First Night Gilbert \& Sullivan}
appeared, Savoyards' interest in textual problems was lively but
unfocused \citep{allen_prologue_1958}.\footnote{Allen was both an insider --- he had
  secured the confidence of the D'Oyly Carte board, of which he became a
  member --- and something of an outsider: an American collector, with
  bibliographic interests, and an awareness of the textual problems from
  which Gilbert's libretti suffered. His attempts to establish the text
  for the first night of each opera, although not always successful,
  constituted an important step in addressing the textual instability of
  the libretti, and the collection of libretti, vocal scores, and other
  material that he assembled and donated to what is now the Morgan
  Library and Museum (formerly the Pierpont Morgan Library) has proven
  an invaluable resource to subsequent editors. On the state of Gilbert
  \& Sullivan textual studies in the 1950s, see Allen's Prologue, pages
  xvii-xviii. Allen's efforts were confined to the libretti; with the
  exception of the vocal scores, the musical sources remained largely
  inaccessible and problematic.} Both performers and Gilbert \& Sullivan
enthusiasts were aware that there were textual problems, that as the
years and decades passed not everything that was being performed had
been intended by the creators, but the ability to address these problems
was hampered by the inaccessibility of sources. D'Oyly Carte jealously
guarded the few full scores in its possession (productions were
conducted from vocal scores),\footnote{Full scores give all the notes sung
  by all the singers (principals and chorus) and played by all the
  instrumentalists, while vocal scores give all the singers' parts but
  reduce the orchestral parts to the two staves of a piano score; vocal
  scores therefore provide much less information than full scores. On
  the problems posed by the Savoy Opera sources, see \citealt{broude_gilbert_2008}.} and
the unauthorized published full scores in circulation had been scored up
from unreliable band parts; the texts of libretti had been uncritically
reproduced for trade editions from a variety of early sources. The
publication of a critical edition of the operas (the first volume
appeared in 1993) and of Ian Bradley's texts of the libretti, \emph{The
Annotated Gilbert \& Sullivan}, were historicizing editions in the sense
that they sought to present texts corresponding to specific states in
which the edited works had existed, and that they were keen to explain to
users the processes by which those texts had changed over
time.\footnote{The first volume in the critical edition was \emph{Trial
  by Jury}, edited by Steven Ledbetter \citep{gilbert_trial_1993}.
  \emph{The Annotated Gilbert and Sullivan}, cited herein as \citealt{bradley_ian_complete_1996}, first appeared in two volumes, the first issued in 1982 and the
  second in 1984; \citealt{bradley_ian_complete_1996} combines the contents of the earlier two
  volumes and provides new material.} The implications of the sorts of
changes discussed in the following paragraphs extend, of course, beyond
the Savoy Operas; every performing genre, whether of drama, dance, or
music, responds to forces that are unique to it, but there are also
certain similarities in the ways all performing works are created,
performed, transmitted, and experienced.

An important category of textual problems presented by the Savoy Operas
requires the editor to distinguish between two sorts of adjustments made
during first productions. On the one hand, there are compromises made to
meet the sorts of unique exigencies that will arise in any production
but that are unlikely ever to arise again. On the other hand, there are
adjustments made for the larger purpose of improving the work.
\emph{Princess Ida}, Gilbert and Sullivan's eighth collaboration, which
opened in 1884, provides a good example of a first-production-only
problem.

Shortly before \emph{Princess Ida} was to go into rehearsal, the tenor
Henry Bracy was signed to sing the part of Prince Hilarion, the male
romantic lead in the opera. It is a convention in this class of work
that when the male romantic lead is a tenor, as he very often is, his is
the highest male part in respect of range. This is the convention upon
which Sullivan relied when he composed No. 20, the first to be composed
of the three pieces in which Hilarion sings with his friends and
confidants Cyril (another tenor) and Florian (a baritone).\footnote{References
  to act, number and measure are to Richard Sher's edition of
  \emph{Princess Ida}, Gilbert \& Sullivan \citeyearpar{gilbert_princess_2021}, and to the original
  vocal score, Gilbert \& Sullivan 1883, first edition, second issue
  (first English issue), which agrees with Sher's edition with respect
  to act and number but which lacks measure numbers. On these events,
  see \citealt{gilbert_princess_2021}.}

Hilarion was Bracy's first role in a Savoy Opera, and it would be his
last, for it soon became apparent that he did not have the high notes
expected of a high tenor. Derward Lely, the Scottish tenor who sang
Cyril, did have those high notes, and so, in the ensemble numbers for
Hilarion, Cyril, and Florian that Sullivan composed after Bracy's
shortcomings had become apparent --- Nos. 12 and 13 --- Cyril (rather than
Hilarion) was given the highest lines. But No. 20, the Finale to Act 2,
which Sullivan had composed the previous summer, contained a
dramatically important solo passage for Hilarion in which, at measure
95, Bracy was required to hit and hold a B-flat above middle C, a note
that a high tenor should have been able to manage with ease. Sullivan,
however, lacked confidence in Bracy's ability to sustain that note
consistently, and so he rewrote Hilarion's line, lowering the B-flat by
a full tone to A-flat, which he trusted Bracy to negotiate. The
orchestral accompaniment was adjusted accordingly. The process of
revision can be traced in Sullivan's holograph.\footnote{See \citealt[fol. 118\textsuperscript{r}]{sullivan_princess_1883}, the tenth measure on the page.}
Presumably, the A-Flat is what Bracy sang in the performances in which
he participated.

However, in the piano vocal score of \emph{Princess Ida}, which was
prepared by George Lowell Tracy and engraved before the B-flat had been
lowered to an A-flat, the B-flat remains, unaltered. And that B-Flat is
what has been sung by Hilarions down to the present day. Now should an editor preparing a critical edition of \emph{Princess Ida}
print the B-flat or the A-flat?

Two approaches supported by familiar (though differing) theories suggest
that it is the A-flat that an editor should print. If an editor adheres
to the view that his or her edition should represent the composer's
latest intentions, then it is the A-flat that clearly represents
Sullivan's latest intentions. On the other hand, if an editor takes a
social approach, then the A-flat represents the socialization of
Sullivan's text, its accommodation to the realities of the institution
for which and within which it was created. But this is really not a
matter of latest intentions or socialized texts.

In deciding what to print in such a case, a conscientious editor will
probably consider several factors. First is what may be called the
narrative, i.e.: the sequence of events that has led to the change,
including the creator's original intentions, the creator's revision, and
the circumstances that led from the former to the latter. Then there is
the evidence of the textual and of the performing traditions --- two
traditions that may not always be in agreement. Finally, there are the
relative artistic merits of the original and revised states. In theory,
this last is not a factor that should influence a critical editor, who
should be concerned not with artistic quality but with what a text
was --- or was intended to be --- at some specific moment in the past. But
where the difference in quality seems significant, artistic merit can
sometimes influence an editor's decision.

In the present case, the narrative is clear: Sullivan originally
intended the B-flat, and changed it only because of a specific singer's
limitations. As for the textual tradition, the B-flat was retained in
the vocal score, even though it could have been changed (although the
degree of Sullivan's involvement in the preparation of the vocal score
is uncertain). The performing tradition is that of the B-flat. And,
happily, Sullivan's original B-flat is musically superior to the altered
line he contrived for Bracy. Richard Sher, the editor of the critical
edition of \emph{Princess Ida} \citep{sher_critical_2021},
prints the B-flat, and discusses the history of the readings in his
critical apparatus.

But the case of Bracy and Hilarion is quite a simple one. The problem
becomes more complicated when the effects of a change made to address
the circumstances of a specific production are not just local or when a
change seriously compromises the artistic quality of a work as a whole.
\emph{H. M. S. Pinafore}, produced in 1878, is an opera in which major
structural changes made to accommodate a particular performer had
unfortunate consequences.

In \emph{Pinafore}, Sir Joseph Porter, the Admiralty's First Lord, has
inappropriately fallen in love with the much younger --- and socially
inferior --- Josephine, who is the daughter of a mere captain. Josephine
has also fallen in love with someone socially beneath her: a sailor
serving aboard the ship her father commands. In the end, when Josephine
gets her sailor and all the principals are being paired off, Sir Joseph
consents to marry his cousin Hebe. As the published texts stand, his
seems a curiously arbitrary decision: Hebe has been a relatively minor
character, who, unlike the other principals, has neither solo nor
dialogue, and who participates in the Act 1 finale for no apparent
dramatic reason.

The role of Cousin Hebe was written for a popular performer, Mrs. Howard
Paul. As originally conceived, the part was a rounded, comic role:
Cousin Hebe wants to marry Sir Joseph herself, she does everything she
can to discourage Sir Joseph's infatuation, and she encourages Josephine
to elope with the sailor whom Josephine loves (hence her presence
onstage during the Act 1 finale, when plans for the elopement are being
laid). The part as originally conceived is to be found in the licensing
copy sent to the Lord Chamberlain three weeks before opening
night \citep{gilbert_h_1878}.\footnote{On the dismissal of Mrs. Howard Paul and
  the adjustments made to accommodate Jessie Bond, see \citealt{young_introduction_2003}, Part
  B, pages 14-15. These events are also recounted by Jessie Bond in \citealt[pages 58--61]{bond_life_1930}, though Bond's objectivity may be questioned.}

Shortly before opening night, however, Mrs. Howard Paul was dismissed,
and it was necessary to find a replacement for her. The replacement whom
Gilbert and Sullivan chose was Jessie Bond, one of the few performers in
London able consistently to get her way with the famously irascible
Gilbert. Bond was a singer, not an actress, and she agreed to take the
role of Cousin Hebe on such short notice only on the condition that she
would have no dialogue to speak; Gilbert accommodated her, and all of
Hebe's dialogue was cut. In all the authorized published texts of
\emph{Pinafore}, Hebe has no dialogue, nor did Hebe ever have any
speaking lines in any production of \emph{Pinafore} mounted by the
D'Oyly Carte Company. One other point: \emph{Pinafore} was revived under
Gilbert's direction in 1887, and had Gilbert wished, he could have
restored Hebe's dialogue for the revival. The role was reprised by
Jessie Bond, who by 1887 was in her tenth year with the D'Oyly Carte
Company, and who by then had become thoroughly accustomed to roles
requiring her to speak dialogue. But Gilbert did not restore the
dialogue. He must certainly have understood that his original conception
of Hebe, as intended for Mrs. Howard Paul, was sound and that in
accommodating a temperamental and manipulative performer, he had
weakened his libretto. There is no documentary or anecdotal evidence to
suggest why the dialogue was not restored, but one reason, no doubt, was
that the opera had been a hit without the suppressed dialogue, and
Gilbert was not a librettist to tinker with something that had proven
successful.

So we may ask: should an editor preparing a critical edition of
\emph{Pinafore} restore Hebe's dialogue? Suppressing Hebe's dialogue
seriously compromises the internal consistency of the action, but both
the textual and the performing traditions argue against restoration.
Percy Young, the editor of the critical edition of \emph{Pinafore}, opts
not to restore the suppressed dialogue in the main body of his edition.
He does, however, print all of the relevant passages in an appendix, so
that productions using his edition can perform the suppressed
dialogue --- and several have done so \citep[B:173--77]{gilbert_h_2003}. So here, as with \emph{Princess Ida,} the editor
has followed the textual and performing traditions, which are in
agreement, and which were clearly in accordance with the creator's
latest wishes. But in the case of \emph{Pinafore}, the editor has done
so believing that the altered version that he has printed is
artistically inferior to what he might have printed.

The dilemma Young faced puts front and center the distinction touched
upon above, that between work and performance (or production). According
to traditional thinking, a work is a stable and continuing entity,
whereas a performance or production is an instantiation of the work.
Provided that the creator's conception be practical --- provided that he
or she asks no more than should be expected of any competent
performer --- it seems perverse to allow the problems that arise in a
particular production to determine forever the shape of the work.

But there is no evidence that Gilbert or Sullivan ever made any
distinction between production-specific adjustments and artistic
improvements. They seem rarely to have thought any farther than the
first run. They were, after all, practical men of the musical theatre,
and they do not seem to have regarded their collaborations as immortal
works of art. Even when, in 1885, they began reviving their earlier
operas (\emph{The Sorcerer,} first produced in 1874\emph{,} was the
first to be revived), they might change the scenery, but they very
rarely modified the libretto or the music; they usually admitted changes
to the published vocal scores and libretti only to correct errors in
earlier states.\footnote{Thus, for example, new scenery was created for
  the revival of \emph{Pinafore} in 1887. In this production the
  audience sat amidship, looking aft towards the quarterdeck. In the
  original production, the audience sat abeam, with the quarterdeck to
  the right and the mainmast to the left. The stage directions, however,
  were not completely revised to suit the new layout in the revival, an
  omission which has caused confusion.

  One of Gilbert's best known efforts at inserting something topical in
  a revival occurred with a reference to the Boer War during a 1901
  revival of \emph{Iolanthe}; the line was considered inappropriate and
  likely to arouse strong political emotions, and Gilbert withdrew it.}
This attitude is no doubt typical of many dramatists and composers, who
are creators confiding their creations to agencies often beyond their
control --- to actors, musicians, dancers, directors and producers --- and
for whom just having seen the piece through to opening night may seem a
significant achievement.

Another sort of problem requiring editorial discretion occurs when the
creator makes changes in response to censorship, whether explicitly
imposed by an agent such as the licensing office of the Lord
Chamberlain, or by critics, producers, or audiences, who exert a
different but no less coercive force. Such cases occur in Gilbert \&
Sullivan's seventh collaboration, \emph{Iolanthe}, which had its
premiere on 25 November 1882, and which has more biting social and
political commentary than any of the other Savoy Operas. \emph{Iolanthe}
deals satirically with the House of Peers, the British legal system, and
the English preoccupation with birth, three seriously flawed
institutions about which Victorian audiences were understandably
sensitive. In this opera, a band of fairies, having been offended by a
particularly arrogant peer, punishes the House of Lords by sending into
Parliament a protégé, who, backed by the fairies' magical powers,
implements legislation that will right certain social wrongs and will
deny the Peers many of their ancient and cherished privileges.

\emph{Iolanthe} had a difficult gestation; Gilbert's sketch-books show
that his first ideas were much more inflammatory than the toned-down
satire of the opera as we know it.\footnote{On the genesis of
  \emph{Iolanthe}, see \citealt{perry_everything_2005}, and \citealt{perry_introduction_2017},
  Part C, pages 2-15.} For \emph{Iolanthe}, Gilbert's polishing process
involved removing the elements with the sharpest edges. Notwithstanding
his reputation as a martinet and his ability to keep performers in
terror of him, Gilbert could be indecisive about his libretti, and the
rewrites that were invariably required during rehearsals were often
quite difficult for him. With \emph{Iolanthe}, even more so than usual,
Gilbert was juggling musical numbers and dialogue right up until opening
night --- and afterwards.

As constituted on opening night, \emph{Iolanthe}'s second act had two
particularly bitter political pieces, the original No. 6, known as the
``De Belville Song, or The Reward of Merit'', and the original No. 9, a
song beginning ``Fold your flapping wings, soaring Legislature''. The
``De Belville Song'' tells of a gentleman gifted as an artist, writer
and inventor, and for whose accomplishments a peerage might have been a
suitable reward. But no peerage was forthcoming until the gentleman
inherited a safe seat in the House of Commons and began making
``inconvenient speeches'' there. At this point, a reward was quickly
found: he was given a peerage and thereby banished to the House of
Lords.\footnote{For the text of the ``De Belville Song'', see \citealt{gilbert_iolanthe_2017}, Part C, pages 198-99. On the song's history, see \citealt{miller_reward_2000}.}

After opening night, Gilbert and Sullivan set about to tighten up the
second act. There were two rounds of cuts separated by a few days; the
alterations can be traced by comparing states of the first English
edition of the libretto, the type for which was kept standing so that
the text could be easily modified until the settled state was
reached.\footnote{The London opening night is reflected by the first
  state of the British libretto (L1b in Gerald Hendrie's edition,
  \citealt{gilbert_iolanthe_2017}), while the second is reflected in the second
  state of the British libretto (Hendrie's L1c). (The edition printed by
  J. M. Stoddart in Philadelphia and run off from plates made from
  stereos sent over from London, is the first published state of the
  libretto, designated by Hendrie L1a.) See the critical apparatus to
  Hendrie's edition of the libretto, \citealt{gilbert_iolanthe_2017}, Part C,
  pages 19-62. References to act, number and measure or line are to this
  edition.} The ``De Belville Song'' was cut during the first round of
revisions. Unfortunately, although the lyric is preserved in the first
two issues of the published libretto, all that survives of the music is
a violin leader part, which provides just enough information to justify
attempting a reconstruction but not enough to support a convincing one,
and it is therefore impossible to assess the quality of
music-cum-words.\footnote{The number was sung in the New York
  production, but in the London production, Rutland Barrington, to whom
  the song had been assigned, was indisposed, and he recited rather than
  sang the number. A reconstruction has been attempted by Bruce Miller
  and Helga Perry; see \citealt{miller_reward_2000}.}

``Fold your flapping wings''\footnote{For an edition of the words and music, see
  \citealt{gilbert_iolanthe_2017}, Part C, pages 181-90; for the words only, see Part C, pages 202-03.} was a victim of the second round of cuts,
but fortunately both its words and music survive. This song contains
some of the angriest passages in all the Savoy Operas. Sung by the
fairies' reform-minded protegé, now an MP, the song deals uncomfortably
with the question of London's underclasses and the system that
perpetuates them. One verse reads:

\begin{figure}[H]
\begin{quote}
Take a wretched thief,\\
Through the city sneaking,\\
Pocket handkerchief,\\
Ever, ever seeking.\\
What is he but I\\
Robbed of all my chances,\\
Picking pockets by\\
Force of circumstances.\\
I might be as bad ---\\
As unlucky, rather ---\\
If I'd only had\\
Fagin for a father.
\end{quote}
\end{figure}

\noindent The number had been well received --- indeed, it had
been encored, and several critics had written very favorably of it. But
other critics had objected that its subject was too serious for a comic
opera.\footnote{The anonymous reviewer for \emph{The Sportsman} praised
  the number, noting that it was ``unanimously
  encored'' \citep[3]{anonymous_review_1882}. On the other hand, William Beatty-Kingston, writing for
  \emph{The Theatre}, asserts that ``it amazes
  and even startles one, like the fall of a red-hot thunderbolt from a
  smiling summer sky'' \citep[22]{beatty-kingston_review_1883}.} In deciding to cut this number, Gilbert was no
doubt responding to the need to compress the second act, but it is
impossible not to believe that he also did so because he feared that
retaining the number would frighten away some potential ticket buyers.
Happily, however, Sullivan was reluctant to see the number cut, and,
contrary to his usual practice of putting discarded numbers aside to be
recycled when Gilbert presented him with a metrically suitable lyric, he
kept the manuscript fascicle containing ``Fold Your Flapping Wings''
intact and had it bound into the end of his holograph\footnote{See \citealt{sullivan_iolanthe_1882}; in the manuscript's present state, it occupies Vol. 2, pages 161-174.} --- both lyric and music, therefore, survive.

Should ``Fold Your Flapping Wings'' be restored in a critical edition of
\emph{Iolanthe}? Should a reconstruction of the ``De Belville Song'' be
offered as part of the edition? Gerald Hendrie, who edited
\emph{Iolanthe} for the critical edition, does not restore ``Fold Your
Flapping Wings'' --- nor any of the other extant material cut either
before or after opening night. But in appendices he does give the
piece --- both words and music --- and as much of the other excised
material --- including the violin leader of the ``De Belville'' song --- as
survives. And both these decisions seem to me to be right: the declared
aim of the Critical Edition is to establish a text that represents the
settled state of each work, the state that Gilbert and Sullivan ---
perhaps misguidedly --- allowed the opera to assume. It is not the
business of the editor of an edition with this aim to second guess the
creators of the work he is editing and to impose his own views, whether
artistic or social, on the edited work. That way lies chaos.

Another sort of problem that confronts editors of performing works is
how to manage changes that have been made, with or without the creators'
approval, to replace material that was topical when a work was written
but that has since become meaningless --- or worse --- offensive. Although
an editor of a poem or novel would not think of making such changes,
they are made in performing works with surprising frequency.\footnote{With
  a performing work, there is a reciprocity between a text and its
  users --- the performers --- that does not exist with a novel or a poem.
  The reader of a poem does not alter a passage that might be improved,
  but a performer not infrequently changes a note or an articulation,
  and conductors have been known to alter the scoring of orchestral
  works. One reason that this is so is that the reader of a novel is the
  novel's audience, while the reader of a performing text is a
  performer, mediating between text and audience, and the audience is
  one remove farther away from the text. Further on this matter, see \citealt[24--33]{broude_performance_2011}.} Several examples of such changes occur in
\emph{The Mikado}, Gilbert and Sullivan's ninth and most popular
collaboration, produced in 1885.

In the second act of \emph{The Mikado} (lines 470-99), the Mikado,
arriving with his royal entourage in the town of Titipu, inquires of the
citizens as to the whereabouts of his runaway son, Nanki-Poo.\footnote{References
  to act and line numbers in \emph{The Mikado} libretto are to Bradley
  1996.} Informed that Nanki-Poo has ``gone abroad'', he asks for his
son's address. The reply is ``Knightsbridge''. A year before \emph{The
Mikado} opened there had been a Japanese exhibition in the London
neighborhood of Knightsbridge, and the reference would have been
understood by \emph{The Mikado}'s first audiences. However, the meaning
was lost when the exposition closed and its memory faded, and so it has
become traditional for performers responding to the Mikado's query to
substitute for ``Knightsbridge'' the name of some locale in or near the
city in which their production is taking place (for example, for the
annual production of a Gilbert \& Sullivan opera at Longwood Gardens
outside Philadelphia, ``Kennett Square'', the town in which Longwood
Gardens is located, might stand in for Knightsbridge). The editor of the
critical edition of \emph{The Mikado} (who is the author of this
article) will print ``Knightsbridge'', but will include a footnote
explaining the practice of posting Nanki-Poo to locales more relevant to
specific productions.

A slightly more complex problem occurs in Act 2, No. 7, in which Ko-Ko,
Pitti-Sing, and Poo-Bah describe for the Mikado an execution that is
supposed to have taken place. Pitti-Sing tells how, noticing her, the
condemned man, who had previously been fearful, pulled himself together:

\begin{figure}[H]
\begin{quote}
He shivered and shook as he gave the sign \\
For the stroke he didn't deserve;\\
When all of a sudden his eye met mine,\\
And it seemed to brace his nerve;\\
For he nodded his head and he kissed his hand\\
And he whistled an air, did he,\\
As the saber true\\
Cut cleanly through\\
\-\hspace{.5cm}His cervical vertebrae!
    \begin{flushright}(lines 427-35)\end{flushright}
\end{quote}
\end{figure}

\noindent At the words ``he whistled an air'', Sullivan inserted in the piccolo
part, as a solo, a snatch from a popular song of the day, the
``Cotillion Waltz''. Presumably, whistling this air would convey to the
audience the cavalier attitude of the man about to be beheaded. But the
``Cotillion Waltz'' soon dropped out of fashion, and since the meaning
of inserting it had been to show that the condemned man whistled a
well-known tune, it was pointless to retain the waltz. It has long been
practice --- exactly when the tradition began is not known --- for the
piccolo to play instead of the ``Cotillion Waltz'' the opening phrase of
``The Girl I left Behind Me'', a traditional military marching song
dating back to the eighteenth century and familiar to most audiences as
such. There is no authority for this except tradition: Sullivan's
holograph \citep{sullivan_mikado_1885} has only the ``Cotillion Waltz'', uncancelled;
the published vocal scores do not notice the piccolo insertion at all;
and the surviving D'Oyly Carte band parts are too late to be reliable
evidence of when this practice started. Yet substituting a traditional
tune such as ``The Girl I left Behind Me'' for a song that has lost its
significance makes good sense. For an editor producing a full score, the
question is whether to follow the holograph or to draw on what is
clearly a longstanding and sensible performing tradition. The critical
edition of \emph{The Mikado} will print the ``Cotillion Waltz'' on the
piccolo line, but will also give ``The Girl I Left Behind Me'' as an
\emph{ossia}, printed on a cue-size staff above the piccolo line. A
footnote will explain the situation.

More delicate is the problem of Gilbert's use of words acceptable in
Victorian England but now regarded as offensive. In \emph{The Mikado},
the n-word\label{nword} occurs twice. The first occurrence is in Act 1, No.
5a, in Koko's little list of people who might be executed without being
missed: among them are ``The [n---] serenader, and others of his race''
(line 252). The second occurrence is in Act 2, No. 6, the Mikado's song
about letting the punishment fit the crime:

\begin{figure}[H]
\begin{quote}
The lady who dyes a chemical yellow\\
Or stains her face with puce\\
Or pinches her figger\\
Is blacked like a [n---]\\
With permanent walnut juice.\\
\begin{flushright}(lines 355-59)\end{flushright}
\end{quote}
\end{figure}

\noindent On May 28, 1948, Rupert D'Oyly Carte, then director of the Company,
wrote to \emph{The London Times}:

\begin{quote}
We recently found that in America much objection was taken by coloured
persons to a word used twice in the Mikado\ldots{}. We consulted the
witty writer on whose shoulders the lyrical mantle of Gilbert may be
said to have fallen {[}A. P. Herbert{]}. He made several suggestions,
one of which we adopted.
\end{quote}

\noindent The first occurrence is now sung as ``The banjo serenader, and others of
his race''. The retention of ``others of his race'' was no doubt
motivated by the desire to tamper as little as possible with what
Gilbert originally wrote, but not modifying the second half of the line
leaves it clear that Gilbert's aversion is a racial matter. The Act 2
occurrence is remedied with these lines by Herbert:

\begin{figure}[H]
\begin{quote}
The lady who dyes a chemical yellow\\
Or stains her face with puce\\
Or pinches her figger\\
Is painted with vigour\\
And permanent walnut juice.
\end{quote}

\end{figure}

\noindent In his communication to \emph{The Times}, D'Oyly Carte blandly added
that ``Gilbert would certainly have approved'' of Herbert's
modifications. This is a highly questionable assumption. Gilbert shared
many of the racial and ethnic prejudices of his Victorian
contemporaries: the same mind that produced the witty lyrics for which
Gilbert is celebrated also entertained many of the assumptions of racial
superiority that helped the British to justify their Empire.

The question facing an editor of \emph{The Mikado} is an instance of a
larger one that is often raised: should works of the past be rewritten
to suppress words or ideas considered offensive or politically incorrect
today? Should racial and ethnic slurs be removed, should pronouns be
made gender neutral, should history be re-written? The present writer
believes that an editor of an historical edition has an obligation to
history and that, distasteful as it may be, what Gilbert wrote must be
retained. A compromise in this case is to print Gilbert's words in the
text underlaid beneath the music in the full and vocal scores, but also
to underlay, as \emph{ossia}, Herbert's alternate versions.\footnote{One
  cannot pretend that Gilbert did not write the offensive words or that
  he was unaware of how offensive they were or might become. No director
  today retains Gilbert's words, although the revisions proposed by Mr.
  Herbert --- the first reflecting only slightly less prejudice than
  Gilbert's original --- are not usually the preferred alternatives.}

The last problem peculiar to performing works to be considered in this essay
arises from performing traditions. Performing traditions are the
unwritten traditions that a performing work acquires as it passes
through time: they include interpretations of characters, readings of
individual lines, and bits of business. As time goes by, a work's
performing tradition often acquires components that are not
authoritative but that audiences familiar with that tradition have come
to regard as integral elements of that work.\footnote{One of the most familiar
examples of a performing tradition unsupported by textual evidence and
no doubt contrary to the playwright's intentions is that of portraying
Shylock as a sympathetic rather than a comic or devilish character.}
Performing traditions are important because they transmit elements that
text does not and/or cannot specify. Until the age of electronic
recording, such traditions could be passed from performer to performer
only by personal contact, with one performer's watching or being
instructed by another.\footnote{Further on performing traditions --- and
  how they sometimes find their way into texts --- see \citealt{broude_performance_2011}, pages
  33-40.} Knowledgeable audience members are usually familiar with the
performing traditions of works they know.

An interesting example of the problems presented by performing
traditions occurs in a late Savoy Opera, \emph{The Yeomen of the Guard}
(1888), a work that hovers so uncertainly between comedy and tragedy
that Gilbert and Sullivan were worried that the first-night audience
would take their opera too seriously. In \emph{Yeomen}, we meet Jack
Point, a shallow jester who aspires to profundity. Point is in love with
and engaged to his protégé, the young and pretty Elsie Maynard. Point,
however, has an unattractive penchant for the main chance, and, in
return for a small payment, he agrees to allow Elsie to marry Colonel
Fairfax, a gallant officer who is shortly to be beheaded on trumped up
charges and who wishes to take a wife in order to thwart the heirs who
have contrived his death. When Fairfax escapes from custody, woos and
wins Elsie, and then is reprieved, Point finds that he has lost Elsie. At the end of \emph{Yeomen}, when the union of Fairfax and Elsie is
celebrated, Gilbert's stage direction reads: ``FAIRFAX embraces ELSIE as
POINT falls insensible at their feet'' (Act 2, l.820).\footnote{References
  to act and line number in \emph{Yeomen} are to \citealt{bradley_ian_complete_1996}.}

The role of Point was created by George Grossmith, who played this final
passage for laughs, twitching comically after he had fallen to the
ground. This interpretation is consistent with Gilbert's text.
Everything about Point points to his being a comic butt: he has
aspirations well beyond his merits; he is far too old for Elsie; and his
sense of values is distinctly deficient. In short, he should be a
thoroughly unsympathetic character whose eventual discomfiture is reason
for laughter. Moreover, since Grossmith created the part under Gilbert's
direction, and since Gilbert controlled as many details as he could in
the realization of his libretti, we must assume that Gilbert conceived
Point's end as deserved and comic; had he not, he would never have
allowed Grossmith to play the clown as the curtain falls.

However, when \emph{Yeomen} went on tour and performers, freed from
Gilbert's watchful eye, felt themselves at liberty to show initiative,
two members of D'Oyly Carte touring companies --- George Thorne and Henry
Lytton --- began to play Point for pathos. Instead of having Point writhe
on the ground in comic frustration, their new interpretation had him
die, presumably of a broken heart (in Lytton's case after having
tenderly kissed the hem of Elsie's dress). This innovation adds depth to
Point's character and is therefore more dramatically effective than a
comic rendering: Point may be an intellectual fraud and an unprincipled
rascal, but his love of Elsie is real, and his death results from the
depth of his feeling.

At some point, Gilbert became aware of Thorne and Lytton's
interpretation, but reports of his response to the innovation differ.
Lytton asserts that Gilbert approved the new interpretation, and Rupert
D'Oyly Carte, who succeeded to the directorship of the D'Oyly Carte
Opera Company a year after Gilbert's death, confirms this. But neither
of these accounts can be considered disinterested: Lytton wanted
authorization for his interpretation, while D'Oyly Carte would have
wanted to protect his company's reputation for preserving the performing
traditions established by Gilbert and Sullivan themselves. A more
objective witness is J. M. Gordon, stage manager for the 1897 revival of
\emph{Yeomen}, who reports that Gilbert refused to say whether Point
dies or not: ``The fate of Jack Point'', Gilbert is supposed to have
said, ``is in the hands of the audience, who may please themselves
whether he lives or dies''.\footnote{For Lytton's version, see \citealt[174]{lytton_secrets_1922}. For a review of responses to Lytton's interpretation, see \citealt[319--22]{bailey_gilbert_1952}. Gordon's report remains in manuscript but is quoted in \citealt[514]{bradley_ian_complete_1996}.}

What is clear is that Gilbert was committed to the stage direction as
printed, for he never changed it, although he certainly could have done
so in the several published forms of the libretto that postdated the
innovation but preceded his death. Significantly, however, he did change
several lines in the finale to alter Elsie's attitude towards Point from
one of contempt to one of pity; the original lines describe Elsie as:

\begin{figure}[H]
\begin{quote}
\-\hspace{1cm}a merrymaid peerly proud,\\
Who loved a lord and who laughed aloud\\
At the moan of the merryman moping mum\ldots{}
\end{quote}
\end{figure}

\newpage
\noindent The revised text describes her as:

\begin{figure}[H]
\begin{quote}
\-\hspace{1cm}a merrymaid nestling near,\\
Who loved a lord --- but who dropped a tear\\
At the moan of the merryman moping mum\ldots{}
\begin{flushright}
(ll.811-12)
\end{flushright}
\end{quote}
\end{figure}


\noindent Gilbert evidently wished to retain the ambiguity of ``Point falls
insensible'', allowing actors the option to interpret the character
either way, but he was sensitive enough to the advantages of the
pathetic interpretation to revise Elsie's lines. To this day, this
passage is played for pathos. Should an editor of \emph{Yeomen} print the earlier or the later version
of Elsie's song? Should the final stage direction be emended to
something like ``POINT dies'' or ``FAIRFAX embraces ELSIE as POINT
expires at their feet?'' After all, if Point is to die, something more
explicit than ``falls insensible'' is required: there is a substantial
difference between insensibility (he may recover and go on to pursue his
career with another protégé) and dying.

In \emph{The Complete Annotated Gilbert and Sullivan}, Bradley prints
the revised lines but preserves Gilbert's original stage direction.
Bradley's text therefore represents not the settled state of
\emph{Yeomen} --- the opera was settled before the touring companies were
formed and sent on the road --- but a revised state. But Bradley's
edition is not committed to representing the settled state; quite the
contrary, for all the historical background that Bradley provides, his
edition follows a longstanding tradition in the publication of dramatic
texts: that the edition reflect what is currently being played rather
than what was originally written.\footnote{For Bradley's statement of
  his policy, see \citealt[xii]{bradley_ian_complete_1996}. On this practice, which applies
  both to opera libretti and to play quartos representing plays ``as
  recently acted by {[}\ldots{}{]}'', see \citealt[34--40]{broude_performance_2011}.} In his
notes, Bradley gives the original lines and recounts the circumstances
behind the change in those lines and in the interpretation of the stage
direction. But no decision that an editor could make in this crux would
be neutral: Bradley's decision to print the revised lines rather than
the original points towards the pathetic ending; printing the original
would favor the comic but would not rule out the pathetic. After all,
Thorne and Lytton introduced the pathetic tradition while the original
lines were still being sung.

So far we have been talking about creators and editors and editions. An
editor of a critical edition of a Savoy Opera makes a sound decision
when he or she decides to produce a text that represents the settled
state of that opera during its first production. But performers and
directors who use that text today are another matter altogether: the job
of the editor of a critical edition is to present a text representing
(insofar as evidence permits) what existed --- or would have existed, had
every mechanical slip been caught in proof --- at some specific moment in
the past, but the job of a producer or director is to make the work
effective in the present, and what may have made an effective production
at one time and place may not excite an audience in other circumstances.
The circumstances under which any revival is mounted are always
different from the circumstances of a first production. This is
especially the case when a work considered a ``classic'' is revived: the
work is revived because it has already demonstrated that it has audience
appeal, and much of the audience may already be familiar with the work.
The case is particularly acute with the Savoy Operas, which have a fan
base much of which is knowledgeable about minute details. Every good
director will take these facts into account.

Gilbert cut Hebe's dialogue to accommodate Jessie Bond; he cut ``Fold
Your Flapping Wings'' because he wanted to quicken the pace of the
second act, and if something had to go, then that something might as
well have been a number that might offend some of his audience and
thereby reduce ticket sales. But 2021 is not 1878 or 1882. Jessie Bond
no longer plays Hebe, and the idea that poverty breeds crime is neither
new nor shocking. Nor is pacing an issue with a repertory work like
\emph{Iolanthe}, which many members of the audience will know by heart.
If anything, today's audience at a Savoy Opera wants more Gilbert and
Sullivan rather than less, and if a director can offer something that is
new but that is also in some way authentic, then so much the better. A
director who wants to give his production something novel but authentic
could ask for nothing better than to restore Hebe's dialogue or ``Fold
Your Flapping Wings''.

Operas that are fortunate enough to enjoy extended lives come to exist
in multiple states. (And this is as true for performing works of all
sorts as it is for the Savoy Operas.) A single text of an opera --- which
is what the text of a critical edition must be --- can represent only one
state. People involved with performing works --- actors, singers,
conductors, directors and producers --- all claim considerable freedom in
realizing texts, and few good performers will regard any text --- the
composer's, the librettist's, or the editor's --- as altogether binding.
An edition of a performing work is not an end in itself but a means to
performance, and so regardless of what state of a work an editor decides
his or her text should represent, it should be understood that editorial
responsibilities include providing, by means of critical apparatus,
footnotes and appendices, enough information about the textual and
performing traditions of the work to enable those who perform it to
exercise intelligently the freedom they will almost certainly claim.

\begin{flushleft}
\bibliography{references/broude}  
\end{flushleft}
\end{paper}