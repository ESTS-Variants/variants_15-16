ibid
\contributor{Lamyk Bekius}
\contribution{The Reconstruction of the Author’s Movement Through the Text, or How to Encode Keystroke Logged Writing Processes in TEI-XML}
\shortcontributor{Lamyk Bekius}
\shortcontribution{The Author’s Movement through the Text}

\hyphenation{ana-lyse}


\begin{paper}
\label{bekius}
\begin{abstract}
This essay demonstrates how the use of the keystroke
logging tool Inputlog allows for a fine-grained analysis of literary
writing processes. But before the writing process can be studied, the
keystroke logging data needs to be transformed into an output that is suitable
for a textual genetic analysis. For this purpose, this essay
investigates the potential of combining text with keystroke
logging data in TEI-conformant XML. Besides discussing how revisions can
be specified in the encoding, the author asks herself how traces of \emph{digital}
writing processes differ from \emph{analogue} traces (and, taking it one step further, how
keystroke logging can be used to record more details about the genesis
of a text), what kind of decisions need to be made when encoding
keystroke logging data, and how the peculiarities of digital authorship leave
their mark on its encoding --- as well as on the interpretation and argumentation
that underlies the transcription. This will demonstrate that the level of
detail that is recorded in keystroke logging data requires us to consider the way in which
the text was typed when we design our encoding schemas. The goal of the TEI-XML encoding of
the keystroke logging data is to provide transcriptions of writing
processes that could be used to analyse (the sequence of) revisions
and text production in each logged writing session in relation to their specific location in
the text.
\end{abstract}

\section{Introduction}
\textsc{This essay illustrates} how TEI-XML encoding of keystroke logging data can be used to arrive at a detailed examination of
the genesis of present-day literary works that are recorded with a keystroke
logger, using the writing processes of the Flemish novelist
Gie Bogaert (1958) as a case study. Bogaert divides his writing process into two stages.
The first part --- the ``creative process'' as Bogaert calls it --- consists
of making notes in a paper notebook. Here, he comes up with the concept and structure of the novel, writes character descriptions, and collects
additional material. The writing of this notebook must be completed
before he can start the second part of the writing process: the
``linguistic creative process''.\footnote{This is how Bogaert, in
  conversation, described his own working method. For a video for the
  Belgian publishing house Standaard Uitgeverij in which Bogaert
  describes his creative writing method, see \citealt{bogaert_gie_2013}.}
This part consists of the actual writing of the novel, for which he uses a word processor. As such, Bogaert works both in a ``traditional'' analogue writing
environment, and in a digital one. This way, his working method aptly
illustrates the challenges textual scholars and genetic critics face
in the (early) twenty-first century. For decades now, geneticists and
editors have had methods and tools at hand to study the traces of the
writing process in the paper notebook: different colours of ink,
interlinear additions, and crossed out words provide valuable clues to gain an insight into the text's
genesis. But how do we analyse the genesis of a text that was written in
a word processor?

Like Bogaert, most present-day authors predominantly work in a digital
environment \citep{kirschenbaum_tracking_2013,van_hulle_modern_2014,buschenhenke_het_2016,kirschenbaum_track_2016,vauthier_genetic_2016,ries_rationale_2018}. Common word
processors tend to hide the author's writing operations, which makes a detailed
reconstruction of the writing process that includes immediate
revisions a difficult endeavour \citep{mathijsen_genetic_2009}. Without a doubt, as Matthew Kirschenbaum and Doug Reside argue in their essay on ``Tracking the Changes'', the analysis of these
``new textual forms require new work habits, new training, new tools,
new practices, and new instincts'' \citep[272]{kirschenbaum_tracking_2013}.
The consequences of the digital work process for genetic criticism are
addressed in the interdisciplinary project \emph{Track Changes: Textual
Scholarship and the Challenge of Digital Literary Writing} that combines
aspects of cognitive writing process research with textual
scholarship.\footnote{This work has been conducted as part of my PhD
  research within the project \emph{Track Changes: Textual Scholarship
  and the Challenge of Digital Literary Writing} (2018-2023), a
  collaboration between Huygens ING (Royal Netherlands Academy of Arts
  and Sciences, Amsterdam) and the University of Antwerp (Antwerp Centre
  for Digital Humanities and Literary Criticism) funded by the Dutch
  Research Council (NWO). Project members include Prof. Karina van
  Dalen-Oskam, Prof. Dirk Van Hulle, Prof. Luuk Van Waes, Dr Mariëlle
  Leijten, Vincent Neyt and Floor Buschenhenke.}

Within cognitive writing process research, tools have been developed to
analyse non-literary writing processes, amongst them what is called
\emph{keystroke logging}: ``as an observational tool, keystroke logging
offers the opportunity to capture details of the activity of writing,
not only for the purpose of the linguistic, textual and cognitive study
of writing, but also for broader applications concerning the development
of language learning, literacy and language pedagogy'' \citep[1]{miller_keystroke_2006}. With the aim of broadening research coverage from
short professional writing processes in educational and corporate
contexts to include long-term literary writing processes, the team
behind the keystroke logging tool Inputlog \citep{leijten_keystroke_2013}
at the University of Antwerp collaborated with Bogaert to log the
writing process of his tenth novel, \emph{Roosevelt} (\citeyear{bogaert_roosevelt_2016}). After
Bogaert's writing process had been recorded, collaboration was
established between the Inputlog team, the Literary Department at
Huygens ING (Amsterdam), and the Centre for Manuscript Genetics
(University of Antwerp) in order to adequately address the analysis of
this literary writing process. This resulted in the interdisciplinary
project \emph{Track Changes}, in which Bogaert's writing process is
examined from the perspective of cognitive writing process research, as well as from
the perspective of genetic criticism. In turn, the collaboration with Bogaert
illustrates the possibilities of keystroke logging for genetic criticism
in future collaborative projects, or when writers choose to log their
writing processes themselves as part of their personal
archive.\footnote{Part of the \emph{Track Changes} project is the
  development of a new keystroke logger based on Inputlog, which improves the usability and the convenience for the authors so that it can be used for their own archival practices.}

The Inputlog team invited Bogaert to record his writing process with
Inputlog. For the purpose of this essay, it is important to keep in mind
that this implies that Bogaert's writing process contained a
researcher intervention. Throughout the writing process itself, however,
this intervention was kept to a minimum: since Inputlog is designed to
be as non-intrusive as possible, the further course of the writing
process was not influenced \citep{leijten_keystroke_2013}. Whilst other
keystroke logging tools (e.g. Scriptlog, see \citealt{wengelin_combined_2009}) were
designed for experimental word-processing environments --- and so cannot
be used to study writing in a more naturalistic setting --- Inputlog logs
the data of writing processes that take place in a word processing
environment that the author is already familiar with: Microsoft Word \citep{leijten_keystroke_2013}.\footnote{For an overview of logging tools,
  see \citealt{van_waes_logging_2011} and \citealt{lindgren_researching_2019}.} Each
time an author activates Inputlog to start a new writing session, the
Word document in which the author is working is saved in the background,
in a folder that contains the session's date and number. Subsequently,
the Word document is saved again when the author ends the writing
session by de-activating Inputlog. This results in a session-version of
the text for each session, which shows the text's gradual expansion. But
Inputlog does not just save Word documents. When the program is running,
every keystroke and mouse movement is recorded with a timestamp \citep{leijten_keystroke_2013}. While writing, authors retain control of the
process: they can start and stop the logging when they choose, and the
data is stored on their local PC or laptop.

Although Inputlog is developed for textual and cognitive study of
writing, the data output from the writing process of Bogaert's
\emph{Roosevelt}, generated in Inputlog, is not immediately suitable for
literary textual research. While Inputlog provides a video replay of the
recorded writing session, some issues emerge when replaying Bogaert's
writing process. Short writing sessions comprising linear text
production are replayed accurately, but as soon as larger segments of text
are relocated or deleted, when the writing is characterized with
non-linearity, or when the logged session is of considerable length, the
replay mode is affected and represents the revisions and text production
at the wrong location in the text. Moreover, relying solely on a video
replay of the writing session for text genetic analysis also seems
undesirable, as one would need to watch a writing session of, say, two
hours in its entirety, while constantly pausing to analyse the effect of
the revisions. A static reconstruction of the writing session --- whether
or not in combination with a video replay, as in Dirk Van Hulle's
proposal for a ``Dynamic Facsimile'' \hyperref[vanhullewip]{in the present issue} \citep{van_hulle_dynamic_2021} --- is favoured to
ensure adequate analysis. Hence, in order to be able to study the
revisions (contained in the keystroke logging data) in their textual
context, the twofold output of Inputlog --- the Word document and the
keystroke logging data --- requires some reassembly.\footnote{Within
  cognitive writing process research, a method has been developed to
  study revisions in context: the S-notation \citep{kollberg_s-notation_1998}. This
  represents the changes in the text at their location and provides
  information about the range, order and structure of the revisions \citep[91]{kollberg_studying_2002}. This computer-based
  notation can be generated using the keystroke logging data from
  Inputlog and is provided within the ``analyse'' feature of the software.
  However, the S-notation was initially developed to visualize revisions
  of short writing processes in experimental settings. As such, it appeared to be unsuitable for the study of longitudinal literary writing processes
  logged in their natural setting. Literary writing processes may take
  up several years and hundreds of writing sessions, with the production
  of an extensive number of words. As a result, the S-notation could not
  be generated using the keystroke data gathered from Bogaert's writing
  process. More generally, the S-notation does not allow for further
  annotation and processing. Another problem concerns the representation
  of deleted text. Since Inputlog logs the position of the event
  according to its position on the x- and y-axes of the MS Word
  document, the deleted text is not always presented correctly (the only
  information the keyboard provides about a deletion is usages of the
  delete or backspace key). This hinders an automatically generated
  visualization of the revisions in their textual context.}

Since TEI-conformant XML is widely used to create a digital form of
humanities data --- texts, manuscripts, archival documents and so on --- I
opted to encode the keystroke logging data in TEI-XML to visualize and
analyse revisions in their textual context \citep{burnard_what_2014}. For me, these transcriptions function as a tool to gain more insight into the textual genesis. They could eventually be used for visualizations of the writing process, but a proper discussion of the latter lies outside the scope of this article. In order to
reflect on how keystroke logging data can be encoded in TEI-conformant
XML, this essay discusses a) the way in which the traces of
\emph{digital} writing processes differ from traces of \emph{analogue} writing processes
(and how keystroke logging can be used to record more details about the
genesis of a text); b) which decisions we need to consider when we encode keystroke logging data; c) the way in which the peculiarities of
digital writing leave their mark on the encoding; and d) the
interpretation and argumentation underlying the transcription. The goal
of the TEI-XML encoding of the keystroke logging data is to provide
transcriptions of the writing processes that could be used to analyse
(the sequence of) the revisions and text production in each logged
writing session in their location in the text.

\section{Born-digital literature and genetic criticism}

The digital environment in which present-day literature is composed
significantly changes the materiality of the sources available for
textual scholarship and genetic criticism. Miriam O'Kane Mara argues
that ``with digital manuscripts, scholars must investigate the methods
that authors use as they save their work as well as the software and
hardware systems through which they compose'' \citep[345]{mara_nuala_2013}. Several explorations of the digital literary writing process have already been
published since then that explore digital files, file formats, and media types.

In \emph{Track Changes. A Literary History of Text Processing} (\citeyear{kirschenbaum_track_2016}), for example,
Matthew Kirschenbaum describes the emergence of the word processor and
its adoption among Anglo-American authors. And in ``The rationale of the born-digital dossier génétique'', Thorsten Ries analysed
born-digital records from the hard drives of the German poet Thomas
Kling (\citealt{ries_rationale_2018}; see also \citealt{,ries_philology_2017}). Ries argues that the digital forensic record of
the writing process comprises ``digital documents'', but also metadata,
automatically saved draft snapshots, recoverable temporary files and
other fragmented traces scattered across the hard drive \citep[417]{ries_rationale_2018}. The use of applications such as a hex editor, a binary parser, an
undelete tool, or a file carver can reveal revision and intermediate
steps in the writing process \citep[418]{ries_rationale_2018}. For example, the so-called
``scratch file'' [\textasciitilde{}WRS0003.tmp] with the first
paragraph of the first chapter of Kling's essay \emph{Herodot} (2005)
``contains an almost complete protocol of the first writing phase of
this paragraph in the form of text additions and textual variants from
the first written line on to a point of time between
[\textasciitilde{}WRL3681.tmp] and
[\textasciitilde{}WRA1775.wbk]'' --- two other temporal and backup
files \citep[141]{ries_philology_2017}. After extracting the fragments, Ries could
reconstruct the nonlinear development process of this paragraph, with
inclusion of editing phases and correction of typing errors
\citep[142]{ries_philology_2017}. Still, as Ries mentions, ``although the relative,
layered sequence of edits can be determined [\ldots{}] due to
textual fragmentation, it is not in all cases possible to determine a
consistent text status at any given time with certainty'' \citep[142]{ries_philology_2017}. Bénédicte Vauthier took on the task of investigating the digital
files the Spanish writer Robert Juan-Cantavella saved during the writing
of his novel \emph{El Dorado} (2008). After comparing digital documents
and analysing the tree structure of the folders and other metadata such
as the file title and creation date, she concluded that, ``although the
dossier does not contain the normal traces of writing --- cancellations,
additions, shifts --- whose absence [\ldots{}] would appear to make
our analysis practically impossible, collating and comparing the digital
documents and files gives us more than a sound basis to allow a
meaningful genetic investigation'' \citep[175]{vauthier_genetic_2016}. Although the
research above has proven that the digital writing process can leave
sufficient traces to ensure genetic analysis, immediate revision and
correction of typing errors remain often --- apart from some exceptions
--- irretrievable.

Most genetic studies of born-digital writing processes work with self-archived born-digital materials
received directly from the authors in question (see, for example, \citealt{vauthier_genetic_2016,crombez_postdramatic_2017,vasari_securing_2019}). This already indicates how important it is to collaborate with authors for this kind of research. Collaboration between writers, scholars and
archivists has already been advocated by Catherine Hobbs with regard to
born-digital archives, ``to understand the relationship between writers,
their documentation and their creative vision'' (Hobbs, qtd. in
\citealt[384]{gooding_forensic_2019}). To be able to study the genesis of a
born-digital work of literature in more detail, the collaboration with
authors may be extended, for example, by logging their writing process
with a keystroke logger. Between 2014 and 2018, the English novelist C.
M. Taylor collaborated with the British Library to record the writing
process of his novel \emph{Staying On} (2018) using the keystroke
logging software Spector Pro. Taylor, driven by both the ``lost drafts'' in
digital writing and the loneliness of the writing process, contacted the
digital curation team of the British Library, and decided together
to record the writing process \citep[n.p.]{taylor_c_2018}. As for copyright, they agreed that the recorded data would belong
to the British Library, while that of the resulting book would belong to Taylor. On the ``English and Drama blog'' of the British Library, Taylor
quoted Jonathan Pledge (curator of contemporary archives at the
British Library) who stated that the used software ``seem[ed] to have
been specifically designed for low-level company surveillance of
employees, potentially without their knowledge'' \citep[n.p.]{taylor_c_2018}.
This use case scenario becomes apparent in the material that was
recorded: the document with the text was not automatically saved after
each writing session, although the software did save a screenshot
``captured every few seconds each time activity on the host computer is
detected'', and Taylor saved some intermediate versions himself
\citep[n.p.]{taylor_c_2018}. Also, the software did not record the location of
the keystroke within the document itself, nor the time of each
individual keystroke. This suggests that while the recorded material
undeniably contains valuable information about Taylor's writing process,
the software that was used seems to make it even more difficult to
pursue a detailed reconstruction of the revisions in their textual
context than is the case with Inputlog.

Although keystroke loggers originated as spyware, both Inputlog and Scriptlog (another
keystroke logging program) have been
developed specifically to observe writing processes for research
purposes \citep{wengelin_combined_2009}. The keystroke logging files logged with Scriptlog can be read
and processed within the Inputlog
environment \citep{van_hoorenbeeck_generic_2015}.\footnote{The same applies to
  eye and handwriting observation software EyeWrite, EyePen and HandSpy
  \citep{van_hoorenbeeck_generic_2015}.} As a result, the proposed encoding
below could also be applied to writing processes logged with Scriptlog.
In addition, if it is possible to represent the revisions made by Taylor
--- as well as revisions logged with other keystroke loggers --- in their
textual context, the revision types given below may also be
distinguished in this material.
  
\section{Keystroke logging data and its use for genetic
criticism}

So what does Inputlog's recorded keystroke logging material consist of?
Bogaert wrote \emph{Roosevelt} in 266 days --- from July 2013 to December 2015 --- during which 447 writing sessions took place, each logged with Inputlog. Of these sessions, 422 were dedicated to the writing of the novel. Luuk Van Waes assisted Bogaert in installing Inputlog, and was available for questions or when problems occurred. The 422 writing sessions resulted in 453 session-versions\footnote{In some sessions the initial document did not
  correspond with the end document from the previous session, so there
  are more session-versions than sessions.} showing the gradual expansion of the text, and 277 hours, 14 minutes and 22 seconds of keystroke logging data.\footnote{For an example of how the successive events in these sessions were logged, see Table \ref{tab:bekius:general} in the Appendix below, which shows a detail from
the General Analysis generated by Inputlog. The General Analysis of each writing session represents every event that was recorded during that  session.}

Let us zoom in on the the composition of a single sentence written by Bogaert in session thirty: 

\begin{quote}
Soms kan hij meer
krijgen dan wat hij voor zo'n kunstwerkje vraagt, maar dat wil hij
nooit. 

[Sometimes he can get more than he asks for such a work of
art, but he never wants that.]\footnote{All the translations of
  Bogaert's sentences are my own. I have tried to stay as close as
  possible to the original Dutch sentences in my translation.}
\end{quote} 

\noindent
All the writing actions
performed in composing this sentence are listed in Table \ref{tab:bekius:soms}. Since all
the different steps Bogaert took to arrive at this sentence are recorded in Inputlog's General Analysis logs, keystroke logging facilitates an
analysis of the text with a granularity that cannot be obtained when working with analogue materials.

In the typology of writing processes of literary works, genetic criticism distinguishes
between exogenetic, endogenetic, and epigenetic writing processes --- all of which
can be studied from a microgenetic point of view, and from a microgenetic one \citep{biasi_what_1996,van_hulle_modelling_2016}. ``Microgenesis'' here includes all
intra-textual processes: ``the processing of a particular exogenetic
source text; the revision history of one specific textual instance
across endogenetic and/or epigenetic versions; the `réécritures' or
revisions within one single version''\citep[50]{van_hulle_modelling_2016}. The
``macrogenesis'' on the other hand, embodies ``the genesis of the work in
its entirety across multiple versions'' \citep[50]{van_hulle_modelling_2016}. When
 examining keystroke logging data, the geneticist can examine the writing process at an unprecedented level of granularity.
The fine-grained data of keystroke logging therefore allows for a new
type of what can be called ``nanogenetic'' research.\footnote{See also
  Dirk Van Hulle's contribution on ``Dynamic Facsimiles'' in the present
  issue of \emph{Variants}.} 

\renewcommand*{\thefootnote}{\fnsymbol{footnote}}
\begin{table}[!ht]
\centering\footnotesize\renewcommand{\arraystretch}{1.5}
\caption[Writing actions in the composition of the sentence ``Soms kan
hij meer krijgen dan wat hij voor zo'n kunstwerkje vraagt, maar dat wil
hij nooit.']{Writing actions in the composition of the sentence ``Soms kan
hij meer krijgen dan wat hij voor zo'n kunstwerkje vraagt, maar dat wil
hij nooit.''\footnotemark[1]}
\begin{tabular}{p{.475\textwidth}p{.475\textwidth}}
\toprule
\textit{Writing action} & \textit{Text}\tabularnewline
\midrule
adds: \verb|Soms beiden| & \emph{\textbf{Soms beiden}}\tabularnewline
adds: \verb|krijgt h| & Soms \emph{\textbf{krijgt
h}}beiden\tabularnewline
adds \verb|kan hij meer| & Soms \emph{\textbf{kan hij meer}} krijgt
hbeiden\tabularnewline
adds: \verb|en dan wat hij voor zon'| \newline \verb|kunstwerkje vraagt| & Soms kan hij
meer \emph{\textbf{krijgen dan wat hij voor zon' kunstwerkje vraagt}}t
hbeiden\tabularnewline
deletes: \verb|n| & Soms kan hij meer krijgen dan wat hij voor zo'
kunstwerkje vraagtt hbeiden\tabularnewline
adds: \verb|n| & Soms kan hij meer krijgen dan wat hij voor
zo'\emph{\textbf{n}} kunstwerkje vraagtt hbeiden\tabularnewline
deletes: \verb|t hbeiden|& Soms kan hij meer krijgen dan wat hij voor
zo'n kunstwerkje vraagt\tabularnewline
adds: \verb|, maar dat il hij niet| & Soms kan hij meer krijgen dan wat
hij voor zo'n kunstwerkje vraagt\emph{\textbf{, maar dat il hij
niet}}\tabularnewline
adds: \verb|w| & Soms kan hij meer krijgen dan wat hij voor zo'n
kunstwerkje vraagt, maar dat \emph{\textbf{w}}il hij niet\tabularnewline
adds: \verb|.| & Soms kan hij meer krijgen dan wat hij voor zo'n
kunstwerkje vraagt, maar dat wil hij
niet\emph{\textbf{.}}\tabularnewline
adds: \verb|ooit| & Soms kan hij meer krijgen dan wat hij voor zo'n
kunstwerkje vraagt, maar dat wil hij
n\emph{\textbf{ooit}}iet.\tabularnewline
deletes: \verb|iet| & Soms kan hij meer krijgen dan wat hij voor zo'n
kunstwerkje vraagt, maar dat wil hij nooit.\tabularnewline
\bottomrule
\end{tabular}
\label{tab:bekius:soms}
\end{table}

Central to a work's nanogenesis would be the author's movement through
the text, as the writing process is taking place. Thanks to keystroke
logging software, this highly detailed form of sequentiality can be
deduced from logged events that allow us to reconstruct the order in
which the text was typed --- for example, whether the author left a
sentence while composing it --- and the way in which words were deleted.
This level of detail of movement through a text cannot be deduced from
an analogue document in which ``the documentary evidence is often so
complex that it becomes impossible to determine the order in which these
revisions were made with any degree of certainty'' \citep[90]{dillen_digital_2015}. \footnotetext[1]{These writing actions are deduced from Inputlog's General Analysis logs. A copy of the original logs can be found in Appendix \ref{app:bekius:generalanalysis} below.}\renewcommand*{\thefootnote}{\arabic{footnote}}The fact that the analysis of keystroke logging data here allows us to succeed
exactly where traces of analogue writing processes crucially fall short warrants the coinage of a new concept such as that of nanogenesis.
In addition, it is important to realize that a work's microgenesis and nanogenesis can be studied separately: one could,
for example, study the variation of one specific paragraph without
taking the fine-grained details about the exact order of the keystrokes
into account, or, conversely, solely focus on the author's movement
through the text.

In the case of digital documents, Mara claims that
``different types of software mould and shape the writing process,
making software a collaborator of sorts'' \citep[344]{mara_nuala_2013}. Although
primarily relying on how the Irish writer Nuala O'Faolain describes
writing in a word processor in her memoirs, Mara indicates changes in
the author's writing process as they move to the digital environment,
``a process that differs from traditional drafting and revision''
\citep[345]{mara_nuala_2013}. For O'Faolain, ``the digital environment provides a
sense of freedom and lack of fear because so-called mistakes can be
easily rectified and revision can be immediate'' \citep[346]{mara_nuala_2013}. Writing on a laptop computer provided O'Faolain with a freedom and
fluidity that ``indicates \emph{for her} a willingness to play with
words and structures that other media might not promote''
\citep[344; emphasis in original]{mara_nuala_2013}. It is exactly this new
mode of writing that is recorded with keystroke logging, and could be
described by means of nanogenetic research.

The disadvantage of keystroke logging data such as those recorded by Inputlog, however,
is that the processes are recorded in such detail that the logs become almost incomprehensible 
to the untrained eye. To make these logs more accessible to researchers, they need to be
presented in a way that captures only the relevant information, and conveys the researcher's 
interpretation of the data in a format that is easy to read and preferably familiar to their peers.
This is exactly the strength of the Text Encoding Initiative (TEI), whose guidelines recommend the use of XML tags
to both transcribe the text as it is recorded on the document, and to encode the researcher's interpretation of that record 
in a human and computer readable format. 

Within the vast realm of possibilities that the TEI allows for, we will do well to check out the work done by its
Work Group on Genetic Editions (part of the TEI's Special Interest Group for Manuscripts) that
produced the ``Encoding Model for Genetic Editions'' \citeyearpar{workgroup_on_genetic_editions_encoding_2010}
to facilitate the encoding of genetic phenomena. The Workgroup focuses on two main requirements: ``the ability to
encode features of the document rather than those of the text, and the
ability to encode time, sequentiality and writing stages in the
transcription'' \citep[81]{dillen_digital_2015}. These two requirements make for a
perfect starting point for a discussion of the differences between
analogue and digital textual genetic material.

First of all, in case of digital writing, the spatial organization of the document no longer contains the majority of the information
about the genesis of the text --- given that an analogue document is a
solid (static, physical) information carrier while the digital document
is ephemeral (dynamic, virtual). In analogue documents ``the dialectic
between a document's physical limitations (as a two-dimensional surface
of limited size) and the internal structure of its different writing
zones on that surface often contains important clues in the
investigation of the text's writing process'' \citep[71]{dillen_digital_2015}. For
example, the position of the two text zones in the top margin of the
notebook page in Figure \ref{fig:bekius:notebook} (within the black squares) and the cramped
position of the word ``heeft'' might indicate that the text in the
right-hand zone was written before the text in the left-hand one. The
dynamic visualization environment of a word processor, on the other
hand, discards this type of information, since it allows text to be inserted at
any given position. In case of keystroke logging, the information about
the text's writing process is saved in a separate file outside of the
main document.

Secondly, the keystroke logging data offers detailed information about
the time and sequentiality of all the activities during writing. The
genetic scholar can only hypothesize about the sequence in which the two
zones in the top margin of the notebook page were written, and can in no
way be entirely certain about the sequence of their writing in relation
to the rest of the text. When the writing process is logged, the
sequence of the writing of the text can be deduced from the keystroke
logging data --- effectively eliminating the need for this type of
guesswork.


\section{What does the document reveal?}

Since the ``Documentary Turn'' in the late twentieth century, the document
has gained a special position in textual criticism --- especially within
the genetic orientation \citep[81]{dillen_digital_2015}. For our understanding of the
text, and to understand its genesis, the documentary context is regarded
as a crucial source of information \citep{pierazzo_putting_2011,dillen_digital_2015}. But for born-digital materials, the spatial organization of the
document becomes less substantial in the analysis of the text's genesis,
since digital documents are essentially distinct from analogue ones. As
Mats Dahlström has noted, digital documents cannot be defined
materially. In digital documents, works are constituted ``by the pattern
of signals and tensions at the binary level of the material carrier''  
\citep[n.p.]{dahlstrom_drowning_2000}. Indeed, the graphical user interface
(GUI), like the print layout in MS Word, only creates the illusion of
the materiality of digital documents \citep[468]{van_hulle_logica_2019}. Yet, at the
same time, as Kirschenbaum notes, ``a digital environment is an abstract
projection supported and sustained by its capacity to propagate the
illusion [\ldots{}] of immaterial behavior: identification without
ambiguity, transmission without loss, repetition without originality'' \citep[11]{kirschenbaum_mechanisms_2008}. Indeed, as Katherine Hayles argued in ``Translating Media: Why We Should Rethink Textuality'', digital documents are always bound to a
material carrier, in which ``data files, programs that call and process
the files, hardware functionalities that interpret or compile the
programs, and so on'' are required to produce the digital document \citep[274]{hayles_translating_2003}. Still, as Ries reminds us, we have to keep in mind that these digital documents ``are not
bound to a single physical entity, not even to a single processing
system context or display application'' \citep[397]{ries_rationale_2018}. The everyday
use of the term ``document'' also seems to complicate its use in textual
scholarship: we ``speak of the `same' digital document when we save `it'
after changing its content, after copying `it' to a pendrive and `open
it' on a different computer with a different word processor which might
display the content in a different way'' \citep[397]{ries_rationale_2018}. In my
comparison with the analogue document, I therefore use the term
``document'' to refer to \emph{each} Word file in the material recorded
with Inputlog: each writing session creates two documents (Word files
with .docx extension) and one XML file with the keystroke information
(.idfx extension).\footnote{How to preserve those files is yet another
  question (see, for example, \citealt{kirschenbaum_digital_2010}).}
Each Word file represents a different stage of the text
(session-version) and can be collated as such to gain a first impression
of the development of the text during each session.

At the document level, there are two main differences between analogue
source materials and born-digital source materials logged with a
keystroke logger: 1) keystroke logging abolishes the distinction between
interdocumentary and intradocumentary variation; and 2) the document
layout no longer contains the primary information about the genesis of
the text.

\subsection{Interdocumentary versus intradocumentary variation}

The first difference arises at the level of intradocumentary and
interdocumentary variation. In writing on paper, the document pages bear
witness of textual genesis through \emph{intra}documentary variation \citep[165]{schauble_encodings_2018}. Besides this stratification visible on
the page, a large part of the textual development also happens \emph{off
the page}, for example in the rewriting of a text \citep[165]{schauble_encodings_2018}.
This kind of \emph{inter}documentary variation can only be made apparent
by means of collation. To account for these differences, Schäuble and
Gabler proposed a distinction between textual \emph{layers} and
\emph{levels}. Textual layers represent the intradocumentary variation
(i.e. the revisions made to a single document) while textual levels
describe the interdocumentary variation (i.e. the differences between two
documents) \citep[169]{schauble_encodings_2018}.

Encoding interdocumentary changes causes a number of interpretative
problems for the editor. When there is no materialization of the change,
rules need to be developed for the encoding of the revision \citep[171]{schauble_encodings_2018}. To illustrate this need, Schäuble and Gabler
discuss how Virginia Woolf changed the phrase ```my mothers
[\emph{sic}] name' (Woolf MS.A.5.b, n3)'' to ```my mothers
[\emph{sic}] laughing nickname' (Woolf TS.A.5.a, 54)'' during her
transcription of her manuscript into a typescript \citep[171]{schauble_encodings_2018}.
This change could be encoded in several ways: 

\begin{quote}
It could be encoded as a
single substitution of the word ``name'' with ``laughing nickname'' or as an
addition of the word ``laughing'' followed by a substitution of ``name''
with ``nickname''. If we tokenise on a finer level of granularity than the
word, it could even be encoded as a single addition of the string
``laughing nick'' that builds a new compound with the following invariant
string ``name''.
\begin{flushright}
\citep[171]{schauble_encodings_2018}
\end{flushright} 
\end{quote}

\noindent Each of these solutions provides a
correct representation of the typescript, but for the editor it is
difficult to decide which encoding models the writing act best \citep[171]{schauble_encodings_2018}.


For digital writing processes that are tracked with
keystroke logging software, the encoding of the writing act would be
less ambiguous, since interdocumentary variation is always preceded by
intradocumentary variation --- which (in most cases) will be recorded
through keystroke logging, and saved in a separate file.\footnote{The
  exceptions to this are the writing sessions in which Bogaert inserted
  fragments of texts, which he had written in Evernote at times when he
  did not have his laptop at his disposal. Of these fragments, the
  keystrokes were not recorded.} When using Inputlog, for example, the author can continue to write in a single
document for the duration of their entire writing process, while
intermediate versions are simultaneously saved as separate documents in the background.
If these are logged consistently and without errors, the keystroke
logging data encompasses all intradocumentary variation, which in turn
provides the information about the interdocumentary variation; in this
case, the difference between the Word documents saved at the beginning
and the end of a writing session can be visualized by means of
collation. An example from Bogaert's writing process may help to clarify
how this would work.

A collation of the Word documents that were saved at the start and the
end of the seventeenth writing session points at an interdocumentary
variant. Here, Bogaert changed the sentence ``Mijn oude huis?'' [\emph{My
old house?}] into ``Mijn oude huid?'' [\emph{My old skin?}]. Following the
reasoning used by Gabler and Schäuble, this could be encoded as a
substitution of ``huis'' with ``huid'' or, at an even finer level, as a
substitution of ``s'' with ``d'' --- both options would be feasible. Since
this substitution was logged with Inputlog, the question of how to
change occurred is no longer an object of speculation. The keystroke
logging data details the sequence of how the revision was carried out:
Bogaert first pressed the key ``d'' and then used the delete key to
remove the ``s'' (see Table \ref{tab:bekius:replacement1}).

\newpage
\renewcommand*{\thefootnote}{\fnsymbol{footnote}}

\begin{table}[H]
\centering\tiny\renewcommand{\arraystretch}{1.5}
\caption[The replacement of ``s'' by ``d'' in the
keystroke data.]{The replacement of ``s'' by ``d'' in the
keystroke data.\footnotemark[2]}
\label{tab:bekius:replacement1}
\begin{tabular}{p{.01\textwidth}p{.12\textwidth}|p{.1\textwidth}|p{.025\textwidth}p{.025\textwidth}p{.025\textwidth}p{.035\textwidth}p{.11\textwidth}p{.03\textwidth}p{.12\textwidth}}
\toprule
\textit{\#id} & \textit{Event Type} & \textit{Output} & \textit{Pos.} & \textit{Doc. Len}. & \textit{CP} & \textit{Start Time} & \textit{Start Clock} & \textit{End Time }& \textit{End Clock}\tabularnewline
\midrule
21	& \verb|keyboard|	& LEFT	& 1838	& 4501	& 4501	& 42308	& 00:00:42.308	& 42355	& 00:00:42.355\tabularnewline

22	& \verb|keyboard|	& d	& 1838	& 4501	& 4501	& 43228	& 00:00:43.228	& 43306	& 00:00:43.306\tabularnewline

23	& \verb|keyboard|	& DELETE	& 2285	& 4501	& 4502	& 43369	& 00:00:43.369	& 43431	& 00:00:43.431\tabularnewline

24	& \verb|replacement|	& [1839:1840]	& 2285	& 4501	& 4502	& 43369	& 00:00:43.369	& 43431	& 00:00:43.431\tabularnewline

26	& \verb|mouse|	& Movement	& 2285	& 4501	& 4502	& 45397	& 00:00:45.397	& 46083	& 00:00:46.083\tabularnewline
\bottomrule
\end{tabular}
\end{table}
\renewcommand*{\thefootnote}{\fnsymbol{footnote}}
\begin{table}[H]
\centering\tiny\renewcommand{\arraystretch}{1.5}
\caption[General Analysis of typing ``Of is het m'' and deleting ``M''.]{General Analysis of typing ``Of is het m'' and deleting ``M''.\footnotemark[2]}
\label{tab:bekius:replacement2}
\begin{tabular}{p{.01\textwidth}p{.12\textwidth}|p{.1\textwidth}|p{.025\textwidth}p{.025\textwidth}p{.025\textwidth}p{.035\textwidth}p{.11\textwidth}p{.03\textwidth}p{.12\textwidth}}
\toprule
\textit{\#id} & \textit{Event Type} & \textit{Output} & \textit{Pos.} & \textit{Doc. Len}. & \textit{CP} & \textit{Start Time} & \textit{Start Clock} & \textit{End Time }& \textit{End Clock}\tabularnewline
\midrule

2460 & \verb|keyboard| & O & 1862 & 4573 & 4774 & 1077967 & 00:17:57.967 & 1078014 & 00:17:58.014\tabularnewline
2461 & \verb|keyboard| & f & 1863 & 4574 & 4775 & 1078295 & 00:17:58.295 & 1078342 & 00:17:58.342\tabularnewline
2462 & \verb|keyboard| & SPACE & 1864 & 4575 & 4776 & 1078498 & 00:17:58.498 & 1078560 & 00:17:58.560\tabularnewline
2463 & \verb|keyboard| & i & 1865 & 4576 & 4777 & 1078747 & 00:17:58.747 & 1078778 & 00:17:58.778\tabularnewline
2464 & \verb|keyboard| & s & 1866 & 4577 & 4778 & 1078825 & 00:17:58.825 & 1078903 & 00:17:58.903\tabularnewline
2465 & \verb|keyboard| & SPACE & 1867 & 4578 & 4779 & 1078966 & 00:17:58.966 & 1079028 & 00:17:59.028\tabularnewline
2466 & \verb|keyboard| & h & 1868 & 4579 & 4780 & 1079184 & 00:17:59.184 & 1079231 & 00:17:59.231\tabularnewline
2467 & \verb|keyboard| & e & 1869 & 4580 & 4781 & 1079324 & 00:17:59.324 & 1079387 & 00:17:59.387\tabularnewline
2468 & \verb|keyboard| & t & 1870 & 4581 & 4782 & 1079418 & 00:17:59.418 & 1079465 & 00:17:59.465\tabularnewline
2469 & \verb|keyboard| & SPACE & 1871 & 4582 & 4783 & 1079730 & 00:17:59.730 & 1079777 & 00:17:59.777\tabularnewline
2470 & \verb|keyboard| & m & 1872 & 4583 & 4784 & 1080666 & 00:18:00.666 & 1080728 & 00:18:00.728\tabularnewline
2471 & \verb|keyboard| & DELETE & 1911 & 4583 & 4785 & 1081243 & 00:18:01.243 & 1081321 & 00:18:01.321\tabularnewline
2472 & \verb|replacement| & [1873:1874] & 1911 & 4583 & 4785 & 1081243 & 00:18:01.243 & 1081321 & 00:18:01.321\tabularnewline
\bottomrule
\end{tabular}
\end{table}
\footnotetext[2]{Unfortunately, Inputlog occasionally gives the wrong position in \emph{Pos.}, as is the case with the deletions in Table \ref{tab:bekius:replacement1} and Table \ref{tab:bekius:replacement2}. The correct positions are given in \emph{Output}, respectively {[}1839:1840{]} and {[}1873:1874{]} (as opposed to the position given in \emph{Pos.}, respectively 2285 and 1911). For a more detailed explanation of what type of data is logged in which of Inputlog's ``General Output'' columns, see Appendix \ref{app:bekius:generalanalysis} below.}

\begin{table}[H]
\centering\tiny\renewcommand{\arraystretch}{1.5}
\caption[General Analysis of typing ``M'' and deleting ``Of is het m''.]{General Analysis of typing ``M'' and deleting ``Of is het m''.\footnotemark[3]}
\label{tab:bekius:replacement3}
\begin{tabular}{p{.01\textwidth}p{.12\textwidth}|p{.1\textwidth}|p{.025\textwidth}p{.025\textwidth}p{.025\textwidth}p{.035\textwidth}p{.11\textwidth}p{.03\textwidth}p{.12\textwidth}}
\toprule
\textit{\#id} & \textit{Event Type} & \textit{Output} & \textit{Pos.} & \textit{Doc. Len}. & \textit{CP} & \textit{Start Time} & \textit{Start Clock} & \textit{End Time }& \textit{End Clock}\tabularnewline
\midrule

8828 & \verb|keyboard| & M & 1834 & 4513 & 5517 & 2647478 & 00:44:07.478 & 2647524 & 00:44:07.524\tabularnewline
8829 & \verb|keyboard| & DELETE & 1835 & 4513 & 5518 & 2648336 & 00:44:08.336 & 2648336 & 00:44:08.336\tabularnewline
8830 & \verb|replacement| & [1835:1836] & 1835 & 4513 & 5518 & 2648336 & 00:44:08.336 & 2648336 & 00:44:08.336\tabularnewline
8832 & \verb|keyboard| & DELETE & 1835 & 4512 & 5518 & 2648835 & 00:44:08.835 & 2648835 & 00:44:08.835\tabularnewline
8833 & \verb|replacement| & [1835:1836] & 1835 & 4512 & 5518 & 2648835 & 00:44:08.835 & 2648835 & 00:44:08.835\tabularnewline
8835 & \verb|keyboard| & DELETE & 1835 & 4511 & 5518 & 2648866 & 00:44:08.866 & 2648866 & 00:44:08.866\tabularnewline
8836 & \verb|replacement| & [1835:1836] & 1835 & 4511 & 5518 & 2648866 & 00:44:08.866 & 2648866 & 00:44:08.866\tabularnewline
8838 & \verb|keyboard| & DELETE & 1835 & 4510 & 5518 & 2648913 & 00:44:08.913 & 2648913 & 00:44:08.913\tabularnewline
8839 & \verb|replacement| & [1835:1836] & 1835 & 4510 & 5518 & 2648913 & 00:44:08.913 & 2648913 & 00:44:08.913\tabularnewline
8841 & \verb|keyboard| & DELETE & 1835 & 4509 & 5518 & 2648944 & 00:44:08.944 & 2648944 & 00:44:08.944\tabularnewline
8842 & \verb|replacement| & [1835:1836] & 1835 & 4509 & 5518 & 2648944 & 00:44:08.944 & 2648944 & 00:44:08.944\tabularnewline
8844 & \verb|keyboard| & DELETE & 1835 & 4508 & 5518 & 2648975 & 00:44:08.975 & 2648975 & 00:44:08.975\tabularnewline
8845 & \verb|replacement| & [1835:1836] & 1835 & 4508 & 5518 & 2648975 & 00:44:08.975 & 2648975 & 00:44:08.975\tabularnewline
8847 & \verb|keyboard| & DELETE & 1835 & 4507 & 5518 & 2649006 & 00:44:09.006 & 2649006 & 00:44:09.006\tabularnewline
8848 & \verb|replacement| & [1835:1836] & 1835 & 4507 & 5518 & 2649006 & 00:44:09.006 & 2649006 & 00:44:09.006\tabularnewline
8850 & \verb|keyboard| & DELETE & 1835 & 4506 & 5518 & 2649053 & 00:44:09.053 & 2649053 & 00:44:09.053\tabularnewline
8851 & \verb|replacement| & [1835:1836] & 1835 & 4506 & 5518 & 2649053 & 00:44:09.053 & 2649053 & 00:44:09.053\tabularnewline
8853 & \verb|keyboard| & DELETE & 1835 & 4505 & 5518 & 2649084 & 00:44:09.084 & 2649084 & 00:44:09.084\tabularnewline
8854 & \verb|replacement| & [1835:1836] & 1835 & 4505 & 5518 & 2649084 & 00:44:09.084 & 2649084 & 00:44:09.084\tabularnewline
8856 & \verb|keyboard| & DELETE & 1835 & 4504 & 5518 & 2649116 & 00:44:09.116 & 2649131 & 00:44:09.131\tabularnewline
8857 & \verb|replacement| & [1835:1836] & 1835 & 4504 & 5518 & 2649116 & 00:44:09.116 & 2649131 & 00:44:09.131\tabularnewline
8859 & \verb|keyboard| & DELETE & 1835 & 4503 & 5518 & 2649553 & 00:44:09.553 & 2649615 & 00:44:09.615\tabularnewline
8860 & \verb|replacement| & [1835:1836] & 1835 & 4503 & 5518 & 2649553 & 00:44:09.553 & 2649615 & 00:44:09.615\tabularnewline
\bottomrule
\end{tabular}
\end{table}
\renewcommand*{\thefootnote}{\arabic{footnote}} 

\noindent The keystroke data appears to contain even more information about the
composition process of this sentence, in that it reveals another
modification as well. Somewhat later in the writing process, Bogaert
adds the clause ``Of is het m'' to the beginning of the sentence and
deletes the capital letter ``M'' (See Table \ref{tab:bekius:replacement2}). The sentence now reads: ``Of
is het mijn oude huid?'' [\emph{Or is it my old skin?}]. But after a while,
Bogaert returns to this sentence and changes it back to its previous
variant by first adding ``M'' and then deleting ``Of is het m'' (see Table \ref{tab:bekius:replacement3}).\footnote{Note that Inputlog's ``General Analysis'' only shows
  keystrokes and mouse movements. This explains why the deleted text is
  not visible in the log of Table \ref{tab:bekius:replacement3}, but only Bogaert's pressing of the
  delete key. Still, the ``General Analysis'' \emph{does} provide
  information about the position of the deleted characters. The letter
  ``M'' is positioned at 1834 and the subsequent letter at position 1835
  is deleted eleven times. This allows us to reconstruct the deleted
  text.} Because this substitution was both performed and undone during
the same session, it is not visible in the end document of this
particular session (the session-version).
\newpage

When we try to encode this sentence, merging the text with the keystroke
data, all the modifications can be put together as follows:
\begin{example}
\label{ex:bekius:1}
\begin{lstlisting}[language=XML]
<seg><add>M</add><del><add>Of is het m</add></del>
<del>M</del>ijn oude hui<add>d</add><del>s</del>?</seg>
\end{lstlisting}
\end{example}

\noindent The keystroke logging data thus provides more information about
interdocumentary revisions than we can extract from analogue material:
all interdocumentary variation is captured as intradocumentary variation
within the keystroke logging data. When a keystroke logger is used while
writing, textual development can no longer occur \emph{off the page} ---
or at least: when authors keep their writing process within the
confines of their Inputlog enabled computer, and when that process is
logged without errors). In Bogaert's case, the records suggest that he
occasionally pasted new or revised textual materials into the document
that were produced in between explicitly logged writing sessions. This means that for these passages, there is no data available about how the variation came to be.

\subsection{Encoding keystroke logging data instead of the
document's
layout}

This leads us to the second difference between analogue and born-digital
writing processes at the document level, which concerns the appearance
of the document page itself. Complex handwritten draft materials often
consist of chaotic pages that contain multiple textual fragments that
were written at different positions and in different directions on the
page. The page of Bogaert's notebook (See Figure \ref{fig:bekius:notebook}) shows text written in
different colours, in the margins of the page, and between two lines. If we want ``to gain insight about
the composition, time of revisions, and flow (flux) of the text'', we therefore need to carefully consider the physical aspects of the document (its layout, the arrangement of the text on the page, and what this tells us about how the text was written), and to inform that reading of the page with a good understanding of the text \citep[§1.3]{workgroup_on_genetic_editions_encoding_2010}. But when Bogaert uses a word
processor, the document remains clean with every modification to the text
(see Figure \ref{fig:bekius:notebook}): additions are always represented as inline insertions, and
deleted text ``disappears'' from the surface. Logging the process with a
keystroke logger that saves these modifications in a separate file prevents them from becoming untraceable. The crucial information about the
genesis of the text thus shifts to the keystroke logging data. This
calls for a different encoding of the revisions, that focuses on exactly those elements that make a difference in the keystroke logging data.

\begin{figure}[!ht]
    \centering
    \includegraphics[width=.45\textwidth]{media/Bekius08.png}
    \includegraphics[width=.45\textwidth]{media/Bekius09.png}    
    \caption{A comparison of Gie Bogaert's analog and digital writing processes. Left: a page in Bogaert's notebook his novel \emph{Roosevelt} (left). Right: a screenshot of a page in one of Bogeart's MS Word documents for the same novel.}
    \label{fig:bekius:notebook}
\end{figure}

When the text is composed in a word processor, revisions cannot be
specified with the attributes used in encoding analogue material.
Instead of indicating the specific writing tool (which may be encoded in \lstinline[language=XML]!@rend!) or the location
in the document (which may be encoded in \lstinline[language=XML]!@place!), digital revisions
(specifically: \lstinline[language=XML]!<add>!s and \lstinline[language=XML]!<del>!s) can be further
specified using their location in the text. These diverging behaviours in the writing of analogue versus digital documents forces us to completely rethink the ontology we use for encoding relative location in our transcriptions. In the following, I therefore propose a list of ``revision types'' on the basis from Lindgren and Sullivan's so-called ``revision taxonomy'' \citeyearpar{lindgren_analysing_2006}. By using their taxonomy as a starting point, we can encode the relative location of additions and deletions in the attributes of \lstinline[language=XML]!<add>! and \lstinline[language=XML]!<del>! tags in a way that is more relevant to born-digital writing processes (see Tables \ref{tab:bekius:encodinganalogue} and \ref{tab:bekius:encodingdigital}).\footnote{For good measure, the attributes for additions and deletions in witnesses of born-digital writing processes (Table \ref{tab:bekius:encodingdigital} are contrasted to those for analogue writing processes (Table \ref{tab:bekius:encodinganalogue}). The specific examples that were used to compile these tables are taken specifically from analogue and digital witnesses to Bogaert's writing process, but could be applied more generally.} 

The first step in adopting Lindgren and Sullivan's taxonomy is to define
revisions according to their relative location in the text, that is ``where and when in the
writing process revision occurred'' \citep[42]{lindgren_writing_2006}.
With regard to this location, the taxonomy distinguishes
``pre-contextual'' revisions (i.e. ``revisions made before an externalised
context is completed'') from ``contextual'' revisions (``revisions made
within a completed externalised context''; see \citealt[159]{lindgren_analysing_2006}).\footnote{Within cognitive writing process research, the
  writing process is generally divided into three consecutive
  components: first the text has to be planned, then these internal
  ideas have to be translated (externalized) into linguistic forms, and
  then those forms have to be evaluated and revised where necessary
 \citep{lindgren_writing_2006}. While writing, different types of
  revision occur, and revision is mostly understood as ``making any
  changes at any point in the writing process'' (\citealt[484]{fitzgerald_research_1987}; \citealt[346]{lindgren_revising_2019}). Two major categories for such revisions
  are \emph{internal} and \emph{external} revisions. The former
  encompasses ``overall, conceptual revision as well as conscious
  evaluative revision and revision of pre-text'' \citep[37]{lindgren_writing_2006}. The latter are all ``visible changes made to the written
  text'' \citep[37]{lindgren_writing_2006}. Inputlog only logs revisions made in already
  externalized text; the encoding therefore covers the external
  revisions.} These location-based revision types best resemble the use
of attributes like \lstinline[language=XML]!@rend! or \lstinline[language=XML]!@place!, as they are based on the
relationship between the keystroke data and the place in the document,
rather than purely on the editor's interpretation.

In the encoding of the keystroke data, these revision types may be
applied to the textual unit of a sentence.\footnote{In the proposed
  encoding scheme, each sentence is encoded with a
  \lstinline[language=XML]!<seg>! tag.} Building upon the taxonomy by
Lindgren and Sullivan, the attribute \lstinline[language=XML]!@type="context"! can be used to
indicate that the revision is a contextual one, that is: a revision made
in a previously written sentence. The attribute \lstinline[language=XML]!@type="pre-context"!, by
contrast, may then be used to indicate revisions made before a sentence
is completed and so concerns the author's most recently typed characters.
Diverging from Lindgren and Sullivan's definition of pre-contextual
revisions, pre-contextual deletions can take place at a point in the
text with externalized text after the deleted text.\footnote{In the
  taxonomy by Lindgren and Sullivan (2006), one feature of a
  pre-contextual revision is that there is no externalized text after
  the place of revision \citep[159]{lindgren_analysing_2006}. As literary writing is often a non-linear
  process, new context can be created in other places than at the end of
  the text. In order to be able to distinguish revision within a
  sentence before this sentence is completed, I also regard these
  revisions as ``pre-contextual''.}

A large number of revisions in digital writing occur as a result of
typographical errors. Within the scope of genetic criticism, such
``typos'' are not the most meaningful entities because they do not
immediately affect the meaning of the text. Within cognitive writing
process research, typos are regarded as a revision type that ``often blur{[}s{]}
the picture of the writing session'' \citep[68]{kollberg_s-notation_1998}. Typographical
errors are ``low-level, and hence less important, types of revision'', and
filtering them out would therefore allow for a more nuanced analysis of
revision \citep[71]{conijn_how_2019}. But the revision of typographical
errors can also break the flow in writing and therefore influence the
writing process \citep[72]{conijn_how_2019}. For this reason, I propose to encode
this type of revisions with a separate \lstinline[language=XML]!@type! attribute: \lstinline[language=XML]!@type="typo"!.
This allows such errors to be filtered out in visualizations where they
are irrelevant, while still allowing us to evaluate their effect on the
writing process.\footnote{Typing errors can be hard to distinguish from
  spelling errors. In the encoding of the typing errors, I therefore
  used a list of criteria \citep[developed by][]{stevenson_revising_2006} for
  distinguishing \emph{typing} revisions from \emph{spelling}
  revisions. According to the checklist developed by Stevenson et
  al., a revision can be identified as a typing revision, if one or more
  of the following applies: ``a. the pre-revision form does not conform
  to the orthographic rules of the language; b. the pre-revision form
  involves a letter string which does not conform to a likely
  pronunciation of the word; c. the semantic context indicates that the
  pre-revision form could not have been intended; d. the same word is
  written correctly at an earlier point in the text; e. a letter is
  replaced by an adjacent letter on the keyboard'' \citep[232]{stevenson_revising_2006}.}

The use of a keystroke logger allows for an exact reconstruction of the
textual development. This includes the moment a new sentence is
produced. Therefore, the production of new sentences can also be
incorporated in the encoding (\lstinline[language=XML]!@type="nt"!; ``new text''), to be able to
differentiate between ``new'' and ``old'' sentences.\footnote{Some text
  segments may also be deleted and added again in a revised form;
  thereby maintaining a semantic relationship with the previously
  deleted text. This is not necessarily ``new'' text and may therefore be
  given another attribute: \lstinline[language=XML]!@type="rt"! (`revised text'). However, the
  encoding of such revised text adds a new level of interpretation to
  the transcription. Whereas ``new text'' is a fairly objective
  interpretation --- as the text is typed into the document for the first
  time --- the classification ``revised text'' relies on the editor's
  interpretation and their understanding of the text.} Writing is not
always a linear process and sentences are not always finished before
modifications are performed elsewhere in the text. The author could, for
example, move away from the point in the writing where new meaning is
produced: the so-called ``leading edge'' \citep{lindgren_revising_2019}. In the
definition formulated by Lindgren et al., the leading edge is
located ``typically at the end of the text produced so far, but can also
occur at the end of insertions within previously written text where a
writer inserts new ideas (not only revises form)'' \citep[347]{lindgren_revising_2019}. Unlike the
point of inscription, which comprises ``all writers' actions in
previously written text as well as at the end of the text produced so
far'', the leading edge is restricted to the creation of new meaning
\citep[347]{lindgren_revising_2019}. During the production of a sentence the author can decide to
leave the sentence produced so far to make a revision elsewhere --- in
the same sentence or at another segment of the text --- after which they
return to the end of the sentence they were writing. This would not be
an addition, because the sentence is not yet completed. However, the
fact that the author moved away from the leading edge is meaningful for
the interpretation of the writing process, as it provides information
about the steps that were taken to write the sentence. To be able to
identify this return to the leading edge, the text can be encoded using
\lstinline[language=XML]!<mod>!. According to the TEI P5 Guidelines, the
\lstinline[language=XML]!<mod>! element may be used to represent ``any kind
of modification identified within a single document'' \citep[§11.3.4.1]{the_tei_consortium_tei_2020}. For
the purpose of analysing digital writing processes, it may also be used
for the ``modification'' of unfinished sentences --- the continuation of
writing the sentence --- using the attribute: \lstinline[language=XML]!@type="continue"!. A
transcription with the inclusion of \lstinline[language=XML]!<mod>! tries
to model the flow of writing.

\begin{figure}[H]
\centering\tiny\renewcommand{\arraystretch}{1.5}
\begin{longtable}[]{@{}p{.05\textwidth}|p{.13\textwidth}|p{.18\textwidth}|p{.24\textwidth}|p{.2\textwidth}@{}}
\caption{Specifications for additions and deletions in analogue
material.\label{tab:bekius:encodinganalogue}} \\
\toprule
\lstinline[language=XML]!<add>!	& \lstinline[language=XML]!@hand="#GB"!	& \lstinline[language=XML]!@rend="#blueInk"!	& \lstinline[language=XML]!@type="#alternative"!	& \lstinline[language=XML]!@place="#marginleft"! \tabularnewline
& & \lstinline[language=XML]!@rend="#blackInk"! & & \lstinline[language=XML]!@place="#supralinear"! \tabularnewline
& & & & \lstinline[language=XML]!@place="#inline"! \tabularnewline\hline
\lstinline[language=XML]!<del>!	& \lstinline[language=XML]!@hand="#GB"!	& \lstinline[language=XML]!@rend="#blackInk"!	& \lstinline[language=XML]!@type="#crossed Out"! & \tabularnewline
& & & \lstinline[language=XML]!@type="#overwritten"! & \tabularnewline
& & & \lstinline[language=XML]!@type="#instant Correction"! & \tabularnewline
& & & \lstinline[language=XML]!@type="#underlined"! & \tabularnewline
\bottomrule
\end{longtable}

\centering\tiny\renewcommand{\arraystretch}{1.5}
\begin{longtable}[]{@{}p{.05\textwidth}|p{.15\textwidth}|p{.2\textwidth}|p{.2\textwidth}|p{.2\textwidth}@{}}
\caption{Specifications for additions and deletions in digital material.\label{tab:bekius:encodingdigital}} \\
\toprule

\lstinline[language=XML]!<add>! &	
\lstinline[language=XML]!@seq="201308230815"!	&
\lstinline[language=XML]!@type="context"! &
\lstinline[language=XML]!@n="15"! &
\lstinline[language=XML]!@evidence="6514-6556"! \tabularnewline
& & \lstinline[language=XML]!@type="pre-context"! & & \tabularnewline
& & \lstinline[language=XML]!@type="typo"! & & \tabularnewline
& & \lstinline[language=XML]!@type="nt"! & & \tabularnewline
& & \lstinline[language=XML]!@type="translocation"! & & \tabularnewline 

\hline	

\lstinline[language=XML]!<del>!	& 
\lstinline[language=XML]!@seq="201308230830"!	& 
\lstinline[language=XML]!@type="context"! & 
\lstinline[language=XML]!@n="16"! &
\lstinline[language=XML]!@evidence="6557-6567"! \tabularnewline
& & \lstinline[language=XML]!@type="pre-context"! & & \tabularnewline
& & \lstinline[language=XML]!@type="typo"! & & \tabularnewline
& & \lstinline[language=XML]!@type="translocation"! & & \tabularnewline

\hline

\lstinline[language=XML]!<mod>!	&
\lstinline[language=XML]!@seq="201308230840"!	&
\lstinline[language=XML]!@type="continue"! &
\lstinline[language=XML]!@n="17"!	& 
\lstinline[language=XML]!@evidence="6576-6599"! \\

\bottomrule
\end{longtable}
\end{figure}


\noindent Most of these revision types can be illustrated using the steps taken by
Bogaert when he wrote the example sentence from Table \ref{tab:bekius:soms} (``Soms kan hij meer krijgen dan wat hij voor zo’n kunstwerkje vraagt, maar dat wil hij nooit''). He started by writing ``\verb|Soms beiden|'', then moved his
cursor between the two words using the left arrow key. There he wrote
``\verb|krijgt h|''. This is a pre-contextual addition, because it takes place
before the sentence is finished.

\begin{example}
\label{ex:bekius:2}
\begin{lstlisting}[language=XML]
<add type="nt">Soms <add type="pre-context">kan hij 
meer </add><add type="pre-context">krijg<mod 
type="continue">en dan wat hij voor zo< type="typo">
n</del>'<add type="typo">n</add> kunstwerkje vraagt
</mod> <del type="pre-context">t h</del></add><del 
type="pre-context">beiden</del><mod type="continue">,
maar dat <add type="typo">w</add>il hij n<add 
type="context">ooit</add><del ="context">iet</del>
</mod><mod type="continue">.</mod></add>
\end{lstlisting}
\end{example}

\noindent Bogaert then continued writing with another pre-contextual addition
between ``\verb|Soms|'' and ``\verb|krijgt|'': ``\verb|kan hij meer|''. After inserting this
fragment, he relocates the cursor between the letter ``\verb|g|'' and the
letter ``\verb|t|'' in the word ``\verb|krijgt|'' and continues writing. Bogaert left
the leading edge (the point where he created new meaning) to make the
pre-contextual additions, but after these insertions a new leading edge
is created as he continues writing the sentence between the letter ``\verb|g|''
and the letter ``\verb|t|''. At the new leading edge he writes: ``\verb|en dan wat hij voor zon' kunstwerkje vraagt|''; the screen would now have displayed
the sentence as: 

\begin{quote}
Soms kan hij meer krijgen dan wat hij voor zon' kunstwerkje vraagtt \\ hbeiden.
\end{quote} 

\noindent The new leading edge was not positioned
at the end of the unfinished sentence, but after the letter \verb|g| in the
word ``\verb|krijgt|''. It was thus followed by ``\verb|t hbeiden|''.

Bogaert then corrects the typo made in the production ``\verb|zon'|'' (the
apostrophe was incorrectly positioned) after which he eventually
deletes the bulk of unused characters at the end of the sentence. These
are all pre-contextual deletions; the sentence is still not finished. Now
the leading edge is positioned at the end of the sentence, where Bogaert
continues writing: ``\verb| , maar dat il hij niet|''. After correcting the
typo with an addition --- he missed the letter ``\verb|w|'' in writing ``\verb|wil|''
--- he types the full stop. This marks the moment the writing of the
sentence is finished. Somewhat later in the session, Bogaert returns to
the sentence to make a contextual revision. He substitutes ``\verb|niet|'' with
``\verb|nooit|'' by adding ``\verb|ooit|'' and deleting ``\verb|iet|''. The writing process
of this sentence illustrates the complexity of digital writing, but also demonstrates that the proposed 
encoding succeeds in capturing every step in the process.\footnote{Bogaert
  explained in conversation that, when writing in the Word document, he
  focuses primarily on finding the right words. For him, this is the
  hardest part of the writing process. The rather complex way in which
  he types this sentence could reflect this. While doing so, he seems
  primarily orientated towards its reformulation. However, this also
  reflects Bogaert's personal typing habits.} Still, this encoding misses an important
aspect of the writing process: time.

\section{Specific encoding of time}\label{specific-encoding-of-time}

Inputlog logs every keystroke and mouse movement in combination with a
timestamp. Unlike analogue writing processes, the keystroke logging data
allows us to incorporate the specific time of writing into the encoding.
Through this temporal aspect, the writer can --- so to speak --- be
followed through the text. Lindgren and Sullivan mention this aspect of
keystroke logging too when they argue that the location of revisions

\begin{quote}
shows how the writers move their points of focus during text
composition; this can be viewed as the route writers take through their
texts. The actions writers perform during composition can, for example,
hint at the writers' developing ideas and associated shifts in text
focus.
\begin{flushright}
\citep[39]{lindgren_writing_2006}
\end{flushright}
\end{quote}

\noindent 
Incorporating the recorded time of the revision
into the encoding thus offers a unique opportunity to study the text's
genesis at a microscopic level --- what I referred to as its
``nanogenesis'' earlier.

The timestamp enables the genetic scholar to investigate the location at
which the author was working before they made a revision at another
place in the text, when (and how quickly) revisions were made, and if there
were certain revision campaigns. To analyse this, the
editor may encode the timestamp for each addition and deletion and every
other event worth mentioning, by using the \lstinline[language=XML]!@seq! attribute (e.g.
\lstinline[language=XML]!@seq="yyyymmddhhmmss"!). The editor can choose to incorporate the dates
of the writing sessions, so as to visualize the chronology of the
writing process, not only within a single session but also across
several (or all) sessions. The hours, minutes and seconds indicate the
time after the start of the session added to the time the session is
started.\footnote{The time is derived from the ``StartClock'' in the
  ``General Analysis'' of Inputlog, which is added to the start time of the session in question. In order to be able to retrieve the
  event in keystroke data, the unique ID of each event in the keystroke
  data needs to be included in \lstinline[language=XML]!@evidence!. As the time of every keystroke
  is given, the editor needs to make a decision as to which time to
  incorporate in the encoding. For a genetic analysis, the time of an
  event's first keystroke may be the most fitting option; for example,
  when the first key is pressed to start production of a new sentence.}
As such, this notation provides the exact time the textual input took
place. The TEI P5 Guidelines propose the attribute \lstinline[language=XML]!@seq! (sequence) for
assigning ``a sequence number related to the order in which the encoded
features carrying this attribute are believed to have occurred''
\citep[§11.3.1.4]{the_tei_consortium_tei_2020}. In the case of the logged writing processes, the \lstinline[language=XML]!@seq!
attribute can be very specific as the data provides information about
the time the deletions and additions were being made.

The timestamp given in \lstinline[language=XML]!@seq! can subsequently be used to number all the
changes in \lstinline[language=XML]!@n!. Using an XSLT script, the events can be listed
chronologically in \lstinline[language=XML]!<listChange>! and allocated a
number.\footnote{I would like to thank Vincent Neyt for writing the XSLT
  script for this purpose.} The \lstinline[language=XML]!@n! includes the number of appearances of
all insertions and deletions, as well as all returns to the leading
edge. From a computational perspective, the number in \lstinline[language=XML]!@n! provides the
same information as is given in \lstinline[language=XML]!@seq!: the chronology of the
modifications. The benefit, however, is for the (human) reader. In the
eventual transcription, the numbers will offer the reader the
possibility to see the sequence of the revision in one glance (see
section \ref{sec:bekius:interpretation}). This is one step into making the complexity of the (digital)
writing process more easily analysable for the reader. The example below
shows the encoding of the same sentence discussed above, with the
inclusion of the time and the order of appearance, starting from 27.


\begin{example}
\label{ex:bekius:3}
\begin{lstlisting}[language=XML]
<add seq="20130826151155" type="nt" evidence="1342" 
n="27">Soms <add seq="20130826151216" type="pre-context"
evidence="1400-1411" n="29">kan hij meer </add><add 
seq="20130826151210" type="pre-context" evidence="1374-
1383" n="28">krijg<mod seq="20130826151221" 
type="continue" evidence="1423" n="30">en dan wat hij 
voor zo<del seq="20130826151232" type="typo" 
evidence="1508-1509" n="31">n</del>'<add 
seq="20130826151233" type="typo" evidence="1513" n="32">
n</add> kunstwerkje vraagt</mod><del seq="20130826151236"
type="pre-context" evidence="1552-1557" n="33">t h</del>
</add><del seq="20130826151237" type="pre-context" 
evidence="1559-1581" n="34">beiden</del><mod 
seq="20130826151241" type="continue" evidence="1595-1643" 
n="35">, maar dat <add seq="20130826151248" type="typo" 
evidence="1643" n="36">w</add>il hij n<add 
seq="20130826154211" type="context" evidence="10622-
10625" n="170">ooit</add><del seq="20130826154212" 
type="context" evidence="10626-10633" n="171">iet</del>
</mod><mod seq="20130826151254" type="continue" 
evidence="1666" n="37">.</mod></add>
\end{lstlisting}
\end{example}

\noindent The number gives the exact order in which the modifications were carried
out while keeping editorial interference to a minimum. This contrasts
with analogue sources, where the complexity of documentary evidence
turns the numbering of revisions into a highly interpretative act
resulting only in speculative readings \citep[90]{dillen_digital_2015}. By comparison,
the keystroke data allows for a detailed reconstruction not only of the
revisions made at sentence level, but those at complete-text level as
well. If the author first made a revision to a sentence in the middle of
the text and then another in a sentence at the top of the text, this
movement through the text can be reconstructed, and also --- crucially ---
referenced in analyses of the writing process.\footnote{The
  transcriptions below (see Figure \ref{fig:bekius:session30} and Figure \ref{fig:bekius:latenstart}) and the discussion
  thereof, provide an example of how the numbers in \lstinline[language=XML]!@n! can be used to
  quote specific revisions.}

\hypertarget{peculiarities-of-digital-writing}{%
\section{Peculiarities of digital
writing}\label{peculiarities-of-digital-writing}}

In the encoding of the keystroke logging data, the way the text is typed
is taken into account. This way, we can distinguish between different typing styles. In this respect, at least two characteristics in digital
writing become apparent: 1) the recycling of words and characters, and
2) the different ways of performing a deletion. These characteristics
may guide editors in the decisions they make in the encoding of
born-digital writing processes.

\subsection{Recycling}\label{recycling}

Although the act of deleting is effectively free of cost in a word processor \citep[256]{sullivan_work_2013}, authors might recycle words and characters in
rewriting their texts. This characteristic is also noted by Py Kollberg
in her study of digital revisions (1998). She discusses how a writer in
her corpus keeps the ``\verb|t|'' in substituting ``\verb|there|'' for ``\verb|it|'':

\begin{quote}
Probably in order to minimize the effort to make this change, the writer
keeps the \emph{t} in \emph{it}, and uses it in the new word
\emph{there}. Two elementary character level revisions performed at
different positions are the result, but the effect of both revisions is
at the word level (and the words are at the same position). Many writers
would have deleted the whole word in this situation. 
\begin{flushright}
\citep[78; emphasis in original]{kollberg_s-notation_1998}
\end{flushright}
\end{quote}

\noindent
Kollberg concludes that people develop personal habits in their use of
the word processor; each writer has their ``own personal set of
organization of operations'' and is used to performing ``certain actions
in certain ways'' \citep[78]{kollberg_s-notation_1998}. Not unlike handwriting, typing
styles contain a ``fingerprint'' of the writer \citep[5]{lindgren_researching_2019}.\footnote{This also depends on the computer or laptop
  used; the keyboards used in a desktop PC set-up are more likely to
  incite use of the delete key, which is not available as a single
  button on many laptop keyboard layouts --- see also section \ref{sec:bekius:deletions}.}
Because genetic criticism is interested in the author's way of working,
the way in which they make use of the word processor also needs to be
apparent in the transcription.

As for Bogaert's way of typing, his recycling of words is very
prominent. Indeed, it is already present in the sample sentence
discussed above (see Examples \ref{ex:bekius:2} and \ref{ex:bekius:3}). Here, Bogaert added the clause ``\verb|en dan wat hij| \verb|voor zon' kunstwerkje vraagt|'' between ``\verb|krijg|'' and the
letter ``\verb|t|'' of the word ``\verb|krijgt|''. As such, he recycles the word part
``\verb|krijg|'', re-using it in the word ``\verb|krijgen|''. This re-use can be
detected at the word level as well, as Bogaert kept the letter ``\verb|n|'' in
the substitution of ``\verb|niet|'' with ``\verb|nooit|''. This is quite
characteristic of Bogaert; as Kollberg remarked in a similar situation quoted above, many others  would have deleted the
entire word.

This recycling of words and characters makes transcription of the
writing process a complex matter, as it makes the concise representation
of the flow of writing more challenging. This characteristic therefore
highlights the importance of encoding the returns to the leading edge
with a different tag. In the genetic transcription (Figure \ref{fig:bekius:session30}), it is
possible to reconstruct that ``\verb|en dat wat hij voor zo'n kunstwerkje vraagt|'' (n30) was written between ``\verb|krijg|'' (n28/1) and ``\verb|t h|'' (n28/2)
while taking into account that this is not a regular addition, but the
writing of the sentence itself. In this visualization, the process of
the writing is emphasized and the singularity of Bogaert's writing
accentuated. Hence, it is important to encode the separate steps in the
writing process in order to be able to reconstruct the flow of writing.

\begin{figure}[H]
    \centering
    \includegraphics[width=.9\textwidth]{media/Bekius10.png}
    \caption[Transcription of a paragraph in session 30, showing all the
different modifications.]{Transcription of a paragraph in session 30, showing all the different modifications. For a legend of the colours and symbols used in this transcription, see Table \ref{tab:bekius:legend} in Appendix \ref{app:bekius:legend} below.}
    \label{fig:bekius:session30}
\end{figure}


\subsection{Deletions}\label{sec:bekius:deletions}

Another way in which typing styles become apparent is the usage of the
keyboard in making a deletion. As Kollberg notes, a delete operation can
be performed in two directions: forwards and backwards \citep[29]{kollberg_s-notation_1998}. A
forward deletion removes characters to the right of the cursor, a
backward deletion those to its left \citep[29]{kollberg_s-notation_1998}. A forward
deletion may be carried out by using the delete key or by selecting the
characters to the right of the cursor and pressing the backspace key.
Using only the backspace key performs a backward deletion. The way the
writer uses the keyboard in performing revision affects the encoding of
the revision.

When a writer uses the backspace key to delete characters in a
substitution --- a backward deletion --- the deleted word will usually
appear in front of the inserted word. If the author uses the backspace
key to delete words during the production of a sentence, the cursor is
continuously positioned at the end of the leading edge. During
production of the clause in Example \ref{ex:bekius:4}, ``\verb|de sheerne khoran kan worden gebracht|' [\emph{the sheerne khoran can be brought}], Bogaert changed the
simple past tense verb ``\verb|kon|'' into the simple present tense ``\verb|kan|''.
After writing ``\verb|kon|'', he deleted ``\verb|on|'' and then continued writing by
typing ``\verb|an|'':

\begin{example}
\label{ex:bekius:4}
\begin{lstlisting}[language=XML]
<seg>[...]de sheerne khoran k<del seq="20140717142431"
type="pre-context" evidence="6689-6690">on</del>an
worden volbracht [...]</seg>
\end{lstlisting}
\end{example}

\noindent
This pre-contextual deletion can be considered as the digital equivalent
of \emph{currente calamo} deletions in analogue material, which
``usually characterize writing produced by an author in the throes of
composition, with corrections or revisions made immediately rather than
later'' \citep[104]{beal_dictionary_2011}. The linearity of the pre-contextual deletions
and the production of the sentence facilitate the readability of the
encoding.

The author may also use the delete key to remove a part of the text --- a
forward deletion. Bogaert prefers this technique: when he makes a
substitution, he writes the addition prior to the deletion so that the
new word appears to the left of the older one --- in the substitution of
``\verb|niet|'' with ``\verb|nooit|'', for instance, the writing of ``\verb|ooit|'' preceded
the deletion of ``\verb|iet|''. The addition therefore appears before the
deletion in the encoding. In the encoding of analogue material, however, Elli Bleeker
notes that a deletion is normally located 

\begin{quote}
before [i.e. ``to the left of''] an addition in a
transcription (regardless of the actual positioning of these elements [on the document]),
simply because --- in the western world --- we read a transcription from
left to right and we usually assume that a word is first deleted and
then replaced. 
\begin{flushright}
\citep[98]{bleeker_future_2015}
\end{flushright}
\end{quote}

\noindent 
This choice is usually guided by the
goal of the transcription of analogue material: to render the text more
readable \citep[98]{bleeker_future_2015}. In the transcription of a digital writing
process, the goal is also to reconstruct that process --- as this is not
visible in the document --- and to capture the author's way of working.
The additions and deletions are therefore best placed in the position at
which they occurred: the way the deletions are performed dictates the
decisions made in the encoding.

\section{Interpretation, selection and
argumentation}\label{sec:bekius:interpretation}

The proposed encoding produces a transcription of the keystroke logging
data in order to provide data output suitable for a genetic analysis.
Specifically, it allows for the examination of revisions and new
text production, their sequences and their effect on the text. This transcription alone is not sufficient to create a digital genetic edition, but it provides a sound basis for the visualization of the writing process. Moreover, the act of encoding the keystroke logging data does coincide with the encoding practice for analogue material in that, here too, ``relatively simple
text encoding forces us to make editorial decisions'' \citep[112]{bleeker_future_2015}. In the case of keystroke logged writing processes, the need for
abstraction (and therefore interpretation) of the recorded material only
increases, because there is so much additional information available to the editor. As the examples above demonstrate, simply converting the writing actions that are recorded in the logging data
to their editorial representations already implies making a series of editorial choices, such as selecting the data and deciding where the encoded
insertions and deletions should be located in the transcription. While the keystroke logging data serves to
make more objective observations about the sequence of the writing, it
also forces the editor to make their interpretation of the material even
more explicit.

The transcription of the keystroke logging data tries to be as objective
as possible. In its proposal to complement a text-oriented approach with
a document-oriented one, the TEI Ms SIG refers to the opposition in
German editorial theory --- as coined by Hans Zeller --- between the
``Befund'' and the ``Deutung''. Respectively, these refer to ``what is there in the source
document, the record'' (Befund), and ``the interpretation of this phenomenon'' (Deutung)  \citep[§1.1]{the_tei_consortium_tei_2020}. The Workgroup notes that one
cannot talk about the record without any interpretation (especially not
in the realm of genetic criticism) but does make a distinction
between different levels of interpretation: 

\begin{quote}
there is an obvious
difference between the interpretation that some trace of ink is indeed a
specific letter and the assumption that a change in one line of a
manuscript must have been made at the same time as a change in another
line because their effects are textually related.
\begin{flushright}
\citep[§1.1]{the_tei_consortium_tei_2020}
\end{flushright}
\end{quote}

\noindent 
The Workgroup therefore proposes making a
distinction between the interpretation of ``what's there''
(document/fact) and ``how does it relate'' (text/interpretation)
 \citep[§1.1]{the_tei_consortium_tei_2020}. A similar distinction is made in research into cognitive
writing process, when it differentiates between what are called \emph{elementary
revisions}, and \emph{interpreted revisions}. According to Kollberg, an elementary revision is a single
deletion or insertion, and the analysis of such elementary
revisions is therefore based only on ``the writer's overt action in
manipulating the text, with a minimum of interpretation of how revisions
may be related according to the writer's intentions'' \citep[16]{kollberg_s-notation_1998}. Interpreted revisions, on the other hand, are revisions which are
analysed at a higher level. For example, if ``two or more elementary
revisions that are seemingly united by the same goal may be combined and
interpreted by the researcher as a unit'' \citep[17]{kollberg_s-notation_1998}. 

Following that logic, the
proposed encoding therefore focuses solely on the elementary revisions.
For example, the distinction between contextual and pre-contextual
revisions rests only on the author's actions as they are recorded in the
keystroke logging data. When a revision is made during the production of
a sentence (before the author presses the full stop), it is marked as a
pre-contextual revision. When the revision is made within a completed
sentence (after the full stop is typed), it is marked as a contextual
revision. In addition, revisions are encoded according to the way they
are performed and the replacement of one word with another is not encoded
as a substitution.\footnote{For example, when ``\texttt{niet}'' is changed to ``\texttt{nooit}'' by adding ``\texttt{ooit}'' and deleting ``\texttt{iet}'', only ``\texttt{ooit}'' and ``\texttt{iet}'' are marked with elements.} Nevertheless, this can only allow for a certain degree
of objectivity, since the selection of the material already involves
interpretation \citep[38]{dillen_editor_2018}.

Selection plays a pivotal part in the encoding of the keystroke logging
data. Although the transcription sets out to represent the author's
movement through the text, many indications of movement have been left
out of the encoding. The encoding marks only the textual output, as
generated by the keyboard, consisting of characters and punctuation
marks. As such, it omits the keyboard events ``\verb|UP|'', ``\verb|DOWN|'', ``\verb|LEFT|'' and ``\verb|RIGHT|''. The same applies to the mouse movements. In focusing on the
textual output, the encoding also ignores data provided about pauses,
their locations and the timing of each action. Moreover, as the only
time that is encoded in the transcription is the start time of any
modification, the time between the end of one revision and the start of
another is omitted as well. This means that the time between two
subsequent revisions cannot be deduced. This does not imply, for
instance, that long pauses or other time indications cannot be encoded
in the transcription, but rather that such an encoding does not lie
within the scope of this particular transcription. The main aim of this
transcription is to help the scholar/reader follow the sequence of text
production and revisions with a focus on the text and its
meaning.\footnote{Providing another transcription in which the pauses
  are encoded would indeed be useful, as it would provide information
  about the fluency of the writing. This may help interpret the sequence
  of the revisions from a cognitive perspective.} By reducing the
presentation of other kinds of information, the scholar/reader is less
distracted from analysing the text. Still, even with a focus on the
textual output, interpretation remains a key factor in the encoding as
``the idea of presenting a text in an objective way is problematic and
arguably impossible'' 
\citep[114]{bleeker_future_2015}. This seems even more true
when encoding keystroke logging data, as the editor is setting out to
reconstruct a state of the text which has never existed in full.

When all the deletions and insertions within a given session are
encoded, we arrive at a state of the text that has never actually
appeared on the author's computer screen as such, and has therefore never
interacted with. This might present us with some issues, such as the question where
to encode insertions and deletions  when several
revisions are located at virtually the same position in the logs \citep[34]{kollberg_s-notation_1998}.
The editor's interpretation is necessary in these instances, especially when the author makes an
insertion next to a place in the document where there was a previous deletion. That is the case because when the author inserts text, the text is inserted at the cursor location. But since any
previously deleted text remains visible in the encoding, there is no straightforward place to locate the insertion in relation to the previously deleted text \citep[34]{kollberg_s-notation_1998}. A protocol for such cases could be that when  it pertains a single insertion, the insertion should be encoded to the right of the previous deletion --- in line with Bleeker's argument for encoding deletions and additions in analogue witnesses. And when it comes to
substitutions, the way in which the insertion and deletion were performed could help make the most accurate decision. For example, when new text was inserted first, and the old text forward deleted afterwards, we could transcribe the insertion first (i.e. to the left) and the deletion second (i.e. to the right).

The editor's interpretation also comes to the fore in the transformation
of the TEI-XML encoding. Joris van
Zundert and Tara Andrews argue that the interface of the digital edition
functions as an argument: ``Our first observation is that a digital
edition's interface is an argument --- not just an argument about the
text, but also an argument about the ``attitude'' of the editor, a window
into his or her take on methodology and the digital edition itself''
\citeyearpar[7]{andrews_what_2018}. The interface of a digital scholarly edition ``is always
closely linked to the data model of the underlying data and the
editorial principles expressed in this data model'', so they function as
``an interpretation of knowledge and provide users with a more or less
``guided'' tour through the data and its general presentational setting'' \citep[VII]{bleier_discussing_2018}. While different transcriptions given below
cannot be considered as a fully developed interface, they already
function as ``an integral part of rhetorical form'' since they
foreground the textual development \citep[8]{andrews_what_2018}. As an example, I
shall discuss some possible transcriptions of the paragraph that includes
the example sentence I refered to above, with a view to guiding attention towards
the dynamics and non-linearity of the writing process --- in this case
within a single writing session.

The first option is to display a transcription that simply presents a
reconstruction of all the textual operations within their textual
context. This transcription promotes reading of the text with all the
modifications made during this session. The different types of
modification are visualized in different colours, which indicate that
the writing of this paragraph proceeded in different steps. At a glance,
one can see the dynamics that underlie the writing process:

\begin{figure}[H]
    \centering
    \includegraphics[width=\textwidth]{media/Bekius11.png}
    \caption{Transcription of another paragraph in session 30, displaying all the different modifications.}
    \label{fig:bekius:latenfull}
\end{figure}

\noindent
The transcription can then be modified to show how the text developed
during the session. By removing all the added text, this transcription
visualizes the text within the paragraph that was already written at the
moment Bogaert started this new session:


\begin{figure}[H]
    \centering
    \includegraphics[width=\textwidth]{media/Bekius12.png}
    \caption{Transcription of the same paragraph in session 30, displaying the text as it was at the beginning of the session.}
    \label{fig:bekius:latenstart}
\end{figure}

\noindent Conversely, the deleted text can also be removed. This enables readers
to see the state of the text at the end of the session. By providing the
option to read the state of the text at the beginning and the end of the
session, one is encouraged to focus on how the text developed and on which
steps were taken during its writing. This shows that the first and the second
sentence were transposed and that Bogaert added eight new sentences:\footnote{In
fact, Bogeart added nine of them: one of the sentences was then deleted during the same
session.}

\begin{figure}[H]
    \centering
    \includegraphics[width=\textwidth]{media/Bekius13.png}
    \caption{Transcription of a paragraph in session 30, displaying the
text at the end of the session.}
    \label{fig:bekius:latenstop}
\end{figure}

\noindent Next, the sequence of all the modifications can be studied by displaying
their numbers in sequential order. The first modifications in this
paragraph are the insertions of four new sentences (n14-n17);\footnote{The numbers (e.g., n14) refer to the chronology of each modification made in this paragraph during this logged writing session and coincide with the \lstinline[language=XML]!@n! attribute in XML (see also Appendix \ref{app:bekius:legend}).} the writing of the fourth of these is interrupted by
typing errors in ``\verb|shilderijtejes| [\emph{sic}]'' (n18-n20;
[\emph{paintings}]), which Bogaert corrects before
finishing the sentence with ``\verb|die hij maakt en verkoopt.|'' (n21; [\emph{which
he makes and sells}]). Hence, if one continues following the sequence of
all the writing operations in chronological order, the nanogenesis can
be analysed. This shows, for example, that Bogaert wrote the seventh
sentence (n22) and then returned to the sixth sentence to insert the
clause, with typos, ``\verb|aan en| [\emph{sic}] \verb|paar vieden| [\emph{sic}] \verb|en famileileden|
[\emph{sic}]'' (n23; [\emph{to a few friends and family
members}]). He then immediately substituted ``\verb|famileileden|
[\emph{sic}]'' (n25; [\emph{family members}]) with ``\verb|verwanten|'' (n24; [\emph{relatives}]). The sentences ``\verb|Maar het is best wel goed|'' (n99; [\emph{But it is pretty good}]) and ``\verb|Het is zo'n jongen van wie je alleen maar kan houden|'' (n150; [\emph{He's the kind of guy you can only love}]) were
inserted later, which indicates that he did not work continuously on
this paragraph but relocated the point of inscription to other locations
in the text in the meantime. The order of some revisions might also
indicate a relationship between them; the quest to correct typos
(n91-n94) and inserting a new sentence after deleting another
(n149-n150). The sequence of the modifications shows the continuous
shaping and reshaping of the text by Bogaert, a process made possible by the word processor:

\begin{figure}[H]
    \centering
    \includegraphics[width=\textwidth]{media/Bekius14.png}
    \caption{Transcription of a paragraph in session 30, displaying the
sequence (numbers) of the modifications. For a legend of the colours and symbols used in this transcription, see Table \ref{tab:bekius:legend} in Appendix \ref{app:bekius:legend} below.}
    \label{fig:bekius:latennumbers}
\end{figure}


\noindent Lastly, symbols can be added to the transcription to provide an
option for colour-blind users to still be able to distinguish the different
modifications (see Appendix \ref{app:bekius:legend}).
Adding these symbols can also enhance readability at the borders of the
insertions and deletions; for example, by distinguishing the contextual
addition ``\verb|verwanten|'' within the contextual addition ``\verb|aan en| [\emph{sic}] \verb|paar vienden| [\emph{sic}] \verb|en famileileden| [\emph{sic}]'':

\begin{figure}[H]
    \centering
    \includegraphics[width=\textwidth]{media/Bekius15.png}
    \caption{Transcription of a paragraph in session 30, displaying the
sequence (numbers) of the modifications and symbols.}
    \label{fig:bekius:latensymbols}
\end{figure}

\noindent Overall, these different transcriptions aim to make the argument that the
sequence of the writing operations and the overall development of the
text are our main points of attention, and can then be used for further
analysis. For example, to examine the effect of the revisions, and how they relate and interact with one another.

\section{New perspectives}

The combination of the Word documents (the session-versions) with the
keystroke logging data (the process) serves to uncover a text that had become
invisible during the writing process because of the
overwriting nature of word processors. The example transcriptions I provided throughout this paper prove that is possible to arrive at a genetic transcription of born-digital works of literature that were composed using a keystroke logger in a way
that also makes it possible to represent all the different actions that were performed as the text was written. By adding the time to the revision, the
sequentiality of all the revisions can also be reconstructed, which
enables a detailed analysis of the way the author moved through the text and how
sentences were produced. According to Elena Pierazzo, such a scholarly
consideration of time plays a pivotal role in the case of modern
autograph drafts and working manuscripts because: 

\newpage
\begin{quote}
the stratification of corrections, deletions and additions can give insights into an author's way of working, into the work itself, the evolution of the author's \emph{Weltanschauung}, the meaning/interpretation of the text.
\begin{flushright}
\citep[171]{pierazzo_digital_2009}
\end{flushright}
\end{quote}

\noindent
Compared with analogue text genetic material, the
keystroke data encompasses detailed information about the process in
which the text was produced. Logging writing processes with a keystroke
logger enables an analysis of the textual genesis at a finer
granularity: at the level of the work's nanogenesis. Future research will have to indicate whether such a nanogenetic
perspective will lead to new perspectives on (the genesis of) a text and
the way present-day authors write their texts. In addition, born-digital
writing processes alter our notions of ``variants'' and ``versions'' \citep{van_hulle_logica_2019}. The transcriptions of the keystroke logging data offer a
starting point for reflection on this question and for help in
redefining these key concepts. In other words, there are still challenges aplenty for textual
scholars in the twenty-first century.

\section*{Appendix}
\begin{subappendices}

\subsection{Genetic Visualization Notation Legend}
\label{app:bekius:legend}
In the transcriptions that were visualized in Figures \ref{fig:bekius:session30}, \ref{fig:bekius:latenfull}, \ref{fig:bekius:latenstart}, \ref{fig:bekius:latenstop}, \ref{fig:bekius:latennumbers} and \ref{fig:bekius:latensymbols} (see pages \pageref{fig:bekius:session30} and \pageref{fig:bekius:latenfull}--\pageref{fig:bekius:latensymbols} above), the different types of modifications are indicated with different colours, and, if preferred, with symbols (see Table \ref{tab:bekius:legend} below). The numbers in superscript refer to the chronology of each modification made during the logged writing session and coincide with the \lstinline[language=XML]!@n! attribute in XML. The numbers belong to nearest textual element in the same colour, and the numbers associated with the insertion of new text and the continuation of unfinished sentences are positioned at the beginning of the relevant text segment, while those associated with the revision types are located at its end. Since insertions can be made within insertions, higher numbers can appear within text elements which have been allocated another (lower) number. The same applies to deletions within previous inserted text. If not interrupted by a number and text segment in orange (indicating the continuation of an unfinished sentence) or text and number in red and bright green (respectively pre-contextual deletions and pre-contextual additions) the production of a new sentence was uninterrupted and therefore runs from start to end.

\begin{table}[H]
    \centering
    \includegraphics[width=.75\textwidth]{media/Bekius16.png}
    \caption{Legend}
    \label{tab:bekius:legend}
\end{table}

\subsection{Inputlog's General Analysis}
\label{app:bekius:generalanalysis}

Table \ref{tab:bekius:general} below shows the General Analysis of the writing process of following sentence in Gie Bogaert's \emph{Roosevelt} (referenced in Table \ref{tab:bekius:soms}, p.\pageref{tab:bekius:soms}; Figure \ref{fig:bekius:session30}, p.\pageref{fig:bekius:session30}; Example \ref{ex:bekius:2}, p.\pageref{ex:bekius:2}; and Example \ref{ex:bekius:3}, p.\pageref{ex:bekius:3}):

\begin{quote}
Soms kan hij meer krijgen dan wat hij voor zo'n 
kunstwerkje vraagt, maar dat wil hij nooit.
\end{quote}

\noindent
In the table, the first column (\textit{\#id})
shows the number of the event (consecutively). In the second column the \textit{Event Type} indicates the kind of event that was recorded, be that of the type \verb|keyboard|, \verb|mouse|, \verb|speech|, \verb|focus|,  \verb|insert| or \verb|replacement|. 
The next column then shows the event's
\textit{Output}. In case of a keyboard event, this output records the typed
letter. The position in the fourth column (\textit{Pos.}) represents the ``cursor
position''. The fifth column (\textit{Doc. Len.}) shows the ``length of the document'' expressed in
characters. This differs from the character production represented in
the sixth column (\textit{CP}), which shows all the characters produced during all
the writing sessions so far. The \textit{Start Time} and \textit{Start Clock} (columns
seven and eight) show the time of the ``key in'' --- respectively in
milliseconds and in clock time --- and the \textit{End Time} and \textit{End Clock}
(columns nine and ten) of each ``key up''. The Action Time (\textit{Act. Time}; column
eleven) represents the time between each key in and key up, the
\textit{Pause Time} (column twelve) the time between two key ins. The location
of the pause is shown in \textit{Pause Location} (column thirteen). 
  
As such, this table reproduces all the information from Inputlog's logging output, except for three columns: \textit{x}, \textit{y}, and another \textit{Pause Location} column (of the same name). In the original output, the \verb|x| and \verb|y| columns  respectively track the location of the mouse on the x and y-axes on the screen \citep{leijten_keystroke_2013}, and are only logged for \verb|mouse| event types (and therefore remained empty for this writing action). The other \textit{Pause Location} column was logged a number code for each of the (written out) location types (such as \verb|BEFORE WORDS|, \verb|WITHIN WORDS|, etc.) --- and therefore carries no additional relevant information. Overall, then, the General Analysis provides information about what
was written where and when, and therefore provides all the details
needed for a fine-grained reconstruction of the writing process.


\begin{center}
\centering\tiny\renewcommand{\arraystretch}{1.5}

\begin{longtable}[]{@{}p{.0075\textwidth}p{.08\textwidth}|p{.075\textwidth}|p{.0075\textwidth}p{.0075\textwidth}p{.0075\textwidth}p{.025\textwidth}p{.06\textwidth}p{.025\textwidth}p{.06\textwidth}p{.01\textwidth}p{.01\textwidth}p{.175\textwidth}@{}}
\caption{\label{tab:bekius:general}}  \\
\toprule
\textit{\#id} & \textit{Event Type} & \textit{Output} & \textit{Pos.} & \textit{Doc. Len}. & \textit{CP} & \textit{Start Time} & \textit{Start Clock} & \textit{End Time }& \textit{End Clock} & \textit{Act. Time} & \textit{Pause Time} & \textit{Pause Location}\tabularnewline
\midrule
\endfirsthead

\caption{continued}  \\
\toprule
\textit{\#id} & \textit{Event Type} & \textit{Output} & \textit{Pos.} & \textit{Doc. Len}. & \textit{CP} & \textit{Start Time} & \textit{Start Clock} & \textit{End Time }& \textit{End Clock} & \textit{Act. Time} & \textit{Pause Time} &\textit{Pause Location}\tabularnewline
\midrule
\endhead

\bottomrule
\endfoot

1342 & \verb|keyboard| & S & 4155 & 6702 & 6737 & 284827 & 00:04:44.827 & 284889 & 00:04:44.889 & 265 & 655 & \verb|BEFORE WORDS|
\tabularnewline 
1343 & \verb|keyboard| & o & 4156 & 6703 & 6738 & 285154 & 00:04:45.154 & 285217 & 00:04:45.217 & 63 & 327 & \verb|WITHIN WORDS|
\tabularnewline 
1344 & \verb|keyboard| & m & 4157 & 6704 & 6739 & 285279 & 00:04:45.279 & 285357 & 00:04:45.357 & 78 & 125 & \verb|WITHIN WORDS|
\tabularnewline 
1345 & \verb|keyboard| & s & 4158 & 6705 & 6740 & 285310 & 00:04:45.310 & 285388 & 00:04:45.388 & 78 & 31 & \verb|WITHIN WORDS|
\tabularnewline 
1346 & \verb|keyboard| & SPACE & 4159 & 6706 & 6741 & 285498 & 00:04:45.498 & 285560 & 00:04:45.560 & 62 & 188 & \verb|AFTER WORDS|
\tabularnewline 
1347 & \verb|keyboard| & b & 4160 & 6707 & 6742 & 285685 & 00:04:45.685 & 285747 & 00:04:45.747 & 62 & 187 & \verb|BEFORE WORDS|
\tabularnewline 
1348 & \verb|keyboard| & e & 4161 & 6708 & 6743 & 285825 & 00:04:45.825 & 285888 & 00:04:45.888 & 63 & 140 & \verb|WITHIN WORDS|
\tabularnewline 
1349 & \verb|keyboard| & i & 4162 & 6709 & 6744 & 285903 & 00:04:45.903 & 285966 & 00:04:45.966 & 63 & 78 & \verb|WITHIN WORDS|
\tabularnewline 
1350 & \verb|keyboard| & d & 4163 & 6710 & 6745 & 286044 & 00:04:46.044 & 286122 & 00:04:46.122 & 78 & 141 & \verb|WITHIN WORDS|
\tabularnewline 
1351 & \verb|keyboard| & e & 4164 & 6711 & 6746 & 286231 & 00:04:46.231 & 286278 & 00:04:46.278 & 47 & 187 & \verb|WITHIN WORDS|
\tabularnewline 
1352 & \verb|keyboard| & n & 4165 & 6712 & 6747 & 286356 & 00:04:46.356 & 286418 & 00:04:46.418 & 62 & 125 & \verb|WITHIN WORDS|
\tabularnewline 
1353 & \verb|keyboard| & SPACE & 4166 & 6713 & 6748 & 286590 & 00:04:46.590 & 286636 & 00:04:46.636 & 46 & 234 & \verb|AFTER WORDS|
\tabularnewline 
1368 & \verb|keyboard| & LEFT & 4160 & 6714 & 6749 & 298368 & 00:04:58.368 & 298399 & 00:04:58.399 & 31 & 11778 & \verb|UNKNOWN|
\tabularnewline 
1370 & \verb|keyboard| & RIGHT & 4159 & 6714 & 6749 & 298773 & 00:04:58.773 & 298898 & 00:04:58.898 & 125 & 405 & \verb|UNKNOWN|
\tabularnewline 
1372 & \verb|keyboard| & RIGHT & 4160 & 6714 & 6749 & 298976 & 00:04:58.976 & 299085 & 00:04:59.085 & 109 & 203 & \verb|UNKNOWN|
\tabularnewline 
1374 & \verb|keyboard| & LEFT & 4161 & 6714 & 6749 & 299319 & 00:04:59.319 & 299429 & 00:04:59.429 & 110 & 343 & \verb|UNKNOWN|
\tabularnewline 
1376 & \verb|keyboard| & k & 4160 & 6714 & 6749 & 300505 & 00:05:00.505 & 300567 & 00:05:00.567 & 62 & 1186 & \verb|BEFORE WORDS|
\tabularnewline 
1377 & \verb|keyboard| & r & 4161 & 6715 & 6750 & 301098 & 00:05:01.098 & 301145 & 00:05:01.145 & 47 & 593 & \verb|WITHIN WORDS|
\tabularnewline 
1378 & \verb|keyboard| & i & 4162 & 6716 & 6751 & 301691 & 00:05:01.691 & 301769 & 00:05:01.769 & 78 & 593 & \verb|WITHIN WORDS|
\tabularnewline 
1379 & \verb|keyboard| & j & 4163 & 6717 & 6752 & 301800 & 00:05:01.800 & 301878 & 00:05:01.878 & 78 & 109 & \verb|WITHIN WORDS|
\tabularnewline 
1380 & \verb|keyboard| & g & 4164 & 6718 & 6753 & 302096 & 00:05:02.096 & 302143 & 00:05:02.143 & 47 & 296 & \verb|WITHIN WORDS|
\tabularnewline 
1381 & \verb|keyboard| & t & 4165 & 6719 & 6754 & 302237 & 00:05:02.237 & 302299 & 00:05:02.299 & 62 & 141 & \verb|WITHIN WORDS|
\tabularnewline 
1382 & \verb|keyboard| & SPACE & 4166 & 6720 & 6755 & 302455 & 00:05:02.455 & 302502 & 00:05:02.502 & 47 & 218 & \verb|AFTER WORDS|
\tabularnewline 
1383 & \verb|keyboard| & h & 4167 & 6721 & 6756 & 302736 & 00:05:02.736 & 302798 & 00:05:02.798 & 62 & 281 & \verb|BEFORE WORDS|
\tabularnewline 
1384 & \verb|keyboard| & LEFT & 4168 & 6722 & 6757 & 304483 & 00:05:04.483 & 304545 & 00:05:04.545 & 62 & 1747 & \verb|UNKNOWN|
\tabularnewline 
1386 & \verb|keyboard| & LEFT & 4167 & 6722 & 6757 & 304670 & 00:05:04.670 & 304717 & 00:05:04.717 & 47 & 187 & \verb|UNKNOWN|
\tabularnewline 
1388 & \verb|keyboard| & LEFT & 4166 & 6722 & 6757 & 304811 & 00:05:04.811 & 304857 & 00:05:04.857 & 46 & 141 & \verb|UNKNOWN|
\tabularnewline 
1390 & \verb|keyboard| & LEFT & 4165 & 6722 & 6757 & 304967 & 00:05:04.967 & 305013 & 00:05:05.013 & 46 & 156 & \verb|UNKNOWN|
\tabularnewline 
1392 & \verb|keyboard| & LEFT & 4164 & 6722 & 6757 & 305107 & 00:05:05.107 & 305169 & 00:05:05.169 & 62 & 140 & \verb|UNKNOWN|
\tabularnewline 
1394 & \verb|keyboard| & LEFT & 4163 & 6722 & 6757 & 305263 & 00:05:05.263 & 305325 & 00:05:05.325 & 62 & 156 & \verb|UNKNOWN|
\tabularnewline 
1396 & \verb|keyboard| & LEFT & 4162 & 6722 & 6757 & 305419 & 00:05:05.419 & 305497 & 00:05:05.497 & 78 & 156 & \verb|UNKNOWN|
\tabularnewline 
1398 & \verb|keyboard| & LEFT & 4161 & 6722 & 6757 & 305622 & 00:05:05.622 & 305684 & 00:05:05.684 & 62 & 203 & \verb|UNKNOWN|
\tabularnewline 
1400 & \verb|keyboard| & k & 4160 & 6722 & 6757 & 306480 & 00:05:06.480 & 306527 & 00:05:06.527 & 47 & 858 & \verb|BEFORE WORDS|
\tabularnewline 
1401 & \verb|keyboard| & a & 4161 & 6723 & 6758 & 306558 & 00:05:06.558 & 306651 & 00:05:06.651 & 93 & 78 & \verb|WITHIN WORDS|
\tabularnewline 
1402 & \verb|keyboard| & n & 4162 & 6724 & 6759 & 306667 & 00:05:06.667 & 306729 & 00:05:06.729 & 62 & 109 & \verb|WITHIN WORDS|
\tabularnewline 
1403 & \verb|keyboard| & SPACE & 4163 & 6725 & 6760 & 306839 & 00:05:06.839 & 306885 & 00:05:06.885 & 46 & 172 & \verb|AFTER WORDS|
\tabularnewline 
1404 & \verb|keyboard| & h & 4164 & 6726 & 6761 & 307041 & 00:05:07.041 & 307088 & 00:05:07.088 & 47 & 202 & \verb|BEFORE WORDS|
\tabularnewline 
1405 & \verb|keyboard| & i & 4165 & 6727 & 6762 & 307197 & 00:05:07.197 & 307307 & 00:05:07.307 & 110 & 156 & \verb|WITHIN WORDS|
\tabularnewline 
1406 & \verb|keyboard| & j & 4166 & 6728 & 6763 & 307291 & 00:05:07.291 & 307369 & 00:05:07.369 & 78 & 94 & \verb|WITHIN WORDS|
\tabularnewline 
1407 & \verb|keyboard| & SPACE & 4167 & 6729 & 6764 & 307494 & 00:05:07.494 & 307572 & 00:05:07.572 & 78 & 203 & \verb|AFTER WORDS|
\tabularnewline 
1408 & \verb|keyboard| & m & 4168 & 6730 & 6765 & 307650 & 00:05:07.650 & 307712 & 00:05:07.712 & 62 & 156 & \verb|BEFORE WORDS|
\tabularnewline 
1409 & \verb|keyboard| & e & 4169 & 6731 & 6766 & 307790 & 00:05:07.790 & 307868 & 00:05:07.868 & 78 & 140 & \verb|WITHIN WORDS|
\tabularnewline 
1410 & \verb|keyboard| & e & 4170 & 6732 & 6767 & 307931 & 00:05:07.931 & 308009 & 00:05:08.009 & 78 & 141 & \verb|WITHIN WORDS|
\tabularnewline 
1411 & \verb|keyboard| & r & 4171 & 6733 & 6768 & 308009 & 00:05:08.009 & 308055 & 00:05:08.055 & 46 & 78 & \verb|WITHIN WORDS|
\tabularnewline 
1412 & \verb|keyboard| & SPACE & 4172 & 6734 & 6769 & 308196 & 00:05:08.196 & 308274 & 00:05:08.274 & 78 & 187 & \verb|AFTER WORDS|
\tabularnewline 
1413 & \verb|keyboard| & RIGHT & 4173 & 6735 & 6770 & 309849 & 00:05:09.849 & 309912 & 00:05:09.912 & 63 & 1653 & \verb|UNKNOWN|
\tabularnewline 
1415 & \verb|keyboard| & RIGHT & 4174 & 6735 & 6770 & 310037 & 00:05:10.037 & 310068 & 00:05:10.068 & 31 & 188 & \verb|UNKNOWN|
\tabularnewline 
1417 & \verb|keyboard| & RIGHT & 4175 & 6735 & 6770 & 310161 & 00:05:10.161 & 310271 & 00:05:10.271 & 110 & 124 & \verb|UNKNOWN|
\tabularnewline 
1419 & \verb|keyboard| & RIGHT & 4176 & 6735 & 6770 & 310349 & 00:05:10.349 & 310458 & 00:05:10.458 & 109 & 188 & \verb|UNKNOWN|
\tabularnewline 
1421 & \verb|keyboard| & RIGHT & 4177 & 6735 & 6770 & 310707 & 00:05:10.707 & 310785 & 00:05:10.785 & 78 & 358 & \verb|UNKNOWN|
\tabularnewline 
1423 & \verb|keyboard| & e & 4178 & 6735 & 6770 & 311409 & 00:05:11.409 & 311487 & 00:05:11.487 & 78 & 702 & \verb|BEFORE WORDS|
\tabularnewline 
1424 & \verb|keyboard| & n & 4179 & 6736 & 6771 & 311550 & 00:05:11.550 & 311597 & 00:05:11.597 & 47 & 141 & \verb|WITHIN WORDS|
\tabularnewline 
1425 & \verb|keyboard| & SPACE & 4180 & 6737 & 6772 & 311706 & 00:05:11.706 & 311784 & 00:05:11.784 & 78 & 156 & \verb|AFTER WORDS|
\tabularnewline 
1426 & \verb|keyboard| & d & 4181 & 6738 & 6773 & 311831 & 00:05:11.831 & 311909 & 00:05:11.909 & 78 & 125 & \verb|BEFORE WORDS|
\tabularnewline 
1427 & \verb|keyboard| & a & 4182 & 6739 & 6774 & 311987 & 00:05:11.987 & 312065 & 00:05:12.065 & 78 & 156 & \verb|WITHIN WORDS|
\tabularnewline 
1428 & \verb|keyboard| & n & 4183 & 6740 & 6775 & 312096 & 00:05:12.096 & 312143 & 00:05:12.143 & 47 & 109 & \verb|WITHIN WORDS|
\tabularnewline 
1429 & \verb|keyboard| & SPACE & 4184 & 6741 & 6776 & 312252 & 00:05:12.252 & 312330 & 00:05:12.330 & 78 & 156 & \verb|AFTER WORDS|
\tabularnewline 
1430 & \verb|keyboard| & w & 4185 & 6742 & 6777 & 312392 & 00:05:12.392 & 312486 & 00:05:12.486 & 94 & 140 & \verb|BEFORE WORDS|
\tabularnewline 
1431 & \verb|keyboard| & a & 4186 & 6743 & 6778 & 312548 & 00:05:12.548 & 312626 & 00:05:12.626 & 78 & 156 & \verb|WITHIN WORDS|
\tabularnewline 
1432 & \verb|keyboard| & t & 4187 & 6744 & 6779 & 312626 & 00:05:12.626 & 312689 & 00:05:12.689 & 63 & 78 & \verb|WITHIN WORDS|
\tabularnewline 
1433 & \verb|keyboard| & SPACE & 4188 & 6745 & 6780 & 312829 & 00:05:12.829 & 312907 & 00:05:12.907 & 78 & 203 & \verb|AFTER WORDS|
\tabularnewline 
1434 & \verb|keyboard| & h & 4189 & 6746 & 6781 & 313188 & 00:05:13.188 & 313250 & 00:05:13.250 & 62 & 359 & \verb|BEFORE WORDS|
\tabularnewline 
1435 & \verb|keyboard| & i & 4190 & 6747 & 6782 & 313359 & 00:05:13.359 & 313469 & 00:05:13.469 & 110 & 171 & \verb|WITHIN WORDS|
\tabularnewline 
1436 & \verb|keyboard| & j & 4191 & 6748 & 6783 & 313453 & 00:05:13.453 & 313547 & 00:05:13.547 & 94 & 94 & \verb|WITHIN WORDS|
\tabularnewline 
1437 & \verb|keyboard| & SPACE & 4192 & 6749 & 6784 & 313687 & 00:05:13.687 & 313765 & 00:05:13.765 & 78 & 234 & \verb|AFTER WORDS|
\tabularnewline 
1438 & \verb|keyboard| & v & 4193 & 6750 & 6785 & 313812 & 00:05:13.812 & 313890 & 00:05:13.890 & 78 & 125 & \verb|BEFORE WORDS|
\tabularnewline 
1439 & \verb|keyboard| & o & 4194 & 6751 & 6786 & 313921 & 00:05:13.921 & 313968 & 00:05:13.968 & 47 & 109 & \verb|WITHIN WORDS|
\tabularnewline 
1440 & \verb|keyboard| & o & 4195 & 6752 & 6787 & 314046 & 00:05:14.046 & 314108 & 00:05:14.108 & 62 & 125 & \verb|WITHIN WORDS|
\tabularnewline 
1441 & \verb|keyboard| & r & 4196 & 6753 & 6788 & 314155 & 00:05:14.155 & 314217 & 00:05:14.217 & 62 & 109 & \verb|WITHIN WORDS|
\tabularnewline 
1442 & \verb|keyboard| & SPACE & 4197 & 6754 & 6789 & 314264 & 00:05:14.264 & 314358 & 00:05:14.358 & 94 & 109 & \verb|AFTER WORDS|
\tabularnewline 
1443 & \verb|keyboard| & z & 4198 & 6755 & 6790 & 314451 & 00:05:14.451 & 314529 & 00:05:14.529 & 78 & 187 & \verb|BEFORE WORDS|
\tabularnewline 
1444 & \verb|keyboard| & o & 4199 & 6756 & 6791 & 314592 & 00:05:14.592 & 314654 & 00:05:14.654 & 62 & 141 & \verb|WITHIN WORDS|
\tabularnewline 
1445 & \verb|keyboard| & n & 4200 & 6757 & 6792 & 314810 & 00:05:14.810 & 314857 & 00:05:14.857 & 47 & 218 & \verb|WITHIN WORDS|
\tabularnewline 
1446 & \verb|keyboard| & ' & 4201 & 6758 & 6793 & 314919 & 00:05:14.919 & 314966 & 00:05:14.966 & 47 & 109 & \verb|WITHIN WORDS|
\tabularnewline 
1447 & \verb|keyboard| & SPACE & 4202 & 6759 & 6794 & 315075 & 00:05:15.075 & 315138 & 00:05:15.138 & 63 & 156 & \verb|AFTER WORDS|
\tabularnewline 
1448 & \verb|keyboard| & k & 4203 & 6760 & 6795 & 315543 & 00:05:15.543 & 315606 & 00:05:15.606 & 63 & 468 & \verb|BEFORE WORDS|
\tabularnewline 
1449 & \verb|keyboard| & u & 4204 & 6761 & 6796 & 315746 & 00:05:15.746 & 315793 & 00:05:15.793 & 47 & 203 & \verb|WITHIN WORDS|
\tabularnewline 
1450 & \verb|keyboard| & n & 4205 & 6762 & 6797 & 315918 & 00:05:15.918 & 315980 & 00:05:15.980 & 62 & 172 & \verb|WITHIN WORDS|
\tabularnewline 
1451 & \verb|keyboard| & s & 4206 & 6763 & 6798 & 316027 & 00:05:16.027 & 316105 & 00:05:16.105 & 78 & 109 & \verb|WITHIN WORDS|
\tabularnewline 
1452 & \verb|keyboard| & t & 4207 & 6764 & 6799 & 316152 & 00:05:16.152 & 316183 & 00:05:16.183 & 31 & 125 & \verb|WITHIN WORDS|
\tabularnewline 
1453 & \verb|keyboard| & w & 4208 & 6765 & 6800 & 316277 & 00:05:16.277 & 316355 & 00:05:16.355 & 78 & 125 & \verb|WITHIN WORDS|
\tabularnewline 
1454 & \verb|keyboard| & e & 4209 & 6766 & 6801 & 316526 & 00:05:16.526 & 316620 & 00:05:16.620 & 94 & 249 & \verb|WITHIN WORDS|
\tabularnewline 
1455 & \verb|keyboard| & r & 4210 & 6767 & 6802 & 316589 & 00:05:16.589 & 316667 & 00:05:16.667 & 78 & 63 & \verb|WITHIN WORDS|
\tabularnewline 
1456 & \verb|keyboard| & k & 4211 & 6768 & 6803 & 316760 & 00:05:16.760 & 316823 & 00:05:16.823 & 63 & 171 & \verb|WITHIN WORDS|
\tabularnewline 
1457 & \verb|keyboard| & j & 4212 & 6769 & 6804 & 316916 & 00:05:16.916 & 316963 & 00:05:16.963 & 47 & 156 & \verb|WITHIN WORDS|
\tabularnewline 
1458 & \verb|keyboard| & e & 4213 & 6770 & 6805 & 317025 & 00:05:17.025 & 317088 & 00:05:17.088 & 63 & 109 & \verb|WITHIN WORDS|
\tabularnewline 
1459 & \verb|keyboard| & SPACE & 4214 & 6771 & 6806 & 317150 & 00:05:17.150 & 317213 & 00:05:17.213 & 63 & 125 & \verb|AFTER WORDS|
\tabularnewline 
1460 & \verb|keyboard| & v & 4215 & 6772 & 6807 & 317369 & 00:05:17.369 & 317415 & 00:05:17.415 & 46 & 219 & \verb|BEFORE WORDS|
\tabularnewline 
1461 & \verb|keyboard| & r & 4216 & 6773 & 6808 & 317540 & 00:05:17.540 & 317587 & 00:05:17.587 & 47 & 171 & \verb|WITHIN WORDS|
\tabularnewline 
1462 & \verb|keyboard| & a & 4217 & 6774 & 6809 & 317634 & 00:05:17.634 & 317696 & 00:05:17.696 & 62 & 94 & \verb|WITHIN WORDS|
\tabularnewline 
1463 & \verb|keyboard| & a & 4218 & 6775 & 6810 & 317790 & 00:05:17.790 & 317868 & 00:05:17.868 & 78 & 156 & \verb|WITHIN WORDS|
\tabularnewline 
1464 & \verb|keyboard| & g & 4219 & 6776 & 6811 & 317961 & 00:05:17.961 & 318008 & 00:05:18.008 & 47 & 171 & \verb|WITHIN WORDS|
\tabularnewline 
1465 & \verb|keyboard| & t & 4220 & 6777 & 6812 & 318180 & 00:05:18.180 & 318211 & 00:05:18.211 & 31 & 219 & \verb|WITHIN WORDS|
\tabularnewline 
1504 & \verb|keyboard| & LEFT & 4202 & 6778 & 6813 & 321581 & 00:05:21.581 & 321596 & 00:05:21.596 & 15 & 3401 & \verb|UNKNOWN|
\tabularnewline 
1506 & \verb|keyboard| & LEFT & 4201 & 6778 & 6813 & 321908 & 00:05:21.908 & 321986 & 00:05:21.986 & 78 & 327 & \verb|UNKNOWN|
\tabularnewline 
1508 & \verb|keyboard| & DELETE & 4200 & 6777 & 6813 & 322298 & 00:05:22.298 & 322329 & 00:05:22.329 & 31 & 390 & \verb|REVISION|
\tabularnewline 
1509 & \verb|replacement| & [4200:4201] & 4200 & 6777 & 6813 & 322298 & 00:05:22.298 & 322329 & 00:05:22.329 & 0 & 0 & \verb|CHANGE|
\tabularnewline 
1511 & \verb|keyboard| & RIGHT & 4200 & 6777 & 6813 & 322517 & 00:05:22.517 & 322595 & 00:05:22.595 & 78 & 219 & \verb|UNKNOWN|
\tabularnewline 
1513 & \verb|keyboard| & n & 4201 & 6777 & 6813 & 323546 & 00:05:23.546 & 323609 & 00:05:23.609 & 63 & 1029 & \verb|BEFORE WORDS|
\tabularnewline 
1546 & \verb|keyboard| & RIGHT & 4218 & 6778 & 6814 & 324903 & 00:05:24.903 & 324935 & 00:05:24.935 & 32 & 1357 & \verb|UNKNOWN|
\tabularnewline 
1548 & \verb|keyboard| & RIGHT & 4219 & 6778 & 6814 & 325231 & 00:05:25.231 & 325309 & 00:05:25.309 & 78 & 328 & \verb|UNKNOWN|
\tabularnewline 
1550 & \verb|keyboard| & RIGHT & 4220 & 6778 & 6814 & 325481 & 00:05:25.481 & 325496 & 00:05:25.496 & 15 & 250 & \verb|UNKNOWN|
\tabularnewline 
1552 & \verb|keyboard| & DELETE & 4221 & 6777 & 6814 & 325949 & 00:05:25.949 & 325995 & 00:05:25.995 & 46 & 468 & \verb|REVISION|
\tabularnewline 
1553 & \verb|replacement| & [4221:4222] & 4221 & 6777 & 6814 & 325949 & 00:05:25.949 & 325995 & 00:05:25.995 & 0 & 0 & \verb|CHANGE|
\tabularnewline 
1555 & \verb|keyboard| & SPACE & 4221 & 6777 & 6814 & 326448 & 00:05:26.448 & 326541 & 00:05:26.541 & 93 & 499 & \verb|AFTER WORDS|
\tabularnewline 
1556 & \verb|keyboard| & DELETE & 4222 & 6777 & 6815 & 326838 & 00:05:26.838 & 326916 & 00:05:26.916 & 78 & 390 & \verb|REVISION|
\tabularnewline 
1557 & \verb|replacement| & [4222:4223] & 4222 & 6777 & 6815 & 326838 & 00:05:26.838 & 326916 & 00:05:26.916 & 0 & 0 & \verb|CHANGE|
\tabularnewline 
1559 & \verb|keyboard| & DELETE & 4222 & 6776 & 6815 & 327041 & 00:05:27.041 & 327103 & 00:05:27.103 & 62 & 203 & \verb|REVISION|
\tabularnewline 
1560 & \verb|replacement| & [4222:4223] & 4222 & 6776 & 6815 & 327041 & 00:05:27.041 & 327103 & 00:05:27.103 & 0 & 0 & \verb|CHANGE|
\tabularnewline 
1562 & \verb|keyboard| & DELETE & 4222 & 6775 & 6815 & 327181 & 00:05:27.181 & 327181 & 00:05:27.181 & 0 & 140 & \verb|REVISION|
\tabularnewline 
1563 & \verb|replacement| & [4222:4223] & 4222 & 6775 & 6815 & 327181 & 00:05:27.181 & 327181 & 00:05:27.181 & 0 & 0 & \verb|CHANGE|
\tabularnewline 
1565 & \verb|keyboard| & DELETE & 4222 & 6774 & 6815 & 327696 & 00:05:27.696 & 327696 & 00:05:27.696 & 0 & 515 & \verb|REVISION|
\tabularnewline 
1566 & \verb|replacement| & [4222:4223] & 4222 & 6774 & 6815 & 327696 & 00:05:27.696 & 327696 & 00:05:27.696 & 0 & 0 & \verb|CHANGE|
\tabularnewline 
1568 & \verb|keyboard| & DELETE & 4222 & 6773 & 6815 & 327727 & 00:05:27.727 & 327727 & 00:05:27.727 & 0 & 31 & \verb|REVISION|
\tabularnewline 
1569 & \verb|replacement| & [4222:4223] & 4222 & 6773 & 6815 & 327727 & 00:05:27.727 & 327727 & 00:05:27.727 & 0 & 0 & \verb|CHANGE|
\tabularnewline 
1571 & \verb|keyboard| & DELETE & 4222 & 6772 & 6815 & 327758 & 00:05:27.758 & 327758 & 00:05:27.758 & 0 & 31 & \verb|REVISION|
\tabularnewline 
1572 & \verb|replacement| & [4222:4223] & 4222 & 6772 & 6815 & 327758 & 00:05:27.758 & 327758 & 00:05:27.758 & 0 & 0 & \verb|CHANGE|
\tabularnewline 
1574 & \verb|keyboard| & DELETE & 4222 & 6771 & 6815 & 327789 & 00:05:27.789 & 327789 & 00:05:27.789 & 0 & 31 & \verb|REVISION|
\tabularnewline 
1575 & \verb|replacement| & [4222:4223] & 4222 & 6771 & 6815 & 327789 & 00:05:27.789 & 327789 & 00:05:27.789 & 0 & 0 & \verb|CHANGE|
\tabularnewline 
1577 & \verb|keyboard| & DELETE & 4222 & 6770 & 6815 & 327836 & 00:05:27.836 & 327836 & 00:05:27.836 & 0 & 47 & \verb|REVISION|
\tabularnewline 
1578 & \verb|replacement| & [4222:4223] & 4222 & 6770 & 6815 & 327836 & 00:05:27.836 & 327836 & 00:05:27.836 & 0 & 0 & \verb|CHANGE|
\tabularnewline 
1580 & \verb|keyboard| & DELETE & 4222 & 6769 & 6815 & 327867 & 00:05:27.867 & 327899 & 00:05:27.899 & 32 & 31 & \verb|REVISION|
\tabularnewline 
1581 & \verb|replacement| & [4222:4223] & 4222 & 6769 & 6815 & 327867 & 00:05:27.867 & 327899 & 00:05:27.899 & 0 & 0 & \verb|CHANGE|
\tabularnewline 
1583 & \verb|keyboard| & RETURN & 4222 & 6769 & 6815 & 328476 & 00:05:28.476 & 328569 & 00:05:28.569 & 93 & 609 & \verb|BEFORE PARAGRAPHS|
\tabularnewline 
1584 & \verb|keyboard| & RETURN & 4223 & 6770 & 6816 & 328647 & 00:05:28.647 & 328710 & 00:05:28.710 & 63 & 171 & \verb|BEFORE WORDS|
\tabularnewline 
1585 & \verb|keyboard| & LEFT & 4224 & 6771 & 6817 & 329147 & 00:05:29.147 & 329240 & 00:05:29.240 & 93 & 500 & \verb|UNKNOWN|
\tabularnewline 
1587 & \verb|keyboard| & LEFT & 4223 & 6771 & 6817 & 329334 & 00:05:29.334 & 329396 & 00:05:29.396 & 62 & 187 & \verb|UNKNOWN|
\tabularnewline 
1589 & \verb|keyboard| & LEFT & 4222 & 6771 & 6817 & 329490 & 00:05:29.490 & 329584 & 00:05:29.584 & 94 & 156 & \verb|UNKNOWN|
\tabularnewline 
1591 & \verb|keyboard| & LEFT & 4221 & 6771 & 6817 & 329693 & 00:05:29.693 & 329771 & 00:05:29.771 & 78 & 203 & \verb|UNKNOWN|
\tabularnewline 
1593 & \verb|keyboard| & RIGHT & 4220 & 6771 & 6817 & 330020 & 00:05:30.020 & 330130 & 00:05:30.130 & 110 & 327 & \verb|UNKNOWN|
\tabularnewline 
1595 & \verb|keyboard| & , & 4221 & 6771 & 6817 & 330847 & 00:05:30.847 & 330894 & 00:05:30.894 & 47 & 827 & \verb|AFTER WORDS|
\tabularnewline 
1596 & \verb|keyboard| & SPACE & 4222 & 6772 & 6818 & 330988 & 00:05:30.988 & 331050 & 00:05:31.050 & 62 & 141 & \verb|AFTER WORDS|
\tabularnewline 
1597 & \verb|keyboard| & m & 4223 & 6773 & 6819 & 331175 & 00:05:31.175 & 331237 & 00:05:31.237 & 62 & 187 & \verb|BEFORE WORDS|
\tabularnewline 
1598 & \verb|keyboard| & a & 4224 & 6774 & 6820 & 331206 & 00:05:31.206 & 331284 & 00:05:31.284 & 78 & 31 & \verb|WITHIN WORDS|
\tabularnewline 
1599 & \verb|keyboard| & a & 4225 & 6775 & 6821 & 331362 & 00:05:31.362 & 331456 & 00:05:31.456 & 94 & 156 & \verb|WITHIN WORDS|
\tabularnewline 
1600 & \verb|keyboard| & r & 4226 & 6776 & 6822 & 331518 & 00:05:31.518 & 331565 & 00:05:31.565 & 47 & 156 & \verb|WITHIN WORDS|
\tabularnewline 
1601 & \verb|keyboard| & SPACE & 4227 & 6777 & 6823 & 331736 & 00:05:31.736 & 331799 & 00:05:31.799 & 63 & 218 & \verb|AFTER WORDS|
\tabularnewline 
1602 & \verb|keyboard| & d & 4228 & 6778 & 6824 & 331830 & 00:05:31.830 & 331892 & 00:05:31.892 & 62 & 94 & \verb|BEFORE WORDS|
\tabularnewline 
1603 & \verb|keyboard| & a & 4229 & 6779 & 6825 & 331986 & 00:05:31.986 & 332064 & 00:05:32.064 & 78 & 156 & \verb|WITHIN WORDS|
\tabularnewline 
1604 & \verb|keyboard| & t & 4230 & 6780 & 6826 & 332111 & 00:05:32.111 & 332142 & 00:05:32.142 & 31 & 125 & \verb|WITHIN WORDS|
\tabularnewline 
1605 & \verb|keyboard| & SPACE & 4231 & 6781 & 6827 & 332282 & 00:05:32.282 & 332345 & 00:05:32.345 & 63 & 171 & \verb|AFTER WORDS|
\tabularnewline 
1606 & \verb|keyboard| & i & 4232 & 6782 & 6828 & 332579 & 00:05:32.579 & 332626 & 00:05:32.626 & 47 & 297 & \verb|BEFORE WORDS|
\tabularnewline 
1607 & \verb|keyboard| & l & 4233 & 6783 & 6829 & 332688 & 00:05:32.688 & 332766 & 00:05:32.766 & 78 & 109 & \verb|WITHIN WORDS|
\tabularnewline 
1608 & \verb|keyboard| & SPACE & 4234 & 6784 & 6830 & 332906 & 00:05:32.906 & 332938 & 00:05:32.938 & 32 & 218 & \verb|AFTER WORDS|
\tabularnewline 
1609 & \verb|keyboard| & h & 4235 & 6785 & 6831 & 333203 & 00:05:33.203 & 333250 & 00:05:33.250 & 47 & 297 & \verb|BEFORE WORDS|
\tabularnewline 
1610 & \verb|keyboard| & i & 4236 & 6786 & 6832 & 333374 & 00:05:33.374 & 333515 & 00:05:33.515 & 141 & 171 & \verb|WITHIN WORDS|
\tabularnewline 
1611 & \verb|keyboard| & j & 4237 & 6787 & 6833 & 333484 & 00:05:33.484 & 333577 & 00:05:33.577 & 93 & 110 & \verb|WITHIN WORDS|
\tabularnewline 
1612 & \verb|keyboard| & SPACE & 4238 & 6788 & 6834 & 333702 & 00:05:33.702 & 333749 & 00:05:33.749 & 47 & 218 & \verb|AFTER WORDS|
\tabularnewline 
1613 & \verb|keyboard| & n & 4239 & 6789 & 6835 & 333889 & 00:05:33.889 & 333952 & 00:05:33.952 & 63 & 187 & \verb|BEFORE WORDS|
\tabularnewline 
1614 & \verb|keyboard| & i & 4240 & 6790 & 6836 & 333998 & 00:05:33.998 & 334076 & 00:05:34.076 & 78 & 109 & \verb|WITHIN WORDS|
\tabularnewline 
1615 & \verb|keyboard| & e & 4241 & 6791 & 6837 & 334123 & 00:05:34.123 & 334186 & 00:05:34.186 & 63 & 125 & \verb|WITHIN WORDS|
\tabularnewline 
1616 & \verb|keyboard| & t & 4242 & 6792 & 6838 & 334279 & 00:05:34.279 & 334342 & 00:05:34.342 & 63 & 156 & \verb|WITHIN WORDS|
\tabularnewline 
1637 & \verb|keyboard| & LEFT & 4233 & 6793 & 6839 & 336775 & 00:05:36.775 & 336791 & 00:05:36.791 & 16 & 2496 & \verb|UNKNOWN|
\tabularnewline 
1639 & \verb|keyboard| & LEFT & 4232 & 6793 & 6839 & 337025 & 00:05:37.025 & 337087 & 00:05:37.087 & 62 & 250 & \verb|UNKNOWN|
\tabularnewline 
1641 & \verb|keyboard| & RIGHT & 4231 & 6793 & 6839 & 337399 & 00:05:37.399 & 337508 & 00:05:37.508 & 109 & 374 & \verb|UNKNOWN|
\tabularnewline 
1643 & \verb|keyboard| & w & 4232 & 6793 & 6839 & 338351 & 00:05:38.351 & 338413 & 00:05:38.413 & 62 & 952 & \verb|BEFORE WORDS|
\tabularnewline 
1644 & \verb|keyboard| & RIGHT & 4233 & 6794 & 6840 & 341580 & 00:05:41.580 & 341658 & 00:05:41.658 & 78 & 3229 & \verb|UNKNOWN|
\tabularnewline 
1646 & \verb|keyboard| & RIGHT & 4234 & 6794 & 6840 & 341767 & 00:05:41.767 & 341767 & 00:05:41.767 & 0 & 187 & \verb|UNKNOWN|
\tabularnewline 
1648 & \verb|keyboard| & RIGHT & 4235 & 6794 & 6840 & 342266 & 00:05:42.266 & 342266 & 00:05:42.266 & 0 & 499 & \verb|UNKNOWN|
\tabularnewline 
1650 & \verb|keyboard| & RIGHT & 4236 & 6794 & 6840 & 342298 & 00:05:42.298 & 342298 & 00:05:42.298 & 0 & 32 & \verb|UNKNOWN|
\tabularnewline 
1652 & \verb|keyboard| & RIGHT & 4237 & 6794 & 6840 & 342344 & 00:05:42.344 & 342344 & 00:05:42.344 & 0 & 46 & \verb|UNKNOWN|
\tabularnewline 
1654 & \verb|keyboard| & RIGHT & 4238 & 6794 & 6840 & 342376 & 00:05:42.376 & 342376 & 00:05:42.376 & 0 & 32 & \verb|UNKNOWN|
\tabularnewline 
1656 & \verb|keyboard| & RIGHT & 4239 & 6794 & 6840 & 342407 & 00:05:42.407 & 342407 & 00:05:42.407 & 0 & 31 & \verb|UNKNOWN|
\tabularnewline 
1658 & \verb|keyboard| & RIGHT & 4240 & 6794 & 6840 & 342438 & 00:05:42.438 & 342438 & 00:05:42.438 & 0 & 31 & \verb|UNKNOWN|
\tabularnewline 
1660 & \verb|keyboard| & RIGHT & 4241 & 6794 & 6840 & 342485 & 00:05:42.485 & 342485 & 00:05:42.485 & 0 & 47 & \verb|UNKNOWN|
\tabularnewline 
1662 & \verb|keyboard| & RIGHT & 4242 & 6794 & 6840 & 342516 & 00:05:42.516 & 342547 & 00:05:42.547 & 31 & 31 & \verb|UNKNOWN|
\tabularnewline 
1664 & \verb|keyboard| & RIGHT & 4243 & 6794 & 6840 & 343000 & 00:05:43 & 343109 & 00:05:43.109 & 109 & 484 & \verb|UNKNOWN|
\tabularnewline 
1666 & \verb|keyboard| & . & 4244 & 6794 & 6840 & 344216 & 00:05:44.216 & 344279 & 00:05:44.279 & 63 & 1216 & \verb|AFTER WORDS|

\end{longtable}
\end{center}
\end{subappendices}

\begin{flushleft}
\bibliography{references/bekius}
\end{flushleft}
\end{paper}